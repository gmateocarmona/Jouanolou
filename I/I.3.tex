%%%%%%%%%%%%%%%%%%%%%%%%%%%%%%%%%%%%
\subsection*{3. $A$-faisceaux de type constant, strict ou $J$-adique.}
\addcontentsline{toc}{subsection}{3. $A$-faisceaux de type constant, strict ou $J$-adique}

Soit $(X, A, J)$ un idéotope.
\vskip .3cm
{
Proposition {\bf 3.1}. --- \it Étant donné un $A$-faisceau $F$ sur $X$, les assertions suivantes sont équivalentes :
\begin{itemize}
    \item[(i)] $F$ est isomorphe (dans $A-\fsc(X)$) à un $A$-faisceau qui est un système projectif strict.
    \item[(ii)] $F$ est localement isomorphe à un $A$-faisceau qui est un système projectif strict.
    \item[(iii)] $F$ vérifie localement la condition de Mittag-Leffler (ML) (EGA$0_{III}$ 13.1.2).
\end{itemize}
}
\vskip .3cm
{\bf Preuve}: Il est clair que (i) $\Rightarrow$ (ii). On voit que (ii) $\Rightarrow$ (iii) en paraphrasant la preuve de (SGA5 V 2.5.1). Si maintenant (iii) est vérifiée, on peut parler localement du système projectif des images universelles de $F$ et ces divers systèmes projectifs d'images universelles se recollent pour donner un $A$-faisceau qui est un sous-système projectif strict $F'$ de $F$. Alors, paraphrasant toujours (SGA5 V 2.5.1), on voit que l'inclusion de $F'$ dans $F$ est localement un isomorphisme, et on conclut par (1.7 (iii)).
\vskip .3cm
{
Définition {\bf 3.2}. --- \it Un $A$-faisceau $F$ qui vérifie les conditions équivalentes de (3.1) est appelé \emph{de type strict}.
}
\vskip .3cm
Si $F$ est de type strict, nous avons vu dans le courant de la démonstration que pour tout entier $n \geq 0$, le système projectif 
$$
(\text{Im}(F_m \to F_n))_{m \geq n}
$$
admet localement, donc aussi globalement, une limite projective $F'_n$ qui s'identifie à un sous-faisceau de $F_n$, et nous avons ainsi défini un sous-$A$-faisceau $F'$ de $F$.
\vskip .3cm
{
Définition {\bf 3.3}. --- \it Si $F$ est un $A$-faisceau de type strict, le $A$-faisceau $F'$ défini ci-dessus est appelé \emph{$A$-faisceau strict associé à $F$}.
}
\vskip .3cm
Rappelons enfin que nous avons vu que l'inclusion canonique $F' \hookrightarrow F$ est un \emph{isomorphisme}.
\vskip .3cm
{
Proposition {\bf 3.4}. --- \it Soit $0 \to E' \to E \to E'' \to 0$ une suite exacte de $A-\fsc(X)$. Alors :
\begin{itemize}
    \item[(i)] Si $E$ est de type strict, $E''$ est de type strict. 
    \item[(ii)] Si $E'$ et $E''$ sont de type strict, il en est de même de $E$.
\end{itemize}
}
\vskip .3cm
{\bf Preuve}: On peut supposer que la suite exacte considérée est l'image d'une suite exacte de $\mathcal{E}(X, J)$ et alors l'énoncé résulte de l'application locale de (EGA $0_{III}$ 13.2.1).
\vskip .3cm
{\bf 3.5}. Convenant de noter 
$$
J-\Mod(X)
$$
la sous-catégorie, \emph{exacte} (i.e. stable par noyaux, conoyaux et extensions), et même épaisse, pleine de $A-\Mod_X$ engendrée par les $A$--Modules localement annulés par une puissance de $J$, associons à tout objet $M$ de $J-\Mod_X$ le $A$-faisceau localement essentiellement constant
$$
\overline{M} = \mathscr{E}_X (M) = (M/J^{n+1}M)_{n \in \mathbf{N}}.
$$
On définit ainsi de fa\c{c}on claire un foncteur additif
$$
J-\Mod(X) \to \mathcal{E}(X, J),
$$
d'où, en composant avec la projection canonique, un foncteur additif
$$
\mathcal{E}_X: J-\Mod(X) \to A-\fsc(X).
\leqno{(3.5.1)}
$$
\vskip .3cm
{
Proposition {\bf 3.6}. --- \it 
\begin{itemize}
    \item[1)] Le foncteur (3.5.1) ci-dessus est exact et pleinement fidèle. Son image essentielle
    $$
    \TC (X, J)
    $$
    est une sous-catégorie abélienne exacte de $A-\fsc(X)$.
    \item[2)] Les assertions suivantes pour un $A$-faisceau $F$ sur $X$ sont équivalentes :
    \begin{itemize}
        \item[(i)] $F$ appartient à l'image essentielle de (3.5.1).
        \item[(ii)] $F$ est isomorphe à un $A$-faisceau localement essentiellement constant.
        \item[(iii)] $F$ est localement isomorphe à un $A$-faisceau essentiellement constant.
    \end{itemize}
\end{itemize}
Un $A$-faisceau vérifiant ces conditions équivalentes est dit \emph{de type constant}. Un $A$-faisceau de type constant est de type strict. La propriété pour un $A$-faisceau d'être de type constant est de nature locale.
}
\vskip .3cm
{\bf Preuve}: 1) Soit $0 \to M \xlongrightarrow{u} N \xlongrightarrow{v} P \to 0$ une suite exacte de $J-\Mod_X$, d'où une suite
$$
0 \to \overline{M} \xlongrightarrow{\overline{u}} \overline{N} \xlongrightarrow{\overline{v}} \overline{P} \to 0
$$
de $\mathcal{E}(X, J)$. Comme les systèmes projectifs $\overline{M}, \overline{N}, \overline{P}$ sont localement essentiellement constants, on voit que, localement, pour $n$ assez grand, 
$$
(\Ker \overline{u})_n = (\Coker \overline{v})_n = (\Ker \overline{v} / \text{Im} \overline{u})_n = 0,
$$
donc les systèmes projectifs $\Ker (\overline{u})$, $\Coker (\overline{v})$, $\Ker (\overline{v})/\text{Im} (\overline{u})$ sont négligeables, d'où l'exactitude de (3.5.1). Montrons qu'il est fidèle. Soit $u: M \to N$ un morphisme de $J-\Mod_X$ tel que $\overline{u}$ définisse le morphisme nul de $A-\fsc(X)$ et montrons que $u = 0$. Dans $\mathcal{E}(X, J)$, le $A$-faisceau $\text{Im}(\overline{u})$ est négligeable ; l'assertion à prouver étant locale, on peut supposer qu'il est essentiellement nul et que $M$ et $N$ sont annulés par une puissance de $J$. Alors, comme $\text{Im}(\overline{u})$ est contenu dans le système projectif essentiellement constant $\overline{N}$, on a évidemment $(\text{Im}(\overline{u}))_n = 0$ pour $n$ assez grand. Mais, toujours pour $n$ assez grand, on a $(\overline{u})_n = 0$, d'où l'assertion.

Soit maintenant $\lambda: \overline{M} \to \overline{N}$ un morphisme de $A$-faisceaux et montrons qu'il existe un morphisme $u: M \to N$ tel que $\lambda = \overline{u}$ dans $A-\fsc(X)$. Comme on sait déjà que (3.5.1) est fidèle, il résulte de (1.7.2) que cette assertion est de nature locale. Par définition (Thèse Gabriel III 1), le morphisme $\lambda$ est la classe d'un triple $(\alpha, \beta, v)$ de $\mathcal{E}(X, J)$-morphismes (cf. diagramme)
\[\begin{tikzcd}
	{\overline{M}} && F \\
	E && {\overline{N},}
	\arrow["\beta"', from=2-3, to=1-3]
	\arrow["v", from=2-1, to=1-3]
	\arrow["\alpha", from=2-1, to=1-1]
\end{tikzcd}\]
avec $\alpha$ un monomorphisme, $\beta$ un épimorphisme et $\Coker(\alpha)$ et $\Ker (\beta)$ négligeables. D'après le caractère local de l'assertion à démontrer, on peut supposer qu'ils sont essentiellement nuls, et que $M$ et $N$ sont annulés par une puissance de $J$. Comme précédemment, on a alors $(\Ker \beta)_n = 0$ pour $n$ assez grand ; de plus, comme $\overline{M}$ est strict, $\alpha$ est un isomorphisme. On vérifie aussitôt que pour $n$ assez grand, le morphisme
$$
(\beta \circ v \circ \alpha^{-1})_n : (\overline{M})_n = M \to N = (\overline{N})_n
$$
ne dépend pas de $n$. Désignant par $u$ la valeur commune, on voit alors sans peine que $\lambda = \overline{u}$. La peine fidélité de (3.5.1) montre que la catégorie $\TC (X, J)$ est stable par noyaux et conoyaux dans $A-\fsc(X)$. Nous verrons la stabilité par extensions après avoir prouvé la deuxième partie.

2) Il est clair que (i) $\Rightarrow$ (ii) $\Rightarrow$ (iii). Montrons que (iii) $\Rightarrow$ (i). Faisons d'abord quelques préliminaires. Étant donné un $A$-faisceau $F$ on définit un nouveau $A$-faisceau $\overline{F}$ par
$$
(\overline{F})_n = (\varprojlim_m (F_m))/J^{n+1}(\varprojlim_m(F_m)),
$$
avec les morphismes de transition évidents. De plus, on a, de fa\c{c}on évidente un $\mathcal{E}(X, J)$-morphisme fonctoriel en $F$
$$
u_F : \overline{F} \to F.
$$
On vérifie sans peine que si $\alpha: F \to G$ est un $\mathcal{E}(X, J)$-morphisme dont le noyau et le conoyau sont négligeables, alors le morphisme correspondant $\overline{\alpha}: \overline{F} \to \overline{G}$ est un isomorphisme. Ceci permet de définir un $A-\fsc(X)$-morphisme, fonctoriel en $F$ lorsque ce dernier parcourt $A-\fsc(X)$,
$$
u_F : \overline{F} \to F.
\leqno{(3.6.1)}
$$
Supposons maintenant que $F$ soit localement dans l'image essentielle de (3.5.1) et montrons qu'il y est globalement. Étant donnés un objet $T$ de $X$, un objet $M$ de $J-\Mod_T$ et un $A-\fsc(T)$-isomorphisme
$$
F|T \isomlong \mathcal{E}_T(M),
$$
on voit sans peine que 
$$
(\varprojlim_m F_m) | T \isom M,
$$
et par suite $\varprojlim_m (F_m)$ est un objet de $J-\Mod_X$. On aura terminé si on preuve que $u_F$ définit un isomorphisme de $A-\fsc(X)$. C'est immédiat si $F = \mathscr{E}_X(M)$ ; dans le cas général, l'assertion étant locale (1.7), on peut supposer donné un isomorphisme
$$
i: F \isomlong \mathscr{E}_X(M) \quad \text{de} \quad A-\fsc(X)
$$
et alors l'assertion résulte de la commutativité du diagramme
\[\begin{tikzcd}
	{\overline{G}} && {\mathscr{E}_X(M) = G} \\
	{\overline{F}} && F
	\arrow["\sim"', from=1-1, to=1-3]
	\arrow["{u_G}", from=1-1, to=1-3]
	\arrow["{u_F}", from=2-1, to=2-3]
	\arrow["{\overline{i}}", from=2-1, to=1-1]
	\arrow["\sim"', from=2-1, to=1-1]
	\arrow["i"', from=2-3, to=1-3]
	\arrow["\sim", from=2-3, to=1-3]
\end{tikzcd}\]
dans $A-\fsc(X)$.

Montrons maintenant que $\TC(X, J)$ est stable par extension dans $A-\fsc(X)$. On se ramène immédiatement à voir que si 
$$
0 \to F' \xlongrightarrow{\alpha} F \xlongrightarrow{\beta} F'' \to 0
$$
est une suite exacte de $\mathscr{E}(X, J)$, avec $F'$ et $F''$ définissant des objets de $\TC(X, J)$, il en est de même pour $F$. L'assertion étant locale d'après l'équivalence (i) $\Leftrightarrow$ (iii), on peut supposer que 
$$
F' \isom \mathscr{E}_X(M') \quad \text{et} \quad F'' \isom \mathscr{E}_X(M'')
$$
dans $A-\fsc(X)$, avec $M'$ et $M''$ annulés par une puissance de $J$. Avec les notations utilisés dans la preuve de 2), on a $\overline{F}' \isom \overline{M}'$ et le morphisme canonique $u_{F'}: \overline{F}' \to F'$ est à noyau et conoyau négligeables et, quitte à localiser, on peut les supposer essentiellement nuls. Dans ce cas, le $A$-faisceau $\Ker(u_{F'})$ vérifie, puisque $\overline{F}'$ est essentiellement constant,
$$
(\Ker u_{F'})_n = 0 \quad \text{pour}~n~\text{assez grand}.
$$
En particulier, $\text{Im}(u_{F'})$ est essentiellement constant. Quitte à remplacer $F''$ par $F/\text{Im}(\alpha \circ u_{F'})$, on peut donc supposer que $F'$ est essentiellement constant. Ceci dit, quitte à remplacer $F''$ par $\overline{F}''$ et $F$ par $\overline{F}'' \times_{F''} F$, on peut de plus supposer que $F''$ est aussi essentiellement constant. Dans ce cas, $F$ l'est également, d'où l'assertion.
\vskip .3cm
{
Corollaire {\bf 3.7}. --- \it La sous-catégorie pleine $\TC(X, J)^+$ de $A-\fsc(X)$ engendrée par les $A$-faisceaux isomorphes à un $A$-faisceau de la forme $\mathcal{E}_X(M)$, avec $M$ un $A_X$--Module annulé par une puissance de $J$, est \emph{exacte}.
}
\vskip .3cm
{\bf Preuve} : Les objets de $\TC(X, J)^+$ sont ceux de $\TC(X, J)$ qui sont annulés par une puissance de l'idéal $J$.
\vskip .3cm
{\bf 3.8}. Soit $J-\ad(X)$ la sous-catégorie pleine de $\mathcal{E}(X, J)$ engendrée par les systèmes projectifs $J$-adiques (SGA5 V 3.1.1) de $A_X$--Modules. Le foncteur canonique
$$
\mathcal{E}(X, F) \to A-\fsc(X)
\leqno{(3.8.1)}
$$
induit un foncteur
$$
J-\ad(X) \to A-\fsc(X)
\leqno{(3.8.2)}
$$
et nous noterons 
$$
\TJ-\ad(X)
$$
l'image essentielle de (3.8.1), i.e. la sous-catégorie pleine de $A-\fsc(X)$ engendrée par les $A$-faisceaux qui sont isomorphes à un $A$-faisceau $J$-adique. Un tel $A$-faisceau sera dit \emph{de type $J$-adique}.
\vskip .3cm
{
Proposition {\bf 3.9}. --- \it Soient $E$ et $F$ deux $A$-faisceaux sur $X$.
\begin{itemize}
    \item[(a)] Si $E$ est $J$-adique, l'application canonique
    $$
    \varphi_X: \Hom_a(E, F) \to \Hom(E, F)
    $$
    est une bijection.
    \item[(b)] Si $E$ est de type $J$-adique, le préfaisceau
    $$
    T \mapsto \Hom(E|T, F|T)
    $$
    sur $X$ est un \emph{faisceau}.
\end{itemize}
}
\vskip .3cm
{\bf Preuve}: Il suffit de montrer a), l'assertion b) en étant conséquence immédiate. Si $u$ et $v: E \rightrightarrows F$ sont deux morphismes de systèmes projectifs vérifiant $\varphi_X(u) = \varphi_X(v)$, alors l'exactitude de (3.8.1) montre que $\text{Im}( v - u)$ est négligeable, donc nul puisque $E$ est strict, d'où $u = v$. Soit maintenant $a: E \to F$ un morphisme de $A-\fsc(X)$ et montrons qu'il est dans l'image de $\varphi_X$. Il résulte de (2.7.1) que localement $a$ est l'image d'un élément de $\varinjlim_\gamma \Hom(\chi_\gamma (E), F)$, donc provient d'un morphisme $u: E \to F$ de $\mathcal{E}(X, J)$, puisque les morphismes canoniques $\chi_\gamma(E) \to E$ sont des isomorphismes. Dans le cas général, on obtient ainsi un recouvrement $(T_i \to e_X)$ de l'objet final $e_X$ de $X$ et des morphismes de systèmes projectifs
$$
v_i: E|T_i \to F|T_i
$$
vérifiant $\varphi_{T_i}(v_i) = a|T_i$. D'après l'injectivité des applications $\varphi_{T_i \times T_j}$, les morphismes $v_i$ se recollent en un morphisme $v$ de $\mathcal{E}(X, J)$ vérifiant $\varphi_X(v) = a$ localement, donc aussi globalement (1.7.2).
\vskip .3cm
{
Corollaire {\bf 3.9.1}. --- \it Le foncteur 
$$
J-\ad(X) \to \TJ-\ad(X)
$$
induit par (3.8.1) est une \emph{équivalence de catégories}.
}
\vskip .3cm
{
Corollaire {\bf 3.9.2}. --- \it La catégorie fibrée 
$$
S \mapsto \TJ-\ad(S)
$$
au-dessus de $X$ est un \emph{champ}, autrement dit on a les propriétés suivantes.
\begin{itemize}
    \item[(i)] Si $F$ et $G$ sont deux $A$-faisceaux de type $J$-adique, le préfaisceau
    $$
    T \mapsto \Hom(F|T, G|T)
    $$
    sur $X$ est un \emph{faisceau}.
    \item[(ii)] Soit $(U_i \to e_X)_{i \in I}$ un recouvrement de l'objet final $e_X$ de $X$. Pour tout couple $(i, j)$ d'éléments de $I$ (resp. tout triple $(i, j, k)$), on pose
    $$
    U_{ij} = U_i \times U_j \quad \text{(resp.}~U_{ijk} = U_i \times U_j \times U_k).
    $$
\end{itemize}
Supposons donné pour tout $i \in I$ un $A$-faisceau de type $J$-adique $F_i$ sur $U_i$ et pour tout couple $(i, j)$ d'éléments de $I$ un isomorphisme
$$
\theta_{ji}: F_i | U_{ij} \isomlong F_j | U_{ij}
$$
de $A-\fsc(U_{ij})$. On suppose que 
\begin{itemize}
    \item[a)] Si $i \in I$, alors $\theta_{ii} = \id$.
    \item[b)] Si $(i, j, k) \in I^3$, $\theta_{ki} = \theta_{kj} \circ \theta_{ji}$ sur $U_{ijk}$.
\end{itemize}
Alors il existe un $A$-faisceau de type $J$-adique $F$ sur $X$ et pour tout $i \in I$ un isomorphisme
$$
\theta_i: F_i \isomlong F|U_i \quad \text{de} \quad A-\fsc(U_i)
$$
tels que pour tout couple $(i, j)$ d'éléments de $I$ on ait
$$
\theta_j \circ \theta_{ji} = \theta_i \quad \text{sur} \quad U_{ij}.
$$
}
\vskip .3cm
{\bf Preuve} : Résulte immédiatement de l'assertion analogue, évidente, pour la catégorie fibrée $T \mapsto J-\ad(T)$ sur $X$.
\vskip .3cm
{
Corollaire {\bf 3.9.3}. --- \it La propriété pour un $A$-faisceau d'être de type $J$-adique est de nature locale.
}
\vskip .3cm
{\bf Preuve} : Soit $F$ un $A$-faisceau localement de type $J$-adique, et $(U_i \to e_X)_{i \in I}$ un recouvrement de l'objet final $e_X$ de $X$ tel que les $A$-faisceaux $F_i = F|U_i$ soient de type $J$-adique. D'après (3.9.2 (ii)), on peut ``recoller'' les $F_i$ suivant un $A$-faisceau de type $J$-adique $F'$. Par ailleurs la proposition (3.9 b)) permet de recoller les morphismes identiques des $F_i$ en un morphisme $F' \to F$, qui est un isomorphisme localement, donc aussi globalement (1.7. (iii)).
\vskip .3cm
{
Proposition {\bf 3.10}. --- \it Soit $F$ un $A$-faisceau sur $X$. Les assertions suivantes sont équivalentes : 
\begin{itemize}
    \item[(i)] $F$ est de type $J$-adique. 
    \item[(ii)] $F$ est de type strict (3.2) et, notant $F'$ le $A$-faisceau strict associé à $F$ (3.3), il existe localement une application croissante $\gamma \geq \id: \mathbf{N} \to \mathbf{N}$ telle que $\chi_\gamma(F')$ soit $J$-adique.
\end{itemize}
De plus ces conditions impliquent la condition (iii) ci-dessous et lui sont équivalentes lorsque l'objet final $e_X$ de $X$ admet un recouvrement par les objets quasicompacts. 
\begin{itemize}
    \item[(iii)] Pour tout entier $r \geq 0$, le $A$-faisceau
    $$
    \tau_r(F) = (F_n \bigotimes_A (A/J^{r+1}))_{n \in \mathbf{N}}
    $$
    est de type constant (3.6).
\end{itemize}
}
\vskip .3cm
{\bf Preuve} : Si $F$ est de type $J$-adique, il existe un recouvrement $(U_i \to e_X)_{i \in I}$ de l'objet final de $X$ tel que pour tout $i \in I$ $F | U_i$ soit isomorphe au sens de $\mathcal{E}_1(U_i, J)$ (2.2) à un $A$-faisceau $J$-adique, et la réciproque est également vraie d'après (3.9.3).

L'équivalence de (i) et (ii) se voit alors en paraphrasant la preuve de (SGA5 V 3.2.3). Montrons que (i) $\Rightarrow$ (iii). Il résultera de (5.1) (le lecteur vérifiera que (3.10) n'est pas utilisé dans la preuve de (5.1)) que si $P$ et $Q$ sont deux $A$-faisceaux isomorphes, alors les $A$-faisceaux $\tau_r (P)$ et $\tau_r(Q)$ sont isomorphes. Par suite on peut supposer que $F$ est $J$-adique, et alors l'assertion est évidente. Pour voir que (iii) $\Rightarrow$ (i) sous l'hypothèse supplémentaire de l'énoncé, on peut grâce à (3.9.3) supposer que $e_X$ est quasicompact. Alors pour tout $r \geq 0$, $\tau_r(F)$ vérifie la condition de Mittagg-Leffler, d'où résulte aussitôt qu'il en est de même pour $F$. On peut donc supposer $F$ strict. Alors pour tout $r \geq 0$, $\tau_r(F)$ est de type $J$-adique, donc essentiellement constant par (i) $\Rightarrow$ (ii). Choisissant alors une application $\gamma \geq \id: \mathbf{N} \to \mathbf{N}$ telle que pour tout $r \geq 0$ le système projectif $\tau_r(F)$ soit constant à partir du rang $\gamma(r)$, il est immédiat que le $A$-faisceau $\chi_\gamma(F)$ est $J$-adique, et donc que $F$ vérifie (ii).
\vskip .3cm
{
Corollaire {\bf 3.11}. --- \it Soit $X$ un topos dont l'objet final est quasicompact. Les assertions suivantes pour un $A$-faisceau $F$ sur $X$ sont équivalentes :
\begin{itemize}
    \item[(i)] $F$ est de type $J$-adique. 
    \item[(ii)] $F$ vérifie la condition de Mittag-Leffler et, désignant par $F'$ le $A$-faisceau strict associé, il existe une application $\gamma \geq \id$ de $\mathbf{N}$ dans $\mathbf{N}$ telle $\chi_\gamma(F')$ soit $J$-adique.
\end{itemize}
}
\vskip .3cm
Signalons enfin l'énoncé suivant, dont la preuve se ramène localement à une paraphrase de celle de (SGA5 V 3.2.4) :
\vskip .3cm
{
Proposition {\bf 3.12}. --- \it Soit 
$$
0 \to F' \to F \to F'' \to 0
$$
une suite exacte de $A$-faisceaux. Alors : 
\begin{itemize}
    \item[(i)] Si $F'$ et $F$ sont respectivement de type strict et de type $J$-adique, alors $F''$ est de type $J$-adique. 
    \item[(ii)] Si $F$ et $F''$ sont respectivement de type strict et de type $J$-adique, alors $F'$ est de type strict.
\end{itemize}
}
\vskip .3cm
{\bf Remarque 3.13}. Bien entendu, comme dans le contexte de (SGA5 V et VI) ce sont là les seules stabilités des notions précédentes. Pour en avoir d'autres, il faudra introduire des conditions de finitude (cf. II).
