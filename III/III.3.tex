%%%%%%%%%%%%%%%%%%%%%%%%%%%%%%%%%%%%
\subsection*{3. Formalisme des fonctions $L$.}
\addcontentsline{toc}{subsection}{3. Formalisme des fonctions $L$}

Soit $p$ un nombre premier $\neq \ell$. On note $f$ l'élément de Frobenius $u \mapsto u^p$ $(u \in \overline{\mathbf{F}}_p)$, qui est un générateur topologique du groupe de Galois $\Gal(\overline{\mathbf{F}}_p/\mathbf{F}_p)$.

Étant donné un schéma $X$ de type fini sur $\mathbf{F}_p$, on note $X^\circ$ l'ensemble des points fermés de $X$, et, pour tout $x \in X^\circ$, on désigne par $d(x)$ le degré résiduel de $x$. Choisissant pour tout $x \in X^\circ$ un point géométrique $\overline{x}$ au-dessus de $x$, on rappelle (SGA 5 XV 3) que la fonction $L$ d'un $\mathbf{Q}_\ell$-faisceau constructible $F$ sur $X$ est définie par la formule
$$
L_F(f) = \prod_{x \in X^\circ} (1/\det (1 - f_{F_{\overline{x}}}^{-d(x)}t^{d(x)})).
\leqno{(3.0)}
$$
Grâce à la propriété de multiplicativité de (SGA 5 XV 3.1 a)), on peut prolonger cette définition à $\D^b_c(X, \mathbf{Q}_\ell)$, en posant pour tout $E \in \D^b_c(X, \mathbf{Q}_\ell)$
$$
L_E(f) = \prod_{i \in \mathbf{Z}} (L_{\mathrm{H}^i(E)}(t))^{(-1)^i}.
\leqno{(3.1)}
$$
\vskip .3cm
{
Proposition {\bf 3.2}. --- \it Soit $X$ un schéma de type fini sur $\mathbf{F}_p$.
\begin{enumerate}
    \item[a)] Pour tout triangle exact
    \[\begin{tikzcd}
	& {E''} \\
	{E'} && E
	\arrow[from=2-1, to=2-3]
	\arrow[from=2-3, to=1-2]
	\arrow[dashed, from=1-2, to=2-1]
    \end{tikzcd}\]
    de $\D^b_c(X, \mathbf{Q}_\ell)$, on a 
    $$
    L_E(t) = L_{E'}(t) L_{E''}(t).
    $$
    En particulier, pour tout $m \in \mathbf{Z}$, on a 
    $$
    L_{E[m]}(t) = (L_E(t))^{(-1)^m}.
    $$
    \item[b)] Soient $Y$ un sous-schéma fermé de $X$, et $U = X - Y$ l'ouvert complémentaire. On a 
    $$
    L_E = L_{E | U} L_{E | Y},
    $$
    pour tout $E \in \D^b_c(X, \mathbf{Q}_\ell)$.
    \item[c)] Soit $h: X \to S$ un morphisme de schémas de type fini sur $\mathbf{F}_p$. Pour tout $E \in \D^b_c(X, \mathbf{Q}_\ell)$, on a 
    $$
    L_E = \prod_{s \in S^\circ} L_{E | X_s}.
    $$
\end{enumerate}
}
\vskip .3cm
{\bf Preuve} : Immédiat à partir des assertions analogues pour les objets de cohomologie (SGA5 XV 3.1).
\vskip .3cm
{
Proposition {\bf 3.3}. --- \it Soient $X$ un schéma de type fini sur $\mathbf{F}_p$, $g: X \to \mathbf{F}_p$ le morphisme structural et $E \in \D^b_c(X, \mathbf{Q}_\ell)$. Alors
$$
L_E = L_{\bRd g_!(E)}.
$$
En particulier, $L_E$ est une fraction rationnelle.
}
\vskip .3cm
{\bf Preuve} : On peut supposer que $E$ est un $\mathbf{Q}_\ell$-faisceau constructible, et alors l'assertion n'est autre que (SGA5 XV 3.2).
\vskip .3cm
{
Corollaire {\bf 3.4}. --- \it Soit $h: X \to S$ un morphisme de schémas de type fini sur $\mathbf{F}_p$. Pour tout $E \in \D^b_c(X, \mathbf{Q}_\ell)$, on a
$$
L_E = L_{\bRd h_! (E)}.
$$
}
\vskip .3cm
Nous allons maintenant déduire de (3.3) une \emph{équation fonctionnelle} pou les fonctions $L$, du moins si $X$ est projectif sur $\mathbf{F}_p$.
\vskip .3cm
{
Définition {\bf 3.5}. --- \it Soient $g: X \to \mathbf{F}_p$ un schéma de type fini sur $\mathbf{F}_p$, et $\overline{X} = X \times_{\mathbf{F}_p} \overline{\mathbf{F}}_p$. Pour tout $E \in \D^b_c(X, \mathbf{Q}_\ell)$, on pose  
$$
\chi(E) = \rang (\bRd g_! E) = \sum_{i \in \mathbf{Z}} (-1)^i [\mathrm{H}^i_c (\overline{X}, \overline{E}):\mathbf{Q}_\ell],
$$
$$
\delta(E) = \det (\bRd g_! (E)) = \prod_{i \in \mathbf{Z}} (\det f_{\mathrm{H}^i_c(\overline{X}, \overline{E})})^{(-1)^i},
$$
où $\overline{E}$ désigne l'image inverse de $E$ au-dessus de $\overline{X}$.
}
\vskip .3cm
D'après les propriétés d'additivité et de multiplicativité respectives de la trace et du déterminant dans la catégorie des $\mathbf{Q}_\ell$-espaces vectoriels de dimension finie, il es clair que pour tout triangle exact
\[\begin{tikzcd}
	& {E''} \\
	{E'} && {E,}
	\arrow[from=2-1, to=2-3]
	\arrow[from=2-3, to=1-2]
	\arrow[dashed, from=1-2, to=2-1]
\end{tikzcd}\]
on a 
$$
\chi(E) = \chi(E') + \chi(E'').
\leqno{(3.5.1)}
$$
$$
\delta(E) = \delta(E')  \delta(E'').
\leqno{(3.5.2)}
$$
En particulier, pour tout $m \in \mathbf{Z}$ et tout $E \in \D^b_c(X, \mathbf{Q}_\ell)$,
$$
\chi(E[m]) = (-1)^m \chi(E) \quad \text{et} \quad \delta (E[m]) = (\delta(E))^{(-1)^m}.
$$
\vskip .3cm
{
Proposition {\bf 3.6}. --- \it Soit $g: X \to \mathbf{F}_p$ un schéma projectif sur $\mathbf{F}_p$. On pose $K_X = \bRd g^! (\mathbf{Q}_\ell)$, et $\D_X = \bRd {\cHom}_{\mathbf{Q}_\ell}(. , K_X)$. Alors, pour tout $E \in \D^b_c(X, \mathbf{Q}_\ell)$, on a l'identité
$$
L_{\D_X(E)}(t) = (-t)^{-\chi(E)} \delta(E) L_E(t^{-1}).
$$
}
\vskip .3cm
{\bf Preuve} : Le second membre a un sens d'après (3.3). Posons $S = \Spec$ et $\D_S = \bRd {\cHom}_{\mathbf{Q}_\ell}(. , \mathbf{Q}_\ell)$. D'après (2.3.2 a)), on a 
$$
\bRd g_* (\D_X E) \isomlong \D_S \bRd g_* (E),
$$
donc (3.3)
$$
L_{\D_X}(E) = L_{\D_S(\bRd g_* (E))}.
$$
Comme $L_E = L_{\bRd g_* (E)}$ (3.3), l'assertion résultera du lemme suivant
\vskip .3cm
{
Lemme {\bf 3.7}. --- \it Si $F \in \D^b_c(S, \mathbf{Q}_\ell)$, on a :
$$
L_{\D_S(F)}(t) = (-t)^{\chi(F)} \delta(F) L_F(t^{-1}).
$$
}
\vskip .3cm
D'après les propriétés d'additivité et de multiplicativité (3.5.1) et (3.5.2), on peut supposer que $F \in \mathbf{Q}_\ell-\fscn (S)$. Alors $F$ correspond (SGA5 VII 1.4.2) à un $\mathbf{Q}_\ell$-espace vectoriel de dimension fini $V$ muni d'une opération continue $f_V$ du Frobenius, et le $\mathbf{Q}_\ell$-faisceau $\D_S (F) = {\cHom}_{\mathbf{Q}_\ell}(F, \mathbf{Q}_\ell)$ correspond (II 1.26) au $\mathbf{Q}_\ell$-espace vectoriel $V^{\nu}$ muni de l'opération continue $(f^{\nu}_V)^{-1}$ du Frobenius. Il suffit alors de montrer que, étant donnés un corps $K$, un $K$-espace vectoriel de dimension finie $V$ et un automorphisme $u$ de $V$, on a l'identité
$$
1/\det(1 - u^{-1} t) = (-t)^{-\text{dim}(V)} \det(u)/\det(1 - ut^{-1})
\leqno{(3.8)}
$$
dans $K(t)$. On peut pour cela supposer $K$ algébriquement clos, donc $u$ triangulable, puis, grâce aux propriétés de multiplicativité du déterminant, que dim$(V) = 1$. Alors $u$ est l'homothétie définie par un scalaire non nul $\lambda$, et (3.8) est l'identité évidente
$$
1/(1-(t/\lambda)) = (-\lambda/t)/(1-(\lambda/t)).
$$
Bien entendu, la formule (3.6) ne présente d'intérêt en pratique que si l'on dispose d'une expression simple pour $\D_X(E)$. Nous allons maintenant donner des cas où il en est ainsi.
\vskip .3cm
{
Proposition {\bf 3.9}. --- \it On suppose $X$ quasiprojectif, lisse et purement de dimension $n$ sur $\mathbf{F}_p$. Posant pour tout $E \in \D^b_c(X, \mathbf{Q}_\ell)$
$$
E^V = {\cHom}^{\bullet}_{\mathbf{Q}_\ell}(E, \mathbf{Q}_\ell),
$$
on a un isomorphisme
$$
\D_X(E) \isom E^V (n) [2n]
$$
dans chacun des cas suivants
\begin{enumerate}
    \item[(i)] $E \in \D^b_t(X, \mathbf{Q}_\ell)$
    \item[(ii)] $X$ est une courbe, et $E$ est un $\mathbf{Q}_\ell$-faisceau constructible de la forme $i_*(M)$, où $i: U \hookrightarrow X$ est l'inclusion d'un ouvert dense de $X$ et $M \in \mathbf{Q}_\ell-\fsct(U)$.
\end{enumerate}
}
\vskip .3cm
{\bf Preuve} : Comme $\D_X(E) = \bRd {{\cHom}}_{\mathbf{Q}_\ell}(E, \mathbf{Q}_\ell(n))[2n]$, le cas (i) résulte du lemme suivant.
\vskip .3cm
{
Lemme {\bf 3.9.1}. --- \it Étant donnés un schéma noethérien $X$, $F \in \mathbf{Q}_\ell-\fsct(X)$ et $G \in \mathbf{Q}_\ell-\fscn(U)$, on a:
$$
{\cExt}^j_{\mathbf{Q}_\ell}(F, G) = 0 \quad (j \geq 1).
$$
}
\vskip .3cm
Il s'agit de voir que si $F \in \mathbf{Z}_\ell-\fsct(X)$ et $G \in \mathbf{Z}_\ell-\fscn(X)$, les $\mathbf{Z}_\ell$-faisceaux $\cExt^j_{\mathbf{Z}_\ell}(F, G)$ $(j \geq 1)$ sont annulés par une puissance de $\ell$. D'après (I 6.4.2) et (II 1.2.1), on peut, quitte à se restreindre à des parties localement fermées convenables de $X$, supposer que $G \in \mathbf{Z}_\ell-\fsct(X)$. Alors, compte tenu de (II 1.26), l'assertion résulte de l'assertion analogue, bien connue, pour les $\mathbf{Z}_\ell$--Modules de type fini. Montrons (ii). 

Il s'agit de voir que
$$
P^j = {\cExt}^j_{\mathbf{Q}_\ell}(E, \mathbf{Q}_\ell(1)) = 0 \quad (j \geq 1).
$$
Comme $M$ est constante tordu constructible, il résulte du cas (i) que $P^j | U = 0$. Il nous suffit donc de voir que pour tout point fermé $x$ de $Y = X - U$ et tout point géométrique $\overline{x}$ au-dessus de $x$, on a $P^j_x = 0$. Le pendant pour les $\mathbf{Q}_\ell$-faisceaux de la variante (SGA5 I 4.6.2) du théorème de dualité locale fournit un accouplement parfait
$$
{\cExt}^j_{\mathbf{Q}_\ell}(E, \mathbf{Q}_\ell(1)) \times \bH^{2-j}_{\overline{x}}(E) \to \mathbf{Q}_{\ell'}
\leqno{(3.9.2)}
$$
avec (SGA5 I 4.5.1)
$$
\bH^{2-j}_{\overline{x}} = (\bH^{2-j}_x(E))_{\overline{x}}.
$$
Comme le morphisme d'adjonction canonique
$$
E \to i_* i^* (E)
$$
est un isomorphisme, il résulte de la première suite exacte de (SGA4 V 4.5) que 
$$
\bH^0_x(E) = \bH^1_x(E) = 0,
$$
d'où aussitôt le résultat annoncé.

Ceci dit, lorsque $X$ est projectif sur $\mathbf{F}_\ell$, la formule (3.6) prend la forme
$$
L_{\check{E}}(p^{-n} t) = (-1)^{- \chi (E)} \delta(E) L_E (t^{-1}),
\leqno{(3.10)}
$$
dans chacun des cas de (3.9). Compte tenu de (3.2 a)), cela résulte du lemme suivant.
\vskip .3cm
{
Lemme {\bf 3.11}. --- \it Soient $X$ un schéma de type fini sur $\mathbf{F}_p$, et $F \in \D^b_c(X, \mathbf{Q}_\ell)$. Posant $F(j) = F \otimes \mathbf{Q}_\ell (j)$ $(j \in \mathbf{Z})$, on a la relation
$$
L_{F(j)}(t) = L_F (p^{-j} t).
$$
}
\vskip .3cm
D'après les propriétés de multiplicativité (3.2 a)), on peut pour le voir supposer que $F$ est un $\mathbf{Q}_\ell$-faisceau constructible; alors, comme le Frobenius opère sur $\mathbf{Q}_\ell(j) \isom \mathbf{Q}_\ell$ (non canoniquement) par l'homothétie de rapport $p^{-j}$, l'assertion est immédiate sur la définition (3.0).

Supposons maintenant qu'on ait de plus un isomorphisme
$$
\check{E} \isomlong E(\rho) \quad \text{pour un}~\rho \in \mathbf{Z}.
$$
Alors la formule (3.10) prend la forme
$$
L_E (p^{-n-\rho}t) = (-t)^{-\chi(E)}\delta(E) L_E(t^{-1}),
$$
ou encore, après avoir posé $q = n + \rho$ et fait le changement de variable $t \mapsto t^{-1}$,
$$
L_E(1/qt) = (-t)^{\chi{E}}\delta(E)L_E(t).
\leqno{(3.12)}
$$
\vskip .3cm
{\bf Remarque 3.13}. Sous les hypothèses de (3.9), l'existence d'un tel entier $p$ est assurée dans les cas suivants
\begin{enumerate}
    \item[cas~(i)] $E \isomlong \mathbf{Q}_\ell(m)$ \quad pour un $m \in \mathbf{Z}$, et alors $\rho = -2m$.
    \item[cas~(ii)] $M \isomlong \mathbf{Q}_\ell(m)$ \quad pour un $m \in \mathbf{Z}$, et alors $\rho = -2m$.
\end{enumerate}
(Pour ce dernier cas, il est immédiat que
$$
i_*(M^\nu) \isom (i_*(M))^\nu. \quad )
$$
Explicitons enfin une relation importante entre les entiers $\chi(E)$ et $\delta(E)$.
\vskip .3cm
{
Proposition {\bf 3.14}. --- \it Soient $X$ un schéma projectif et lisse purement de dimension $n$ sur $\mathbf{Z}_p$ et $E \in \D^b_c(X, \mathbf{Q}_\ell)$. On suppose qu'il existe un entier $m$ tel que 
$$
\D_X(E) \isomlong E(m),
$$
et on pose $q= p^m$. Alors, on a l'égalité
$$
\delta(E)^2 = q^{\chi(E)}.
$$
}
\vskip .3cm
{\bf Preuve} : La substitution $t \mapsto 1/qt$ dans (3.12) fournit l'équation fonctionnelle
$$
L_E(t) = (-1/qt)^{\chi(E)}\delta(E)L_E(1/qt).
\leqno{(3.12~\text{bis})}
$$
Multipliant (3.12) et (3.12 bis) membre à membre, on obtient l'identité 
$$
L_E(t)L_E(1/qt) = q^{-\chi(E)}(\delta(E))^2 L_E(t) L_E(1/qt),
$$
d'où aussitôt la relation désirée, compte tenu du fait que $L_E$ n'est pas identiquement nulle, comme il est clair sur sa définition (3.0).
