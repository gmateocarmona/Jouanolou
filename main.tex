 \documentclass[12pt,twoside]{report} %openright
	\usepackage[utf8]{inputenc}
 	\usepackage[greek,english]{babel}
    \usepackage{yfonts}
	\usepackage{latexsym}
	\usepackage{amsmath}
	\usepackage{amssymb}
	\usepackage{amsthm}
	\usepackage{stmaryrd}
	\usepackage{pb-diagram}
	\usepackage{amscd}
	\usepackage{mathrsfs}
	\usepackage{enumerate}
	\usepackage{color}
	\usepackage{cite}
	\usepackage{graphicx}
	\usepackage[all]{xy}
	
	 \usepackage[svgnames]{xcolor}
	\usepackage{hyperref}
\hypersetup{
    colorlinks=true,
    linkcolor=[RGB]{61,82,62},
    filecolor=[RGB]{61,82,62},      
    urlcolor=[RGB]{61,82,62},
}
	\usepackage{graphicx}
\graphicspath{ {./images/} }
	
	\usepackage{fixltx2e}
	\usepackage{extarrows}
	\usepackage{datetime}
	\usepackage{mathtools}
    \usepackage{remreset}
    \usepackage{afterpage}
    \usepackage{verbatim}
    \usepackage{cleveref}
    \usepackage{csquotes}
    \usepackage{etoc}

\usepackage{pdflscape}
	
	\usepackage[urw-garamond]{mathdesign}
	\usepackage[T1]{fontenc}
	\usepackage[utf8]{inputenc}
    \usepackage{frcursive}
	\usepackage{float}
	\usepackage[12pt]{moresize}
	\usepackage{titlesec}
	\usepackage[twoside,top=1.5in,bottom=1.5in,left=1in,right=1.3in,headheight=16pt,headsep=30pt,footskip=40pt]{geometry}

	\linespread{1.25}
%	\linespread{1.05}


\NeedsTeXFormat{LaTeX2e}
\ProvidesPackage{quiver}[2021/01/11 quiver]

% `tikz-cd` is necessary to draw commutative diagrams.
\RequirePackage{tikz-cd}
% `amssymb` is necessary for `\lrcorner` and `\ulcorner`.
\RequirePackage{amssymb}
% `calc` is necessary to draw curved arrows.
\usetikzlibrary{calc}
% `pathmorphing` is necessary to draw squiggly arrows.
\usetikzlibrary{decorations.pathmorphing}

% A TikZ style for curved arrows of a fixed height, due to AndréC.
\tikzset{curve/.style={settings={#1},to path={(\tikztostart)
    .. controls ($(\tikztostart)!\pv{pos}!(\tikztotarget)!\pv{height}!270:(\tikztotarget)$)
    and ($(\tikztostart)!1-\pv{pos}!(\tikztotarget)!\pv{height}!270:(\tikztotarget)$)
    .. (\tikztotarget)\tikztonodes}},
    settings/.code={\tikzset{quiver/.cd,#1}
        \def\pv##1{\pgfkeysvalueof{/tikz/quiver/##1}}},
    quiver/.cd,pos/.initial=0.35,height/.initial=0}

% TikZ arrowhead/tail styles.
\tikzset{tail reversed/.code={\pgfsetarrowsstart{tikzcd to}}}
\tikzset{2tail/.code={\pgfsetarrowsstart{Implies[reversed]}}}
\tikzset{2tail reversed/.code={\pgfsetarrowsstart{Implies}}}
% TikZ arrow styles.
\tikzset{no body/.style={/tikz/dash pattern=on 0 off 1mm}}

	% use osf for figures in math mode
	\DeclareSymbolFont{digits}{\encodingdefault}{\rmdefault}{m}{n}
	\SetSymbolFont{digits}{normal}{\encodingdefault}{\rmdefault}{m}{n}
	\DeclareMathSymbol{0}{\mathalpha}{digits}{"30}
	\DeclareMathSymbol{1}{\mathalpha}{digits}{"31}
	\DeclareMathSymbol{2}{\mathalpha}{digits}{"32}
	\DeclareMathSymbol{3}{\mathalpha}{digits}{"33}
	\DeclareMathSymbol{4}{\mathalpha}{digits}{"34}
	\DeclareMathSymbol{5}{\mathalpha}{digits}{"35}
	\DeclareMathSymbol{6}{\mathalpha}{digits}{"36}
	\DeclareMathSymbol{7}{\mathalpha}{digits}{"37}
	\DeclareMathSymbol{8}{\mathalpha}{digits}{"38}
	\DeclareMathSymbol{9}{\mathalpha}{digits}{"39}
	% helvetica for sans in math
	\DeclareMathAlphabet{\mathsf}{\encodingdefault}{phv}{m}{n}
	%\usepackage{showkeys}
	%\usepackage{showframe}
	\usepackage{fancyhdr}
	%\fancyhead[CO]{\rightmark}
	\fancyhead[CO]{}
	%\fancyhead[CE]{\leftmark}
	\fancyhead[CE]{}
	\fancyhead[RE]{}
	\fancyhead[LO]{}
	\fancyhead[RO]{}
	\fancyhead[LE]{}
	\cfoot{\thepage}
	\renewcommand{\headrulewidth}{0.0pt}
	\titleformat\section{\normalfont\Large\centerline{\hss\rule{1.3in}{.01in}\hss}}{\thesection}{1em}{}
    \titleformat{\chapter}[display]{}{\huge\filcenter { \thechapter}}{12pt}{\large \filcenter}

\newcommand\hmmax{0}
\newcommand\bmmax{0}

\hypersetup{% 
  pdftitle={\emph{Catégories Dérivées en Cohomologie $\ell$-adique}, J.-P. Jouanolou. (Ed.) M. Carmona et al},
  pdfauthor={Alexander Grothendieck},
  pdflang={fr}
}



%%%%%%%%%%%%%%%%%%%%%%%%%%%%%%%%%%%%%%%%%%%%%%

\newcommand*\isommap{%
  \xrightarrow{\raisebox{-0.2 em}{\smash{\ensuremath{\sim}}}}%
}

\newcommand*\isomlong{%
  \xlongrightarrow{\raisebox{-0.2 em}{\smash{\ensuremath{\sim}}}}%
}
	% letters
	\newcommand{\PP}{\mathbf{P}}
%	\newcommand{\RR}{\mathbf{R}}
	\renewcommand{\AA}{\mathbf{A}}
	\newcommand{\ZZ}{\mathbf{Z}}
    \newcommand{\Q}{\mathbf{Q}}
	%\newcommand{\NN}{\mathbf{N}}	
	%\newcommand{\QQ}{\mathbb{Q}}	
	\newcommand{\QQ}{\mathbf{Q}}	
	\newcommand{\Aa}{\mathscr{A}}	
	\newcommand{\Kk}{\mathscr{K}}	
	%\newcommand{\BB}{\mathbb{B}}	
	\newcommand{\FF}{\mathbf{F}}	
	\newcommand{\Hh}{\mathscr{H}}	
	\newcommand{\Xx}{\mathscr{X}}	
	\newcommand{\Gg}{\mathscr{G}}	
	\newcommand{\GG}{\mathbf{G}}	
	\newcommand{\Ee}{\mathscr{E}}	
	\newcommand{\Ll}{\mathscr{L}}	
	\newcommand{\Mm}{\mathscr{M}}	
	\newcommand{\mm}{\mathfrak{m}} %{\mathfrak{m}}	
    \newcommand{\pp}{\wp}	
	\newcommand{\Ff}{\mathscr{F}}	
	\newcommand{\Oo}{\mathscr{O}}	
	\newcommand{\Cc}{\mathscr{C}}	
	\newcommand{\CC}{\mathbf{C}}
    
    \newcommand{\bfGr}{\mathbf{Gr}}
    \newcommand{\bfFl}{\mathbf{Fl}}
    \newcommand{\bfCart}{\mathbf{Cart}}
    
	\newcommand{\bari}{\bar\imath}
	\newcommand{\barj}{\bar\jmath}
	%\newcommand{\u1}{\underline{1}}
	

	
 
    %%%%%
    \def\Z{\Bbb Z}
    \def\gT{{\textfrak T}}
    \def\gG{{\textfrak G}}
    \def\gS{{\textfrak S}}
    \def\gA{{\textfrak A}}
	\def\gP{{\textfrak P}}
	\def\cD{{\cal D}}
	\def\cA{{\cal A}}
	\def\cB{{\cal B}}
	\def\cC{{\cal C}}
	\def\cP{{\cal P}}
	\def\cQ{{\cal Q}}
	\def\cN{{\cal N}}
	\def\cM{{\cal M}}
	\def\cU{{\cal U}}
	\def\cG{{\cal G}}
	\def\cH{{\cal H}}
	\def\cJ{{\cal J}}
	\def\cE{{\cal E}}
	\def\cT{{\cal T}}
	\def\gn{{g,\nu}}
	\def\GG{{{\rm I}\!\Gamma}}
    
    \def\spc{{\qquad\qquad\qquad\qquad}}
	\def\C{\Bbb C}
	%\def\N{\Bbb N}
	\def\O{\Bbb O}
	\def\M{\Bbb M}
	\def\P{\Bbb P}
	\def\E{{\Bbb E}}
	\def\F{{\Bbb F}}
	\def\Q{\Bbb Q}
	\def\glq{{{\rm Gal}(\overline\Q/\Q)}}
	\def\R{\Bbb R}
	\def\U{\Bbb U}

	\renewcommand{\P}{\mathbb{P}}
    
	% operators
	\renewcommand{\bar}{\overline}
	\newcommand{\et}{{\rm\acute{e}t}}
	\newcommand{\an}{{\rm an}}
	\newcommand{\ket}{{\rm k\et}}
	\newcommand{\fet}{{\rm f\et}}
	\newcommand{\Et}{{\acute{E}t}}
	\DeclareMathOperator{\Gal}{Gal}
	\DeclareMathOperator{\Sh}{Sh}
	\DeclareMathOperator{\gr}{gr}
	\DeclareMathOperator{\ev}{ev}
	\DeclareMathOperator{\Img}{Im}
	\DeclareMathOperator{\cone}{cone}
	\DeclareMathOperator{\Star}{Star}
	\DeclareMathOperator{\coker}{coker}
	\DeclareMathOperator{\chara}{char}
	\DeclareMathOperator{\Hom}{Hom}
	\DeclareMathOperator{\Hhom}{{\mathscr{H}om}}
	\DeclareMathOperator{\Ext}{Ext}
	\DeclareMathOperator{\Tor}{Tor}
	\DeclareMathOperator{\Frac}{Frac}
	\DeclareMathOperator{\Eext}{\Ee{}xt}
	\DeclareMathOperator{\rank}{rank}
	\DeclareMathOperator{\Spec}{Spec}
	\DeclareMathOperator{\Gr}{Gr}
	\DeclareMathOperator{\Proj}{Proj}
	\DeclareMathOperator{\Pic}{Pic}
	\DeclareMathOperator{\Cl}{Cl}
	\DeclareMathOperator{\height}{ht}
	\DeclareMathOperator{\Div}{Div}
	\DeclareMathOperator{\codim}{codim}
	\DeclareMathOperator{\Def}{Def}
	\DeclareMathOperator{\Inf}{Inf}
	\DeclareMathOperator{\Set}{Set}
	\DeclareMathOperator{\Ab}{Ab}
	\DeclareMathOperator{\colim}{colim}
	\DeclareMathOperator{\Art}{Art}
	\DeclareMathOperator{\ass}{ass}
	\DeclareMathOperator{\supp}{supp}
	\DeclareMathOperator{\ArtFk}{Art^F_k}
	\DeclareMathOperator{\ArtFW}{Art^F_W}
	\DeclareMathOperator{\Tube}{Tube}
	\DeclareMathOperator{\Swan}{Swan}
    \DeclareMathOperator{\B}{B}
    \DeclareMathOperator{\Sch}{Sch}
    \DeclareMathOperator{\Mor}{Mor}
    \DeclareMathOperator{\Cart}{Cart}
    \DeclareMathOperator{\CART}{CART}
    \DeclareMathOperator{\Arr}{Arr}
    \DeclareMathOperator{\Ob}{Ob}
    \DeclareMathOperator{\MOR}{MOR}
    \DeclareMathOperator{\HH}{H}
    \DeclareMathOperator{\RR}{R}
    %\DeclareMathOperator{\Htp}{Htp}
    \DeclareMathOperator{\Aut}{Aut}
    \DeclareMathOperator{\Mod}{Mod}
    \DeclareMathOperator{\Fl}{Fl}
    \DeclareMathOperator{\Id}{Id}
    \DeclareMathOperator{\End}{End}
    \DeclareMathOperator{\Card}{Card}
    \DeclareMathOperator{\Ens}{Ens}
    \DeclareMathOperator{\Bil}{Bil}
    \DeclareMathOperator{\Cat}{Cat}
    \DeclareMathOperator{\Diag}{Diag}
     \DeclareMathOperator{\Diagcomm}{Diagcomm}
     \DeclareMathOperator{\pr}{pr}
     \DeclareMathOperator{\inj}{inj}
     \DeclareMathOperator{\Rev}{Rev}
     \DeclareMathOperator{\Ker}{Ker}
     \DeclareMathOperator{\Coker}{Coker}
     \DeclareMathOperator{\Oub}{Oub}
     \DeclareMathOperator{\Top}{Top}
     \DeclareMathOperator{\Topcomp}{Topcomp}
     \DeclareMathOperator{\diag}{diag}
     \DeclareMathOperator{\Coim}{Coim}
     \DeclareMathOperator{\Codiag}{Codiag}
    \DeclareMathOperator{\D}{D}
    \DeclareMathOperator{\K}{K}
    \DeclareMathOperator{\Ld}{\mathrel{L}}
    \DeclareMathOperator{\Rd}{\mathrel{R}}
    %\DeclareMathOperator{\cHom}{\mathcal{H}\text{om}}
    %\DeclareMathOperator*{\bRd}{\mathbf{\textbf{R}}}
    \DeclareMathOperator*{\bRd}{\mathrm{\mathbf{R}}} 
    \DeclareMathOperator*{\bLd}{\mathrm{\mathbf{L}}}
    %\DeclareMathOperator*{\bH}{\mathbf{\textbf{H}}}
    %\newcommand{\bH}[1]{\mathbf{\textbf{H}}^{#1}}
    %\newcommand{\bH}[1]{\boldsymbol{\mathrm{H}}^{#1}}
    \DeclareMathOperator{\bH}{\mathbf{H}}
    \DeclareMathOperator{\id}{id}
    \DeclareMathOperator{\Tr}{Tr}
    \DeclareMathOperator{\fscn}{fscn}
    \DeclareMathOperator{\fsct}{fsct}
    \DeclareMathOperator{\fsc}{fsc}
    \DeclareMathOperator{\rang}{rang}
    \DeclareMathOperator{\Elc}{Elc}
    \DeclareMathOperator{\Res}{Res}
    \DeclareMathOperator{\adt}{adt}
    \DeclareMathOperator{\adn}{adn}
    \DeclareMathOperator{\pt}{pt}
    \DeclareMathOperator{\grs}{grs}
    %\DeclareMathOperator{\gr}{gr}
    \DeclareMathOperator{\modn}{modn}
    \DeclareMathOperator{\parf}{parf}
    \DeclareMathOperator{\coh}{coh}
    \DeclareMathOperator{\tr}{tr}
    \DeclareMathOperator{\ad}{ad}
    \DeclareMathOperator{\dimtops}{dimtops}
    %\DeclareMathOperator{\dp}{dp}
    \DeclareMathOperator{\ql}{ql}
    \DeclareMathOperator{\fl}{fl}
    %\DeclareMathOperator{\ht}{ht}
    \DeclareMathOperator{\pre}{pre}
    \DeclareMathOperator{\prefsc}{prefsc}
    \DeclareMathOperator{\TC}{TC}
    \DeclareMathOperator{\TJ}{TJ}
    \DeclareMathOperator{\can}{can}
    \DeclareMathOperator{\Pro}{Pro}
    \DeclareMathOperator{\Ass}{Ass}
    \DeclareMathOperator{\ext}{ext}
    \DeclareMathOperator{\ann}{ann}
    \DeclareMathOperator{\torf}{torf}
    \DeclareMathOperator{\Modn}{Modn}
    \DeclareMathOperator{\Modt}{Modt}
    \DeclareMathOperator{\Modlc}{Modlc}

	% symbols

 \newcommand{\cHom}{\mathrel{\mathscr{H}\mkern-2mu\text{\upshape om}}}
 \newcommand{\cTor}{\mathrel{\mathscr{T}\mkern-2mu\text{\upshape or}}}
 \newcommand{\cExt}{\mathrel{\mathscr{E}\mkern-2mu\text{\upshape xt}}}
	
	\newcommand{\isom}{\simeq}
	\newcommand{\xto}{\xrightarrow}
	\newcommand{\dual}{\vee}
	%\newcommand{\pt}{{pt}}
	\renewcommand{\phi}{\varphi}
	\renewcommand{\emptyset}{\varnothing}
	\renewcommand{\epsilon}{\varepsilon}
	\renewcommand{\tilde}{\widetilde}
	\newcommand{\sto}{{\xto{}}}
	\renewcommand{\to}{\longrightarrow}
	\newcommand{\und}{\underline}
	\newcommand{\indown}{\mathrel{\rotatebox[origin=c]{-90}{$\in$}}}

	% comments
	%\newcommand{\comment}[1]{\textcolor{blue}{[#1]}}

	% stacks project references
	\newcommand{\stacks}[1]{\cite[\href{http://stacks.math.columbia.edu/tag/#1}{Tag #1}]{stacks-project}}

	% environments
	\theoremstyle{plain}
		\newtheorem{lemma}{Lemma}[section]
		\newtheorem{corollary}[lemma]{Corollary}
		\newtheorem{proposition}[lemma]{Proposition}
		\newtheorem{theorem}[lemma]{Theorem}
		\newtheorem*{theorem*}{Theorem}
		\newtheorem*{corollary*}{Corollary}
		\newtheorem{conjecture}{Conjecture}
	\theoremstyle{definition}
		\newtheorem{definition}[lemma]{Definition}
		\newtheorem{example}[lemma]{Example}
		\newtheorem{remark}[lemma]{Remark}
		\newtheorem{construction}[lemma]{Construction}
        
        \def\rinto{\HorizontalMap\rthooka-\empty-\rhla}%%
		\def\linto{\HorizontalMap\lhla-\empty-\lthooka}%%
		\def\dinto{\VerticalMap\dthookb|\empty|\dhla}%%
		\def\uinto{\VerticalMap\uhla|\empty|\uthookb}%%
        


%%%%%%%%%%%%%%%%%%%%%%%%%%%%%%%%%%%%%%%%%%%%%%

\makeatletter
\newcommand*{\toccontents}{\@starttoc{toc}}
\makeatother

\makeatletter
\@addtoreset{footnote}{section}
\makeatother


\begin{document}
\setcounter{tocdepth}{2}
\sloppy


\newgeometry{left=1.5in,right=1.5in, bottom=1.5in}


\thispagestyle{empty}
\vfill
\vfill
\noindent\makebox[\linewidth]{\rule{4in}{0.01in}}
\vfill
\centerline{{\Huge Catégories Dérivées en}} 
\vskip0.5em
\centerline{{\Huge Cohomologie $\ell$-adique}}
\vskip1em
\centerline{{\Large par}}
\vskip1em
\centerline{{\Large Jean-Pierre JOUANOLOU}}
\vfill
\noindent\makebox[\linewidth]{\rule{4in}{0.01in}}

\begin{tikzpicture}[remember picture,overlay]
    \draw[line width=10pt,color=DarkKhaki]
        ([shift={(-0.5\pgflinewidth,-0.5\pgflinewidth)}]current page.north west)
        rectangle
        ([shift={(0.5\pgflinewidth,0.5\pgflinewidth)}]current page.south east);
\end{tikzpicture}






\newpage 
\thispagestyle{empty}
\vfill
\vfill
\centerline{N$^\circ$ d'enregistrement}
\vskip0.5em
\centerline{au C.N.R.S}
\vskip0.5em
\centerline{A.0.3374}
\vskip4em
\centerline{{THÈSE DE DOCTORAT D'ÉTAT}}
\vskip0.5em
\centerline{{ès SCIENCES MATHÉMATIQUES}}
\vskip0.5em
\centerline{{présentée}}
\vskip0.5em
\centerline{{À LA FACULTÉ DES SCIENCES DE PARIS}}
\vskip2em
\centerline{{par}}
\vskip2em
\centerline{{M. JOUANOLOU Jean-Pierre}}
\vskip1em
\centerline{{pour obtenir le grade Docteur ès-Sciences}}
\vfill
\centerline{{\bf Sujet de thèse : {\Large Catégories Dérivées en Cohomologie $\ell$-adique}}} 
\vskip0.5em
\centerline{{soutenue le : 3 Juillet 1969 devant la Commission d'examen}} 
\vfill
\centerline{{MM. SAMUEL \quad Président}} 
\vskip2em
\centerline{{GROTHENDIECK}} 
\vskip0.5em
\centerline{{\qquad\qquad\qquad\qquad VERDIER \qquad\qquad Examinateurs}} 
\vskip0.5em
\centerline{{DIXMIER}} 
\vfill
\vfill
\vfill

\newpage
\mbox{}








%%%%%%%%%%%%%%%%%%%%%%%%%%%%%%%%%%%%%
\chapter*{NOTICE TO THE READER}
\label{ch:pref}
\section*{}

This thesis, under Grothendieck, was defended on July 3, 1969, at IHP (Paris), in front of a jury chaired by P. Samuel, with examiners J. Dixmier, A. Grothendieck, and J. L. Verdier. 

An important precursor to the thesis was the Seminar SGA5 (1965/66), during which J. P. Jouanolou delivered three consecutive exposés. These exposés elaborated on key concepts and techniques that would directly influence his own thesis work. In Exposé III Intro, we find the following statement:
\begin{quote}
    ``\emph{Faute de disposer d'une bonne catégorie dérivée de faisceaux $\ell$-adiques (la thèse de Jouanolou n'ayant malheureusement pas été publiée), nous travaillons systématiquement avec des coefficients de torsion (première aux caractéristiques résiduelles).}''
\end{quote}
The thesis holds significant historical value as it reflects, in some sense, a language and an approach developed during Grothendieck's active years. In the 1980s, he wrote in ``Récoltes et Semailles'':
\begin{quote}
    ``\emph{De toutes façons, c'est aujourd'hui encore le seul texte au monde qui présente la théorie des coefficients $\ell$-adiques, version catégories dérivées --- et un texte introuvable par dessus le marché, pour mettre la joie à son comble.}''
\end{quote}
This project, involving the transcription of J.-P. Jouanolou's thesis, was conducted with his authorization. It was typeset in \LaTeX{} under the direction of Mateo Carmona, with the collaboration of Niels Borne, along with volunteers. The transcription aims to be as faithful as possible to the original. Its edition is considered provisional, and we welcome any remarks, comments, and corrections. For more details about the project, please visit \\
\url{https://github.com/carmonamateo/Jouanolou}

\begin{flushright}
Mateo Carmona \\ 
mateo.carmona@csg.igrothendieck.org \\
Coordinator of the ``Centre of Grothendieck Studies (CSG)''
\end{flushright}


%%%%%%%%%%%%%%%%%%%%%%%%%%%%%%%%%%%%%


\renewcommand{\contentsname}{TABLE DE MATIÈRES}
\tableofcontents\thispagestyle{empty}




%%%%%%%%%%%%%%%%%%%%%%%%%%%%%%%%%%%%%%%%%

%Begin






%%%%%%%%%%%%%%%%%%%%%%%%%%%%%%%%%%%%%%%%%%%%%%%%%%%%%%%%%%%%%%%
\chapter*{\S \space I. --- CATÉGORIES DES FAISCEAUX SUR UN IDÉOTOPE}
\addcontentsline{toc}{section}{I. Catégorie des faisceaux sur un idéotope}
\label{ch:1}
\section*{}


%%%%%%%%%%%%%%%%%%%%%%%%%%%%%%%%%%%%
\subsection*{1. Généralités.}
\addcontentsline{toc}{subsection}{1. Généralités}

\vskip .3cm
{
Définition {\bf 1.1}. --- \it On appelle \emph{idéotope} un triple $(X, A, J)$ formé d'un topos $X$, d'un anneau commutatif unifère $A$ et d'un idéal propre $J$ de $A$.
}
\vskip .3cm
On suppose donné dans la suite du paragraphe un idéotope $(X, A, J)$. On note $A-\Mod_X$ la catégorie des faisceaux de $A_X$-Modules et 
$$
\underline{\Hom}(\mathbf{N}^\circ, A-\Mod_X)
$$
la catégorie abélienne des systèmes projectifs indexés par $\mathbf{N}$ de $A_X$-Modules.
\vskip .3cm
{
Définition {\bf 1.2}. --- \it On appelle $(A, J)$-\emph{faisceau} sur $X$, ou s'il n'y a pas de confusion possible $A$-\emph{faisceau} sur $X$, un système projectif
$$
F = (F_n, u_{m, n})_{(m, n) \in \mathbf{N} \times \mathbf{N}, m \geq n}
$$
de $A_X$-Modules, vérifiant
$$
J^{n + 1} F_n = 0
$$
pour tout entier $n \geq 0$. On note $\mathcal{E}(X, J)$ la sous-catégorie, abélienne, pleine de $\underline{\Hom}(\mathbf{N}^\circ, A-\Mod_X)$ engendrée par les $A$-faisceaux.
}
\vskip .3cm
Pour des raisons qui apparaîtront par la suite, la catégorie $\mathcal{E}(X, J)$ ne mérite pas le nom de catégorie des $A$-faisceaux sur $X$; c'est seulement une catégorie quotient de la précédente que nous baptiserons ainsi. Aussi, pour éviter le risque de confusion, nous arrivera-t-il, étant donnés deux $A$-faisceaux $E$ et $F$, de noter
$$
\Hom_a(E, F)
$$
($a$ pour anodin) l'ensemble des $\mathcal{E}(X, J)$-morphismes de $E$ dans $F$.

Notons pour tout objet $T$ de $X$ par $\mathbf{T}$, ou même $T$ s'il n'y a pas de confusion possible, le topos $X/T$. Le foncteur restriction pour les faisceaux de $A$-Modules induit de fa\c{c}on évidente un foncteur restriction
$$
\mathcal{E}(X, J) \to \mathcal{E}(T, J)
$$
$$
E \mapsto E | T.
$$
\vskip .3cm
{
Proposition-définition {\bf 1.4}. --- \it Soit $E = (E_n)_{n \in \mathbf{N}}$ un $A$-faisceau sur $X$:
\begin{enumerate}
    \item[1)] On dit que $E$ est \emph{essentiellement nul} s'il est nul en tant que pro-objet, ce qui revient à dire que pour tout entier $n \geq 0$, il existe un entier $p \geq 0$ tel que le morphisme de transition
    $$
    E_{n + p} \to E_n
    $$
    soit nul.
    \item[2)] On dit que $E$ est \emph{négligeable} s'il vérifie l'une des relations équivalentes suivantes~:
    \begin{enumerate}
        \item[(i)] Il existe un recouvrement $(T_i \to e_X)_{i \in I}$ de l'objet final $e_X$ de $X$ tel que les $A$-faisceaux $E|T_i$ soient essentiellement nuls.
        \item[(ii)] Idem, mais en supposant de plus que les $T_i$ sont des ouverts de $X$.
    \end{enumerate}
\end{enumerate}
}
\vskip .3cm
{\bf Preuve}: Pour voir l'équivalence de (i) et (ii), il suffit d'observer que pour tout $i \in I$, le faisceau image $U_i$ de $T_i$ par le morphisme canonique $T_i \to e_X$ est tel que le morphisme restriction
$$
\mathbf{U}_i \to \mathbf{T}_i
$$
soit fidèle.

Il est clair que lorsque l'objet final de $X$ est quasicompact (SGA4 VI 1.1), il revient au même pour un $A$-faisceau de dire qu'il est essentiellement nul ou qu'il est négligeable. Il est par ailleurs immédiat que la sous-catégorie pleine
$$
N(X, J) \quad \text{(ou plus simplement}~N_X)
\leqno{(1.4.1)}
$$
de $\mathcal{E}(X, J)$ engendré par les $A$-faisceaux négligeables est \emph{épaisse} dans $\mathcal{E}(X, J)$.
\vskip .3cm
{
Définition {\bf 1.5}. --- \it Soit $(X, A, J)$ un idéotope. On appelle \emph{catégorie} des $(A, J)$-\emph{faisceaux} (ou $A$-faisceaux s'il n'y a pas de confusion possible) sur $X$ et on note
$$
(A, J)-\fsc(X) \quad \text{(ou plus simplement}~A-\fsc(X))
$$
la catégorie abélienne quotient (thèse Gabriel III.1)
$$
\mathcal{E}(X, J)/N_X.
$$
}
\vskip .3cm
{\bf 1.6}. Soit $T$ un objet de $X$. Il est clair que le foncteur restriction (1.3) est exact et envoie $N_X$ dans $N_T$, d'où par passage au quotient un foncteur exact, appelé encore \emph{restriction},
$$
r_{T, X}: A-\fsc(X) \to A-\fsc(T).
\leqno{(1.6.1)}
$$
Soient maintenant $T$ et $T'$ deux objets de $X$ et $f: T \to T'$ un morphisme. Se pla\c{c}ant dans le topos $\mathbf{T}'$, on déduit de (1.6.1) un foncteur exact
$$
f^*: A-\fsc(T') \to A-\fsc(T),
\leqno{(1.6.2)}
$$
vérifiant les propriétés de transitivité habituelles.

Ces remarques étant faites, nous utiliserons dans la suite sans plus d'explications le langage local pour les $A$-faisceaux.
\vskip .3cm
{
Proposition {\bf 1.7}. --- \it Les propriétés suivantes sont de nature locale pour la topologie de $X$. 
\begin{enumerate}
    \item[(i)] La propriété pour un $A$-faisceau d'être nul, i.e. isomorphe au système projectif nul.
    \item[(ii)] La propriété pour une suite
    $$
    E' \xlongrightarrow{u} E \xlongrightarrow{v} E''
    $$
    de $A$-faisceaux d'être exacte.
    \item[(iii)] La propriété pour un morphisme $u: E \to F$ de $A$-faisceaux d'être un monomorphisme (resp. un épimorphisme, resp. un isomorphisme).
    \item[(iv)] La propriété pour deux morphismes $u, v: E \rightrightarrows F$ d'être égaux.
\end{enumerate}
}
\vskip .3cm
{\bf Preuve} : L'assertion (i) est immédiate. On en déduit (ii) en l'appliquant successivement à $\text{Im}(v \circ u)$ et à $\Ker(v)/\text{Im}(u)$. L'assertion (iii) est un cas particulier de (ii). Enfin (iv) s'obtient en appliquant (i) à $\text{Im}(v-u)$.
\vskip .3cm
{
Corollaire {\bf 1.7.1}. --- \it Soient $T$ et $T'$ deux objets de $X$ et $f: T \to T'$ un épimorphisme. Le foncteur
$$
f^*: A-\fsc(T') \to A-\fsc(T)
$$
est fidèle.
}
\vskip .3cm
{\bf Preuve}: Appliquer 1.7 (i) au topos $\mathbf{T}'$.
\vskip .3cm
{
Corollaire {\bf 1.7.2}. --- \it Soient $E$ et $F$ deux $A$-faisceaux sur $X$. Lorsque $T$ parcourt les objets de $X$, le préfaisceau
$$
T \mapsto \Hom(E|T, F|T)
$$
est séparé.
}
\vskip .3cm
{\bf Preuve}: Simple traduction de 1.7 (iv).
\vskip .3cm
{\bf Remarque 1.7.3}. En général, le préfaisceau précédent n'est pas un faisceau. Nous verrons toutefois qu'il en est ainsi, à peu de choses près, lorsque le topos $X$ est noethérien (SGA4 VI 2.11), ou lorsque $E$ est de type $J$-adique (3).








%End

%Begin














%%%%%%%%%%%%%%%%%%%%%%%%%%%%%%%%%%%%
\subsection*{2. Cas où l'objet final de $X$ est quasicompact.}
\addcontentsline{toc}{subsection}{2. Cas où l'objet final de $X$ est quasicompact}

On se propose maintenant de donner un certain nombre de catégories équivalentes à $A-\fsc(X)$, lorsque l'objet final de $X$ est quasicompact. Nous aurons besoin pour cela d'un certain nombre de lemmes techniques, dont la plupart n'utilisent pas cette hypothèse.

\vskip .3cm
{\bf 2.1}. Soit $(X, A, J)$ un idéotope. Étant donnés un objet $M$ de $\underline{\Hom}(N^\circ, A-\Mod_X)$, et une application croissante $\gamma \geq \id: \mathbf{N} \to \mathbf{N}$, on définit un nouveau système projectif $c_\gamma(M)$ en posant
$$
c_\gamma(M)_n = M_{\gamma(n)} \quad (n \geq 0),
$$
avec les morphismes de transition évidents. De plus, si $\gamma$ et $\delta$ sont deux applications de ce type, avec $\delta \geq \gamma$, on a un morphisme évident de systèmes projectifs
$$
c_\delta(M) \to c_\gamma(M).
$$
Ceci permet de définir une nouvelle catégorie, notée
$$
\underline{\Hom}_1(\mathbf{N}^\circ, A-\Mod_X), \quad \text{comme suit}.
$$
\begin{itemize}
    \item[(i)] Ses objets sont ceux de $\underline{\Hom}(N^\circ, A-\Mod_X)$. 
    \item[(ii)] Si $M$ et $M'$ sont deux objets de $\underline{\Hom}(N^\circ, A-\Mod_X)$, l'ensemble des morphismes de $M$ dans $M'$ est
    $$
    \Hom_1(M, M') = \varinjlim_\gamma \Hom(c_\gamma(M), M'),
    $$
    avec la loi de composition évidente.
\end{itemize}
Il est clair qu'un morphisme de $\underline{\Hom}(N^\circ, A-\Mod_X)$ définit un morphisme de $\underline{\Hom}_1(N^\circ, A-\Mod_X)$, d'où un foncteur ``projection''
$$
q: \underline{\Hom}(N^\circ, A-\Mod_X) \to \underline{\Hom}_1(N^\circ, A-\Mod_X),
$$
qui est une bijection sur les objets.
\vskip .3cm
{\bf 2.2}. Soient maintenant $M$ un objet de $\mathcal{E}(X, J)$ et $\gamma \geq \id: \mathbf{N} \to \mathbf{N}$ une application croissante. On définit un nouvel objet $\chi_\gamma(M)$ de $\mathcal{E}(X, J)$, fonctoriel en $M$, par
$$
(\chi_\gamma (M))_n = M_{\gamma(n)}/J^{n+1}M_{\gamma(n)} \quad (n \geq 0),
$$
avec les morphismes de transition évidents. De plus, si $\delta$ est une application de même type, avec $\delta \geq \gamma$, on a un morphisme canonique
$$
\chi_\delta (M) \to \chi_\gamma (M).
$$
Ceci permet de définir une nouvelle catégorie $\mathcal{E}_1(X, J)$ comme suit.
\begin{itemize}
    \item[(i)] Ses objets sont ceux de $\mathcal{E}(X, J)$.
    \item[(ii)] Si $M$ et $N$ sont deux objets de $\mathcal{E}(X, J)$, l'ensemble des $\mathcal{E}_1(X, J)$-morphismes de $M$ dans $N$ est
    $$
    \Hom_1(M, N) = \varinjlim_\gamma \Hom(\chi_\gamma(M), N),
    $$
    avec la loi de composition évidente.
\end{itemize}
De même que précédemment, on a un foncteur projection
$$
q: \mathcal{E}(X, J) \to \mathcal{E}_1(X, J).
$$

Il est clair que la catégorie $\mathcal{E}_1(X, J)$ s'identifie à la sous-catégorie pleine de $\underline{\Hom}_1(\mathbf{N}^\circ, A-\Mod_X)$ engendrée pas les $A$-faisceaux, et que le diagramme
\[\begin{tikzcd}
	{\underline{\Hom}(\mathbf{N}^\circ, A-\Mod_X)} && {\underline{\Hom}_1(\mathbf{N}^\circ, A-\Mod_X)} \\
	{\mathcal{E}(X, J)} && {\mathcal{E}_1(X, J),}
	\arrow["q", from=1-1, to=1-3]
	\arrow["q", from=2-1, to=2-3]
	\arrow[hook', from=2-1, to=1-1]
	\arrow[hook', from=2-3, to=1-3]
\end{tikzcd}\]
dans lequel les flèches verticales sont les inclusions canoniques, est commutatif.

Enfin, on vérifie, comme dans (SGA5 V 2.4.1), que la famille des flèches canoniques du type $\chi_\gamma(M) \to M$ permet un calcul de fractions à droite, ce qui permet d'identifier $\mathcal{E}_1(X, J)$ à la catégorie obtenue à partir de $\mathcal{E}(X, J)$ en inversant ces flèches.
\vskip .3cm
{
Lemme {\bf 2.3}. --- \it La catégorie $\underline{\Hom}_1(\mathbf{N}^\circ, A-\Mod_X)$ (resp. $\mathcal{E}_1(X, J)$) est abélienne et le foncteur $q$ est exact.
}
\vskip .3cm
{\bf Preuve}~: Montrons tout d'abord l'assertion non respée. Il est évident que la catégorie $\underline{\Hom}_1(\mathbf{N}^\circ, A-\Mod_X)$ est additive et que le foncteur $q$ rend inversibles toutes les flèches canoniques de la forme $c_\gamma(M) \to M$. Si $u: P \to Q$ est un élément de $\Hom_1(P, Q)$, alors $u$ est la classe d'une flèche $f: c_\gamma (P) \to Q$ et, en notant $c: c_\gamma (P) \to P$ la flèche canonique, on a donc $u = q(f) \circ q(c)^{-1}$ ; par suite, toute flèche de $\underline{\Hom}_1(\mathbf{N}^\circ, A-\Mod_X)$ est isomorphe à l'image par $q$ d'une flèche de $\underline{\Hom}(\mathbf{N}^\circ, A-\Mod_X)$. Il en résulte que pour voir que $\underline{\Hom}_1(\mathbf{N}^\circ, A-\Mod_X)$ est abélienne, il suffit de montrer que le foncteur $q$ est exact. Soient donc $P$ et
$$
0 \to M' \to M \to M'' \to 0
$$
respectivement un objet et une suite exacte de $\underline{\Hom}(\mathbf{N}^\circ, A-\Mod_X)$.

Comme les foncteurs $c_\gamma$ sont exacts et les limites inductives filtrantes de groupes abéliens sont exactes, les suites évidentes
$$
0 \to \varinjlim_\gamma \Hom(c_\gamma(P), M') \to \varinjlim_\gamma(c_\gamma(P), M) \to \varinjlim_\gamma \Hom(c_\gamma(P), M'')
$$
et
$$
0 \to \varinjlim_\gamma \Hom(c_\gamma(M''), P) \to \varinjlim_\gamma \Hom(c_\gamma(M), P) \to \varinjlim_\gamma \Hom(c_\gamma(M'), P)
$$
sont exactes, d'où l'assertion. L'assertion respée se voit de fa\c{c}on analogue; en fait, on montre en même temps que le foncteur d'inclusion
$$
\mathscr{E}_1(X, J) \to \underline{\Hom}_1(\mathbf{N}^\circ, A-\Mod_X)
$$
est exact
\vskip .3cm
{\bf 2.4}. Soient $P$ et $Q$ deux objets de $\underline{\Hom}(\mathbf{N}^\circ, A-\Mod_X)$, $\gamma \geq \id$ une application croissante de $\mathbf{N}$ dans $\mathbf{N}$ et $f: c_\gamma(P) \to Q$ un morphisme de systèmes projectifs. Désignant pour tout $n \in \mathbf{N}$ par $\xi_n$ la classe de $f_n$ dans $\varinjlim_m \Hom(P_m, Q_n)$, il est clair que l'ensemble des $\xi_n$ $(n \in \mathbf{N})$ définit un élément de 
$$
\varprojlim_n \varinjlim_m \Hom(P_m, Q_n),
$$
qui ne dépend que de la classe de $f$ dans $\Hom_1(P, Q)$. On définit ainsi un foncteur
$$
\rho: \underline{\Hom}_1(\mathbf{N}^\circ, A-\Mod_X) \to \Pro(A-\Mod_X),
$$
d'où un foncteur
$$
\delta: \mathcal{E}_1(X, J) \to \Pro(A-\Mod_X),
$$
obtenu par restriction de $\rho$ à $\mathcal{E}_1(X, J)$.
\vskip .3cm
{
Lemme {\bf 2.5}. --- \it Les foncteurs $\rho$ et $\delta$ ci-dessus sont des injections sur les objets et sont pleinement fidèles.
}
\vskip .3cm
{\bf Preuve}~: Il suffit de le voir pour $\rho$. Dans ce cas, la seule assertion non tautologique est que $\rho$ est pleinement fidèle. Soient donc $P$ et $Q$ deux objets de $\underline{\Hom}(\mathbf{N}^\circ, A-\Mod_X)$, et montrons que l'application canonique
$$
\varinjlim_\gamma \Hom(c_\gamma(P), Q) \to \varprojlim_n \varinjlim_m \Hom(P_m, Q_n)
$$
est bijective.

{\bf Elle est injective}.  Soient $\gamma$ et $\gamma'$ deux applications croissantes $\geq \id$ de $\mathbf{N}$ dans $\mathbf{N}$, et
$$
f: c_\gamma(P) \to Q \quad \text{et} \quad f': c_{\gamma'}(P) \to Q
$$
deux morphismes ayant même image dans $\Pro \Hom(P, Q)$. Par hypothèse, pour tout entier $n \geq 0$, les morphismes $f_n$ et $f'_n$ définissent le même élément de $\varinjlim_m \Hom(P_m, Q_n)$, donc il existe un entier $\varphi(n) \geq \max(\gamma(n), \gamma'(n))$ tel que les composés 
$$
\begin{cases}
    P_{\varphi(n)} \xlongrightarrow{\can} P_{\gamma(n)} \xlongrightarrow{f_n} Q_n \\
    P_{\varphi(n)} \xlongrightarrow{\can} P_{\gamma'(n)} \xlongrightarrow{f'_n} Q_n
\end{cases}
$$
soient égaux. Il est alors immédiat que l'application $\gamma'': \mathbf{N} \to \mathbf{N}$ définie par 
$$
\gamma''(n) = \max_{n' \leq n}\delta(n')
$$
est croissante supérieure à l'identité, et que pour tout $n \geq 0$ les composés
$$
P_{\gamma''(n)} \xlongrightarrow{\can} P_{\gamma(n)} \xlongrightarrow{f_n} Q_n
$$
et
$$
P_{\gamma''(n)} \xlongrightarrow{\can} P_{\gamma'(n)} \xlongrightarrow{f'_n} Q_n
$$
sont égaux.

{\bf Elle est surjective}. Soient donnés
\begin{itemize}
    \item[a)] Une application $\theta: \mathbf{N} \to \mathbf{N}$.
    \item[b)] Une application $\lambda: \mathbf{N} \times \mathbf{N} \to \mathbf{N}$ vérifiant 
    $$
    \lambda(j, k) \geq \max (\theta(j), \theta(k))
    $$
    pour tout couple $(j, k)$.
    \item[c)] Pour tout entier $j$, un morphisme
    $$
    \xi_j: P_{\theta(j)} \to Q_j
    $$
    tel que, dès que $k \geq j$, le diagramme évident
    \[\begin{tikzcd}
	& {P_{\theta(j)}} && {Q_j} \\
	{P_{\lambda(j, k)}} \\
	& {P_{\theta(k)}} && {Q_k}
	\arrow["\can"', from=3-4, to=1-4]
	\arrow["{\xi_k}"', from=3-2, to=3-4]
	\arrow["{\xi_j}", from=1-2, to=1-4]
	\arrow["\can", from=2-1, to=1-2]
	\arrow["\can"', from=2-1, to=3-2]
    \end{tikzcd}\]
    soit commutatif.
\end{itemize}
Il s'agit de trouver une application croissante $\gamma \geq \id$ de $\mathbf{N}$ dans $\mathbf{N}$ et un morphisme $f: c_\gamma(P) \to Q$ tel que pour tout $j \in \mathbf{N}$, les morphismes $f_j$ et $\xi_j$ aient même classe dans 
$$
\varinjlim_i \Hom(P_i, Q_j).
$$
On vérifie aisément que le couple $(\gamma, f)$ défini par 
$$
\begin{cases}
    \gamma(j) = \sup_{k, l \leq j} \lambda(k, l) \\ 
    f_j P_{\gamma(j)} \xlongrightarrow{\can} P_{\theta(j)} \xlongrightarrow{\xi_j} Q_j 
\end{cases}
$$
répond à la question.
\vskip .3cm
{\bf 2.6}. Supposons maintenant que l'objet final de $X$ soit quasicompact. Il est clair que le foncteur $q: \mathcal{E}(X, J) \to \mathcal{E}_1(X, J)$ envoie sur $O$ les $A$-faisceaux essentiellement nuls, ou, ce qui renvient au même négligeables. D'après la propriété universelle des catégories abéliennes quotients (Thèse Gabriel III 1 Cor.2), il admet donc une factorisation unique
\[\begin{tikzcd}
	{\mathcal{E}(X, J)} && {\mathcal{E}_1(X, J)}\arrow["{\overline{q}}"', from=2-2, to=1-3] \\
	& {A-\fsc(X)}
	\arrow["q", from=1-1, to=1-3]	
	\arrow["{\pi_X}"', from=1-1, to=2-2]
\end{tikzcd}\]
avec $\overline{q}$ un foncteur exact et $\pi_X$ le foncteur canonique de passage au quotient.
\vskip .3cm
{
Lemme {\bf 2.7}. --- \it Le foncteur $\overline{q}$ ci-dessus est un isomorphisme de catégories.
}
\vskip .3cm
{\bf Preuve}~: Tenant compte de l'interprétation de $\mathcal{E}_1(X, J)$ comme catégorie de fractions (2.2), il s'agit de voir que si $C$ est une catégorie abélienne et
$$
F: \mathcal{E}(X, J) \to C
$$
un foncteur exact, les assertions suivantes sont équivalentes :
\begin{itemize}
    \item[(i)] $F$ annulle tout $A$-faisceau essentiellement nul. 
    \item[(ii)] $F$ rend inversible toute flèche canonique $\chi_\gamma(M) \to M$.
\end{itemize}
Si $M$ est un $A$-faisceau essentiellement nul, il existe une application croissante $\gamma \geq \id: \mathbf{N} \to \mathbf{N}$ telle que la flèche canonique $\chi_\gamma(M) \to M$ soit nulle, d'où (ii) $\Rightarrow$ (i). Prouvons (i) $\Rightarrow$ (ii).

Si $P$ et $Q$ sont définis par l'exactitude de la suite
$0 \to P \to \chi_\gamma(M) \xlongrightarrow{\can} M \to Q \to 0$, il est immédiat que les morphismes canoniques $\chi_\gamma(P) \to P$ et $\chi_\gamma(Q) \to Q$ sont nuls, d'où aussitôt l'assertion.
\vskip .3cm
{\bf Remarque 2.7.1}. Plus généralement, l'argument précédent montre, sans hypothèse sur $X$, que $\mathcal{E}_1(X, J)$ est la catégorie abélienne quotient de $\mathcal{E}(X, J)$ par la sous-catégorie, abélienne engendrée par les $A$-faisceaux essentiellement nuls. On en déduit grâce à la description des morphismes d'une catégorie abélienne quotient, que tout morphisme (resp. isomorphisme) de $A-\fsc(X)$ est localement l'image d'un morphisme de $\mathcal{E}(X, J)$.
\vskip .3cm
{
Proposition {\bf 2.8}. --- \it Soit $(X, A, J)$ un idéotope. On suppose que l'objet final de $X$ est quasicompact. Alors le foncteur canonique
$$
\sigma \circ \overline{q}: A-\fsc(X) \to \Pro(A-\Mod_X)
$$
induit un isomorphisme de $A-\fsc(X)$ sur la sous-catégorie pleine de $\Pro(A-\Mod_X)$ engendrée par les $A$-faisceaux.
}
\vskip .3cm
{\bf Preuve}: Résulte immédiatement de (2.5) et (2.7).
\vskip .3cm
{\bf Remarques 2.8.1}. La proposition précédente s'applique notamment lorsque $X$ est le topos ponctuel.
\vskip .3cm
{\bf 2.8.2}. La preuve qu'on a donnée de (2.8) n'utilise pas le fait que la catégorie des pro-objets d'une catégorie abélienne, et il ne semble pas que l'utilisation de ce fait apporte des simplifications notables.

Soit $X$ un topos localement noethérien (SGA4 VI 2.11).

On rappelle que la sous-catégorie pleine de $X$ engendrée par les objets noethériens est stable par produits fibrés finis et que munissant $C$ de la structure de site induite par la topologie de $X$, le foncteur canonique
$$
X \to \widetilde{C}
$$
est une équivalence de catégories.
\vskip .3cm
{
Lemme {\bf 2.9}. --- \it Soient $(X, A, J)$ un idéotope, avec $X$ un topos localement noethérien, et $E$ et $F$ deux $A$-faisceaux sur $X$. La restriction au site des objets noethériens de $X$ du préfaisceau
$$
T \to \Hom(E|T, F|T)
$$
est un \emph{faisceau}.
}
\vskip .3cm
{\bf Preuve}~: On est ramené à voir que si $T$ est un objet noethérien de $X$ et $(T_i \to T)_{i \in I}$ est un recouvrement \emph{fini} de $T$ par des objets noethériens, alors la suite canonique
$$
0 \to \varinjlim_\gamma \Hom_a(\chi_\gamma (E)|T, F|T) \to \prod \varinjlim_\gamma \Hom_a(\chi_\gamma(E)|T, F|T),
$$
$$
\prod_j \varinjlim_\gamma \Hom_a(\chi_\gamma (E)| T_i \times_T T_j, F|T_i \times_T T_j)
$$
est exacte (2.7). Comme les produits finis sont des sommes directes et les limites inductives filtrantes de groupes abéliens sont exactes, l'assertion résulte de l'exactitude des suites canoniques 
$$
0 \to \Hom_a(\chi_\gamma (E) | T, F|T) \to \prod_i \Hom_a(\chi_\gamma(E)|T_i, F|T_i) \to 
$$
$$
\to \prod_{i, j} \Hom_a(\chi_\gamma (E) | T_i \times_T T_j, F|T_i \times_T T_j).
$$








% End
% Begin












%%%%%%%%%%%%%%%%%%%%%%%%%%%%%%%%%%%%
\subsection*{3. $A$-faisceaux de type constant, strict ou $J$-adique.}
\addcontentsline{toc}{subsection}{3. $A$-faisceaux de type constant, strict ou $J$-adique}

Soit $(X, A, J)$ un idéotope.
\vskip .3cm
{
Proposition {\bf 3.1}. --- \it Étant donné un $A$-faisceau $F$ sur $X$, les assertions suivantes sont équivalentes :
\begin{itemize}
    \item[(i)] $F$ est isomorphe (dans $A-\fsc(X)$) à un $A$-faisceau qui est un système projectif strict.
    \item[(ii)] $F$ est localement isomorphe à un $A$-faisceau qui est un système projectif strict.
    \item[(iii)] $F$ vérifie localement la condition de Mittag-Leffler (ML) (EGA$0_{III}$ 13.1.2).
\end{itemize}
}
\vskip .3cm
{\bf Preuve}: Il est clair que (i) $\Rightarrow$ (ii). On voit que (ii) $\Rightarrow$ (iii) en paraphrasant la preuve de (SGA5 V 2.5.1). Si maintenant (iii) est vérifiée, on peut parler localement du système projectif des images universelles de $F$ et ces divers systèmes projectifs d'images universelles se recollent pour donner un $A$-faisceau qui est un sous-système projectif strict $F'$ de $F$. Alors, paraphrasant toujours (SGA5 V 2.5.1), on voit que l'inclusion de $F'$ dans $F$ est localement un isomorphisme, et on conclut par (1.7 (iii)).
\vskip .3cm
{
Définition {\bf 3.2}. --- \it Un $A$-faisceau $F$ qui vérifie les conditions équivalentes de (3.1) est appelé \emph{de type strict}.
}
\vskip .3cm
Si $F$ est de type strict, nous avons vu dans le courant de la démonstration que pour tout entier $n \geq 0$, le système projectif 
$$
(\text{Im}(F_m \to F_n))_{m \geq n}
$$
admet localement, donc aussi globalement, une limite projective $F'_n$ qui s'identifie à un sous-faisceau de $F_n$, et nous avons ainsi défini un sous-$A$-faisceau $F'$ de $F$.
\vskip .3cm
{
Définition {\bf 3.3}. --- \it Si $F$ est un $A$-faisceau de type strict, le $A$-faisceau $F'$ défini ci-dessus est appelé \emph{$A$-faisceau strict associé à $F$}.
}
\vskip .3cm
Rappelons enfin que nous avons vu que l'inclusion canonique $F' \hookrightarrow F$ est un \emph{isomorphisme}.
\vskip .3cm
{
Proposition {\bf 3.4}. --- \it Soit $0 \to E' \to E \to E'' \to 0$ une suite exacte de $A-\fsc(X)$. Alors :
\begin{itemize}
    \item[(i)] Si $E$ est de type strict, $E''$ est de type strict. 
    \item[(ii)] Si $E'$ et $E''$ sont de type strict, il en est de même de $E$.
\end{itemize}
}
\vskip .3cm
{\bf Preuve}: On peut supposer que la suite exacte considérée est l'image d'une suite exacte de $\mathcal{E}(X, J)$ et alors l'énoncé résulte de l'application locale de (EGA $0_{III}$ 13.2.1).
\vskip .3cm
{\bf 3.5}. Convenant de noter 
$$
J-\Mod(X)
$$
la sous-catégorie, \emph{exacte} (i.e. stable par noyaux, conoyaux et extensions), et même épaisse, pleine de $A-\Mod_X$ engendrée par les $A$--Modules localement annulés par une puissance de $J$, associons à tout objet $M$ de $J-\Mod_X$ le $A$-faisceau localement essentiellement constant
$$
\overline{M} = \mathscr{E}_X (M) = (M/J^{n+1}M)_{n \in \mathbf{N}}.
$$
On définit ainsi de fa\c{c}on claire un foncteur additif
$$
J-\Mod(X) \to \mathcal{E}(X, J),
$$
d'où, en composant avec la projection canonique, un foncteur additif
$$
\mathcal{E}_X: J-\Mod(X) \to A-\fsc(X).
\leqno{(3.5.1)}
$$
\vskip .3cm
{
Proposition {\bf 3.6}. --- \it 
\begin{itemize}
    \item[1)] Le foncteur (3.5.1) ci-dessus est exact et pleinement fidèle. Son image essentielle
    $$
    \TC (X, J)
    $$
    est une sous-catégorie abélienne exacte de $A-\fsc(X)$.
    \item[2)] Les assertions suivantes pour un $A$-faisceau $F$ sur $X$ sont équivalentes :
    \begin{itemize}
        \item[(i)] $F$ appartient à l'image essentielle de (3.5.1).
        \item[(ii)] $F$ est isomorphe à un $A$-faisceau localement essentiellement constant.
        \item[(iii)] $F$ est localement isomorphe à un $A$-faisceau essentiellement constant.
    \end{itemize}
\end{itemize}
Un $A$-faisceau vérifiant ces conditions équivalentes est dit \emph{de type constant}. Un $A$-faisceau de type constant est de type strict. La propriété pour un $A$-faisceau d'être de type constant est de nature locale.
}
\vskip .3cm
{\bf Preuve}: 1) Soit $0 \to M \xlongrightarrow{u} N \xlongrightarrow{v} P \to 0$ une suite exacte de $J-\Mod_X$, d'où une suite
$$
0 \to \overline{M} \xlongrightarrow{\overline{u}} \overline{N} \xlongrightarrow{\overline{v}} \overline{P} \to 0
$$
de $\mathcal{E}(X, J)$. Comme les systèmes projectifs $\overline{M}, \overline{N}, \overline{P}$ sont localement essentiellement constants, on voit que, localement, pour $n$ assez grand, 
$$
(\Ker \overline{u})_n = (\Coker \overline{v})_n = (\Ker \overline{v} / \text{Im} \overline{u})_n = 0,
$$
donc les systèmes projectifs $\Ker (\overline{u})$, $\Coker (\overline{v})$, $\Ker (\overline{v})/\text{Im} (\overline{u})$ sont négligeables, d'où l'exactitude de (3.5.1). Montrons qu'il est fidèle. Soit $u: M \to N$ un morphisme de $J-\Mod_X$ tel que $\overline{u}$ définisse le morphisme nul de $A-\fsc(X)$ et montrons que $u = 0$. Dans $\mathcal{E}(X, J)$, le $A$-faisceau $\text{Im}(\overline{u})$ est négligeable ; l'assertion à prouver étant locale, on peut supposer qu'il est essentiellement nul et que $M$ et $N$ sont annulés par une puissance de $J$. Alors, comme $\text{Im}(\overline{u})$ est contenu dans le système projectif essentiellement constant $\overline{N}$, on a évidemment $(\text{Im}(\overline{u}))_n = 0$ pour $n$ assez grand. Mais, toujours pour $n$ assez grand, on a $(\overline{u})_n = 0$, d'où l'assertion.

Soit maintenant $\lambda: \overline{M} \to \overline{N}$ un morphisme de $A$-faisceaux et montrons qu'il existe un morphisme $u: M \to N$ tel que $\lambda = \overline{u}$ dans $A-\fsc(X)$. Comme on sait déjà que (3.5.1) est fidèle, il résulte de (1.7.2) que cette assertion est de nature locale. Par définition (Thèse Gabriel III 1), le morphisme $\lambda$ est la classe d'un triple $(\alpha, \beta, v)$ de $\mathcal{E}(X, J)$-morphismes (cf. diagramme)
\[\begin{tikzcd}
	{\overline{M}} && F \\
	E && {\overline{N},}
	\arrow["\beta"', from=2-3, to=1-3]
	\arrow["v", from=2-1, to=1-3]
	\arrow["\alpha", from=2-1, to=1-1]
\end{tikzcd}\]
avec $\alpha$ un monomorphisme, $\beta$ un épimorphisme et $\Coker(\alpha)$ et $\Ker (\beta)$ négligeables. D'après le caractère local de l'assertion à démontrer, on peut supposer qu'ils sont essentiellement nuls, et que $M$ et $N$ sont annulés par une puissance de $J$. Comme précédemment, on a alors $(\Ker \beta)_n = 0$ pour $n$ assez grand ; de plus, comme $\overline{M}$ est strict, $\alpha$ est un isomorphisme. On vérifie aussitôt que pour $n$ assez grand, le morphisme
$$
(\beta \circ v \circ \alpha^{-1})_n : (\overline{M})_n = M \to N = (\overline{N})_n
$$
ne dépend pas de $n$. Désignant par $u$ la valeur commune, on voit alors sans peine que $\lambda = \overline{u}$. La peine fidélité de (3.5.1) montre que la catégorie $\TC (X, J)$ est stable par noyaux et conoyaux dans $A-\fsc(X)$. Nous verrons la stabilité par extensions après avoir prouvé la deuxième partie.

2) Il est clair que (i) $\Rightarrow$ (ii) $\Rightarrow$ (iii). Montrons que (iii) $\Rightarrow$ (i). Faisons d'abord quelques préliminaires. Étant donné un $A$-faisceau $F$ on définit un nouveau $A$-faisceau $\overline{F}$ par
$$
(\overline{F})_n = (\varprojlim_m (F_m))/J^{n+1}(\varprojlim_m(F_m)),
$$
avec les morphismes de transition évidents. De plus, on a, de fa\c{c}on évidente un $\mathcal{E}(X, J)$-morphisme fonctoriel en $F$
$$
u_F : \overline{F} \to F.
$$
On vérifie sans peine que si $\alpha: F \to G$ est un $\mathcal{E}(X, J)$-morphisme dont le noyau et le conoyau sont négligeables, alors le morphisme correspondant $\overline{\alpha}: \overline{F} \to \overline{G}$ est un isomorphisme. Ceci permet de définir un $A-\fsc(X)$-morphisme, fonctoriel en $F$ lorsque ce dernier parcourt $A-\fsc(X)$,
$$
u_F : \overline{F} \to F.
\leqno{(3.6.1)}
$$
Supposons maintenant que $F$ soit localement dans l'image essentielle de (3.5.1) et montrons qu'il y est globalement. Étant donnés un objet $T$ de $X$, un objet $M$ de $J-\Mod_T$ et un $A-\fsc(T)$-isomorphisme
$$
F|T \isomlong \mathcal{E}_T(M),
$$
on voit sans peine que 
$$
(\varprojlim_m F_m) | T \isom M,
$$
et par suite $\varprojlim_m (F_m)$ est un objet de $J-\Mod_X$. On aura terminé si on prouve que $u_F$ définit un isomorphisme de $A-\fsc(X)$. C'est immédiat si $F = \mathscr{E}_X(M)$ ; dans le cas général, l'assertion étant locale (1.7), on peut supposer donné un isomorphisme
$$
i: F \isomlong \mathscr{E}_X(M) \quad \text{de} \quad A-\fsc(X)
$$
et alors l'assertion résulte de la commutativité du diagramme
\[\begin{tikzcd}
	{\overline{G}} && {\mathscr{E}_X(M) = G} \\
	{\overline{F}} && F
	\arrow["\sim"', from=1-1, to=1-3]
	\arrow["{u_G}", from=1-1, to=1-3]
	\arrow["{u_F}", from=2-1, to=2-3]
	\arrow["{\overline{i}}", from=2-1, to=1-1]
	\arrow["\sim"', from=2-1, to=1-1]
	\arrow["i"', from=2-3, to=1-3]
	\arrow["\sim", from=2-3, to=1-3]
\end{tikzcd}\]
dans $A-\fsc(X)$.

Montrons maintenant que $\TC(X, J)$ est stable par extensions dans $A-\fsc(X)$. On se ramène immédiatement à voir que si 
$$
0 \to F' \xlongrightarrow{\alpha} F \xlongrightarrow{\beta} F'' \to 0
$$
est une suite exacte de $\mathscr{E}(X, J)$, avec $F'$ et $F''$ définissant des objets de $\TC(X, J)$, il en est de même pour $F$. L'assertion étant locale d'après l'équivalence (i) $\Leftrightarrow$ (iii), on peut supposer que 
$$
F' \isom \mathscr{E}_X(M') \quad \text{et} \quad F'' \isom \mathscr{E}_X(M'')
$$
dans $A-\fsc(X)$, avec $M'$ et $M''$ annulés par une puissance de $J$. Avec les notations utilisés dans la preuve de 2), on a $\overline{F'} \isom \overline{M'}$ et le morphisme canonique $u_{F'}: \overline{F'} \to F'$ est à noyau et conoyau négligeables et, quitte à localiser, on peut les supposer essentiellement nuls. Dans ce cas, le $A$-faisceau $\Ker(u_{F'})$ vérifie, puisque $\overline{F'}$ est essentiellement constant,
$$
(\Ker u_{F'})_n = 0 \quad \text{pour}~n~\text{assez grand}.
$$
En particulier, $\text{Im}(u_{F'})$ est essentiellement constant. Quitte à remplacer $F''$ par $F/\text{Im}(\alpha \circ u_{F'})$, on peut donc supposer que $F'$ est essentiellement constant. Ceci dit, quitte à remplacer $F''$ par $\overline{F''}$ et $F$ par $\overline{F''} \times_{F''} F$, on peut de plus supposer que $F''$ est aussi essentiellement constant. Dans ce cas, $F$ l'est également, d'où l'assertion.
\vskip .3cm
{
Corollaire {\bf 3.7}. --- \it La sous-catégorie pleine $\TC(X, J)^+$ de $A-\fsc(X)$ engendrée par les $A$-faisceaux isomorphes à un $A$-faisceau de la forme $\mathcal{E}_X(M)$, avec $M$ un $A_X$--Module annulé par une puissance de $J$, est \emph{exacte}.
}
\vskip .3cm
{\bf Preuve} : Les objets de $\TC(X, J)^+$ sont ceux de $\TC(X, J)$ qui sont annulés par une puissance de l'idéal $J$.
\vskip .3cm
{\bf 3.8}. Soit $J-\ad(X)$ la sous-catégorie pleine de $\mathcal{E}(X, J)$ engendrée par les systèmes projectifs $J$-adiques (SGA5 V 3.1.1) de $A_X$--Modules. Le foncteur canonique
$$
\mathcal{E}(X, F) \to A-\fsc(X)
\leqno{(3.8.1)}
$$
induit un foncteur
$$
J-\ad(X) \to A-\fsc(X)
\leqno{(3.8.2)}
$$
et nous noterons 
$$
\TJ-\ad(X)
$$
l'image essentielle de (3.8.1), i.e. la sous-catégorie pleine de $A-\fsc(X)$ engendrée par les $A$-faisceaux qui sont isomorphes à un $A$-faisceau $J$-adique. Un tel $A$-faisceau sera dit \emph{de type $J$-adique}.
\vskip .3cm
{
Proposition {\bf 3.9}. --- \it Soient $E$ et $F$ deux $A$-faisceaux sur $X$.
\begin{itemize}
    \item[(a)] Si $E$ est $J$-adique, l'application canonique
    $$
    \varphi_X: \Hom_a(E, F) \to \Hom(E, F)
    $$
    est une bijection.
    \item[(b)] Si $E$ est de type $J$-adique, le préfaisceau
    $$
    T \mapsto \Hom(E|T, F|T)
    $$
    sur $X$ est un \emph{faisceau}.
\end{itemize}
}
\vskip .3cm
{\bf Preuve}: Il suffit de montrer a), l'assertion b) en étant conséquence immédiate. Si $u$ et $v: E \rightrightarrows F$ sont deux morphismes de systèmes projectifs vérifiant $\varphi_X(u) = \varphi_X(v)$, alors l'exactitude de (3.8.1) montre que $\text{Im}( v - u)$ est négligeable, donc nul puisque $E$ est strict, d'où $u = v$. Soit maintenant $a: E \to F$ un morphisme de $A-\fsc(X)$ et montrons qu'il est dans l'image de $\varphi_X$. Il résulte de (2.7.1) que localement $a$ est l'image d'un élément de $\varinjlim_\gamma \Hom(\chi_\gamma (E), F)$, donc provient d'un morphisme $u: E \to F$ de $\mathcal{E}(X, J)$, puisque les morphismes canoniques $\chi_\gamma(E) \to E$ sont des isomorphismes. Dans le cas général, on obtient ainsi un recouvrement $(T_i \to e_X)$ de l'objet final $e_X$ de $X$ et des morphismes de systèmes projectifs
$$
v_i: E|T_i \to F|T_i
$$
vérifiant $\varphi_{T_i}(v_i) = a|T_i$. D'après l'injectivité des applications $\varphi_{T_i \times T_j}$, les morphismes $v_i$ se recollent en un morphisme $v$ de $\mathcal{E}(X, J)$ vérifiant $\varphi_X(v) = a$ localement, donc aussi globalement (1.7.2).
\vskip .3cm
{
Corollaire {\bf 3.9.1}. --- \it Le foncteur 
$$
J-\ad(X) \to \TJ-\ad(X)
$$
induit par (3.8.1) est une \emph{équivalence de catégories}.
}
\vskip .3cm
{
Corollaire {\bf 3.9.2}. --- \it La catégorie fibrée 
$$
S \mapsto \TJ-\ad(S)
$$
au-dessus de $X$ est un \emph{champ}, autrement dit on a les propriétés suivantes.
\begin{itemize}
    \item[(i)] Si $F$ et $G$ sont deux $A$-faisceaux de type $J$-adique, le préfaisceau
    $$
    T \mapsto \Hom(F|T, G|T)
    $$
    sur $X$ est un \emph{faisceau}.
    \item[(ii)] Soit $(U_i \to e_X)_{i \in I}$ un recouvrement de l'objet final $e_X$ de $X$. Pour tout couple $(i, j)$ d'éléments de $I$ (resp. tout triple $(i, j, k)$), on pose
    $$
    U_{ij} = U_i \times U_j \quad \text{(resp.}~U_{ijk} = U_i \times U_j \times U_k).
    $$
\end{itemize}
Supposons donné pour tout $i \in I$ un $A$-faisceau de type $J$-adique $F_i$ sur $U_i$ et pour tout couple $(i, j)$ d'éléments de $I$ un isomorphisme
$$
\theta_{ji}: F_i | U_{ij} \isomlong F_j | U_{ij}
$$
de $A-\fsc(U_{ij})$. On suppose que 
\begin{itemize}
    \item[a)] Si $i \in I$, alors $\theta_{ii} = \id$.
    \item[b)] Si $(i, j, k) \in I^3$, $\theta_{ki} = \theta_{kj} \circ \theta_{ji}$ sur $U_{ijk}$.
\end{itemize}
Alors il existe un $A$-faisceau de type $J$-adique $F$ sur $X$ et pour tout $i \in I$ un isomorphisme
$$
\theta_i: F_i \isomlong F|U_i \quad \text{de} \quad A-\fsc(U_i)
$$
tels que pour tout couple $(i, j)$ d'éléments de $I$ on ait
$$
\theta_j \circ \theta_{ij} = \theta_i \quad \text{sur} \quad U_{ij}.
$$
}
\vskip .3cm
{\bf Preuve} : Résulte immédiatement de l'assertion analogue, évidente, pour la catégorie fibrée $T \mapsto J-\ad(T)$ sur $X$.
\vskip .3cm
{
Corollaire {\bf 3.9.3}. --- \it La propriété pour un $A$-faisceau d'être de type $J$-adique est de nature locale.
}
\vskip .3cm
{\bf Preuve} : Soit $F$ un $A$-faisceau localement de type $J$-adique, et $(U_i \to e_X)_{i \in I}$ un recouvrement de l'objet final $e_X$ de $X$ tel que les $A$-faisceaux $F_i = F|U_i$ soient de type $J$-adique. D'après (3.9.2 (ii)), on peut ``recoller'' les $F_i$ suivant un $A$-faisceau de type $J$-adique $F'$. Par ailleurs la proposition (3.9 b)) permet de recoller les morphismes identiques des $F_i$ en un morphisme $F' \to F$, qui est un isomorphisme localement, donc aussi globalement (1.7. (iii)).
\vskip .3cm
{
Proposition {\bf 3.10}. --- \it Soit $F$ un $A$-faisceau sur $X$. Les assertions suivantes sont équivalentes : 
\begin{itemize}
    \item[(i)] $F$ est de type $J$-adique. 
    \item[(ii)] $F$ est de type strict (3.2) et, notant $F'$ le $A$-faisceau strict associé à $F$ (3.3), il existe localement une application croissante $\gamma \geq \id: \mathbf{N} \to \mathbf{N}$ telle que $\chi_\gamma(F')$ soit $J$-adique.
\end{itemize}
De plus ces conditions impliquent la condition (iii) ci-dessous et lui sont équivalentes lorsque l'objet final $e_X$ de $X$ admet un recouvrement par les objets quasicompacts. 
\begin{itemize}
    \item[(iii)] Pour tout entier $r \geq 0$, le $A$-faisceau
    $$
    \tau_r(F) = (F_n \bigotimes_A (A/J^{r+1}))_{n \in \mathbf{N}}
    $$
    est de type constant (3.6).
\end{itemize}
}
\vskip .3cm
{\bf Preuve} : Si $F$ est de type $J$-adique, il existe un recouvrement $(U_i \to e_X)_{i \in I}$ de l'objet final de $X$ tel que pour tout $i \in I$ $F | U_i$ soit isomorphe au sens de $\mathcal{E}_1(U_i, J)$ (2.2) à un $A$-faisceau $J$-adique, et la réciproque est également vraie d'après (3.9.3).

L'équivalence de (i) et (ii) se voit alors en paraphrasant la preuve de (SGA5 V 3.2.3). Montrons que (i) $\Rightarrow$ (iii). Il résultera de (5.1) (le lecteur vérifiera que (3.10) n'est pas utilisé dans la preuve de (5.1)) que si $P$ et $Q$ sont deux $A$-faisceaux isomorphes, alors les $A$-faisceaux $\tau_r (P)$ et $\tau_r(Q)$ sont isomorphes. Par suite on peut supposer que $F$ est $J$-adique, et alors l'assertion est évidente. Pour voir que (iii) $\Rightarrow$ (i) sous l'hypothèse supplémentaire de l'énoncé, on peut grâce à (3.9.3) supposer que $e_X$ est quasicompact. Alors pour tout $r \geq 0$, $\tau_r(F)$ vérifie la condition de Mittagg-Leffler, d'où résulte aussitôt qu'il en est de même pour $F$. On peut donc supposer $F$ strict. Alors pour tout $r \geq 0$, $\tau_r(F)$ est de type $J$-adique, donc essentiellement constant par (i) $\Rightarrow$ (ii). Choisissant alors une application $\gamma \geq \id: \mathbf{N} \to \mathbf{N}$ telle que pour tout $r \geq 0$ le système projectif $\tau_r(F)$ soit constant à partir du rang $\gamma(r)$, il est immédiat que le $A$-faisceau $\chi_\gamma(F)$ est $J$-adique, et donc que $F$ vérifie (ii).
\vskip .3cm
{
Corollaire {\bf 3.11}. --- \it Soit $X$ un topos dont l'objet final est quasicompact. Les assertions suivantes pour un $A$-faisceau $F$ sur $X$ sont équivalentes :
\begin{itemize}
    \item[(i)] $F$ est de type $J$-adique. 
    \item[(ii)] $F$ vérifie la condition de Mittag-Leffler et, désignant par $F'$ le $A$-faisceau strict associé, il existe une application $\gamma \geq \id$ de $\mathbf{N}$ dans $\mathbf{N}$ telle $\chi_\gamma(F')$ soit $J$-adique.
\end{itemize}
}
\vskip .3cm
Signalons enfin l'énoncé suivant, dont la preuve se ramène localement à une paraphrase de celle de (SGA5 V 3.2.4) :
\vskip .3cm
{
Proposition {\bf 3.12}. --- \it Soit 
$$
0 \to F' \to F \to F'' \to 0
$$
une suite exacte de $A$-faisceaux. Alors : 
\begin{itemize}
    \item[(i)] Si $F'$ et $F$ sont respectivement de type strict et de type $J$-adique, alors $F''$ est de type $J$-adique. 
    \item[(ii)] Si $F$ et $F''$ sont respectivement de type strict et de type $J$-adique, alors $F'$ est de type strict.
\end{itemize}
}
\vskip .3cm
{\bf Remarque 3.13}. Bien entendu, comme dans le contexte de (SGA5 V et VI) ce sont là les seules stabilités des notions précédentes. Pour en avoir d'autres, il faudra introduire des conditions de finitude (cf. II).









% End
% Begin















%%%%%%%%%%%%%%%%%%%%%%%%%%%%%%%%%%%%
\subsection*{4. Opérations externes.}
\addcontentsline{toc}{subsection}{4. Opérations externes}

On suppose donné dans ce paragraphe un anneau commutatif unifère $A$ et un idéal propre $J$ de $A$.

\vskip .3cm
{\bf 4.1}. Soient $X$ et $Y$ deux topos et $f: X \to Y$ un morphisme de topos. Ayant choisi un foncteur image réciproque
$$
f^*: A-\Mod_Y \to A-\Mod_X,
$$
on définit un foncteur exact, noté de même,
$$
f^*: \mathcal{E}(Y, J) \to \mathcal{E}(X, J)
$$
en posant pour tout $A$-faisceau $F = (F_n)_{n \in \mathbf{N}}$ sur $Y$
$$
f^*(F) = (f^*(F_n))_{n \in \mathbf{N}}
$$
et pour tout $\mathcal{E}(Y, J)$-morphisme $u = (u_n)_{n \in \mathbf{N}}$,
$$
f^*(u) = (f^*(u_n))_{n \in \mathbf{N}}.
$$
Ce foncteur est exact et transforme évidemment $A$-faisceau négligeable en $A$-faisceau négligeable. On en déduit par passage au quotient un foncteur exact, appelé foncteur \emph{image réciproque par $f$}, 
$$
f^*: A-\fsc(Y) \to A-\fsc(X).
\leqno{(4.1.1)}
$$
Il est clair que deux foncteurs images réciproques de $A-\Mod_Y$ dans $A-\Mod_X$, étant isomorphes, définissent des foncteurs isomorphes de $A-\fsc(Y)$ dans $A-\fsc(X)$ ; par suite, on pourra parler, sans plus d'ambiguïté que dans le cas des faisceaux de $A$--Modules, ``du'' foncteur image réciproque. 
\vskip .3cm
{\bf Exemple 4.1.2}. Le foncteur restriction (1.6.2) associé à un morphisme $f: T \to T'$ d'objets d'un topos $X$ n'est autre que le foncteur image réciproque associé au morphisme de topos
$$
X/T \to X/T'
$$
correspondant.
\vskip .3cm
{\bf 4.1.3}. Si maintenant $f: X \to Y$ et $g: Y \to Z$ sont deux morphismes de topos, on définit sans peine, argument par argument, un isomorphisme de foncteurs
$$
(g \circ f)^* \isom f^* \circ g^*,
$$
vérifiant la condition de cocycles habituelle.
\vskip .3cm
{\bf 4.1.4}. Notant ``$\pt$'' le topos ponctuel, i.e. la catégorie des ensembles munie de la topologie canonique, on rappelle (2.8.1) que la catégorie $A-\fsc(\pt)$ s'identifie à la sous-catégorie pleine de $\Pro(A-\mod)$ engendrée par les systèmes projectifs $M = (M_n)_{n \in \mathbf{N}}$ de $A$--modules vérifiant $J^{n+1}M_n = 0$ pour tout $n \geq 0$. Si maintenant $X$ est un topos et 
$$
p: X \to \pt 
$$
le morphisme de topos canonique, le foncteur $p^*$ associe à tout système projectif $M$ comme ci-dessus un $A$-faisceau sur $X$, qui sera noté de même s'il n'y a pas de confusion possible.

Supposons maintenant que $A$ soit \emph{noethérien}. Il résulte du lemme d'Artin-Rees (EGA $0_I$ 7.3.2.1). que le foncteur
$$
M \mapsto (M/J^{n+1}M)_{n \in \mathbf{N}}
\leqno{(4.1.4.1)}
$$
de la catégorie $A-\modn$ des $A$--modules de type fini dans $A-\fsc(\pt)$ est exact et fidèle,et même pleinement fidèle lorsque $A$ est complet pour la topologie $J$-adique (EGA $0_I$ 7.8.2). Composant avec le foncteur $p^*$, on en déduit un foncteur exact et fidèle
$$
A-\modn \to A-\fsc(\pt),
\leqno{(4.1.4.2)}
$$
qui est de même pleinement fidèle lorsque $A$ est complet pour la topologie $J$-adique. Dans la suite, on identifiera si aucune confusion n'en résulte un $A$--module de type fini et le système projectif n'en résulte un $A$--module de type fini et le système projectif associé au moyen du foncteur (4.1.4.2). 
\vskip .3cm
{\bf 4.1.5}. Soit $X$ un topos. Tout foncteur point
$$
i: \pt \to X
$$
de $X$ définit un foncteur exact
$$
i^*: A-\fsc(X) \to A-\fsc(\pt) \hookrightarrow \Pro(A-\mod),
$$
qu'on appellera \emph{foncteur fibre associé à $i$}. On prendra garde que si $X$ admet une famille conservative $(i_r)_{r \in R}$ de foncteurs points, alors la famille de foncteurs exacts
$$
i^*_r: A-\fsc(X) \to A-\fsc(\pt)
$$
n'est pas en général conservative. Soient en effet $X$ un espace topologique quasicompact et $(x_r)_{r \in \mathbf{N}}$ une infinité dénombrable de points fermés de $X$, et notons $i_r: x_r \to X$ les inclusions canoniques. On peut choisir une infinité dénombrable $(P_r)_{r \in \mathbf{N}}$ de $A$-faisceaux essentiellement nuls sur le   topos ponctuel, telle qu'il n'existe aucune application croissante $\gamma \geq \id: \mathbf{N} \to \mathbf{N}$ pour laquelle les morphismes canoniques $\chi_\gamma(P_r) \to P_r$ soient simultanément nuls. Alors le système projectif de $A_X$--Modules
$$
P = \bigoplus_{r \in \mathbf{N}}(i_r)_* (P_r)
$$
n'est pas essentiellement nul et définit pourtant un $A$-faisceau qui est envoyé sur le $A$-faisceau nul par tous les foncteurs fibres associés aux points de $X$. Nous verrons cependant plus loin que ce genre d'inconvénient ne se produit plus lorsqu'on fait des hypothèse de finitude convenables sur les $A$-faisceaux envisagés.
\vskip .3cm
{\bf 4.2}. Soient $X$ et $Y$ deux topos et $f: X \to Y$ un morphisme. Avec les abus de langage usuels, on définit pour tout entier $p$ un foncteur additif
$$
\Rd^pf_*: \mathscr{E}(X, J) \to \mathscr{E}(Y, J)
$$
par la formule
$$
\Rd^p f_* ((F_n)_{n \in \mathbf{N}}) = (\Rd^p f_* (F_n))_{n \in \mathbf{N}}.
$$
Mieux, la collection des foncteurs $(\Rd^p f_*)_{p \in \mathbf{Z}}$ est munie de fa\c{c}on évidente d'une structure de foncteur cohomologique de $\mathscr{E}(X, J)$ dans $\mathscr{E}(Y, J)$.
\vskip .3cm
{
Lemme {\bf 4.2.1}. --- \it On suppose $f$ \emph{quasicompact} (SGA4 VI 3.1). Soit $u: F \to G$ un morphisme de $\mathscr{E}(X, J)$ dont le noyau et le conoyau sont négligeables. Alors pour tout entier $p \in \mathbf{Z}$, le morphisme 
$$
\Rd^p f_* (u): \Rd^p f_* (F) \to \Rd^p f_* (G)
$$
est à noyau et conoyau négligeables.
}
\vskip .3cm
{\bf Preuve} : Comme $f$ est quasicompact, on voit en se ramenant au cas où l'objet final de $X$ est quasicompact que pour tout $q \in \mathbf{Z}$ le foncteur $\Rd^q f_*$ transforme $A$-faisceau négligeant en $A$-faisceaux négligeable. Le lemme s'en déduit en utilisant la structure de foncteur cohomologique sur $(\Rd^p f_*)_{p \in \mathbf{Z}}$.

D'après les propriétés générales des catégories abéliennes quotients, la catégorie $A-\fsc(X)$ est obtenue à partir de $\mathcal{E}(X, J)$ en inversant les flèches dont le noyau et le conoyau sont négligeables. Il résulte de (4.2.1) que lorsque $f$ est \emph{quasicompact} le foncteur cohomologique $(\Rd^p f_*)_{p \in \mathbf{Z}}$ définit par passage au quotient un foncteur cohomologique noté de même
$$
(\Rd^p f_*)_{p \in \mathbf{Z}}: A-\fsc(X) \to A-\fsc(Y).
\leqno{(4.2.2)}
$$
Bien entendu, $\Rd^i f_* = 0$ pour $i < 0$, et en particulier le foncteur \emph{image directe} $f_* = \Rd^0 f_*$ est exact à gauche.
\vskip .3cm
{
Définition {\bf 4.2.3}. --- \it Soit $X$ un topos. On dit qu'un $A$-faisceau $F = (F_n)_{n \in \mathbf{N}}$ est \emph{flasque} si chacun des $F_n$ est un $A$--Module (ou, ce qui revient au même, un faisceau abélien) flasque.
}
\vskip .3cm
{
Proposition {\bf 4.2.4}. --- \it 
\begin{enumerate}
    \item[(i)] Soit $X$ un topos. Tout $A$-faisceau sur $X$ se plonge dans un $A$-faisceau flasque. 
    \item[(ii)] Soient $X$ et $Y$ deux topos et $f: X \to Y$ un morphisme quasicompact. Pour tout $A$-faisceau flasque $F$ sur $X$, le $A$-faisceau $f_*(F)$ est flasque et on a 
    $$
    \Rd^p f_* (F) = 0 \quad (p \geq 1),~\text{dans}~A-\fsc(Y).
    $$
    En particulier, pour tout entier $p \geq 1$, le foncteur $\Rd^p f_*$ est effa\c{c}able.
\end{enumerate}
}
\vskip .3cm
{\bf Preuve} : Lorsque $X$ admet suffisamment de points, le caractère fonctoriel de la ``résolution de Godement'' (SGA4 XII 3.4) permet de la prolonger aux systèmes projectifs, ce qui montre (i) dans ce cas.

Dans le cas général, on laisse au lecteur le soin de faire la construction pas à pas. D'ailleurs, nous verrons plus loin (6.6.3) que la catégorie $\mathcal{E}(X, J)$ possède suffisamment d'injectifs et que ceux-ci sont flasques, ce qui prouvera également le résultat annoncé. Quant à l'assertion (ii), elle est immédiate.
\vskip .3cm
{\bf 4.3}. Soient $X$ et $Y$ deux topos et $f: X \to Y$ un morphisme quasicompact. Soient $F$ un $A$-faisceau sur $Y$ et $G$ un $A$-faisceau sur $X$.

On définit de fa\c{c}on évidente, composant par composant, des ``morphismes d'adjonction''
$$
a_F: F \to f_* f^*(F)
\leqno{(4.3.1)}
$$
$$
b_G: f^* f_*(G) \to G
\leqno{(4.3.2)}
$$
fonctoriels en $F$ et $G$ respectivement. On en déduit des applications fonctorielles (Sém. CARTAN 11 Exp. 7 par 3)
$$
\varphi: \Hom(f^* F, G) \to \Hom(F, f_*(G))
\leqno{(4.3.3)}
$$
$$
\psi: \Hom(F, f_*(G)) \to \Hom(f^* F, G).
\leqno{(4.3.4)}
$$
\vskip .3cm
{
Proposition {\bf 4.3.5}. --- \it Les applications $\varphi$ et $\psi$ précédentes sont des bijections inverses l'une de l'autre. En particulier, le foncteur
$$
f^*: A-\fsc(Y) \to A-\fsc(X)
$$
est adjoint à gauche du foncteur
$$
f_*: A-\fsc(X) \to A-\fsc(Y).
$$
}
\vskip .3cm
{\bf Preuve} : Il suffit (cf. loc. cit.) de montrer que les composés
$$
f_* (G) \xlongrightarrow{a_{f_* (G)}} f_* f^* f_* (G) \xlongrightarrow{f_*(b_G)} f_* (G)
$$
et
$$
f^*(F) \xlongrightarrow{f^*(a_F)} f^* f_* f^* (F) \xlongrightarrow{b_{f^*(F)}} f^* (F)
$$
sont respectivement l'identité de $f_*(G)$ et celle de $f^*(F)$. Or cela est vrai au stade des composants, d'où l'assertion.
\vskip .3cm
{\bf 4.4}. Soit $X$ un topos dont l'\emph{objet final est quasicompact}, de sorte que (SGA4 VI 3.2), le morphisme canonique
$$
p: X \to \pt
$$
est \emph{quasicompact}. On définit de fa\c{c}on évidente un foncteur cohomologique
$$
(\overline{\mathrm{H}}^i (X, .))_{i \in \mathbf{Z}}: \mathscr{E}(X, J) \to \mathscr{E}(\pt, J) \quad (\to \Pro(A-\mod)),
$$
en posant pour tout $A$-faisceau $F = (F_n)_{n \in \mathbf{N}}$
$$
\overline{\mathrm{H}}^i (X, F) = (\overline{\mathrm{H}}^i(X, F_n))_{n \in \mathbf{N}}
\leqno{(4.4.1)}
$$
avec les morphismes de transition évidents.

Le même raisonnement qu'en (4.2) montre qu'il définit par passage au quotient un nouveau foncteur cohomologique, noté sans inconvénient de la même manière,
$$
(\overline{\mathrm{H}}^i (X, .))_{i \in \mathbf{Z}}: A-\fsc(X) \to A-\fsc(\pt) \hookrightarrow \Pro(A-\mod).
\leqno{(4.4.2)}
$$
Comme précédemment, $\overline{\mathrm{H}}^i = 0$ pour $i < 0$, et on pose $\overline{\mathrm{H}}^0 (X, .) = \overline{\Gamma}(X, .)$.

Identifiant de la fa\c{c}on habituelle les foncteurs $\Gamma$ et $p_*$ pour les $A_X$--Modules, on obtient une identification canonique entre les foncteurs $\overline{\mathrm{H}}^i (X, .)$ et $\Rd^i p_* (.)$, de sorte que les énoncés précédents peuvent être considérés comme une redite de (4.2). 
\vskip .3cm
{\bf 4.4.3}. Soient $X$ et $Y$ deux topos dont l'objet final est quasicompact et $f: X \to Y$ un morphisme. Si $F$ est un $A$-faisceau sur $Y$, l'image réciproque en cohomologie définit de fa\c{c}on évidente un morphisme de foncteurs cohomologiques
$$
f^*: \overline{\mathrm{H}}^p (Y, F) \to \overline{\mathrm{H}}^p (X, f^*(F))
$$
avec les propriétés habituelles (isomorphisme canonique pour le composé, avec condition de cocycles).
\vskip .3cm
{\bf 4.5}. Soient $X$ un topos et $f: T \to T'$ un morphisme quasicompact (SGA4 VI 1.7) de $X$. Le foncteur exact (SGA4 III 6.8)
$$
f_!: A-\Mod_T \to A-\Mod_{T'}
$$
définit de fa\c{c}on claire un foncteur exact
$$
f_!: \mathscr{E}(T, J) \to \mathscr{E}(T', J)
\leqno{(4.5.1)}
$$
en posant pour tout $A$-faisceau $F = (F_n)_{n \in \mathbf{N}}$ sur $\mathbf{T}$
$$
f_!(F) = (f_!(F_n))_{n \in \mathbf{N}}.
\leqno{(4.5.2)}
$$
Comme $f$ est quasicompact, le foncteur (4.5.1) transforme $A$-faisceau négligeable en $A$-faisceau négligeable, et définit par suite par passage au quotient un foncteur exact, noté de même,
$$
f_!: A-\fsc(T) \to A-\fsc(T').
\leqno{(4.5.3)}
$$
\vskip .3cm
{\bf 4.5.4}. Si $g: T' \to T''$ est un autre morphisme quasicompact de $X$, on définit, composant par composant sur les systèmes projectifs, un isomorphisme
$$
(g f)_! \isomlong g_! f_!
$$
vérifiant la condition de cocycles habituelle.
\vskip .3cm
{
Proposition {\bf 4.5.5}. --- \it Soient $X$ un topos et $f: T \to T'$ un morphisme \emph{quasicompact} de $X$. Le foncteur
$$
f_!: A-\fsc(T) \to A-\fsc(T')
$$
est adjoint à gauche du foncteur
$$
f^*: A-\fsc(T') \to A-\fsc(T).
$$
}
\vskip .3cm
{\bf Preuve} : On se ramène comme dans la preuve de (4.3.5) à l'assertion analogue pour les $A$--Modules.
\vskip .3cm
{\bf 4.6}. Soient $X$ un topos, $U$ un ouvert de $X$ et $Y$ le fermé complémentaire de $U$ (SGA4 IV 3.3). On note
$$
j: Y \to X
$$
le morphisme de topos canonique, et on rappelle que $j$ est quasicompact. Sur le modèle de (4.2), le foncteur cohomologique (\quad)
$$
(\Rd^p j^!)_{p \in \mathbf{Z}}: A-\Mod_X \to A-\Mod_Y
$$
permet de définir un foncteur cohomologique, noté de même, 
$$
(\Rd^p j^!)_{p \in \mathbf{Z}}: A-\fsc(X) \to A-\fsc(Y).
\leqno{(4.6.1)}
$$
On a $\Rd^i j^! = 0$ pour $i < 0$, et on pose $j^! = \Rd^0 j^!$. Le foncteur $j^!$ est \emph{exact à gauche}.
\vskip .3cm
{\bf 4.6.2}. Si $k: Z \to Y$ est une autre immersion fermée de topos, on a un isomorphisme canonique
$$
(jk)^! \isomlong (k^!)(j^!),
$$
vérifiant la condition de cocycles habituelle.
\vskip .3cm
{
Proposition {\bf 4.6.3}. --- \it Le foncteur
$$
j_*: A-\fsc(Y) \to A-\fsc(X)
$$
est adjoint à gauche du foncteur
$$
j^!: A-\fsc(X) \to A-\fsc(Y).
$$
}
\vskip .3cm
{\bf Preuve} : Analogue à celle de (4.3.5), compte tenu de (SGA4 IV 3.6).
\vskip .3cm
{
Proposition {\bf 4.6.4}. --- \it On suppose que le morphisme canonique $i: U \to X$ est \emph{quasicompact}. Alors on a pour tout $A$-faisceau $F$ sur $X$ des suites exactes de $\mathcal{E}(X, J)$, donc aussi de $A-\fsc(X)$, fonctorielles en $F$,
$$
0 \to i_! i^*(F) \to F \to j_* j^*(F) \to 0
\leqno{(i)}
$$
$$
0 \to j_* j^! (F) \to F \to i_* i^*(F),
\leqno{(ii)}
$$
dans lesquelles les flèches non évidentes désignent les morphismes d'adjonction. 
}
\vskip .3cm
{\bf Preuve} : Résulte aussitôt de (SGA4 IV 3.7) appliqué aux composants de $F$.
\vskip .3cm
{\bf 4.6.5}. Signalons enfin que toutes les opérations que nous venons de définir transforment évidemment $A$-faisceau de type constant en $A$-faisceau de type constant, et que les foncteurs image réciproque et prolongement par zéro (4.5.3), étant exacts, transforment $A$-faisceau de type $J$-adique en $A$-faisceau de type $J$-adique. Nous verrons dans le chapitre II, moyennant des conditions de finitude convenables, d'autres propriétés de stabilité pour ces notions.









% End
% Begin

















%%%%%%%%%%%%%%%%%%%%%%%%%%%%%%%%%%%%
\subsection*{5. Produit tensoriel.}
\addcontentsline{toc}{subsection}{5. Produit tensoriel}

On suppose donné dans ce paragraphe un idéotope $(X, A, J)$, et on convient de poser pour tout entier $n \geq 0$
$$
A_n = A/J^{n+1}.
$$
\vskip .3cm
{\bf 5.1}. Soient $E = (E_n)_{n \in \mathbf{N}}$ et $F = (F_n)_{n \in \mathbf{N}}$ deux $A$-faisceaux sur $X$. Pour tout entier $i \in \mathbf{Z}$, on définit comme suit un nouveau $A$-faisceau, noté
$$
\cTor^A_i(E, F).
$$
Si $n$ est un entier $\geq 0$, le $n^{\text{ème}}$ composant est
$$
\cTor^A_i(E, F)_n = \cTor^{A_n}_i(E_n, F_n),
$$
est le morphisme de transition
$$
\cTor^A_i(E, F)_{n+1} \to \cTor^A_i(E, F)_n
$$
est le composé du morphisme canonique
$$
\cTor^{A_{n+1}}_i(E_{n+1}, F_{n+1}) \to \cTor^{A_{n+1}}_i(E_n, F_n),
$$
déduit des morphismes de transition de $E$ et $F$ respectivement, et du morphisme de changement d'anneau (CE VI 4)
$$
\cTor^{A_{n+1}}_i(E_n, F_n) \to \cTor^{A_n}_i(E_n, F_n).
$$
Si $u: E \to E'$ et $v: F \to F'$ sont deux morphismes de $\mathcal{E}(X, J)$, on définit un $\mathcal{E}(X, J)$-morphisme
$$
\cTor^A_i(u, v): \cTor^A_i (E, F) \to \cTor^A_i(E', F')
$$
en posant
$$
(\cTor^A_i(u, v))_n = \cTor^{A_n}_i(u_n, v_n).
$$
Mieux, on peut, grâce à (CE p.119, Remarque), définir des morphismes bords composant par composant et munir ainsi la collection des $\cTor^A_i$ d'une structure de bifoncteur cohomologique
$$
\mathscr{E}(X, J) \times \mathscr{E}(X, J) \to \mathscr{E}(X, J).
$$
Bien entendu, le bifoncteur $\cTor^A_0$ sera noté $\otimes_A$ et appelé \emph{produit tensoriel}. Il est clair qu'il est exact à droite, commutatif et associatif. Enfin, on a pour tout entier $i$ un isomorphisme ``canonique'' de bifoncteurs.
$$
\cTor^A_i(E, F) \isom \cTor^A_i (F, E).
\leqno{(5.1.1)}
$$
\vskip .3cm
{
Proposition {\bf 5.2}. --- \it Il existe un unique bifoncteur cohomologique, noté encore $(\cTor^A_i)_{i \in \mathbf{Z}}$
$$
(\cTor^A_i)_{i \in \mathbf{Z}}: A-\fsc(X) \times A-\fsc(X) \to A-\fsc(X)
$$
vérifiant les propriétés suivantes.
\begin{itemize}
    \item[(a)] Pour tout entier $i \in \mathbf{Z}$, le diagramme
    \[\begin{tikzcd}
	{\mathscr{E}(X, J) \times \mathscr{E}(X, J)} && {\mathscr{E}(X, J)} \\
	{A-\fsc(X) \times A-\fsc(X)} && {A-\fsc(X)}
	\arrow["{\pi \times \pi}", from=1-1, to=2-1]
	\arrow["\pi", from=1-3, to=2-3]
	\arrow["{\cTor^A_i}", from=2-1, to=2-3]
	\arrow["{\cTor^A_i}", from=1-1, to=1-3]
    \end{tikzcd}\]
    dans lequel $\pi$ désigne le foncteur canonique, est commutatif.
    \item[(b)] Pour tout suite exacte $0 \to E' \to E \to E'' \to 0$ (resp. $0 \to F' \to F \to F'' \to 0$) de $\mathcal{E}(X, J)$, le foncteur $\pi$ transforme les morphismes bords
    $$
    \cTor^A_{i+1}(E'', F) \to \cTor^A_i(E', F)
    $$
    $$
    \text{(resp.} \quad \cTor^A_{i+1}(E, F'') \to \cTor^A_i(E, F') \quad )
    $$
    de $\mathscr{E}(X, J)$ en les morphismes bords correspondants dans $A-\fsc(X)$.
\end{itemize}
}
\vskip .3cm
{\bf Preuve} : Comme toute flèche de $A-\fsc(X)$ peut se mettre sous la forme $\pi(u) \circ \pi(s)^{-1}$, où $u$ et $s$ sont deux flèches de $\mathscr{E}(X, J)$, l'unicité est immédiate en ce qui concerne les bifoncteurs $\cTor^A_i$. Pour ce qui est des opérateurs bords, elle résulte de ce que toute suite exacte de $A-\fsc(X)$ est, d'après les propriétés générales des catégories abéliennes quotients, isomorphe à l'image par $\pi$ d'une suite exacte de $\mathscr{E}(X, J)$. Quant à l'existence, on est essentiellement réduit à montrer que si $u: E \to E'$ est un morphisme de $\mathscr{E}(X, J)$ dont le noyau et le conoyau sont négligeables, alors il en est de même pour tout entier $i$ et tout $A$-faisceau $F$ des morphismes
$$
\cTor^A_i(u, \id_F): \cTor^A_i(E, F) \to \cTor^A_i(E', F).
$$
Compte tenu du fait que les $\cTor^A_i$ sont munis d'une structure de foncteur cohomologique sur $\mathscr{E}(X, J)$, on est ramené à voir qu'ils transforment $A$-faisceau négligeable en $A$-faisceau négligeable, ce qui est évident. 
\vskip .3cm
{
Définition {\bf 5.3}. --- \it Le bifoncteur $\cTor^A_0$ est appelé \emph{produit tensoriel} et noté $\otimes_A$.
}
\vskip .3cm
Le bifoncteur produit tensoriel est évidemment exact à droite, commutatif et associatif. De plus on a pour tout entier $i$ des isomorphismes fonctoriels
$$
\cTor^A_i(E, F) \isomlong \cTor^A_i (F, E).
$$
\vskip .3cm
{\bf Convention 5.4}. Comme l'anneau $A$ est fixé dans tout le paragraphe, on le supprime à partir de maintenant des notations, afin d'alléger le texte.
\vskip .3cm
{
Définition {\bf 5.5}. --- \it Soit $E$ un $A$-faisceau sur $X$. On dit que $E$ est \emph{plat} si pour toute suite exacte
$$
0 \to F' \xlongrightarrow{u} F \xlongrightarrow{v} F'' \to 0
$$
de $A-\fsc(X)$, la suite correspondante
$$
0 \to E \otimes F' \xlongrightarrow{\id_E \otimes u} E \otimes F \xlongrightarrow{\id_E \otimes v} E \otimes F'' \to 0
$$
est exacte.
}
\vskip .3cm
On voit immédiatement que le produit tensoriel de deux $A$-faisceaux plats est un $A$-faisceau plat.
\vskip .3cm
{
Lemme {\bf 5.6}. --- \it Soit $E$ un $A$-faisceau. On suppose que pour toute suite exacte de $\mathcal{E}(X, J)$
$$
0 \to F' \xlongrightarrow{u} F \xlongrightarrow{v} F'' \to 0
$$
la suite correspondante
$$
0 \to E \otimes F' \xlongrightarrow{\id_E \otimes u} E \otimes F \xlongrightarrow{\id_E \otimes v} E \otimes F'' \to 0
$$
de $\mathcal{E}(X, J)$ soit exacte. Alors $E$ est un $A$-faisceau plat.
}
\vskip .3cm
{\bf Preuve} : On utilise le fait que toute suite exacte de $A-\fsc(X)$ est l'image d'une suite exacte de $\mathscr{E}(X, J)$.

En particulier, un $A$-faisceau $E = (E_n)_{n \in \mathbf{N}}$, dont pour tout entier $n \geq 0$ le composant $E_n$ est un $A_n$--Module plat, est plat.
\vskip .3cm
{
Définition {\bf 5.7}. --- \it Soit $X$ un topos. Étant donné un système projectif indexé par $\mathbf{N}$ d'objets de $X$
$$
H = (H_n)_{n \in \mathbf{N}},
$$
on appelle $A$-faisceau \emph{quasilibre engendré par $H$}, et on note $A_H$, le $A$-faisceau
$$
A_H = ((A_n)_{H_n})_{n \in \mathbf{N}},
$$
dans lequel les morphismes de transition sont ceux déduits des morphismes de transition de $H$. On dit qu'un $A$-faisceau est \emph{quasilibre} s'il est isomorphe dans $\mathscr{E}(X, J)$  à un $A$-faisceau de la forme $A_H$.
}
\vskip .3cm
D'après 5.6, un \emph{$A$-faisceau quasilibre est plat}.
\vskip .3cm
{
Proposition {\bf 5.8}. --- \it  
\begin{itemize}
    \item[(a)] Soient $H$ un système projectif indexé par $\mathbf{N}$ d'objets de $X$, et $E$ un $A$-faisceau sur $X$. On a un isomorphisme fonctoriel en $E$
    $$
    \Hom(A_H, E) \isomlong \Pro \Hom(H, E).
    $$
    \item[(b)] Si $H$ et $K$ sont deux systèmes projectifs indexés par $N$ d'objets de $X$, on a un isomorphisme canonique
    $$
    A_{H \times K} \isomlong A_H \otimes_A A_K
    $$
    \item[(c)] Si $E$ est un $A$-faisceau sur $X$, on a dans $\mathscr{E}(X, J)$, donc aussi dans $A-\fsc(X)$, un épimorphisme fonctoriel en $E$
    $$
    A_E \to E \to 0. 
    $$
    En particulier, tout $A$-faisceau est quotient d'un $A$-faisceau plat.
\end{itemize}
}
\vskip .3cm
{\bf Preuve} : Montrons d'abord a). Si l'objet final de $X$ est quasicompact, le premier s'identifie à l'ensemble de morphismes de $A_H$ dans $E$ dans la catégorie $\Pro(A-\Mod_X)$ (2.8). On en déduit le résultat grâce à (SGA4 IV 2.13 1) a)). L'isomorphisme de b) est la collection des isomorphismes canoniques sur les composants (SGA4 IV 2.13. 1) b)). Enfin l'épimorphisme de c) est la collection des épimorphismes canoniques (d'adjonction)
$$
(A_n)_{E_n} \to E_n.
$$
\vskip .3cm
{
Proposition {\bf 5.9}. --- \it Soit $F$ un $A$-faisceau sur $X$. Les assertions suivantes sont équivalentes : 
\begin{itemize}
    \item[(i)] $F$ est plat.
    \item[(ii)] Pour toute suite exacte de $A$-faisceau
    $$
    0 \to G \xlongrightarrow{u} H \xlongrightarrow{v} F \to 0
    $$
    et tout $A$-faisceau $E$, la suite
    $$
    0 \to E \otimes G \xlongrightarrow{\id_E \otimes u} E \otimes H \xlongrightarrow{\id_E \otimes v} E \otimes F \to 0
    $$
    est exacte.
    \item[(iii)] Pour tout $A$-faisceau $G$, on a :
    $$
    \cTor_1(F, G) = 0.
    $$
    \item[(iv)] Pour tout $A$-faisceau $G$, on a 
    $$
    \cTor_i (G, F) = 0 \quad (i \geq 1).
    $$
\end{itemize}
}
\vskip .3cm
{\bf Preuve} : (i) $\Rightarrow$ (ii). Soit $0 \to R \to L \to E \to 0$ une suite exacte, avec $L$ plat (par exemple quasilibre). On en déduit de fa\c{c}on évidente un diagramme commutatif exact
\[\begin{tikzcd}
	{G \otimes R} & {H \otimes R} & {F \otimes R} & 0 \\
	{G \otimes L} & {H \otimes L} & {F \otimes L} & 0 \\
	{G \otimes E} & {H \otimes E} \\
	0 & 0 && {.}
	\arrow[from=2-3, to=2-4]
	\arrow[from=1-3, to=1-4]
	\arrow[from=2-2, to=2-3]
	\arrow[from=1-2, to=1-3]
	\arrow[from=1-2, to=2-2]
	\arrow[from=1-1, to=2-1]
	\arrow[from=1-1, to=1-2]
	\arrow[from=2-1, to=3-1]
	\arrow[from=3-1, to=4-1]
	\arrow[from=3-2, to=4-2]
	\arrow[from=2-2, to=3-2]
	\arrow["{u \otimes \id_E}", from=3-1, to=3-2]
	\arrow["\alpha", from=2-1, to=2-2]
	\arrow["\beta", from=1-3, to=2-3]
\end{tikzcd}\]
Comme $L$ est plat, la flèche $\alpha$ est un monomorphisme. On peut donc appliquer le lemme du serpent au diagramme délimité par les deux lignes du haut. Il fournit une suite exacte
$$
\Ker(\beta) \to G \otimes E \xlongrightarrow{u \otimes \id_E} H \otimes E.
$$
Comme $F$ est plat, $\Ker(\beta) = 0$, donc $u \otimes \id_E$ est un monomorphisme, ce qui prouve (ii). Montrons (ii) $\Rightarrow$ (iii). Soit
$$
0 \to K \to L \to F \to 0
$$
une suite exacte, avec $L$ un $A$-faisceau quasilibre. On en déduit une suite exacte
$$
\cTor_1(G, L) \to \cTor_1(G, F) \to G \otimes K \to G \otimes L \to G \otimes F \to 0,
$$
d'où d'après (ii) un épimorphisme
$$
\cTor_1(G, L) \to \cTor_1(G, F) \to 0.
$$
Nous aurons donc montré que $\cTor_1(G, F) = 0$ si nous prouvons le lemme suivant.
\vskip .3cm
{
Lemme {\bf 5.9.1}. --- \it Soit $L$ un $A$-faisceau quasilibre. Pour tout $A$-faisceau $G$, on a 
$$
\cTor_i(L, G) = 0 \quad (i \geq 1), \quad \text{dans} \quad \mathscr{E}(X, J).
$$
}
\vskip .3cm
En effet, comme pour tout entier $n \geq 0$, $L_n$ est un $A_n$--Module plat, il est évident que $\cTor^{A_n}_i(L_n, G_n) = 0$, d'où l'assertion.

Montrons que (iii) $\Rightarrow$ (iv). Nous allons le voir par récurrence croissante sur l'entier $i \geq 1$. Supposons donc prouvé que pour tout $A$-faisceau $H$, on ait $\cTor_i (F, H) = 0$, et soit $G$ un $A$-faisceau. Choisissons une suite exacte
$$
0 \to H \to L \to G \to 0,
$$
avec $L$ quasilibre. On en déduit une suite exacte
$$
\cTor_{i+1}(F, L) \to \cTor_{i + 1}(F, G) \to \cTor_i(F, H) \to \cTor_i(F, L),
$$
dans laquelle les termes extrêmes sont nuls d'après (5.9.1). L'hypothèse de récurrence appliquée à $H$ montre alors que
$$
\cTor_{i + 1}(F, G) \isom \cTor_i(F, H) = 0.
$$
Montrons que (iv) $\Rightarrow$ (i). soit $0 \to E' \xlongrightarrow{u} E \xlongrightarrow{v} E'' \to 0$ une suite exacte de $A$-faisceaux et montrons que $\id_F \otimes u$ est un monomorphisme. Comme $\cTor_1(F, E'') = 0$ par hypothèse, l'assertion résulte de la suite exacte canonique
$$
\cTor_1(F, E'') \to F \otimes E' \xlongrightarrow{\id_F \otimes u} F \otimes E \xlongrightarrow{\id_F \otimes v} F \otimes E'' \to 0.
$$
\vskip .3cm
{
Corollaire {\bf 5.9.2}. --- \it Soit $0 \to F' \to F \to F'' \to 0$ une suite exacte de $A$-faisceaux.
\begin{itemize}
    \item[(i)] Si $F'$ et $F''$ sont plats, alors $F$ est plat.
    \item[(ii)] Si $F$ et $F''$ sont plats, alors $F'$ est plat.
\end{itemize}
}
\vskip .3cm
{
Corollaire {\bf 5.9.3}. --- \it La propriété pour un $A$-faisceau $F$ d'être plat est de nature locale.
}
\vskip .3cm
{\bf Preuve} : Soit $G$ un autre $A$-faisceau. L'assertion $\cTor_1(G, F) = 0$ est de nature locale (1.7 (i)). Par ailleurs il est immédiat (cf. aussi (5.17)) que si $T$ est un objet de $X$, on a
$$
\cTor_1(G, F)|T \isom \cTor_1(G|T, F|T).
$$
\vskip .3cm
{\bf 5.10}. Soient $E, F, G$ trois $A$-faisceaux. Étant données trois résolutions plates $L, M, N$ de $E, F, G$ respectivement, on pose
$$
\cTor_i(E, F, G) = \mathrm{H}_i(L \otimes M \otimes N).
$$
Les propriétés (5.9) et (5.9.2) des $A$-faisceaux plats impliquent de fa\c{c}on classique que la définition précédente ne dépend pas des résolutions plates choisies.
\vskip .3cm
{
Proposition {\bf 5.11}. --- \it Soient $E, F, G$ trois $A$-faisceaux et $L, M, N$ des résolutions plates de $E, F, G$ respectivement.
\begin{itemize}
    \item[(i)] On a des isomorphismes canoniques   
    $$
    \cTor_i(E, F) \isom \mathrm{H}_i(L \otimes F) \isom \mathrm{H}_i(E \otimes M) \isom \mathrm{H}_i(L \otimes M).
    $$
    \item[(ii)] On a trois suites spectrales birégulières
    \[\begin{tikzcd}
	{\cTor_i(E, \cTor_j(F, G))} & {\cTor_{i+j}(E, F, G)} \\
	{\cTor_i(F, \cTor_j(G, E))} & {\cTor_{i+j}(E, F, G)} \\
	{\cTor_i(G, \cTor_j(E, F))} & {\cTor_{i+j}(E, F, G).}
	\arrow[Rightarrow, from=1-1, to=1-2]
	\arrow[Rightarrow, from=2-1, to=2-2]
	\arrow[Rightarrow, from=3-1, to=3-2]
    \end{tikzcd}\]
\end{itemize}
}
\vskip .3cm
{\bf Preuve} : Classique à partir des propriétés énoncées des $A$-faisceaux plats (voir thèse de VERDIER). En ce qui concerne les suites spectrales, on fera une démonstration analogue en (\quad).
\vskip .3cm
{
Proposition {\bf 5.12} (Lemme de NAKAYAMA). --- \it Soit $F$ un $A$-faisceau. Les assertions suivantes sont équivalentes : 
\begin{itemize}
    \item[(i)] $F/JF = 0$.
    \item[(ii)] $F = 0$.
\end{itemize}
}
\vskip .3cm
{\bf Preuve} : On a seulement à voir que (i) entraîne (ii). Quitte à localiser, on peut supposer le système projectif
$$
F/JF = (F_n/JF_n)_{n \in \mathbf{N}}
$$
essentiellement nul. Alors il existe une application croissante $\gamma \geq \id: \mathbf{N} \to \mathbf{N}$ telle que pour tout entier $n \geq 1$ le morphisme canonique $F_{\gamma(n)}/JF_{\gamma(n)} \to F_n/JF_n$ soit nul. Autrement dit, l'image de $F_{\gamma(n)}$ par le morphisme canonique
$$
F_{\gamma(n)} \to F_n
$$
est contenue dans $JF_n$. On en déduit par récurrence que pour tout entier $r \geq 1$, l'image du morphisme canonique
$$
F_{\gamma^{r}(n)} \to F_n
$$
est contenue dans $J^r F_n$. En particulier, comme $J^{n+1} F_n=0$, le morphisme canonique
$$
F_{\gamma^{n+1}(n)} \to F_n
$$

À partir de maintenant et jusqu'à mention expresse du contraire nous allons supposer que $A$ est \emph{noethérien} et que $J$ est un idéal \emph{maximal} de $A$.
\vskip .3cm
{
Proposition {\bf 5.13}. --- \it Soit $(X, A, J)$ un idéotope, avec $A$ un anneau noethérien et $J$ un idéal maximal de $A$. Les assertions suivantes pour un $A$-faisceau $F$ sont équivalentes~:
\begin{itemize}
    \item[(i)] $\cTor_1(A/J, F) = 0$. 
    \item[(ii)] Pour tout $A$--module de type fini $M$, on a 
    $$
    \cTor_i(M, F) = 0 \quad (i \geq 1).
    $$
    \item[(iii)] Pour tout $A$-faisceau $G$ annulé par une puissance de $J$, on a 
    $$
    \cTor_i(F, G) = 0 \quad (i \geq 1).
    $$
    (Les notations de l'énoncé sont celles de 4.1.4).
\end{itemize}
}
\vskip .3cm
{\bf Preuve} : La catégorie des $A$-faisceaux ne changeant pas si on remplace $A$ par $A_J$, nous pouvons supposer $A$ local d'idéal maximal $J$.
Il est clair que (ii) et (iii) impliquent chacun (i). Montrons que (i) $\Rightarrow$ (ii). Tout d'abord, utilisant une résolution de $M$ par des $A$--modules libres de type fini, on voit par récurrence qu'il suffit de montrer que $\cTor_1(M, F) =  0$, pour tout $A$--module de type fini $M$. Nous allons voir ce dernier point par récurrence croissante sur la dimension de $M$. Si dim$(M) = 0$, alors $M$ admet une filtration finie par des $A$--modules isomorphes à $A/J$ et l'assertion résulte de (i) par récurrence sur la longueur, grâce à la suite exacte des $\cTor$. Dans le cas général, $M$ admet (Bourbaki Alg. Comm. IV 4.1) une filtration finie dont les quotients consécutifs sont isomorphes à des $A$--modules de la forme $A/P$, où $P$ est un idéal premier de $A$, de sorte que l'on peut supposer que $M$ est de cette forme. Si $P = J$, c'est terminé. Sinon, il existe un élément $a$ non nul de $J/P$, d'où une suite exacte
$$
0 \to A/P \xlongrightarrow{a} A/P \to A/P+aA \to 0.
$$
Comme $\cTor_1(A/P+aA, F) = 0$ par récurrence, on déduit de cette suite exacte que la multiplication par $a$
$$
\cTor_1(A/P, F) \xlongrightarrow{a} \cTor_1(A/P, F)
$$
est un épimorphisme. On conclut alors par le lemme de Nakayama 5.12.

Montrons enfin que (i) $\Rightarrow$ (iii). Le $A$-faisceau $G$ admet une filtration finie dont les quotients consécutifs sont annulés par $J$, de sorte qu'on peut supposer que $G$ lui-même est annulé par $J$.
\vskip .3cm
{
Lemme {\bf 5.13.1}. --- \it Soit (S) $0 \to E' \xlongrightarrow{u} E \xlongrightarrow{v} E'' \to 0$ une suite exacte de $A$-faisceaux, avec $JE = 0$. Pour tout $A$-faisceau $P$, la suite correspondante
$$
0 \to P \otimes E' \to P \otimes E \to P \otimes E'' \to 0
$$
est exacte.
}
\vskip .3cm
On se ramène immédiatement au cas où les morphismes $u$ et $v$ sont images de morphismes de $\mathcal{E}(X, J)$. De plus, on peut remplacer $E$, $E''$ et $E'$ respectivement par $E/JE$, $E''/JE''$ et $\text{Im}(E'/JE' \to E/JE)$, ces expressions étant entendues au sens de $\mathcal{E}(X, J)$ ; dans ce cas, on a pour tout entier $n \geq 1$ 
$$
JE'_n = J E_n = JE''_n = 0.
$$
Il suffit alors pour voir le lemme de prouver que les suites évidentes
$$
0 \to P_n \otimes_{A_n} E'_n \to P_n \otimes_{A_n} E_n \to P_n \otimes_{A_n} E''_n \to 0
$$
sont exactes. Or on peut pour le voir remplacer $P_n$ par $P_n/JP_n$, et alors c'est évident car on a affaire à des faisceaux de $A/J$-espaces vectoriels.
\vskip .3cm
{
Lemme {\bf 5.13.2}. --- \it Soient $F$ et $G$ deux $A$-faisceaux. On suppose que $F$ vérifie (i) et que $JG = 0$. Alors $\cTor_1(F, G) = 0$. 
}
\vskip .3cm
Soit $0 \to M \to L \to G \to 0$ une suite exacte de $A$-faisceaux, avec $L$ plat. Il est clair que $JL \subset M$. On en déduit un diagramme commutatif et exact :
\[\begin{tikzcd}
	& 0 & 0 \\
	& JL & JL \\
	0 & M & L & G & 0 \\
	0 & {M/JL} & {L/JL} & G & 0 \\
	& 0 & 0
	\arrow[from=1-2, to=2-2]
	\arrow[from=1-3, to=2-3]
	\arrow[from=2-2, to=3-2]
	\arrow[from=3-2, to=4-2]
	\arrow[from=4-2, to=5-2]
	\arrow[from=3-1, to=3-2]
	\arrow[from=4-1, to=4-2]
	\arrow[from=3-2, to=3-3]
	\arrow[from=4-2, to=4-3]
	\arrow[from=3-3, to=3-4]
	\arrow[from=2-3, to=3-3]
	\arrow[from=4-3, to=4-4]
	\arrow[from=3-3, to=4-3]
	\arrow[from=4-3, to=5-3]
	\arrow[from=3-4, to=4-4]
	\arrow[from=3-4, to=3-5]
	\arrow[from=4-4, to=4-5]
	\arrow[from=2-2, to=2-3]
\end{tikzcd}\leqno{(D)}\]
En tensorisant par $F$, on en déduit un diagramme commutatif et exact
\[\begin{tikzcd}
	& 0 \\
	{F \otimes JL} & {F \otimes JL} & 0 \\
	{F \otimes M} & {F \otimes L} & {F \otimes G} & 0 \\
	{F \otimes (M/JM)} & {F \otimes (L/JL)} & {F \otimes G} & 0 \\
	0 & 0 &&& {.}
	\arrow[from=1-2, to=2-2]
	\arrow[from=2-2, to=2-3]
	\arrow[from=2-3, to=3-3]
	\arrow[from=3-2, to=3-3]
	\arrow[from=2-2, to=3-2]
	\arrow[from=3-2, to=4-2]
	\arrow[from=4-2, to=4-3]
	\arrow[from=4-2, to=5-2]
	\arrow[from=4-1, to=5-1]
	\arrow[from=4-1, to=4-2]
	\arrow[from=3-1, to=3-2]
	\arrow[from=2-1, to=2-2]
	\arrow[from=2-1, to=3-1]
	\arrow[from=3-1, to=4-1]
	\arrow[from=3-3, to=4-3]
	\arrow[from=3-3, to=3-4]
	\arrow[from=4-3, to=4-4]
\end{tikzcd}\leqno{(D')}\]
En effet, comme $F$ vérifie (i), la suite canonique
$$
0 \to F \otimes J \to F \to F/JF \to 0
\leqno{(U)}
$$
est exacte, et il en est de même de la colonne centrale de (D') qui se déduit de (U) par tensorisation par le $A$-faisceau plat $L$. Par ailleurs, la ligne du bas de (D') est exacte d'après le lemme (5.13.1) appliqué à la ligne correspondante de (D). L'assertion (5.13.2) résulte alors de (D') en appliquant le lemme du serpent au diagramme défini par la colonne de gauche et celle du centre~: on obtient ainsi que la suite canonique 
$$
0 \to F \otimes M \to F \otimes L \to F \otimes G \to 0
$$
est exacte, d'où l'assertion par la suite exacte des $\cTor$.

On achève la preuve de (iii) comme suit. Il est clair que $G$ admet une résolution gauche par des $A$-faisceaux de la forme $L/JL$, avec $L$ un $A$-faisceau plat. L'assertion se verra par récurrence croissante sur l'entier $i$ à partir du cas $i = 1$ si on montre que
$$
\cTor_i(F, L/JL) = 0 \quad  (i \geq 1)
$$
pour tout $A$-faisceau plat $L$. Or la considération des deux suites spectrales (5.11)
\[\begin{tikzcd}
	{\cTor_i(F, \cTor_j(A/J, L))} & {\cTor_{i+j}(F, A/J, L)} \\
	{\cTor_i(L, \cTor_j(A/J, F))} & {\cTor_{i+j}(F, A/J, L)}
	\arrow[Rightarrow, from=1-1, to=1-2]
	\arrow[Rightarrow, from=2-1, to=2-2]
\end{tikzcd}\]
montre, compte tenu du fait que $L$ est plat, que l'on a un isomorphisme
$$
\cTor_i (F, L/JL) \isom L \otimes \cTor_i (A/J, F),
$$
d'où le résultat désiré grâce à l'implication (i) $\Rightarrow$ (ii).
\vskip .3cm
{
Définition {\bf 5.14}. --- \it Un $A$-faisceau vérifiant les conditions équivalentes de (5.13) sera dit \emph{presque plat}.
}
\vskip .3cm
{\bf Remarque 5.15}. Lorsque $A$ est un anneau de valuation discrète d'idéal maximal $J$, alors, désignant par $u$ une uniformisante locale de $A$, on voit facilement, par application de la suite exacte des $\cTor$ à la suite exacte canonique $0 \to A \xlongrightarrow{xu} A \to A/J \to 0$ que les $A$-faisceaux presque plats sont ceux qui sont \emph{sans torsion}, i.e. ceux pour lesquels la multiplication par un élément de $A$ est un monomorphisme.
\vskip .3cm
{
Proposition {\bf 5.15}. --- \it 
\begin{itemize}
    \item[1)] Soit $0 \to F' \to F \to F'' \to 0$ une suite exacte de $A$-faisceaux. Si $F'$ et $F''$ (resp. $F$ et $F''$) sont presque plats, alors il en est de même pour $F$ (resp. $F'$). 
    \item[2)] Si $E$ et $F$ sont deux $A$-faisceaux presque plats, alors $E \otimes_A F$ est presque plat et l'on a
    $$
    \cTor_i(E, F) = 0 \quad (i \geq 1).
    $$
\end{itemize}
}
\vskip .3cm
{\bf Preuve} : L'assertion (1) se voit de fa\c{c}on classique en utilisant la suite exacte des $\cTor$. Montrons (2). Compte tenu de (5.13 (iii)), la suite spectrale
$$
\cTor_i(F, \cTor_j(A/J, E)) \Rightarrow \cTor_{i+j}(F, A/J, E)
\leqno{(5.15.1)}
$$
montre que
$$
\cTor_p(F, A/J, E) = 0 \quad (p \geq 1).
$$
Utilisant ce résultat, on déduit de la suite spectrale
$$
\cTor_i(A/J, \cTor_j(E, F)) \Rightarrow \cTor_{i+j}(A/J, E, F)
\leqno{(5.15.2)}
$$
l'égalité
$$
\cTor_1(A/J, E \otimes F) = 0,
$$
donc que $E \otimes F$ est presque plat. Mais alors (5.13), on a 
$$
\cTor_i(A/J, E \otimes F) = 0 \quad (i \geq 1)
$$
et un nouvel examen de (5.15.2) montre que 
$$
(A/J) \otimes \cTor_1(E, F) = 0,
$$
d'où $\cTor_1(E, F) = 0$ par le lemme de Nakayama. Nous allons maintenant voir le fait que $\cTor_i(E, F) = 0$ $(i \geq 1)$ par récurrence croissante sur l'entier $i \geq 1$. Supposons donc la vraie pour $i$ et montrons qu'elle est vraie pour $i + 1$. Soit pour cela
$$
0 \to F' \to L \to F \to 0
\leqno{(S)}
$$
une suite exacte de $A$-faisceaux, avec $L$ plat. D'après (1), $F'$ est presque plat, d'où $\cTor_i (E, F') = 0$ par hypothèse de récurrence. On conclut par la suite exacte des $\cTor_i(E, .)$ appliquée à la suite (S).
\vskip .3cm
{
Proposition {\bf 5.16}. --- \it Soit $(X, A, J)$ un idéotope, avec $A$ un anneau local \emph{régulier de dimension $r$} et $J$ son idéal maximal. Étant donnés deux $A$-faisceaux $E$ et $F$ sur $X$, on a 
$$
\cTor_i (E, F) = 0 \quad (i \geq 2r+1)
$$
Si de plus $F$ est presque plat, alors
$$
\cTor_i(E, F) = 0 \quad (i \geq r+1)
$$
}
\vskip .3cm
{\bf Preuve} : Nous utiliserons le lemme suivant, qui peut-être utile en soi.
\vskip .3cm
{
Lemme {\bf 5.16.1}. --- \it Pour tout $A$-faisceau $G$ et tout $A$-module de type fini $M$, on a 
$$
\cTor_i(M, G) = 0 \quad (i > \mathrm{dp}_A(M)).
$$
}
\vskip .3cm
Pour le voir, on utilise (5.11), en prenant une résolution plate de $M$ définie par une résolution de longueur $\text{dp}_A(M)$ du $M$--module $M$ par des $A$--modules libres de type fini.

Le lemme montre que pour tout $A$-faisceau $F$ et toute suite exacte
$$
0\to Z_r \to L_{r-1} \to L_{r-2} \to \dots \to L_1 \to L_0 \to F \to 0,
$$
avec les $L_i$ des $A$-faisceaux plats, le $A$-faisceau $Z_r$ est presque plat. Faisant de même pour $E$, ce qui fournit un $A$-faisceau presque plat $Y_r$, on a de fa\c{c}on classique des isomorphismes
$$
\cTor_{i+2r}(E, F) \isom \cTor_i(Y_r, Z_r) \quad (i \geq 1)
$$
d'où le fait que 
$$
\cTor_i(E, F) = 0 \quad (i \geq 2r+1)
$$
grâce à (5.15.2). L'assertion analogue lorsque $F$ est presque plat se voit de même.

\vskip .3cm
{
Corollaire {\bf 5.16.1}. --- \it Sous les hypothèses de (5.16), il existe pour tout $A$-faisceau $F$ sur $X$ une suite exacte
$$
0 \to P_{2r} \to P_{2r-1} \to \dots \to P_{1} \to P_{0} \to F \to 0
$$
de $A$-faisceaux, avec $P_i$ plat sur $i \in [0, 2r]$.
}
\vskip .3cm

\emph{A partir de maintenant, l'anneau $A$ et l'idéal $J$ sont de nouveau quelconques}. 
\vskip .3cm
{\bf 5.17}. Soient $X$ et $Y$ deux topos et $f: X \to Y$ un morphisme de topos. Choisissons un foncteur image réciproque
$$
f^*: A-\Mod_Y \to A-\Mod_X.
$$
On rappelle qu'on en déduit pour tout couple $(P, Q)$ de $A_{nY}$--Modules ($n$ entier $\geq 0$) et tout entier $i \geq 0$ un isomorphisme fonctoriel
$$
\alpha_i: \cTor^{A_n}_i(f^* P, f^* Q) \isomlong f^* \cTor^{A_n}_i(P, Q),
$$
et que la collection des $\alpha_i$ définit un isomorphisme de foncteurs cohomologiques de $A_n-\Mod_X$. Appliquant ces isomorphismes aux composants des systèmes projectifs, on en déduit sans peine un isomorphisme analogue de bifoncteurs cohomologiques de $\mathcal{E}(Y, J)$ dans $\mathcal{E}(X, J)$. Enfin, un passage au quotient immédiat fournit un \emph{isomorphisme de bifoncteurs cohomologiques} de $A-\fsc(Y)$ dans $A-\fsc(X)$~:
$$
(\cTor_i(f^*E, f^* F) \isomlong f^* \cTor_i(E, F))_{i \in \mathbf{N}}.
\leqno{(5.17.1)}
$$
\vskip .3cm
{
Proposition {\bf 5.17.2}. --- \it Pour tout $A$-faisceau quasilibre (5.7) $L$ sur $Y$, le $A$-faisceau $f^*(L)$ est quasilibre.
}
\vskip .3cm
{\bf Preuve} : L'assertion résultera du fait, appliqué aux composants de $L$, que si $B$ est un anneau et $H$ un faisceau d'ensembles sur $Y$, alors $f^*(B_H) \isom B_{f^* (H)}$. Faute de référence, montrons comment on voit ce dernier point. Si $M$ est un $B_X$--Module, on a la suite d'isomorphismes
\[\begin{tikzcd}
	{(1)} && {\Hom_B(f^*(B_H), M) \isom \Hom_B(B_H, f_*(M))} \\
	{(2)} && {\Hom_B(B_H, f_*(M)) \isom \Hom_{\Ens}(H, f_*(M))} \\
	{(3)} && {\Hom_{\Ens}(H, f_*(M)) \isom \Hom_{\Ens}(f^*(H), M)} \\
	{(4)} && {\Hom_{\Ens}(f^*(H), M) \isom \Hom_B(B_{f_*(H)}, M),}
\end{tikzcd}\]
les isomorphismes d'ordre impair provenant de la formule d'adjonction entre $f_*$ et $f^*$, et ceux d'ordre pair traduisant les définitions respectives de $B_H$ et $B_{f^*(H)}$. L'égalité annoncée en résulte aussitôt. 
\vskip .3cm
{\bf 5.18}. Soient $X$ un topos, $T$ et $T'$ deux objets de $X$ et $i: T \to T'$ un morphisme \emph{quasicompact}. On note de même le morphisme de topos
$$
i: X/T \to X/T'
$$
correspondant. Si $E = (E_n)_{n \in \mathbf{N}}$ et $F = (F_n)_{n \in \mathbf{N}}$ sont respectivement un $A$-faisceau sur $T$ et un $A$-faisceau sur $T'$, les isomorphismes de projection (SGA4 IV 2.13)
$$
i_!(E_n \otimes_A i^* (F_n)) \isomlong i_!(E_n) \otimes_A F_n
$$
définissent un morphisme de $\mathcal{E}(T', J)$
$$
i_!(E \otimes i^* (F)) \isomlong i_!(E) \otimes F,
\leqno{(5.18.1)}
$$
d'où un isomorphisme bifonctoriel de $A-\fsc(T')$.

En particulier, on en déduit pour tout $A$-faisceau $F$ sur $T'$ un isomorphisme fonctoriel en $F$
$$
F \otimes i_!(A) \isom i_! i^* (F).
\leqno{(5.18.2)}
$$
De cette dernière formule, on déduit pour tout  couple $(F, G)$ de $A$-faisceaux sur $T'$ un isomorphisme fonctoriel
$$
i_! i^*(F \otimes G) \isom (i_! i^* F) \otimes G.
\leqno{(5.18.3)}
$$
Supposons maintenant que $i$ soit un monomorphisme, autrement dit que $T$ définisse un ouvert de $X/T'$, auquel cas nous dirons que $i$ est une \emph{immersion ouverte}. Alors on définit comme suit pour tout couple $(F, F')$ de $A$-faisceaux sur $T$ un isomorphisme fonctoriel
$$
i_!(F \otimes F') \isom i_!(F) \otimes i_!(F').
\leqno{(5.18.4)}
$$
L'isomorphisme évident $i^* i_! \isom \id$ fournit un isomorphisme
$$
i_!(F \otimes F') \isom i_!(F \otimes i^* i_! F')
$$
qui, composé avec l'isomorphisme de projection (5.18.1)
$$
i_!(F \otimes i^* i_! F') \isom i_!(F) \otimes i_!(F'),
$$
donne l'isomorphisme désiré.
\vskip .3cm
{
Proposition {\bf 5.18.5}. --- \it Soient $X$ un topos et $i: T \to T'$ un morphisme quasicompact d'objets de $X$.
\begin{itemize}
    \item[(i)] Pour tout $A$-faisceau quasilibre (resp. plat) $L$ sur $T$, le $A$-faisceau $i_!(L)$ est quasilibre (resp. plat). 
    \item[(ii)] Si de plus $i$ est une immersion ouverte, alors pour tout $A$-faisceau plat $P$ sur $T'$, le $A$-faisceau $i^* (P)$ est plat.
\end{itemize}
}
\vskip .3cm
{\bf Preuve} : (i) Supposons d'abord $L$ quasilibre. Pour voir que $i_! L$ l'est également, il suffit de montrer que pour tout anneau $B$ et tout objet $u: H \to T$ de $X/T$, on a un isomorphisme
$$
i_!(B_u) \isom B_{i \circ u},
$$
ce qui se voit comme en (5.17.2), en utilisant cette fois l'adjonction entre $i_!$ et $i^*$. Montrons maintenant que si $L$ est plat, alors $i_!(L)$ l'est aussi.

Par (5.18.1), le foncteur $M \mapsto i_!(L) \otimes M$ est isomorphe au foncteur $M \mapsto i_!(L \otimes i^*(M))$. Or ce dernier est exact, puisque $i_!$, $i^*$ et le foncteur $P \to L \otimes P$ sont exacts. D'où l'assertion. (ii) Il s'agit de voir que si $u: L \to M$ est un monomorphisme de $A$-faisceaux sur $T'$, le morphisme
$$
L \otimes i^*(P) \xlongrightarrow{u \otimes \id} M \otimes i^*(P)
$$
est un monomorphisme. Comme $i^* i_! \isom \id$, il suffit de voir que le morphisme $i_!(u \otimes \id)$ est un monomorphisme. Or celui-ci est isomorphe d'après (5.18.1) au morphisme
$$
i_!(L) \otimes P \xlongrightarrow{i_! (u) \otimes \id} i_! (M) \otimes P,
$$
qui est un monomorphisme puisque $i_!$ est exact à gauche et $P$ plat.



\vskip .3cm
{\bf 5.19}. Soient $X$ un topos, $U$ un ouvert de $X$ et $Y$ le topos fermé complémentaire. On note
$$
j: Y \to X
$$
le morphisme canonique. Dans ces conditions, mutatis mutandis, le formulaire (5.18.1) à (5.18.4) reste valable, à condition d'y remplacer le formule $i_!$ par le foncteur $j_*$ et $i^*$ par $j^*$. (Preuves identiques).

De même on a l'énoncé suivant.
\vskip .3cm
{
Proposition {\bf 5.19.1}. --- \it  
\begin{itemize}
    \item[(i)] Si $L$ est un $A$-faisceau plat sur $Y$, le $A$-faisceau $j_*(L)$ est plat. 
    \item[(ii)] Si $P$ est un $A$-faisceau plat sur $X$, le $A$-faisceau $j^*(P)$ est plat.
\end{itemize}
}
\vskip .3cm
{\bf Preuve} : Analogue à celle de (5.18.5), à condition encore de remplacer $i_!$ par $j_*$.









% End
% Begin















%%%%%%%%%%%%%%%%%%%%%%%%%%%%%%%%%%%%
\subsection*{6. Foncteurs associés aux homomorphismes.}
\addcontentsline{toc}{subsection}{6. Foncteurs associés aux homomorphismes}

\vskip .3cm
{\bf 6.1}. Soient $E = (E_n)_{n \in \mathbf{N}}$ et $F = (F_n)_{n \in \mathbf{N}}$ deux $A$-faisceaux sur un topos $X$. Pour tout entier $i \in \mathbf{Z}$, on définit comme suit un nouveau $A$-faisceau, noté
$$
\cExt^i_A (E, F),
$$
la mention de l'anneau $A$ pouvant être éventuellement supprimée s'il n'y a pas de confusion possible. Soient $m' \geq m \geq n$ trois entiers $\geq 0$. En composant le morphisme d'extension d'anneau (CE VI 4 cas3)
$$
\cExt^i_{A_m}(E_m, F_n) \to \cExt^i_{A_{m'}}(E_m, F_n)
$$
avec le morphisme 
$$
\cExt^i_{A_{m'}}(E_m, F_n) \to \cExt^i_{A_{m'}}(E_{m'}, F_n)
$$
déduit du morphisme de transition $E_{m'} \to E_{m}$ on obtient un morphisme de $(A_n)_X$--Modules
$$
\varphi_{m, {m'}}: \cExt^i_{A_m}(E_m, F_n) \to \cExt^i_{A_{m'}}(E_{m'}, F_n).
$$
D'où, l'entier $n$ étant fixé, un système inductif dont on note
$$
\varinjlim_{m \geq n} \cExt^i_{A_m}(E_m, F_n)
$$
la limite inductive. Si maintenant $n' \geq n$ est un autre entier, le morphisme de transition $F_{n'} \to F_n$ définit pour tout entier $m \geq n'$ un morphisme de faisceaux de $A_X$--Modules
$$
\cExt^i_{A_m}(E_{m}, F_{n'}) \to \cExt^i_{A_m}(E_m, F_n).
$$
Par passage à la limite inductive, on en déduit un morphisme
$$
u_{n, n'}: \varinjlim_m \cExt^i_{A_m}(E_m, F_{n'}) \to \varinjlim_m \cExt^i_{A_m}(E_m, F_n)
$$
et il est évident que si $n'' \geq n' \geq n$, on a 
$$
u_{n, n''} = u_{n, n'} \circ u_{n', n''}.
$$
On définit alors le $A$-faisceau $\cExt^i_A(E, F)$ par la formule
$$
(\cExt^i_A (E, F))_n = \varinjlim_m \cExt^i_{A_m}(E_m, F_n) \quad (n \geq 0),
$$
les morphismes de transition étant les $u_{n, n'}$.

Si maintenant $u: E' \to E$ et $v: F \to F'$ sont deux morphismes de $\mathcal{E}(X, J)$, on obtient un $\mathcal{E}(X, J)$-morphisme
$$
\cExt^i_A(u, v): \cExt^i_A (E, F) \to \cExt^i_A(E', F')
$$
en posant pour tout entier $n \geq 0$
$$
\cExt^i_A(u, v)_n = \varinjlim_m \cExt^i_{A_m}(u_m, v_n).
$$
Mieux, on peut, grâce à (CE VI 4.4 a) et Remarque), définir des opérateurs bords, composant par composant pour le second argument et par une limite inductive évidente pour le premier. On munit ainsi la collection des $\cExt^i_A$ d'une structure de bifoncteur cohomologique
$$
\mathcal{E}(X, J)^\circ \times \mathcal{E}(X, J) \to \mathcal{E}(X, J).
$$
Bien entendu, le bifoncteur $\cExt^0_A$ sera aussi noté $\cHom_A$. Il est clair qu'il exact à gauche.
\vskip .3cm
{
Proposition {\bf 6.1.1}. --- \it Il existe un unique bifoncteur cohomologique, noté encore $(\cExt^i_A)_{i \in \mathbf{Z}}$,
$$
(\cExt^i_A)_{i \in \mathbf{Z}}: (A-\fsc(X))^\circ \times (A-\fsc(X)) \to A-\fsc(X)
$$
vérifiant les propriétés suivantes.
\begin{itemize}
    \item[a)] Pour tout entier $i \in \mathbf{Z}$, le diagramme
    \[\begin{tikzcd}
	{\mathcal{E}(X, J)^\circ \times \mathcal{E}(X, J)} && {\mathcal{E}(X, J)} \\
	{A-\fsc(X)^\circ \times A-\fsc(X)} && {A-\fsc(X),}
	\arrow["{\pi^\circ\times \pi}", from=1-1, to=2-1]
	\arrow["\pi", from=1-3, to=2-3]
	\arrow["{\cExt^i_A}", from=2-1, to=2-3]
	\arrow["{\cExt^i_A}", from=1-1, to=1-3]
    \end{tikzcd}\]
    dans lequel $\pi$ désigne le foncteur de projection canonique, est commutatif.
    \item[b)] Pour toute suite exacte $0 \to E' \to E \to E'' \to 0$ (resp. $0 \to F' \to F \to F'' \to 0$) de $\mathcal{E}(X, J)$, le foncteur $\pi$ transforme le morphisme bord
    $$
    \cExt^i_A(E', F) \to \cExt^{i+1}_A(E'', F)
    $$
    $$
    \text{(resp.} \quad \cExt^i_A(E, F'') \to \cExt^{i+1}_A(E, F') \quad)
    $$
    dans $\mathcal{E}(X, J)$ en le morphisme bord correspondant dans $A-\fsc(X)$.
\end{itemize}
}
\vskip .3cm
{\bf Preuve} : L'unicité se montre comme en (5.2). Pour voir l'existence, on est ramené comme dans loc.cit. à voir que si $E$ ou $F$ est négligeable, alors (dans $\mathcal{E}(X, J)$) les $A$-faisceaux $\cExt^i_A(E, F)$ sont négligeables. C'est évident lorsque $F$ est négligeable ; lorsque $E$ est négligeable, on a même, vu la définition des $\cExt^i_A(E, F)_n$,
$$
\cExt^i_A(E, F) = 0,
$$
comme objet de $\mathcal{E}(X, J)$.
\vskip .3cm
{\bf 6.1.2}. On a en particulier $\cExt^i_A = 0$ pour $i < 0$, et le foncteur $\cExt^0_A$, aussi noté de plus souvent $\cHom_A$, est exact à gauche.
\vskip .3cm
{
Définition {\bf 6.1.3}. --- \it Si $E$ et $F$ sont deux $A$-faisceaux sur $X$, le $A$-faisceau $\cHom_A(E, F) = \cExt^0_A(E, F)$ est appelé \emph{$A$-faisceau des homomorphismes de $E$ dans $F$}.
}
\vskip .3cm
{\bf 6.2}. Soient $X$ un topos \emph{dont l'objet final est quasicompact} et $E$ et $F$ deux $A$-faisceaux sur $X$. On définit comme en (6.1) des bifoncteurs cohomologiques, notés tous deux $(\overline{\Ext}^i_A)_{i \in \mathbf{Z}}$,
\[\begin{tikzcd}
	{\mathcal{E}(X, J)^\circ \times \mathcal{E}(X, J)} & {\mathcal{E}(\pt, J)} \\
	{A-\fsc(X)^\circ \times A-\fsc(X)} & {A-\fsc(\pt) \subset \Pro(A-\mod)}
	\arrow[from=1-1, to=1-2]
	\arrow[from=2-1, to=2-2]
\end{tikzcd}\leqno{(6.2.1)}\]
par les formules 
$$
\overline{\Ext}^i_A(E, F)_n = \varinjlim_m \Ext^i_{A_m}(E_m, F_n),
$$
avec les morphismes de transition évidents.

Le bifoncteur $\overline{\Ext}^0_A$ est aussi noté $\overline{\Hom}$. Il est exact à gauche.

Supposons maintenant que le topos $X$ soit \emph{cohérent} (SGA4 VI 2.) Il résulte alors immédiatement du fait que le foncteur $\Gamma(X, .)$ commute aux limites inductives filtrantes que l'on a l'égalité
$$
\overline{\Hom}_A(E, F) \isom \overline{\Gamma}(X, \cHom_A(E, F))
\leqno{(6.2.2)}
$$
(voir (4.4) pour la notation $\overline{\Gamma}$).


\vskip .3cm
{\bf 6.3 ``Formules chères à CARTAN''}.

Soient $X$ un topos dont l'objet final est quasicompact, et $E$, $F$ et $G$ trois $A$-faisceaux sur $X$. On définit comme suit un morphisme de $\mathcal{E}(\pt, J)$, donc aussi de $A-\fsc(\pt)$,
$$
\overline{\Hom}_A(E \otimes F, G) \to \overline{\Hom}(E, \cHom_A(F, G))
\leqno{(6.3.1)}
$$
fonctoriel en $E, F$ et $G$. Pour tout entier $n \geq 0$, le n$^{\text{ème}}$ composant du premier membre de (6.3.1) est 
$$
\varinjlim_q \Hom_A(E_q \otimes_A F_q, G_n).
$$
Comme le diagonale est cofinale dans l'ensemble ordonné $\mathbf{N} \times \mathbf{N}$, l'application canonique 
$$
\varinjlim_{m, p} \Hom_A (E_m \otimes_A F_p, G_n) \to \varinjlim_q \Hom_A (E_q \otimes_A F_q, G_n)
$$
est un isomorphisme. Compte tenu des isomorphismes canoniques
$$
\Hom_A(E_m \otimes_A F_p, G_n) \isomlong \Hom_A (E_m, \cHom_A (F_p, G_n)),
$$
on en déduit un isomorphisme 
$$
\overline{\Hom}_A(E \otimes F, G)_n \isomlong \varinjlim_m \varinjlim_p \Hom_A (E_m, \cHom_A (F_p, G_n)).
$$
Enfin, utilisant les morphismes canoniques
$$
\varinjlim_p \Hom_A (E_m, \cHom_A (F_p, G_n)) \to \Hom_A (E_m, \varinjlim_p \cHom_A(F_p, G_n))
\leqno{(6.3.2)}
$$
on obtient pour tout entier $n \geq 0$ une application
$$
\overline{\Hom}_A (E \otimes F, G)_n \to \overline{\Hom}_A (E, \cHom_A(F, G))_n,
$$
et cela de fa\c{c}on compatible avec les morphismes de transition, d'où le morphisme (6.3.1) annoncé.

Passant à la limite projective, on déduit, grâce à (2.8), du morphisme de systèmes projectifs (6.3.1), une application
$$
\Hom(E \otimes F, G) \to \Hom(E, \cHom_A(F, G)),
\leqno{(6.3.3)}
$$
fonctorielle en $E$, $F$ et $G$.
\vskip .3cm
{
Proposition {\bf 6.3.4}. --- \it Soient $X$ un topos dont \emph{l'objet final est quasicompact} et $E$, $F$ et $G$ trois $A$-faisceaux sur $X$. On suppose que les composants de $E$ sont des $A$--modules \emph{noethériens}. Alors les morphismes fonctoriels (6.3.1) et (6.3.3) sont des isomorphismes.
}
\vskip .3cm
{\bf Preuve} : Comme $E_m$ est un $A$--Module noethérien, l'application (6.3.2) est une bijection.

Soient maintenant $X$ un topos arbitraire et $E$, $F$ et $G$ trois $A$-faisceaux sur $X$. Nous allons définir un morphisme fonctoriel
$$
\cHom_A(E \otimes F, G) \to \cHom_A (E, \cHom_A(F, G))
\leqno{(6.3.5)}
$$
de $\mathcal{E}(X, J)$, donc aussi de $A-\fsc(X)$. Il s'agit de définir pour tout entier $n \geq 0$ un morphisme
$$
\cHom_A(E \otimes F, G)_n \to \cHom_A(E, \cHom_A(F, G))_n
\leqno{(6.3.6)}
$$
commutant avec les morphismes de transition. Comme précédemment, on voit que le premier membre de (6.3.6) est isomorphe à 
$$
\varinjlim_m \varinjlim_p \cHom_A(E_m, \cHom_A (F_p, G_n)).
\leqno{(6.3.6)}
$$
Le morphisme (6.3.6) est obtenu à partir de là en utilisant les morphismes canoniques
$$
\varinjlim_p \cHom_A(E_m, \cHom_A (F_p, G_n)) \to \cHom_A (E_m, \varinjlim_p \cHom_A(F_p, G_n)).
\leqno{(6.3.7)}
$$
\vskip .3cm
{
Proposition {\bf 6.3.8}. --- \it Soient $X$ un topos localement noethérien et $E$, $F$ et $G$ trois $A$-faisceaux sur $X$. On suppose que les composants de $E$ sont des $A$--Modules \emph{constructibles}. Alors le morphisme (6.3.5) de $\mathcal{E}(X, J)$ est un isomorphisme.
}
\vskip .3cm
{\bf Preuve} : Sous les hypothèses de (6.3.8), il résulte du lemme suivant que le morphisme (6.3.7) est un isomorphisme.
\vskip .3cm
{
Lemme {\bf 6.3.9}. --- \it Soient $X$ un topos localement noethérien, $E$ un $A$--Module constructible sur $X$, et
$$
(F_i, u_{i, j})_{(i, j) \in I \times I}
$$
un système inductif filtrant de $A$--Modules sur $X$. Le morphisme canonique
$$
\varinjlim_i \cHom_A (E, F_i) \to \cHom_A(E, \varinjlim_i (F_i))
$$
est un isomorphisme.
}
\vskip .3cm
Il suffit de voir que pour tout objet noethérien $T$ de $X$, le morphisme canonique
$$
\Gamma(T, \varinjlim_i \cHom_A (E, F_i)) \to \Gamma (T, \cHom_A(E, \varinjlim_i (F_i)))
\leqno{(6.3.10)}
$$
est un isomorphisme. Comme $T$ est noethérien, le foncteur $\Gamma(T, .)$ commute aux limites inductives filtrantes, de sorte que (6.3.10) est isomorphe au morphisme canonique
$$
\varinjlim_i \Hom_A (E|T, F_i|T) \to \Hom_A(E|T, \varinjlim_i (F_i|T)).
\leqno{(6.3.11)}
$$
Mais le $A_T$--Module $E|T$ est constructible, donc noethérien, et par suite (6.3.11) est une bijection, d'où l'assertion.
\vskip .3cm
{
Proposition {\bf 6.3.12}. --- \it Soient $X$ un topos cohérent et $E$ et $F$ deux $A$-faisceaux sur $X$. On a un isomorphisme fonctoriel en $E$ et $F$ :
$$
\Hom(E, F) \isom \Hom(A, \cHom_A(E, F)).
$$
}
\vskip .3cm
{\bf Preuve} : Il provient par passage à la limite projective de (6.2.2).
\vskip .3cm
{\bf 6.3.13}. Nous allons maintenant généraliser à un topos arbitraire le morphisme (6.3.3). Soient $E = (E_n)_{n \in \mathbf{N}}$ et $F = (F_n)_{n \in \mathbf{N}}$ deux $A$-faisceaux sur un topos $X$. On définit comme suit un morphisme d'``adjonction''
$$
E \to \cHom_A(F, E \otimes_A F).
\leqno{(6.3.13.1)}
$$
Pour tout entier $n \geq 0$, on a un morphisme canonique
$$
E_n \to \cHom_A(F_n, E_n \otimes_A F_n)
\leqno{(6.3.13.2)}
$$
réalisant l'adjonction entre les foncteurs $\Hom_A(E_n \otimes_A F_n, .)$ et $\Hom_A (E_n, \cHom_A (F_n, .))$. Composant (6.3.13.2) avec le morphisme évident
$$
\cHom_A (F_n, E_n \otimes_A F_n) \to \varinjlim_{m \geq n} \cHom_A (F_m, E_n \otimes_A F_n),
$$
on obtient un morphisme de $A$--Modules
$$
E_n \to \cHom_A (F, E \otimes_A F)_n,
$$
et la collection de ces morphismes pour les divers entiers $n$ fournit le morphisme (6.3.13.1) désiré. De (6.3.13.1), on déduit de fa\c{c}on évidente une application
$$
\Hom(E \otimes_A F, G) \to \Hom(E, \cHom_A (F, G))
\leqno{(6.3.13.3)}
$$
pour tout $A$-faisceau $G$. On laisse au lecteur le soin de vérifier que lorsque l'objet final de $X$ est quasicompact, les morphismes (6.3.3) et (6.3.13.3) sont égaux.



\vskip .3cm
{\bf 6.4. Comportement vis à vis des morphismes}.
\vskip .3cm
{\bf 6.4.1}. Soient $f: X \to Y$ un morphisme de topos et $E = (E_n)_{n \in \mathbf{N}}$ et $F = (F_n)_{n \in \mathbf{N}}$ deux $A$-faisceaux sur $Y$. Nous allons définir un morphisme de bifoncteurs cohomologiques de $\mathcal{E}(Y, J)^\circ \times \mathcal{E}(Y, J)$ dans $\mathcal{E}(X, J)$ :
$$
f^* \cExt^\bullet_A (E, F) \to \cExt^\bullet_A (f^* E, f^* F).
\leqno{(6.4.1.1)}
$$
Soit $i \in \mathbf{Z}$ un entier. L'entier $n \geq 0$ étant fixé, on a pour tout entier $m \geq n$ un morphisme canonique (EGA $0_{III}$ 12.3.4)
$$
f^* \cExt^i_{A_m} (E_m, F_n) \to \cExt^i_{A_m}(f^* E_m, f^* F_n),
\leqno{(6.4.1.2)}
$$
d'où un morphisme 
$$
\alpha_n: \varinjlim_m f^* \cExt^i_{A_m}(E_m, F_n) \to \varinjlim_m \cExt^i_{A_m}(f^* E_m, f^* F_n).
$$
Par ailleurs, comme le foncteur $f^*$ commute aux limites inductives, le morphisme canonique
$$
\beta_n: \varinjlim_m f^* \cExt^i_{A_m}(E_m, F_n) \to f^*(\varinjlim_m \cExt^i_{A_m}(E_m, F_n))
$$
est un isomorphisme. Ceci permet de définir un morphisme de $A_{nX}$--Modules
$$
\mu_n = \alpha_n \circ \beta^{-1}_n: f^* (\cExt^i_A(E, F)_n) \to \cExt^i_A(f^* E, f^* F)_n.
$$
On s'assure sans peine que $(\mu_n)_{n \in \mathbf{N}}$ définit un morphisme de systèmes projectifs
$$
\mu^i: f^* (\cExt^i_A (E, F)) \to \cExt^i_A (f^* E, f^* F)
$$
et que la collection des $\mu^i$ est un morphisme de bifoncteurs cohomologiques.

On en déduit aussitôt un morphisme analogue de bifoncteurs cohomologiques de $A-\fsc(Y)^\circ \times A-\fsc(Y)$ dans $A-\fsc(X)$.

Si $g: Y \to Z$ est un autre morphisme de topos, le morphisme (6.4.1.1) associé au composé $g \circ f$ est le composé des morphismes analogues associés à $g$ et $f$ respectivement.
\vskip .3cm
{
Proposition {\bf 6.4.2}. --- \it On suppose $A$ noethérien. Soient $f: X \to Y$ un morphisme de topos et $E$ et $F$ deux $A$-faisceaux sur $Y$. On suppose que les composants de $E$ sont \emph{localement constants de type fini}. Alors les morphismes 
$$
f^* \cExt^i_A (E, F) \to \cExt^i_A (f^* E, f^* F)
\leqno{(6.4.1.1)}
$$
sont des isomorphismes.
}
\vskip .3cm
{\bf Preuve} : Comme pour tout entier $m$, le $A$--Module $E_m$ est localement constant de type fini, les morphismes (6.4.1.2) sont des isomorphismes (EGA $0_{III}$ 12.3.5). L'assertion en résulte sans peine. 
\vskip .3cm
{\bf 6.4.3}. Soient $X$ et $Y$ deux topos dont l'objet final est quasicompact et $f: X \to Y$ un morphisme. Soient $E = (E_n)_{n \in \mathbf{N}}$ et $F = (F_n)_{n \in \mathbf{N}}$ deux $A$-faisceaux sur $Y$. On a pour tout triple $(m, n, i)$ d'entiers, avec $m \geq 0$ une application canonique
$$
\Ext^i_{A_m}(E_m, F_n) \to \Ext^i_{A_m}(f^* E_m, f^* F_n).
$$
On en déduit immédiatement un morphisme de bifoncteurs cohomologiques
$$
f^* : \overline{\Ext}^i_A(E, F) \to \overline{\Ext}^i_A(f^* E, f^* F),
$$
fonctoriel en $f$.
\vskip .3cm
{\bf 6.4.4}. Soient $f: X \to Y$ un morphisme de topos et $E$ et $F$ deux $A$-faisceaux sur $Y$ et $X$ respectivement. On définit comme suit un morphisme de bifoncteur
$$
\cHom_A(E, f_* F) \to f_* \cHom_A(f^* E, F).
\leqno{(6.4.4.1)}
$$
Soit $n \geq 0$ un entier. Pour tout entier $m \geq n$, on a un isomorphisme canonique (SGA4 IV 2.2.8)
$$
\cHom_A(E_m, f^*(F_n)) \to f_* \cHom_A(f^* E_m, F_n),
$$
d'où par passage à la limite suivant $m$, un isomorphisme
$$
\gamma_n: \cHom_A(E, f_* F)_n \isomlong \varinjlim_m f_* \cHom_A (f^* E_m, F_n)
$$
qui, composé avec le morphisme canonique
$$
\delta_n : \varinjlim_m f_* \cHom_A(f^* E_m, F_n) \to f_* (\varinjlim_m \cHom_A (f^* E_m, F_n)),
$$
fournit un morphisme de $A$--Modules
$$
\delta_n \circ \gamma_n: \cHom_A (E, f_* F)_n \to f_* (\cHom_A (f^* E, F)_n).
$$
Il est immédiat que la collection $(\delta_n \circ \gamma_n)_{n \in \mathbf{N}}$ définit un morphisme de $\mathcal{E}(Y, J)$ répondant à la question.

On en déduit aussitôt, lorsque $f$ est \emph{quasicompact}, un morphisme analogue de foncteurs de $A-\fsc(Y)^\circ \times A-\fsc(X)$ dans $A-\fsc(Y)$.
\vskip .3cm
{
Proposition {\bf 6.4.5}. --- \it Soit $f: X \to Y$ un morphisme cohérent de topos (SGA4 VI 3.1). Étant donnés un $A$-faisceau $E$ sur $Y$ et un $A$-faisceau $F$ sur $X$, le morphisme
$$
\cHom_A (E, f_* F) \to f_* \cHom_A (f^* E, F)
\leqno{(6.4.1.1)}
$$
est un isomorphisme.
}
\vskip .3cm
{\bf Preuve} : Si $f$ est cohérent, le foncteur $f_*$ commute aux limites inductives filtrantes (SGA4 VI \quad), donc les morphismes $\delta_n$ sont des isomorphismes. L'assertion en résulte aussitôt.
\vskip .3cm
{\bf 6.4.6}. Soient $X$ et $Y$ deux topos dont l'objet final est quasicompact et $f: X \to Y$ un morphisme quasicompact. Étant donnés deux $A$-faisceaux $E$ et $F$ sur $Y$ et $X$ respectivement, les morphismes d'adjonction usuels sur les $A$--Modules définissent de fa\c{c}on évidente un isomorphisme de bifoncteurs
$$
\overline{\Hom}_A (f^* E, F) \isomlong \overline{\Hom}_A (E, f_* F).
$$
\vskip .3cm
{\bf 6.4.7}. Soient $X$ un topos, $T$ et $T'$ deux objets de $X$ et $f: T \to T'$ un morphisme. On note de même le morphisme de topos
$$
X/T \to X/T'
$$
correspondant. On définit comme précédemment, à partir du morphisme analogue pour les $A$--Modules, un morphisme de foncteurs de $\mathcal{E}(T, J)^\circ \times \mathcal{E}(T', J)$ dans $\mathcal{E}(T', J)$
$$
f_* \cHom_A (E, f^* F) \longleftarrow \cHom_A (f_! E, F).
\leqno{(6.4.7.1)}
$$
Lorsque $f$ est \emph{quasicompact}, le morphisme (6.4.7.1) définit un morphisme de foncteurs de $A-\fsc(T)^\circ \times A-\fsc(T')$ dans $A-\fsc(T')$.

Lorsque $f$ est \emph{cohérent}, le morphisme (6.4.7.1) est un \emph{isomorphisme}.

Supposons maintenant que $T$, $T'$ et $f$ soient quasicompacts, la dernière hypothèse résultant des deux premières si de plus $T'$ est quasiséparé (SGA4 VI 1.14). Alors les morphismes d'adjonction usuels pour les $A$--Modules définissent de fa\c{c}on évidente un isomorphisme de bifoncteurs
$$
\overline{\Hom}_A(E, f^* F) \isomlong \overline{\Hom}_A (f_! E, F),
\leqno{(6.4.7.2)}
$$
$E$ désignant un $A$-faisceau sur $T$ et $F$ un $A$-faisceau sur $T'$.
\vskip .3cm
{\bf 6.4.8}. Soient $X$ un topos, $U$un ouvert de $X$ et $Y$ le topos fermé complémentaire. On note $j: Y \to X$ le morphisme de topos canonique~: on rappelle (SGA4 \quad) que $j$ est \emph{cohérent}. Étant donnés deux $A$-faisceaux $E$ et $F$ sur $Y$ respectivement, on définit comme précédemment un \emph{isomorphisme} de bifoncteurs
$$
j_* \cHom_A (E, j^! F) \isomlong \cHom_A (j_* E, F).
\leqno{(6.4.8.1)}
$$
Si de plus l'objet final de $X$ est quasicompact, on a un \emph{isomorphisme} de bifoncteurs
$$
\overline{\Hom}_A(E, j^! F) \isomlong \overline{\Hom}_A(j_* E, F)
\leqno{(6.4.8.2)}
$$
obtenu de fa\c{c}on évidente à partir de l'isomorphisme analogue pour les $A$--Modules.



\vskip .3cm
{\bf 6.5. Cohomologie à support dans un fermé}.

Soient $X$ un topos, $U$ un ouvert de $X$, $Y$ le topos fermé complémentaire et $j: Y \to X$ le morphisme canonique.
\vskip .3cm
{
Proposition {\bf 6.5.1}. --- \it Soit $F$ un $A$-faisceau sur $X$. On a dans $\mathcal{E}(Y, J)$ et $\mathcal{E}(\pt, J)$ respectivement des isomorphismes fonctoriels en $F$
$$
j^* \cExt^i_A (j_* A, F) \isom (\mathbf{\mathrm{H}}^1_Y(F_n))_{n \in \mathbf{N}}
\leqno{(6.5.1.1)}
$$
$$
\overline{\Ext}^i_A(j_* A, F) \isom (\overline{\mathrm{H}}^i_Y(X, F_n))_{n \in \mathbf{N}}.
\leqno{(6.5.1.2)}
$$
}
\vskip .3cm
{\bf Preuve} : Compte tenu des définitions des seconds membres, il suffit de prouver que si $m$ et $n$ sont deux entiers, avec $m \geq n$, les morphismes canoniques
$$
\cExt^i_{A_n}(j_* A_n, F_n) \to \cExt^i_{A_m}(j_* A_m, F_n)
$$
et
$$
\Ext^i_{A_n}(j_* A_n, F_n) \to \Ext^i_{A_m}(j_* A_m, F_n)
$$
sont des isomorphismes. C'est là un fait bien connu, qui résulte aussi du lemme suivant qui nous sera utile dans la suite.
\vskip .3cm
{
Lemme {\bf 6.5.2}. --- \it Soient $X$ un topos, $B$ et $C$ deux Anneaux de $X$ et $u: B \to C$ un morphisme d'Anneaux. Pour tout $B$--Module plat $M$, tout $C$--Module $N$ et tout entier $p$, le morphisme de $C$--Modules (resp. de groupes abéliens)
\[\begin{tikzcd}
	{\cExt^p_C(M \otimes_B C, N) \to \cExt^p_B(M, N)} \\
	{\text{(resp.} \quad \Ext^p_C(M \otimes_B C, N)) \to \Ext^p_B(M, N) \quad )}
\end{tikzcd}\]
composé de morphisme d'extension d'Anneau
\[\begin{tikzcd}
	{\cExt^p_C(M \otimes_B C, N) \to \cExt^p_B(M \otimes_B C, N)} \\
	{\text{(resp.} \quad \Ext^p_C(M \otimes_B C, N)) \to \Ext^p_B(M \otimes_B C, N) \quad )}
\end{tikzcd}\]
et du morphisme canonique déduit de $\id \otimes u: M \to M \otimes_B C$
\[\begin{tikzcd}
	{\cExt^p_B(M \otimes_B C, N) \to \cExt^p_B(M, N)} \\
	{\text{(resp.} \quad \Ext^p_B(M \otimes_B C, N)) \to \Ext^p_B(M, N) \quad )}
\end{tikzcd}\]
est un \emph{isomorphisme}.
}
\vskip .3cm
{\bf Preuve} : Nous montrerons seulement l'assertion non respée, l'assertion respée se prouvant de fa\c{c}on analogue. Fixons $M$. Faisant varier $N$, on obtient un morphisme de foncteurs cohomologiques 
$$
\cExt^\bullet_C (M \otimes_B C, .) \to \cExt^\bullet_B (M, .)
$$
qui est un isomorphisme en degré zéro et dont la source est effa\c{c}able. Il suffit pour voir le lemme de montrer que si $N$ est un $C$--Module injectif, on a $\cExt^p_B(M, N) = 0$ $(p \geq 1)$. Lorsque $M$ est un $B$--Module libre de base un faisceau d'ensembles, cela résulte du lemme (6.5.3) ci dessous, compte tenu du fait que $N$ est un $C$--Module flasque, donc un $B$--Module flasque. Dans le cas général, on va raisonner par récurrence. Commen\c{c}ons par le cas $p = 1$. Soit
$$
0 \to M' \to L \to M \to 0
\leqno{(S)}
$$
une suite exacte de $B$--Modules, avec $L$ un $B$--Module libre. Comme $M$ est plat, la suite obtenue en tensorisant (S) par $C$ est exacte. Puisque $N$ est un $C$--Module injectif, on en déduit que la suite évidente
$$
0 \to \cHom_C(M \otimes_B C, N) \to \cHom_C(L \otimes_B C, N) \to \cHom_C(M' \otimes_B C, N) \to 0
\leqno{(T)}
$$
est exacte. Mais (T) est isomorphe à la suite canonique
$$
0 \to \cHom_B (M, N) \to \cHom_B (L, N) \to \cHom_B (M', N) \to 0.
$$
Comme $\cExt^1_B(L, N) = 0$, on en déduit que $\cExt^1_B(M, N) = 0$. Supposons maintenant établi que pour tout $B$--Module plat $M$ et tout entier $i \in [1, p]$ on a $\cExt^i_B (M, N) = 0$. Considérons à nouveau la suite exacte (S), dans laquelle il est clair que $M'$ est $B$-plat. On déduit de (S) des isomorphismes
$$
\cExt^i_B (M, N) \isom \cExt^{i-1}_B(M', N) \quad (i \geq 2)
$$
d'où, grâce à l'hypothèse de récurrence appliquée à $M'$,
$$
\cExt^i_B(M, N) = 0 \quad (2 \leq i \leq p+1)
$$
ce qui achève la démonstration puis que l'on sait déjà cette égalité vraie pour $i = 1$.
\vskip .3cm
{
Lemme {\bf 6.5.3}. --- \it Soit $(X, B)$ un topos annelé. Pour tout $B$--Module libre $L$ et tout $B$--Module flasque $M$, on a 
$$
\cExt^p_B (L, M) = 0 \quad (p \geq 1)
$$
$$
\Ext^p_B (L, M) = 0 \quad (p \geq 1).
$$
}
\vskip .3cm
{\bf Preuve} : Soit $H$ un faisceau d'ensembles tel que $L \isom B_H$. On a 
$$
\Ext^p_B(L, M) \isom \mathrm{H}^p(H, M),
$$
et la deuxième assertion résulte alors de l'acyclicité de $M$ (SGA4 V 3.7). Maintenant le $B$--Module $\cExt^p_B (L, M)$ est associé au préfaisceau
$$
T \to \Ext^p_B (L|T, M|T),
$$
d'où la première assertion, puisque $L|T$ et $M|T$ sont respectivement libre et flasque.
\vskip .3cm
{
Définition {\bf 6.5.4}. --- \it Soient $X$ un topos, $U$ un ouvert de $X$, $Y$ le topos fermé complémentaire, $j: Y \to X$ le morphisme canonique. Étant donné un $A$-faisceau $F$ sur $X$, on pose pour tout entier $i$
\[\begin{tikzcd}
	{\mathbf{\mathrm{H}}^i_Y (F) = \cExt^i_A(j_* A, F) = (\mathbf{\mathrm{H}}^i_Y(F_n))_{n \in \mathbf{N}}.} \\
	{\overline{\mathrm{H}}^i_Y (X, F) = \overline{\Ext}^i_A(j_* A, F) = (\mathrm{H}^i_Y(F_n))_{n \in \mathbf{N}}.}
\end{tikzcd}\]
}
\vskip .3cm
{
Proposition {\bf 6.5.5}. --- \it Soit $F$ un $A$-faisceau sur $X$.
\begin{itemize}
    \item[a)] Si l'immersion canonique $i: U \to X$ est quasicompacte, on a une suite exacte, fonctorielle en $F$, de $A-\fsc(X)$
    $$
    0 \to \mathbf{\mathrm{H}}^0_Y (F) \to F \to i_* (F|U) \to \mathbf{\mathrm{H}}^1_Y (F) \to 0
    $$
    et des isomorphismes fonctoriels en $F$
    $$
    \mathbf{\mathrm{H}}^p_Y (F) \isom \Rd^{p-1}i_* (F|u) \quad (p \geq 2).
    $$
    \item[b)] Si $U$ et $X$ ont des objets finaux quasicompacts, on a une suite exacte fonctorielle en $F$ de $A-\fsc(\pt)$
    $$
    \dots \to \overline{\mathrm{H}}^p_Y (X, F) \to \overline{\mathrm{H}}^p(X, F) \to \overline{\mathrm{H}}^p(U, F|U) \xlongrightarrow{\delta} \overline{\mathrm{H}}^{p+1}_Y (X, F) \to \dots 
    $$
\end{itemize}
}
\vskip .3cm
{\bf Preuve} : Appliquant l'assertion analogue pour les $A$--Modules aux composants de $F$, on obtient de telles suites exactes dans $\mathcal{E}(X, J)$ (resp. $\mathcal{E}(\pt, J)$).




\vskip .3cm
{\bf 6.6. Objets injectifs et pseudo-injectifs}.

Soit $(X, A, J)$ un idéotope. 
\vskip .3cm
{
Définition {\bf 6.6.1}. --- \it On dit qu'un $A$-faisceaux $I$ sur $X$ est \emph{pseudo-injectif} si c'est un objet injectif de la catégorie $\mathcal{E}(X, J)$.
}
\vskip .3cm
Le lecteur peut se demander pourquoi considérer cette notion plutôt que la notion d'injectif de $A-\fsc(X)$ ; c'est parce que nous verrons plus loin (au moins lorsque $A = \mathbf{Z}_{\ell}$) que le seul $A$-faisceau injectif est le $A$-faisceau nul.

On vérifie par un raisonnement classique d'adjonction que pour tout objet $T$ de $X$ et tout $A$-faisceau pseudo-injectif $I$ sur $X$, le $A$-faisceau $I | T$ est pseudo-injectif.
\vskip .3cm
{
Définition {\bf 6.6.2}. --- \it Soit $C$ une catégorie abélienne. Un système projectif $E$ d'objets de $C$ est dit \emph{directement strict} si tous ses morphismes de transition sont des épimorphismes directs.
}
\vskip .3cm
{
Proposition {\bf 6.6.3}. --- \it 
\begin{itemize}
    \item[1)] Soit $I = (I_n, u_n)_{n \in \mathbf{N}}$ un $A$-faisceau pseudo-injectif sur $X$. Alors~:
    \begin{itemize}
        \item[(i)] $I$ est flasque (4.2.3) et directement strict.
        \item[(ii)] Pour tout entier $n \geq 0$, le $A_n$--Module $\Ker(u_n)$ est injectif.
        \item[(iii)] Pour tout entier $n \geq 0$, on a un isomorphisme
        $$
        I_n \isom \bigoplus_{0 \leq p \leq n} \Ker(u_n).
        $$
        (avec la convention $u_0 = 0$).
    \end{itemize}
    \item[2)] Pour tout $A$-faisceau $E$ sur $X$, il existe un monomorphisme de $\mathcal{E}(X, J)$
    $$
    m: E \to I,
    $$
    avec $I$ un $A$-faisceau pseudo-injectif.
\end{itemize}
}
\vskip .3cm
{\bf Preuve} : Pour montrer 2), on peut référer au résultat général de (Tohoku). Nous allons toutefois donner une preuve directe, qui sera utile pour montrer 1). Soient $n \geq 0$ un entier et $M$ un $A_n$--Module. On définit un $A$-faisceau noté ${}_n\widetilde{M}$ par les formules
$$
({}_n\widetilde{M})_p =
\begin{cases}
    0 \quad \text{si} \quad p < n \\
    M \quad \text{si} \quad p \geq n,
\end{cases}
$$
les morphismes de transition étant l'identité en degrés $\geq n$ et 0 ailleurs. Choisissions maintenant pour tout entier $n \geq 0$ un monomorphisme
$$
\xi_n: E_n \to I_n,
$$
avec $I_n$ un $(A_n)_X$--Module injectif. Le morphisme $\xi_n$ définit de fa\c{c}on évidente un morphisme 
$$
\widetilde{\xi_n}: E \to {}_n\widetilde{(I_n)}
$$
de $\mathcal{E}(X, J)$. On en déduit un monomorphisme de $\mathcal{E}(X, J)$
$$
m = \prod_n \widetilde{(\xi_n)}: E \to \prod_n {}_n\widetilde{I_n} = I,
$$
dont on vérifie aisément qu'il répond à la question. De plus, le $A$-faisceau $I$ est par construction flasque et directement strict. Montrons maintenant 1). Si $I$ est pseudo-injectif, il résulte de la partie 2) qu'il est facteur direct (dans $\mathcal{E}(X, J)$) d'un $A$-faisceau flasque et directement strict, d'où aussitôt (i) et (iii). Pour voir que $\Ker(u_n)$ est un $(A_n)_X$--Module injectif, il suffit d'exprimer que le foncteur
$$
M \mapsto \Hom_a(M, I)
$$
est exact lorsque $M$ parcourt la catégorie des $A$-faisceaux vérifiant $M_p = 0$ pour $p \neq n$. En effet, on a évidemment dans ce cas
$$
\Hom_a(M, I) = \Hom_{A_n}(M, \Ker(u_n)).
$$
\vskip .3cm
{
Proposition {\bf 6.6.4}. --- \it Soit $(X, A, J)$ un idéotope. On suppose que $X$ est localement algébrique (SGA4 VI 2.3), et que $A$ est un anneau de valuation discrète et $J$ son idéal maximal. Alors tout $A$-faisceau injectif sur $X$ est nul.
}
\vskip .3cm
{\bf Preuve} : Soit $F$ un $A$-faisceau injectif. D'après (6.6.3), il existe un monomorphisme $u: F \to P$, avec $P$ un $A$-faisceau directement strict. Comme $F$ est injectif, le morphisme $u$ admet dans $A-\fsc(X)$ une rétraction
$$
v: P \to F,
\leqno{(6.6.5)}
$$
qui est un épimorphisme. En particulier, $F$ est de type strict.

Montrons d'abord la proposition lorsque $X$ est le topos ponctuel. Si $s$ désigne une uniformisante locale de $A$, on obtient en écrivant l'exactitude du foncteur $\Hom(\cdot, F)$ pour la suite exacte
$$
0 \to A \xlongrightarrow{s} A \to A/J \to 0
$$
une suite exacte
$$
0 \to \Hom(A/J, F) \to \varprojlim (F) \xlongrightarrow{s} \varprojlim (F) \to 0.
\leqno{(6.6.6)}
$$
Par ailleurs, posant $K = sF$, on a des suites exactes
\[\begin{tikzcd}
	0 & {{}_sF} & F & K & 0 \\
	0 & K & F & {F/sF} & {0,}
	\arrow[from=1-1, to=1-2]
	\arrow[from=1-2, to=1-3]
	\arrow[from=1-3, to=1-4]
	\arrow[from=1-4, to=1-5]
	\arrow[from=2-1, to=2-2]
	\arrow[from=2-2, to=2-3]
	\arrow[from=2-3, to=2-4]
	\arrow[from=2-4, to=2-5]
\end{tikzcd}\]
avec $K$ et $F/sF$ vérifiant la condition de Mittag-Leffler. On en déduit une suite exacte
$$
0 \to \varprojlim K \xlongrightarrow{g} \varprojlim F \to \varprojlim (F/sF) \to 0.
$$
Le morphisme composé naturel
$$
\varprojlim (F) \to \varprojlim (K) \to \varprojlim (F)
$$
étant un épimorphisme (6.6.6), il en est de même de $g$, donc
$$
\varprojlim (F/sF) = 0.
$$
Se ramenant au cas strict, on voit sans peine qu'un système projectif de groupes abéliens vérifiant la condition de Mittag-Leffler et dont la limite projective est nulle, est essentiellement nul. Par suite $F/sF = 0$, et $F = 0$ par le lemme de Nakayama (5.12).

Passons au cas général. Comme il est immédiat par adjonction que la restriction d'un $A$-faisceau injectif à un objet $T$ de $X$ est un $A$-faisceau injectif, on peut supposer que l'objet final de $X$ est quasicompact.
\vskip .3cm
{
Lemme {\bf 6.6.7}. --- \it Soit $U$ un objet quasicompact de $X$. Le système projectif
$$
\overline{\mathrm{H}}^0(U, F) = (\mathrm{H}^0(U, F_n))_{n \in \mathbf{N}}
$$
est essentiellement nul.
}
\vskip .3cm
En effet, notant $p: U \to \pt$ le morphisme de topos canonique, le foncteur $P_*$ (qui existe puisque $U$ est quasicompact) est adjoint à droite du foncteur exact $p^*$. Il transforme donc $A$-faisceaux injectifs en $A$-faisceaux injectifs, et en particulier
$$
p_*(F) = \overline{\mathrm{H}}^0(U, F)
$$
est un injectif de $A-\fsc(\pt)$, donc est essentiellement nul d'après ce qui précède.
\vskip .3cm
{
Lemme {\bf 6.6.8}. --- \it Soient $F$ un $A$-faisceau injectif et strict sur $X$, et $\sigma$ une section de $F_n$ $(n \geq 0)$. Il existe un recouvrement $(U_i)_{i \in I}$ de l'objet final de $X$ tel que pour tout $i \in I$, la section $\sigma | U_i$ appartienne à l'image de $\Gamma(U_i, F_{n+r})$ pour tout entier $r \geq 0$.
}
\vskip .3cm
Comme $F$ est strict, il existe une application $\gamma \geq \id: \mathbf{N} \to \mathbf{N}$ telle que le morphisme (6.6.5) provienne d'un épimorphisme
$$
\chi_\gamma (P) \to F \to 0
$$
de $\mathcal{E}(X, J)$, ou encore d'un épimorphisme de systèmes projectifs
$$
h_\gamma(P) \to F \to 0,
$$
où $(h_\gamma (P))_p = P_{\gamma(p)}$ pour tout $p \geq 0$. Localement, la section de $F_n$ est l'image d'une section de $P_{(n)}$, donc d'une section de $P_{\gamma(n+r)}$ pour tout entier $r \geq 0$ ($h_\gamma(P)$ est \emph{directement} strict). Le lemme en résulte aussitôt.

Montrons maintenant comment les lemmes précédents impliquent la proposition. Comme $F$ est de type strict, on peut supposer qu'il est strict. Soit $n \geq 0$ un entier et montrons que $F_n = 0$. Soit $\sigma$ une section de $F_n$ au-dessus d'un objet $T$ de $X$ et montrons qu'elle est nulle. Quitte à localiser, on peut supposer que $T$ est quasicompact et que (6.6.8)
$$
\sigma \in \text{Im}(\Gamma(U, F_{n+r})) \to \Gamma(U, F_n) \quad \text{pour tout}~r \geq 0.
$$
Comme (6.6.7) le système projectif $(\Gamma(U, F_p))_{p \in \mathbf{N}}$ est essentiellement nul, on en déduit bien que $\sigma = 0$.




\vskip .3cm
{\bf 6.7. Relations d'orthogonalité}.
\vskip .3cm
{
Définition {\bf 6.7.1}. --- \it Soit $(X, A, J)$ un idéotope. On dit qu'un $A$-faisceau sur $X$ est \emph{fortement plat} s'il est isomorphe dans $A-\fsc(X)$ à un $A$-faisceau $F = (F_n)_{n \in \mathbf{N}}$ tel que pour tout $n \in \mathbf{N}$ le $A_{nX}$--Module $F_n$ soit plat. 
}
\vskip .3cm
Un $A$-faisceau fortement plat est plat (5.6). Un $A$-faisceau quasi-libre est fortement plat.
\vskip .3cm
{
Proposition {\bf 6.7.2}. --- \it Soient $E$ et $F$ deux $A$-faisceaux sur $X$.
\begin{itemize}
    \item[(i)] Si $E$ est quasilibre et $F$ flasque, alors
    \begin{itemize}
        \item[(a)] $\cExt^i_A(E, F) = 0$ \quad $(i \geq 1)$ \quad dans \quad $\mathscr{E}(X, J)$.
        \item[(b)] $\overline{\Ext}^i_A(E, F) = 0$ \quad $(i \geq 1)$ \quad dans \quad $\mathscr{E}(\pt, J)$.
    \end{itemize}
    \item[(ii)] Si $E$ est fortement plat et $F$ pseudo-injectif, alors
    \begin{itemize}
        \item[(a)] $\cExt^i_A(E, F) = 0$ \quad $(i \geq 1)$ \quad dans \quad $A-\fsc(X)$.
        \item[(b)] $\overline{\Ext}^i_A(E, F) = 0$ \quad $(i \geq 1)$ \quad dans \quad $A-\fsc(\pt)$,
    \end{itemize}
    si l'objet final de $X$ est quasicompact.
\end{itemize}
}
\vskip .3cm
{\bf Preuve} : Les assertions (i) résultent immédiatement des définitions et de (6.5.3). Montrons par exemple la première partie de (ii), la seconde se voyant de fa\c{c}on analogue. On peut supposer que pour tout entier $n \geq 0$ le $n^{\text{ème}}$ composant $E_n$ de $E$ est un $A_n$--Module plat. Si on note $(u_p)_{p \in \mathbf{N}}$ les morphismes de transition de $F$, on a (6.6.3) 
$$
F_n \isom \bigoplus_{0 \leq j \leq n} \Ker(u_j)
$$
et les $\Ker(u_j)$ sont des $A_j$--Modules injectifs. On est donc ramené à voir que pour tout entier $j \geq 0$ et tout $A_j$--Module injectif $J$, on a 
$$
\varinjlim_{q \geq j} \cExt^i_{A_q}(E_q, J) = 0.
$$
Mais, comme $E_q$ est un $A_q$--Module plat, on a (6.5.2)
$$
\cExt^i_{A_q}(E_q, J) \isom \cExt^i_{A_j}(E_q \otimes_{A_q} A_j, J) = 0,
$$
la dernière égalité provenant du fait que $J$ est un $A_j$--Module injectif. D'où le lemme.







% End
% Begin

















%%%%%%%%%%%%%%%%%%%%%%%%%%%%%%%%%%%%
\subsection*{7. Catégories dérivées.}
\addcontentsline{toc}{subsection}{7. Catégories dérivées}

\vskip .3cm
{\bf 7.1}. Soit $(X, A, J)$ un idéotope. On convient de poser, désignant l'un des symboles $+, -, b$ ou ``vide''
$$
\K^*(X, A) = \K^*(A-\fsc(X))
$$
$$
\D^*(X, A) = \D^*(A-\fsc(X)).
$$
On prendra garde de ne pas confondre ces catégories avec les catégories $\K^*(A-\Mod_X)$ et $\D^*(A-\Mod_X)$, pour lesquelles on fait d'habitude des conventions analogues.

Reprenant les notations de (3.5), le foncteur canonique (3.5.1)
$$
\mathscr{E}_X: J-\Mod_X \to A-\fsc(X)
$$
est exact et induit donc des foncteurs exacts
$$
\D^*(\mathscr{E}_X): \D^*(J-\Mod_X) \to \D^*(X, A).
\leqno{(7.1.1)}
$$
\vskip .3cm
{
Proposition {\bf 7.1.2}. --- \it On suppose que l'objet final de $X$ est quasicompact. Alors le foncteur (7.1.1)
$$
\D^+(\mathscr{E}_X): \D^+(J-\Mod_X) \to \D^+(X, A)
$$
$$
\text{(resp.} \quad \D^b(\mathscr{E}_X): \D^b(J-\Mod_X) \to \D^b(X, A) \quad )
$$
est pleinement fidèle et induit une équivalence avec la sous-catégorie, triangulée, pleine de $\D^+(X, A)$ (resp. $\D^b(X, A)$) engendrée par les complexes dont la cohomologie appartient à $\TC(X, J)$ (3.6).
}
\vskip .3cm
{\bf Preuve} : Comme $\TC(X, J)$ est une sous-catégorie exacte de $A-\fsc(X)$ les sous-catégories pleines engendrés par les complexes à cohomologie dans $\TC(X, J)$ sont des sous-catégories triangulées de $\D^+(X, A)$ et $\D^b(X, A)$ respectivement. Pour montrer la proposition il suffit de voir que les objets de $\TC(X, J)$ vérifient relativement aux $A$-faisceaux les conditions duales de EGA $0_{III}$ II.9.I).
\vskip .3cm
{
Lemme {\bf 7.1.3}. --- \it Pour tout monomorphisme $u: F \to G$ de $A-\fsc(X)$ avec $F$ un objet de $\TC(X, J)$, il existe un morphisme $v: G \to K$, avec $K$ un objet de $\TC(X, J)$ tel que le morphisme composé $v \circ u$ soit un monomorphisme.  
}
\vskip .3cm
On peut supposer $F$ de la forme $\mathcal{E}_X (M)$, avec $M$ un $A_X$--Module annulé par une puissance de $J$. Alors (3.9), le morphisme $u$ est l'image d'un morphisme de $\mathcal{E}(X, J)$, qu'on notera de même. Dans $\mathcal{E}(X, J)$, le $A$-faisceau $\Ker(u)$ est négligeable, donc essentiellement nul puisque l'objet final de $X$ est quasicompact. Comme $\mathcal{E}_X(M)$ est essentiellement constant, il existe donc un entier $p$ tel que 
$$
\Ker(u)_n = 0 \quad \text{pour} \quad n \geq p.
$$
Soit alors $K$ le $A$-faisceau défini par 
$$
K_n = 
\begin{cases}
    G_p \quad n \geq p \\
    0 \quad n < p
\end{cases}
$$
avec pour morphismes de transition l'identité en degrés $\geq p$ et $0$ ailleurs. Les morphismes de transition de $G$ définissent de fa\c{c}on évidente dans $\mathcal{E}(X, J)$ un morphisme
$$
v: G \to K,
$$
tel que le composé $v \circ u$ définisse un monomorphisme de $A-\fsc(X)$. Par ailleurs $K = \mathcal{E}_X(G_p)$ dans $A-\fsc(X)$, d'où le lemme.
\vskip .3cm
{\bf 7.2}. Les propriétés générales des $A$-faisceaux plats (5.9) et (5.9.1) permettent de leur appliquer les arguments de ((CD) p.41 th.2.2.) et ((H) II 4). Ainsi, on peut définir au moyen des résolutions plates un bifoncteur dérivé du produit tensoriel
$$
\boldsymbol{\otimes}: \D^-(X, A) \times \D^-(X, A) \to \D^-(X, A),
\leqno{(7.2.1)}
$$
vérifiant les propriétés de commutativité et d'associativité habituelles.
\vskip .3cm
{\bf 7.2.2}. Soit $K$ un complexe borné supérieurement de $A$-faisceaux. On dit que $K$ est \emph{de tor-dimension finie} s'il vérifie l'une des propriétés équivalentes suivantes : 
\begin{itemize}
    \item[(i)] Il existe deux entiers $m$ et $n$ tels que l'on ait
    $$
    \mathrm{H}^i(K \boldsymbol{\otimes} E) = 0 \quad \text{pour} \quad i \notin [m, n]
    $$
    et pour tout $A$-faisceau $E$.
    \item[(ii)] Il existe un quasi-isomorphisme $u: L \to K$, avec $L$ un complexe borné et à composants plats.
\end{itemize}
L'équivalence se montre en paraphrasant la preuve de (SGA4 \quad). La sous-catégorie pleine de $\D^-(X, A)$ engendrée par les complexes de tor-dimension finie est une sous-catégorie triangulée, notée 
$$
\D^-(X, A)_{\torf}.
$$
Utilisant des résolutions plates bornées pour le composant de gauche, on définit un bifoncteur dérivé du produit tensoriel
$$
\boldsymbol{\otimes}: \D^-(X, A)_{\torf} \times \D (X, A) \to \D(X, A)
\leqno{(7.2.3)}
$$
coïncidant avec (7.2.1) sur leur domaine commun de définition.
\vskip .3cm
{\bf 7.2.4}. Supposons maintenant que $A$ soit un anneau local régulier et $J$ son idéal maximal. Il résulte alors de (5.16.1) et de ((CD) p.25 lemme 1) que pour tout complexe $K$ de $A$-faisceaux sur $X$, il existe un complexe $P$ de $A$-faisceaux plats et un quasi-isomorphisme
$$
P \to K,
$$
le complexe $P$ pouvant être pris borné (resp. borné inférieurement, resp. borné supérieurement) si $K$ l'est. En particulier, les complexes parfaits s'identifient à isomorphisme près aux complexes bornés. A partir de là, les arguments de ((CD) p.41 th.2.2) et de ((H) II 4) permettent de définir au moyen d'une résolution plate du terme de gauche, des bifoncteurs dérivés du produit tensoriel
\[\begin{tikzcd}
	{\D^+(X, A) \times \D^+(X, A)} & {\D^+(X, A)} \\
	{\D^b(X, A) \times \D^*(X, A)} & {\D^*(X, A),}
	\arrow[from=1-1, to=1-2]
	\arrow[from=2-1, to=2-2]
\end{tikzcd}\leqno{\boldsymbol{\otimes}}\]
coïncidant avec (7.2.1) sur leur domaine commun de définition, et vérifiant les propriétés de commutativité et d'associativité habituelles.
\vskip .3cm
{\bf 7.2.5}. Soient $K$ et $L$ deux complexes de $A$-faisceaux sur $X$. Dans chacun des cas où nous avons défini $K \boldsymbol{\otimes} L$, nous poserons
$$
\cTor^A_i(K, L) = \mathrm{H}^{-i}(K \boldsymbol{\otimes} L) \quad (i \in \mathbf{Z}).
$$
Si $E$ et $F$ sont deux $A$-faisceaux, alors, notant de même les complexes de degré zéro associés, on retrouve à isomorphisme près (5.11) les bifoncteurs $\cTor^A_i(E, F)$ définis en (5.1).
\vskip .3cm
{\bf 7.3}. Soient $K$ et $L$ deux complexes de $A$-faisceaux. On suppose que 
\begin{itemize}
    \item ou bien $K$ est borné supérieurement et $L$ borné inférieurement. 
    \item ou bien $K$ est borné inférieurement et $L$ borné supérieurement.
    \item ou bien l'un des deux est borné.
\end{itemize}
On définit alors un nouveau complexe
$$
\cHom^\bullet_A (K, L)
$$
comme suit. Pour tout $n \in \mathbf{Z}$, son $n^{\text{ème}}$ terme est 
$$
(\cHom^\bullet_A (K, L))^n = \bigoplus_{p \in \mathbf{Z}} \cHom_A (K^p, L^{p+n})
$$
(somme directe finie), et sa différentielle de degré $n$ est donnée de fa\c{c}on claire par la formule
$$
d^n_{\cHom^\bullet_A(K, L)} = \cHom_A (d_K, \id_L) + (-1)^{n+1}\cHom_A (\id_K, d_L).
$$
Comme d'habitude, on en déduit des bifoncteurs exacts, notés $\cHom^\bullet_A:$
\[\begin{tikzcd}
	{\K^+(X, A)^\circ \times \K^-(X, A)} & {\K^-(X, A),} \\
	{\K^-(X, A)^\circ \times \K^+(X, A)} & {\K^+(X, A),} \\
	{\K^b(X, A)^\circ \times \K(X, A)} & {\K(X, A),} \\
	{\K(X, A)^\circ \times \K^b(X, A)} & {\K(X, A),}
	\arrow[from=1-1, to=1-2]
	\arrow[from=2-1, to=2-2]
	\arrow[from=3-1, to=3-2]
	\arrow[from=4-1, to=4-2]
\end{tikzcd}\]
qui coïncident sur leurs domaines communs de définition.

Soit $\K^-(X, A)_{\ql}$ la sous-catégorie triangulée pleine de $\K^-(K, A)$ engendrée par les complexes bornés supérieurement et quasilibres en tout degré, et $\D^-(X, A)_{\ql}$ la catégorie triangulée obtenue à partir de $\K^-(X, A)_{\ql}$ en inversant les quasi-isomorphismes. Comme il y a suffisamment de $A$-faisceaux quasilibres, le foncteur canonique
$$
\D^-(X, A)_{\ql} \to \D^-(X, A)
\leqno{(7.3.1)}
$$
est une équivalence de catégories ((CD) cor.2 p.43).

De même, notons $\K^+(X, A)_{\fl}$ la sous-catégorie triangulée pleine de $\K^+(X, A)$ engendrée par les complexes bornés inférieurement et flasques en tout degré et $\D^+(X, A)_{\fl}$ la catégorie triangulée que l'on en déduit en inversant les quasi-isomorphismes. Comme il y a suffisamment de $A$-faisceaux flasques, le foncteur canonique
$$
\D^+(X, A)_{\fl} \to \D^+(X, A)
\leqno{(7.3.2)}
$$
est une équivalence de catégories.
\vskip .3cm
{
Lemme {\bf 7.3.3}. --- \it Soient $K$ et $L$ deux complexes de $A$-faisceaux sur $X$, avec $K$ borné supérieurement et flasque, et $L$ borné inférieurement et flasque. Si l'un d'eux est acyclique, alors le complexe $\cHom^\bullet_A (K, L)$ est acyclique. 
}
\vskip .3cm
{\bf Preuve} : Montrons-le lorsque $L$ est acyclique, le cas où $K$ est acyclique se traitant de fa\c{c}on analogue. Lorsque $K$ est ``réduit au degré zéro'', on le voit par un argument de récurrence classique à partir de (6.7.2.(i)). Soit maintenant $M^{\bullet \bullet}$ le double complexe de terme général
$$
M^{p, q} = \cHom_A (K^{-p}, L^q),
$$
et dont les deux différentielles sont déduites de fa\c{c}on évidente de celles de $K$ et $L$ respectivement. Rappelant que $\cHom^\bullet_A (K, L)$ est le complexe simple associé à $M^{\bullet \bullet}$, l'assertion résulte dans le cas général de la suite spectrale birégulière
$$
\mathrm{H}^p_I \mathrm{H}^{q}_{II} (M^{\bullet \bullet}) \Rightarrow \mathrm{H}^{p+q}(\cHom^\bullet_A(K, L)).
$$
Le lemme (7.3.3) permet par passage au quotient de définir un bifoncteur exact
$$
\D^-(X, A)^\circ_{\ql} \times \D^+(X, A)_{\fl} \to \D^+(X, A),
$$
d'où, compte tenu des équivalences (7.3.1) et (7.3.2), un bifoncteur exact : 
$$
\bRd \cHom_A : \D^-(X, A)^\circ \times \D^+(X, A) \to \D^+(X, A).
\leqno{(7.3.4)}
$$
\vskip .3cm
{
Proposition {\bf 7.3.5}. --- \it Étant donnés deux $A$-faisceaux $E$ et $F$ sur $X$, on a pour tout $i \in \mathbf{Z}$ un isomorphisme fonctoriel
$$
\cExt^i_A(E, F) \isomlong \mathrm{H}^i(\bRd \cHom_A (E, F)).
$$
}
\vskip .3cm
{\bf Preuve} : Soit, dans $\mathcal{E}(X, J)$, $U^\bullet$ (resp. $V^\bullet$) une résolution quasilibre (resp. flasque) de $E$ (resp. $F$). Par définition, la deuxième membre est isomorphe à 
$$
\mathrm{H}^i(\cHom^\bullet_A (U^\bullet, V^\bullet)),
$$
soit, comme les limites inductives filtrantes de faisceaux abéliens commutent aux suites exactes, au $A$-faisceau
$$
(\varinjlim_m \mathrm{H}^i \cHom^\bullet_A ((U^\bullet)_m , (V^\bullet)_n))_{n \in \mathbf{N}}.
$$
A des vérifications immédiates de commutativité de diagrammes près, on est ramené à voir que 
$$
\mathrm{H}^i(\cHom^\bullet_A ((U^\bullet)_m , (V^\bullet)_n)) = \cExt^i_{A_m}(E_m, F_n).
$$
Utilisant (6.5.3), on voit que le premier membre ne change pas si on y remplace $(V^\bullet)_n$ par une résolution $A_m$-injective de $F_{n}$, d'où l'assertion.

Soient $K \in \D^-(X, A)$ et $L \in \D^+(X, A)$. La proposition (7.3.5) permet de poser sans ambiguïté
$$
\cExt^i_A(K, L) = \mathrm{H}^i (\bRd \cHom_A (K, L)).
\leqno{(7.3.6)}
$$
\vskip .3cm
{
Proposition {\bf 7.3.7}. --- \it Soient $K \in \D^-(X, A)$ et $L \in \D^+(X, A)$. On a des suites spectrales birégulières~:  
$$
E^{p, q}_1 = \cExt^q_A (K, L^p) \Rightarrow \cExt^{p+q}_A (K, L).
\leqno{(7.3.7.1)}
$$
$$
E^{p, q}_2 = \cExt^p_A (K, \mathrm{H}^q(L)) \Rightarrow \cExt^{p+q}_A (K, L).
\leqno{(7.3.7.2)}
$$
$$
E^{p, q}_1 = \cExt^q_A (K^{-p}, L) \Rightarrow \cExt^{p+q}_A (K, L).
\leqno{(7.3.7.3)}
$$
$$
E^{p, q}_2 = \cExt^p_A (\mathrm{H}^{-q}(K), L) \Rightarrow \cExt^{p+q}_A (K, L).
\leqno{(7.3.7.4)}
$$
}
\vskip .3cm
{\bf Preuve} : Nous montrerons les deux premières, les deux autres se montrant de fa\c{c}on duale. Nous nous appuierons sur le lemme suivant, bien connu lorsque le foncteur cohomologique en question est la cohomologie des complexes, et qui se démontre par la méthode des couples exacts.
\vskip .3cm
{
Lemme {\bf 7.3.8}. --- \it Soient $C$ une catégorie abélienne, et $(T^i)_{i \in \mathbf{Z}}$ un foncteur cohomologique de $C$ dans une catégorie abélienne $D$. Soit $K$ un objet de $C$ muni d'une filtration décroissante $(F^p K)_{p \in \mathbf{Z}}$. On suppose que : 
\begin{itemize}
    \item[(i)] Pour tout $n \in \mathbf{Z}$, il existe un $p(n) \in \mathbf{Z}$ tel que le morphisme 
    $$
    T^n(F^{p(n)} K) \to T^n(K)
    $$
    déduit de l'inclusion soit un isomorphisme.
    \item[(ii)] Pour tout $n \in \mathbf{Z}$, il existe un $q(n) \in \mathbf{Z}$ tel que
    $$
    T^n(F^p K) = 0 \quad \text{pour} \quad p \geq q(n).
    $$
    Alors, il existe une suite spectrale birégulière
    $$
    E^{p, q}_1 = T^{p+q}(F^p K / F^{p+1}K) \Rightarrow T^{p+q}(K).
    $$
\end{itemize}
}
\vskip .3cm
Montrons (7.3.7.1). On prend pour catégorie $C$ la catégorie $C^+(A-\fsc(X))$ des complexes bornés inférieurement de $A$-faisceaux, comme foncteur cohomologique $(\cExt^i_A(K, .))_{i \in \mathbf{Z}}$ et on munit $L$ de la filtration
$$
F^p(L) = \dots \to 0 \to 0 \to \dots \to 0 \to L^p \to L^{p+1} \to \dots \to L^i \to \dots 
$$
Alors, $L/F^p(L) = \dots \to L^i \to \dots \to L^{p-1} \to 0 \to 0 \to 0 \dots$. Étant donné un entier $r$, on voit sans peine, pour des raisons de degrés, que $T^r(L/F^p L) = 0$ pour $p$ assez petit. D'où l'assertion (i) dans ce cas. L'assertion (ii) se voit de même. On en déduit donc une suite spectrale birégulière
$$
E^{p, q}_1 = \cExt^{p+q}_A (K, F^p L / F^{p+1} L) \Rightarrow \cExt^{p+q}_A (K, L),
$$
d'où l'assertion car $F^p L / F^{p+1} L \isom L^p (-p)$.

Montrons (7.3.7.2). On prend le même foncteur cohomologique et on munit cette fois $L$ de la filtration 
$$
F^p (L) = \dots \to L^i \to \dots \to L^{-p-1} \to \Ker(d^{-p}) \to 0 \to 0 \dots 
$$
de sorte que $F^pL/F^{p+1} L \isom \mathrm{H}^{-p}(L)(p)$. On vérifie sans peine que les conditions du lemme sont encore vérifiées dans ce cas, d'où une suite spectrale birégulière
$$
\overline{E}^{p, q}_1 = \cExt^{2p+q}_A (K, \mathrm{H}^{-q}(L)) \Rightarrow \cExt^{p+q}_A (K, L) = \overline{E}^{p+q}.
$$
On en déduit la suite spectrale annoncée en posant
$$
E^n = \overline{E}^n \quad (n \in \mathbf{Z}).
$$
$$
F^p E^n = F^{p-n}(\overline{E}^n)
$$
$$
E^{p, q}_r = \overline{E}^{-q, p+2q}_{r-1} \quad (r \geq 2),
$$
sans changer les différentielles.

Nous allons maintenant indiquer, surtout à titre d'exercice, un autre cas où, étant donnés deux complexes de $A$-faisceaux sur $X$, on peut définir $\bRd \cHom_A (K, L)$.
\vskip .3cm
{
Définition {\bf 7.3.8}. --- \it On dit qu'un $A$-faisceaux $E = (E_n)_{n \in \mathbf{N}}$ sur $X$ est \emph{pseudolibre} si pour tout $n \in \mathbf{N}$, le $A_n$--Module $E_n$ est localement libre de type fini.
}
\vskip .3cm
{
Lemme {\bf 7.3.9}. --- \it Soient $K$ et $L$ deux complexes de $A$-faisceaux sur $X$, dont l'un est acyclique. On suppose que $K$ est borné (resp. borné supérieurement) et à composants pseudolibres, et que $L$ est arbitraire (resp. borné inférieurement). Alors le complexe $\cHom^\bullet_A(K, L)$ est acyclique.
}
\vskip .3cm
{\bf Preuve} : Analogue à celle de (7.3.3), compte tenu du fait, évident sur les définitions, que si $E$ et $F$ sont deux $A$-faisceaux sur $X$, avec $E$ pseudolibre, alors on a 
$$
\cExt^i_A (E, F) = 0 \quad (i \geq 1).
$$
On déduit du lemme que, étant donné un complexe de $A$-faisceaux $K$ borné (resp. borné supérieurement) à composants des $A$-faisceaux pseudolibres, le foncteur $\cHom^\bullet_A$ est dérivable, d'où un foncteur exact
$$
\bRd \cHom_A (K, .) : \D(X, A) \to \D(X, A)
\leqno{(7.3.10)}
$$
$$
\text{(resp.} \quad \D^+(X, A) \to \D^+(X, A)).
$$
\vskip .3cm
{
Proposition {\bf 7.3.11}. --- \it Soit $K$ un complexe borné supérieurement à composants des $A$-faisceaux pseudolibres sur $X$. Les deux foncteurs exacts
$$
\bRd \cHom_A (K, .) : \D^+(X, A) \to \D^+(X, A)
$$
induits par (7.3.4) et (7.3.10) respectivement, sont égaux.
}
\vskip .3cm
{\bf Preuve} : Soit $L$ un complexe borné supérieurement, et soient $u: P \to K$ et $v: L \to F$ respectivement une résolution quasilibre de $K$ et une résolution flasque de $L$. Nous allons voir que le morphisme de complexes
$$
\cHom^\bullet_A (u, v): \cHom^\bullet_A (K, L) \to \cHom^\bullet_A(P, F)
$$
est un quasi-isomorphisme. De (7.3.9) résulte que le foncteur $\cHom^\bullet_A(K, .)$ transforme quasi-isomorphismes en quasi-isomorphismes, de sorte que l'on peut supposer que $L = F$. Utilisant alors le mapping-cylinder de $u$, on est ramené à prouver le lemme suivant, qui est une légère amélioration de (7.3.3).
\vskip .3cm
{
Lemme {\bf 7.3.12}. --- \it Soient $K$ un complexe borné supérieurement et acyclique et $L$ un complexe flasque et borné inférieurement. On suppose que pour tout $n \in \mathbf{N}$, les composants du $A$-faisceau $K^n$ sont localement libres (de base un faisceau d'ensembles). Alors le complexe $\cHom^\bullet_A (K, L)$ est acyclique.
}
\vskip .3cm
Montrons le lemme. Comme dans la preuve de (7.3.3), on est ramené à voir que si $Q$ est un $A$-faisceau flasque, alors pour tout $n \in \mathbf{Z}$
$$
\cExt^i_A (K^n, Q) = 0 \quad (i \geq 1).
$$
Or cela résulte sans peine de la définition (6.1) et de (6.5.3).
\vskip .3cm
{
Corollaire {\bf 7.3.13}. --- \it Pour tout complexe de $A$-faisceaux $L$ sur $X$, on a un isomorphisme fonctoriel en $L$
$$
L \isom \bRd \cHom_A (A, L).
$$
}
\vskip .3cm
{\bf 7.3.14}. Bien entendu, on étend dans le nouveau contexte la définition (7.3.6). On laisse au lecteur le soin d'adapter (7.3.7).
\vskip .3cm
{\bf 7.4}. Soit $X$ un topos dont \emph{l'objet  final est quasicompact}.
\vskip .3cm
{\bf 7.4.1}. Soient $K$ et $L$ deux complexes de $A$-faisceaux sur $X$, satisfaisant à l'une des hypothèses suivantes sur les degrés.
\[\begin{tikzcd}
	K & {-} & {+} & 0 & b \\
	L & {+} & {-} & b & 0 & {.}
\end{tikzcd}\]
On définit alors comme suit un complexe, noté $\overline{\Hom}^\bullet_A (K, L)$, d'objets de $A-\fsc(\pt)$. Pour tout $n \in \mathbf{Z}$, son $n^{\text{ème}}$ terme est
$$
(\overline{\Hom}^\bullet_A (K, L))^n = \bigoplus_{p \in \mathbf{Z}} \overline{\Hom}_A (K^p, L^{p+n}).
\leqno{(6.2)}
$$
(somme directe finie), et sa différentielle de degré $n$ est définie de fa\c{c}on claire par la formule
$$
d^n = \overline{\Hom}_A(d_K, \id_L) + (-1)^{n+1} \overline{\Hom}_A (\id_K, d_L).
$$
On en déduit comme d'habitude des bifoncteurs exacts, notés $\overline{\Hom}_A$,
\[\begin{tikzcd}
	{\K^+(X, A)^\circ \times \K^-(X, A)} & {\K^-(\pt, A)} \\
	{\K^-(X, A)^\circ \times \K^+(X, A)} & {\K^+(\pt, A)} \\
	{\K^b(X, A)^\circ \times \K(X, A)} & {\K(\pt, A)} \\
	{\K(X, A)^\circ \times \K^b(X, A)} & {\K(\pt, A).}
	\arrow[from=1-1, to=1-2]
	\arrow[from=2-1, to=2-2]
	\arrow[from=3-1, to=3-2]
	\arrow[from=4-1, to=4-2]
\end{tikzcd}\]
\vskip .3cm
{
Lemme {\bf 7.4.2}. --- \it Soient $K$ et $L$ deux complexes de $A$-faisceaux sur $X$. On suppose que 
\begin{itemize}
    \item[a)] $K \in \K^-(X, A)$ et est à composants quasilibres. 
    \item[b)] $L \in \K^+(X, A)$ et est à composants flasques.
\end{itemize}
Alors si l'un des deux est acyclique, le complexe $\overline{\Hom}^\bullet_A(K, L)$ est acyclique.
}
\vskip .3cm
{\bf Preuve} : Analogue à celle de (7.3.3), en utilisant cette fois (6.7.2. (i) b)) au lieu de (6.7.2. (i) a)).

Le lemme (7.4.2) permet de définir, exactement comme dans (7.3) un bifoncteur exact
$$
\bRd \overline{\Hom}_A : \D^-(X, A)^\circ \times \D^+(X, A) \to \D^+(\pt, A).
\leqno{(7.4.3)}
$$
\vskip .3cm
{
Lemme {\bf 7.4.4}. --- \it Soient $K$ et $L$ deux complexes de $A$-faisceaux sur $X$. On suppose que :
\begin{itemize}
    \item[a)] $K \in \K^-(X, A)$ et est à composants quasilibres. 
    \item[b)] $L \in \K^+(X, A)$ et est à composants flasques et directement stricts.
\end{itemize}
Alors si $K$ ou $L$ est acyclique, le complexe $\Hom^\bullet_A (K, L)$ est acyclique.
}
\vskip .3cm
{\bf Preuve} : D'après (2.8), le complexe $\Hom^\bullet_A (K, L)$ est obtenu en appliquant le foncteur $\varprojlim_{n \in \mathbf{N}}$ au complexe $\overline{\Hom}^\bullet_A(K, L)$. Or ce dernier est acyclique (7.4.2) et l'hypothèse sur $L$ implique que ses composants sont directement stricts. On conclut grâce à (EGA $0_{III}$ 13.2.2). 

Comme précédemment, compte tenu de (6.6.3), le lemme (7.4.4) permet de définir un bifoncteur exact
$$
\bRd \Hom_A : \D^-(X, A)^\circ \times \D^+(X, A) \to \D^+(A-\mod). 
\leqno{(7.4.5)}
$$
\vskip .3cm
{
Proposition {\bf 7.4.6}. --- \it  
\begin{itemize}
    \item[(i)] Soient $E$ et $F$ deux $A$-faisceaux sur $X$. On a un isomorphisme de foncteurs cohomologiques
    $$
    \overline{\Ext}^i_A (E, F) \isomlong \mathrm{H}^i(\bRd \overline{\Hom}_A (E, F)) \quad (i \in \mathbf{Z}).
    $$
    \item[(ii)] Soient $K \in \K^-(X, A)$ et $L \in \K^+(X, A)$. On a un isomorphisme de foncteurs cohomologiques
    $$
    \Ext^i_A (K, L) \isomlong \mathrm{H}^i(\bRd \Hom_A (K, L)) \quad (i \in \mathbf{Z}).
    $$
\end{itemize}
}
\vskip .3cm
{\bf Preuve} : Précisons tout d'abord que l'on a posé
$$
\Ext^i_A (K, L) \isom \Ext^i (K, L)
$$
pour uniformiser les notations. La preuve de (i), semblable à celle de (7.3.5), est laissée au lecteur. Montrons (ii) dans le cas $i = 0$, le cas général s'en déduisant aussitôt. On peut supposer $K$ quasilibre et $L$ à composants flasques et directement stricts. Alors, notant $\Hom_{\text{ht}}(K, L)$ les morphismes de $K$ dans $L$ dans la catégories $\K(X, A)$, on est ramené à prouver le corollaire suivant : 
\vskip .3cm
{
Corollaire {\bf 7.4.7}. --- \it Soient $K \in \K^-(X, A)$ et $L \in \K^+(X, A)$. On suppose que $K$ est à composants quasilibres et $L$ à composants flasques et directement stricts. Alors l'application canonique
$$
\Hom_{\text{ht}}(K, L) \to \Hom_A (K, L)
$$ 
est une bijection.
}
\vskip .3cm
Par définition, le deuxième membre est égal à 
$$
\varinjlim_{\sigma : K' \to K} \Hom_{\text{ht}}(K', L),
$$
où $\sigma$ parcourt les quasi-isomorphismes de but $K$. Comme parmi ces quasi-isomorphismes, ceux de source un complexe borné supérieurement et à composants quasilibres forment une famille cofinale, on est ramené à voir que pour tout quasi-isomorphisme $u: K' \to K$, avec un $K'$ borné supérieurement et quasilibre, le morphisme correspondant
$$
\Hom_{\text{ht}}(K, L) \to \Hom_{\text{ht}}(K', L)
$$
est une bijection. Désignant par $M$ le mapping-cylinder de $u$, on a un triangle exact de $\K(A-\mod)$ : 
\[\begin{tikzcd}
	& {\Hom^\bullet_A (M, L)(-1)} \\
	{\Hom^\bullet_A (K, L)} && {\Hom^\bullet_A (K', L),}
	\arrow[from=2-3, to=1-2]
	\arrow["{d^\circ 1}"', dashed, from=1-2, to=2-1]
	\arrow["\alpha"', from=2-1, to=2-3]
	\arrow["{\Hom^\bullet_A(u, \id)}", from=2-1, to=2-3]
\end{tikzcd}\]
et il s'agit de voir que $\mathrm{H}^0(\alpha)$ est un isomorphisme. Or $M$ est quasilibre et acyclique, donc (7.4.4) $\Hom^\bullet_A (M, L)$ est acyclique, d'où l'assertion.
\vskip .3cm
{
Corollaire {\bf 7.4.8}. --- \it Soient $E$ et $F$ deux $A$-faisceaux sur $X$. On suppose que l'on est dans l'un des cas suivants : 
\begin{itemize}
    \item[(i)] $E$ est quasilibre et $F$ est flasque et directement strict. 
    \item[(ii)] $E$ est fortement plat (6.7.1) et $F$ pseudo-injectif (6.6.1).
\end{itemize}
Alors on a 
$$
\Ext^i_A (E, F) = 0 \quad (i \geq 1).
$$
}
\vskip .3cm
{\bf Preuve} : Dans le premier cas, $\Rd \Hom_A (E, F) = \Hom_A (E, F)$ est réduit au degré 0. Dans le deuxième cas, soit $L^\bullet \to E$ une résolution quasilibre de $E$. De (6.7.2 (ii)) résulte que le complexe
$$
P = \overline{\Hom}^\bullet_A (L^\bullet, F)
$$
est acyclique en degrés $\geq 1$, et il s'agit de voir qu'il en est de même pour le complexe déduit de celui-ci en appliquant le foncteur $\varprojlim_{n \in \mathbf{N}}$. Comme les composants de $P$ sont directement stricts ainsi que $\mathrm{H}^0(P) = \overline{\Hom}_A (E, F)$, l'assertion résulte de (EGA $0_{III}$ 13.2.2)
\vskip .3cm
{
Définition {\bf 7.4.9}. --- \it Soient $K \in \K^-(X, A)$ et $L \in \K^+(X, A)$. Pour tout $i \in \mathbf{Z}$ on pose
$$
\overline{\Ext}^i(K, L) = \mathrm{H}^i(\bRd \overline{\Hom}_A (K, L)).
$$
}
\vskip .3cm
Il résulte de (7.4.6 (i)) que cela ne prête pas à confusion.
\vskip .3cm
{
Définition {\bf 7.4.10}. --- \it On note 
$$
\Rd \Gamma : \D^+(X, A) \to \D^+(A-\mod)
$$
$$
\text{(resp.} \quad \Rd \overline{\Gamma} : \D^+(X, A) \to \D^+(\pt, A) \quad )
$$
le foncteur exact
$$
K \mapsto \bRd \Hom_A (A, K) = \bRd \Gamma (K) = \bRd \Gamma (X, K).
$$
$$
\text{(resp.} \quad K \mapsto \bRd \overline{\Hom}_A (A, K) = \bRd \overline{\Gamma} (K) = \bRd \overline{\Gamma} (X, K). \quad )
$$
}
\vskip .3cm
Soit $E$ un $A$-faisceau sur $X$. Il résulte de (7.4.6 (i)) et (6.5.4) que l'on a des isomorphismes canoniques
$$
\overline{\mathrm{H}}^i(X, E) \isom \mathrm{H}^i(\bRd \overline{\Gamma}(E)) \quad (i \in \mathbf{Z}).
$$
Ceci permet de poser sans ambiguïté pour tout complexe borné inférieurement $K$
$$
\overline{\mathrm{H}}^i(X, K) = \mathrm{H}^i(\bRd \overline{\Gamma}(K)) \quad (i \in \mathbf{Z}).
\leqno{(7.4.11)}
$$
Enfin, on posera de même
$$
\mathrm{H}^i(X, K) = \mathrm{H}^i(\bRd \Gamma(K)) \quad (i \in \mathbf{Z}).
\leqno{(7.4.12)}
$$
\vskip .3cm
{\bf 7.4.13}. On sait que la catégorie $\Pro(A-\mod)$ est abélienne et possède suf\-fisamment d'objets $\varprojlim_A$ acycliques, ce qui permet de définir les foncteurs dérivés
$$
\varprojlim^{(i)}_A : \Pro(A-\mod) \to A-\mod 
$$
du foncteur limite projective. Plus précisément, on définit un foncteur dérivé
$$
\bRd (\varprojlim_A) : \D^+(\Pro(A-\mod)) \to \D^+(A-\mod)
$$
et pour tout pro-$A$-module $E$, on a 
$$
\varprojlim^{(i)}_A (E) \isom \mathrm{H}^i (\bRd \varprojlim_A (E)) \quad (i \in \mathbf{Z}).
$$
On rappelle que la structure de groupe abélien sous-jacente à $\varprojlim^{(i)}_A (E)$ ne dépend que de la structure de pro-groupe abélien sous-jacente à $E$. Si l'ensemble indexant $E$ est $\mathbf{N}$, on a 
$$
\varprojlim^{(i)}_A (E) = 0 \quad (i \geq 2),
$$
et même 
$$
\varprojlim^{(i)}_A (E) = 0 \quad (i \geq 1)
$$
lorsque $E$ vérifie la condition de Mittag-Leffler. On en déduit en particulier que lorsque $E$ est indexé par $\mathbf{N}$, on peut calculer les $\varprojlim^{(i)}_A (E)$ au moyen d'une résolution de $E$ par des pro-$A$-modules indexés par $\mathbf{N}$ et vérifiant la condition de Mittag-Leffler. Il résulte de là que le diagramme
\[\begin{tikzcd}
	{\D^+(\pt, A)} && {\D^+(\Pro(A-\mod))} \\
	& {\D^+(A-\mod)}
	\arrow["{\bRd \Gamma}"', from=1-1, to=2-2]
	\arrow["{\bRd (\varprojlim_A)}", from=1-3, to=2-2]
	\arrow["{\D^+(c)}", from=1-1, to=1-3]
\end{tikzcd}\]
dans lequel $C$ désigne l'inclusion canonique, est ommutatif. Ceci nous permettra dans ce cas d'identifier le plus souvent le foncteur $\bRd \Gamma$ (resp. $\mathrm{H}^i(\pt, .)$ et le foncteur $\bRd \varprojlim_A$ (resp. $\varprojlim^{(i)}_A$). 
\vskip .3cm
{
Proposition {\bf 7.4.14}. --- \it Soient $K \in \D^-(X, A)$ et $L \in \D^+(X, A)$. On a un isomorphisme canonique
$$
\bRd \Hom_A (K, L) \isomlong \bRd (\varprojlim) \circ \Rd \overline{\Hom}_A (K, L).
$$
}
\vskip .3cm
{\bf Preuve} : On peut supposer $K$ quasilibre et $L$ à composants flasques et directement stricts. Alors la formule résulte de l'isomorphisme 
$$
\Hom^\bullet_A (K, L) \isom \varprojlim_{n \in \mathbf{N}}\overline{\Hom}^\bullet_A (K, L)
\leqno{(2.8)}
$$
et du fait que les composants de $\overline{\Hom}^\bullet_A (K, L)$ sont directement stricts.
\vskip .3cm
{
Corollaire {\bf 7.4.15}. --- \it Soit $K \in \D^+(X, A)$. On a un isomorphisme fonctoriel
$$
\bRd \Gamma \isom \Rd (\varprojlim) \circ \Rd \overline{\Gamma}(K).
$$
}
\vskip .3cm
{
Corollaire {\bf 7.4.16}. --- \it Soient $K \in \D^-(X, A)$ et $L \in \D^+(X, A)$. On a pour tout $i \in \mathbf{Z}$ des suites exactes de $A$--modules
\begin{itemize}
    \item[a)] $0 \to \varprojlim^{(1)}_A \overline{\Ext}^{i-1}_A (K, L) \to \Ext^i_A(K, L) \to \varprojlim_A \overline{\Ext}^i_A (K, L) \to 0$ 
    \item[b)] $0 \to \varprojlim^{(1)}_A \overline{\mathrm{H}}^{i-1} (X, L) \to \mathrm{H}^i(X, L) \to \varprojlim_A \overline{\mathrm{H}}^i (X, L) \to 0$.
\end{itemize}
}
\vskip .3cm
{\bf Preuve} : La suite exacte b) étant un cas particulier de a), nous avons seulement à montrer cette dernière. L'isomorphisme (7.4.14) donne lieu par la méthode des couples exacts (cf. thèse Verdier ou la preuve de 7.3.7) à une suite spectrale birégulière
$$
E^{i, j}_2 = \varprojlim^{(i)}_A \overline{\Ext}^j_A (K, L) \Rightarrow \Ext^{i+j}_A (K, L).
$$
Comme les systèmes projectifs considérés sont indexés par $\mathbf{N}$, on a $E^{i, j}_2 = 0$ pour $i \geq 2$, d'où aussitôt la suite exacte annoncée.
\vskip .3cm
{
Corollaire {\bf 7.4.17}. --- \it Soient $E$ et $F$ deux $A$-faisceaux sur $X$, avec $F$ directement strict. Pour tout $i \in \mathbf{Z}$, le morphisme canonique
$$
\Ext^{i}_A(E, F) \to \varprojlim \overline{\Ext}^i_A(E, F)
$$
est un isomorphisme. En particulier, les morphismes canoniques
$$
\mathrm{H}^i(X, F) \to \varprojlim \overline{\mathrm{H}}^i(X, F)
$$
sont des isomorphismes.
}
\vskip .3cm
{\bf Preuve} : L'hypothèse sur $F$ entraîne que pour tout $i \in \mathbf{Z}$, le système projectif $\overline{\Ext}^i_A (E, F)$ vérifie la condition de Mittag-Leffler, donc annule les foncteurs $\varprojlim^{(p)}_A$ pour $p \geq 1$. L'assertion résulte alors immédiatement de (7.4.16).
\vskip .3cm
{
Proposition {\bf 7.4.18}. --- \it On suppose le topos $X$ et l'anneau $A$ \emph{noethériens}. Soient $K \in \D^-(X, A)$ et $L \in \D^+(X, A)$. On a des isomorphismes fonctoriels : 
\begin{itemize}
    \item[(i)] $\Rd \overline{\Hom}_A (K, L) \isom \bRd \overline{\Gamma}(X, \bRd \cHom_A (K, L))$.
    \item[(ii)] $\bRd \Hom_A (K, L) \isom \bRd \Gamma(X, \bRd \cHom_A (K, L))$.
    \item[(iii)] $\Hom_A (K, L) \isom \Hom_A(A, \bRd \cHom_A (K, L))$.
\end{itemize}
}
\vskip .3cm
{\bf Preuve} : On peut supposer que les composants de $K$ sont quasilibres et ceux de $L$ pseudo-injectifs (6.6.1), donc directement stricts (6.6.3). Dans ce cas, les composants de $\cHom^\bullet_A (K, L)$ sont directement stricts et le lemme (7.4.19) ci-dessous montre qu'ils sont flasques. Alors les assertions (i) et (ii) proviennent de (6.2.2) et (6.3.12) respectivement. Grâce à (7.4.6 (ii)), on déduit (iii) de (ii) en appliquant le foncteur $\mathrm{H}^0$ aux deux membres.
\vskip .3cm
{
Lemme {\bf 7.4.19}. --- \it Soient $T$ un topos localement noethérien, $A$ un anneau noethérien et $J$ un idéal de $A$. Étant donnés un $A$-faisceau quasilibre $E$ et un $A$-faisceau pseudo-injectif $F$ sur $T$, le $A$-faisceau
$$
\mathrm{H} = \cHom_A (E, F)
$$
est flasque.
}
\vskip .3cm
Soit $n$ un entier $\geq 0$. D'après (6.6.3), le $n^{\text{ème}}$ composant de $H$ est de la forme
$$
\varinjlim_{p \geq n} \cHom_A (E_p , \bigoplus_{q \leq n} I_q)
$$
où pour tout $q$, $I_q$ est un $A_q$--Moule injectif. L'entier $q \geq n$ étant fixé, pour tout entier $p \leq n$, le $A_q$--Module $E_p/J^{q+1}E_p$ est plat, donc $\cHom_A (E_p, I_q)$ est un $A_q$--Module injectif. Les hypothèses sur $T$ et $A$ impliquent que la catégorie $A_q-\Mod_T$ est localement noethérienne, et par suite qu'une limite inductive filtrante de $A_q$--Modules injectifs est un $A_q$--Module injectif. En particulier
$$
\varinjlim_{p \geq n} \cHom_A (E_p, I_q)
$$
est un $A_q$--Module injectif, donc flasque. L'assertion en résulte aussitôt.
\vskip .3cm
{\bf 7.4.20}. Soit $F = (F_n, u_n)_{n \in \mathbf{N}}$ un $A$-faisceau flasque et strict. On suppose que pour tout $n \geq 0$, le $A$--Module $\Ker(u_n)$ est flasque. Alors $\overline{\mathrm{H}}^i(X, F) = 0$ pour $i \geq 1$ et $\overline{\mathrm{H}}^0(X, F)$ est strict. Par (7.4.16), on en déduit que 
$$
\mathrm{H}^i(X, F) = 0 \quad (i \geq 1).
$$
Supposons maintenant que le topos $X$ possède suffisamment de points. Alors on voit, en utilisant ce qui précède et en se ramenant au cas où $F$ est strict, que pour tout $A$-faisceau $F$ vérifiant la condition de Mittag-Leffler, on a 
$$
\bRd \Gamma (X, F) \isom \Hom^\bullet_A (A, C^\bullet(F)),
$$
où $C^\bullet(F)$ désigne la résolution de Godement de $F$.
\vskip .3cm
{\bf 7.5}. Soit $(X, A, J)$ un idéotope. On rappelle que la catégorie $\Pro(A-\Mod_X)$ est abélienne et possède suffisamment d'objets $\varprojlim$ acycliques, ce qui permet de définir les foncteurs dérivés
$$
\varprojlim^{(i)}_A : \Pro(A-\Mod_X) \to A-\Mod_X \quad (i \geq 0)
$$
du foncteur limite projective. Plus précisément, on définit un foncteur dérivé à droite
$$
\bRd (\varprojlim_A): \D^+(\Pro(A-\Mod_X)) \to \D^+(A-\Mod_X),
$$
et pour tout $A$-faisceau $F$, on a 
$$
\varprojlim^{(i)}_A (F) = \mathrm{H}^i(\bRd \varprojlim_A (F)) \quad (i \in \mathbf{N}).
$$
\vskip .3cm
{
Lemme {\bf 7.5.1}. --- \it Soit $F = (F_n, u_n)_{n \in \mathbf{N}}$ un système projectif strict de $A$--Modules, dont les composants sont flasques. On suppose que pour tout $n$, de $A$--Module $\Ker(u_n)$ soit flasque. Alors on a :
$$
\varprojlim^{(i)}_A (F) = 0 \quad (i \geq 1).
$$
}
\vskip .3cm
{\bf Preuve} : Nous allons le voir par récurrence croissante sur l'entier $i$. Choisissons pour tout $n \geq 0$ un monomorphisme $F_n \to I_n$, avec $I_n$ un $A$--Module injectif. Posant pour tout $n \geq 0$
$$
G_n = \bigoplus_{i \leq n} I_i,
$$
on définit, avec les morphismes de transition évidents, un système projectif $G \varprojlim_A$-acyclique et un monomorphisme de $\underline{\Hom}(\mathbf{N}^0, A-\Mod_X)$
$$
v: F \to G.
$$
Notant $K$ le conoyau de $v$ dans $\underline{\Hom}(\mathbf{N}^0, A-\Mod_X)$, nous allons voir que la suite canonique
$$
0 \to \varprojlim_A (F) \to \varprojlim_A (G) \to \varprojlim_A (K) \to 0
\leqno{(S)}
$$
est exacte, ce qui prouvera le lemme lorsque $i = 1$. Comme $F$ est à composants flasques, pour tout objet $T$ de $X$ la suite canonique 
$$
0 \to \overline{\Gamma}(T, F) \to \overline{\Gamma}(T, G) \to \overline{\Gamma}(T, K) \to 0
\leqno{(T)}
$$
est exacte. Comme les noyaux des morphismes de transition de $F$ sont flasques, le système projectif de $A$--modules $\overline{\Gamma}(T, F)$ est strict ; par suite (EGA $0_{III}$ 13.2.2), la suite obtenue à partir de (T) en passant à la limite projective est encore exacte. Autrement dit, la suite canonique
$$
0 \to \Gamma(T, \varprojlim_A (F)) \to \Gamma (T, \varprojlim_A (G)) \to \Gamma(T, \varprojlim_A (K)) \to 0
$$
est exacte, d'où aussitôt l'exactitude de (S). On termine par récurrence en remarquant que $K$ vérifie les conditions du lemme et que l'on a des isomorphismes 
$$
\varprojlim^{(i)}_A (F) \isom \varprojlim^{(i-1)}_A (K) \quad (i \geq 2).
$$
Les conditions du lemme (7.5.1) étant satisfaites en particulier par les $A$-faisceaux pseudo-injectives, on peut, en utilisant des résolutions pseudo-injectifs, définir un foncteur dérivé
$$
\bRd \varprojlim_A : \D^+(X, A) \to \D^+(A-\Mod_X),
\leqno{(7.5.2)}
$$
et le diagramme
\[\begin{tikzcd}
	{\D^+(X, A)} && {\D^+(\Pro(A-\Mod_X))} \\
	& {\D^+(A-\Mod_X)}
	\arrow["{\D^+(c)}", from=1-1, to=1-3]
	\arrow["{\bRd \varprojlim_A}"', from=1-1, to=2-2]
	\arrow["{\bRd \varprojlim_A}", from=1-3, to=2-2]
\end{tikzcd}, \]
dans lequel $c$ désigne le foncteur canonique $A-\fsc(X) \to \Pro(A-\Mod_X)$, est commutatif.
\vskip .3cm
{
Proposition {\bf 7.5.3}. --- \it On suppose qu'il existe une sous-catégorie pleine génératrice de $X$, formée d'objets quasicompacts (par exemple que $X$ est \emph{localement algébrique}). Pour tout $A$-faisceau $F$ sur $X$ et tout $i \in \mathbf{N}$ le $A$--Module $\varprojlim^{(i)}_A(F)$ est le faisceau associé au préfaisceau
$$
T \mapsto \mathrm{H}^i(T, F)
$$
sur la famille génératrice des objets quasicompacts de $X$.
}
\vskip .3cm
{\bf Preuve} : Soit $J^\bullet$ une résolution pseudo-injective de $F$. Le $A$--Module $\varprojlim^{(i)}_A(F) = \mathrm{H}^i(\varprojlim_A J^\bullet)$ est le faisceau associé au préfaisceau
$$
T \mapsto \mathrm{H}^i\Gamma (T, \varprojlim_A J^\bullet).
$$
Lorsque $T$ est quasicompact, il est clair que
$$
\Gamma (T, \varprojlim_A J^\bullet) = \Gamma(T, J^\bullet)
$$
d'où, comme la propriété pour un $A$-faisceau d'être pseudo-injectif est stable par restriction à un objet du topos,
$$
\mathrm{H}^i (T, \varprojlim_A J^\bullet) = \mathrm{H}^i(T, F|T),
$$
ce qui entraîne manifestement le résultat.
\vskip .3cm
{
Corollaire {\bf 7.5.4}. --- \it Soit $F$ un $A$-faisceau flasque sur un topos $X$ engendré par ses objets quasicompacts. On a 
$$
\varprojlim^{(i)}_A (F) = 0 \quad (i \geq 2).
$$
}
\vskip .3cm
{\bf Preuve} : Il résulte sans peine de (7.4.16) que l'on a $\mathrm{H}^i(T, F) = 0$ $(i \geq 2)$ pour tout objet quasicompact $T$ de $X$.
\vskip .3cm
{
Proposition {\bf 7.5.5}. --- \it Soit $X$ un topos dont l'objet final est quasicompact. On a pour tout objet $K$ de $\D^+(X, A)$ un isomorphisme fonctoriel en $K$ :
$$
\bRd \Gamma (X, \bRd \varprojlim_A (K)) \isomlong \bRd \Gamma (X, K).
$$
}
\vskip .3cm
{\bf Preuve} : On peut supposer que les composants de $K$ sont pseudo-injectifs, et alors l'assertion a été vue dans la preuve de (7.5.3).
\vskip .3cm
{
Corollaire {\bf 7.5.6}. --- \it Soit $X$ un topos dont l'objet final est quasicompact. Pour tout $A$-faisceau $F$ sur $X$, on a une suite spectrale birégulière
$$
E^{p, q}_2 = \mathrm{H}^p(X, \varprojlim^{(q)})_A (F)) \Rightarrow \mathrm{H}^{p+q}(X, F).
$$
}
\vskip .3cm
Nous allons maintenant nous intéresser aux \emph{théorèmes $\check{C}$echistes de la cohomologie} sur un topos $X$ dont l'objet final est quasicompact.
\vskip .3cm
{
Définition {\bf 7.5.7}. --- \it Soit $X$ un topos. On appelle \emph{$A$-préfaisceau sur $X$} un système projectif indexé par $\mathbf{N}$, $F = (F_n)_{n \in \mathbf{N}}$ de préfaisceaux de $A$--Modules vérifiant les relations
$$
J^{n+1}F_n = 0 \quad (n \geq 0).
$$
}
\vskip .3cm
Dans ce qui suit, nous noterons
$$
\pre-\mathscr{E}(X, J)
$$
la sous-catégorie, exacte, pleine de $\underline{\Hom}(\mathbf{N}^\circ, \pre-A-\Mod_X)$ engendrée par les $A$-préfaisceaux. Comme pour les $A$-faisceaux, on définit la notion de $A$-préfaisceau essentiellement nul (resp. négligeable), et on appelle catégorie des $A$-préfaisceaux et on note
$$
A-\prefsc(X)
$$
la catégorie abélienne quotient de $\pre-\mathscr{E}(X, J)$ par la sous-catégorie abélienne épaisse engendrée par les $A$-faisceaux négligeables.

On définit, composant par composant, la notion de $A$-faisceau associé à un $A$-préfaisceau. Il est clair que le $A$-faisceau associé à un $A$-préfaisceau négligeable est négligeable, ce qui permet de définir un foncteur \emph{exact}, appelé foncteur faisceau associé : 
$$
a: A-\prefsc(X) \to A-\fsc(X).
$$
D'autre part, le foncteur inclusion $i: A-\Mod_X \to \pre-A-\Mod_X$ définit, composant par composant, un foncteur noté de même
$$
i : \mathscr{E}(X, J) \to \pre-\mathscr{E}(X, J).
$$
Plus généralement, on définit un foncteur cohomologique $(\Rd^p i)_{p \in \mathbf{Z}}$ en posant pour tout $A$-faisceau $F$ et tout entier $p$
$$
\Rd^p i(F) = (\Rd^p i(F_n))_{n \in \mathbf{N}}.
$$
Il est immédiat que pour tout morphisme $u: E \to F$ de $\mathscr{E}(X, J)$ dont le noyau et le conoyau sont négligeables, le noyau et le conoyau des morphismes $\Rd^p i(u)$ sont négligeables. Ceci permet de définir un foncteur cohomologique, noté de même
$$
(\Rd^p i)_{p \in \mathbf{Z}}: A-\fsc(X) \to A-\prefsc(X).
$$
Bien entendu, $\Rd^p i = 0$ pour $p < 0$, et on note $i$ le foncteur exact à gauche $\Rd^0 i$.
\vskip .3cm
{
Lemme {\bf 7.5.8}. --- \it Pour tout $A$-faisceau flasque $F$, on a : 
$$
\Rd^p i (F) = 0 \quad (p \geq 1).
$$
}
\vskip .3cm
{\bf Preuve} : On rappelle que si $M$ est un $A$--Module, $\Rd^p i (M)$ est le faisceau associé au préfaisceau $T \mapsto \mathrm{H}^p(T, M)$ ; en particulier, $\Rd^p i(M) = 0$ $(p \geq 1)$ lorsque $M$ est flasque. D'où le lemme, en appliquant ce résultat aux composants de $F$.

Ce lemme permet, en utilisant des résolutions flasques, de dériver le foncteur $i$ en un foncteur exact
$$
\bRd i: \D^+(X, A) \to \D^+(A-\prefsc(X)),
$$
et pour tout $A$-faisceau $F$, on a des isomorphismes
$$
\Rd^p i(F) \isom \mathrm{H}^p(\bRd i (F)) \quad (p \in \mathbf{Z}).
$$
\vskip .3cm
{
Proposition {\bf 7.5.9}. --- \it Soient $K \in \D(A-\prefsc(X))$ et $L \in \D^+(X, A)$. On a un isomorphisme fonctoriel
$$
\Hom(aK, L) \isomlong \Hom(K, \Rd i (L)).
$$
}
\vskip .3cm
{\bf Preuve} : Tout d'abord, étant donnés un $A$-préfaisceau $E$ et un $A$-faisceau $F$, on définit, au moyen des morphismes d'adjonction usuels sur les composants, des morphismes ``d'adjonction''
$$
u_E: E \to i \circ a(E).
\leqno{(1)}
$$
$$
v_F: a \circ i(F) \to F.
\leqno{(2)}
$$
Se ramenant au cas où $L$ est flasque, on définit en appliquant le morphisme (2) aux composants du complexe $L$ un morphisme ``d'adjonction'' fonctoriel
$$
v_L: a \circ \bRd i (L) \to L.
\leqno{(2\text{bis})}
$$
D'autre part, lorsque $K$ est \emph{borné inférieurement}, le morphisme (1) appliqué aux composants de $K$, composé avec $i(r)$, où $r$ désigne une résolution flasque de $a(K)$, fournit un morphisme ``d'adjonction'' fonctoriel
$$
u_K: K \to \bRd i (aK).
\leqno{(1\text{bis})}
$$
Le morphisme (2bis) définit de fa\c{c}on claire un morphisme de bifoncteurs cohomologiques
$$
\Hom(K, \bRd i (L)) \to \Hom(aK, L),
\leqno{(3)}
$$
et nous allons voir que c'est un isomorphisme. Quitte à décomposer $K$ en parties positive et négative, on est ramené à le voir lorsque $K$ appartient respectivement à $\D^+(A-\prefsc(X))$ et $\D^-(A-\prefsc(X))$. Dans ce dernier cas, on se ramène grâce à la suite spectrale
$$
E^{p, q}_1 = \Ext^q(U^{-p}, V) \Rightarrow \Ext^{p, q}(U, V),
$$
qui se montre comme (7.3.7.3) en utilisant le lemme (7.3.8), au cas où $K$ a un seul composant non nul. Finalement, on peut supposer que $K \in \D^+(A-\prefsc(X))$. Alors le morphisme (1bis) permet de définir un nouveau morphisme de bifoncteurs cohomologiques
$$
\Hom(aK, L) \to \Hom(K, \bRd i (L)),
\leqno{(4)}
$$
dont nous allons voir qu'il est inverse de (3). Pour cela, il suffit (Sém. CARTAN 11 Exp. 7) de montrer que les morphismes composés
$$
\bRd i (L) \xlongrightarrow{u_{\bRd i (L)}} \bRd i \circ a \circ \bRd i (L) \xlongrightarrow{\bRd i (v_L)} \bRd i (L)
$$
$$
a(K) \xlongrightarrow{a(u_K)} a \circ \bRd i \circ a(K) \xlongrightarrow{v_{a(K)}} a(K)
$$
sont respectivement l'identité de $\bRd i (L)$ et celle de $a(K)$. Pour cela, à des commutations de diagrammes près, il suffit de voir qu'étant donnés un $A$-préfaisceau $E$ et un $A$-faisceau $F$, les morphismes composés
$$
i(F) \xlongrightarrow{u_i(F)} i \circ a \circ i (F) \xlongrightarrow{i(v_F)} i (F)
$$
$$
a(E) \xlongrightarrow{a(u_E)} a \circ i \circ a(E) \xlongrightarrow{v_{a(E)}} a(E)
$$
sont respectivement l'identité de $i(F)$ et celle de $a(E)$. Or cela est vrai au stade des composants, d'où l'assertion.
\vskip .3cm
{
Définition {\bf 7.5.10}. --- \it On dit qu'un $A$-préfaisceau $F$ sur $X$ est \emph{pseudo-injectif} si c'est un objet injectif de $\pre-\mathscr{E}(X, J)$.
}
\vskip .3cm
Paraphrasant la preuve de (6.6.3), on voit que si $F = (F_n, u_n)_{n \in \mathbf{N}}$ est un $A$-préfaisceau pseudo-injectif, alors pour tout entier $n \geq 0$, le $\pre-A_n$--Module $\Ker(u_n)$ est injectif et on a un isomorphisme
$$
F_n \simeq \bigoplus_{0 \leq p \leq n} \Ker(u_n).
$$
De plus, il y a suffisamment de $A$-préfaisceaux pseudo-injectifs. Enfin, on voit facilement par adjonction, compte tenu du fait que le foncteur faisceau associé $\pre-\mathscr{E}(X, J) \to \mathscr{E}(X, J)$ est exact, que tout $A$-faisceau pseudo-injectif est également pseudo-injectif en tant que $A$-préfaisceau.

Supposons maintenant que l'\emph{objet final de $X$ soit quasicompact}. On définit alors par les arguments habituels un foncteur cohomologique
$$
(\overline{\check{\mathrm{H}}}^p(X, .))_{p \in \mathbf{Z}}: A-\prefsc(X) \to A-\fsc(\pt)
$$
en posant pour tout $A$-préfaisceau $F = (F_n)_{n \in \mathbf{N}}$ et tout entier $p$
$$
\overline{\check{\mathrm{H}}}^p(X, F) = (\check{\mathrm{H}}^p(X, F_n))_{n \in \mathbf{N}},
$$
avec les morphismes de transition évidents. On a bien sûr
$$
\overline{\check{\mathrm{H}}}^p(X, .) = 0 \quad (p < 0)
$$
et on pose 
$$
\overline{\check{\mathrm{H}}}^0(X, .) = \overline{\check{\Gamma}}(X, .).
$$
\vskip .3cm
{
Lemme {\bf 7.5.11}. --- \it Si $F$ est un $A$-préfaisceau pseudo-injectif, alors 
$$
\overline{\check{\mathrm{H}}}^p(X, F) = 0 \quad (p \geq 1).
$$
}
\vskip .3cm
{\bf Preuve} : Vu la structure des composants des $A$-préfaisceaux pseudo-injectifs, on est ramené à voir que si $M$ est un $\pre-A_n$--Module injectif, alors $\check{\mathrm{H}}^p(X, M) = 0$ $(p \geq 1)$. Or il résulte de (SGA4 V 2.1 formule (15)) que les foncteurs $\check{\mathrm{H}}^p(X, .)$ dépendent seulement de la structure de préfaisceau abélien, de sorte que les $\check{\mathrm{H}}^p(X, M)$ sont aussi les dérivés de $\check{\mathrm{H}}^0(X, .)$ dans la catégorie des $\pre-A_n$--Modules. D'où l'assertion.

Le lemme (7.5.11) permet de dériver le foncteur $\overline{\check{\Gamma}}(X, .)$, au moyen de résolutions pseudo-injectives, en un foncteur exact
$$
\bRd \overline{\check{\Gamma}}(X, .): \D^+(A-\prefsc(X)) \to \D^+(\pt, A).
$$
Si $F$ est un $A$-préfaisceau, on a 
$$
\overline{\check{\mathrm{H}}}^p(X, F) = \mathrm{H}^p(\bRd \overline{\check{\Gamma}}(X, F)) \quad (p \in \mathbf{Z}).
$$
Pour tout $A$-faisceau $F$, on convient de poser par définition 
$$
\overline{\check{\mathrm{H}}}^p(X, F) = \overline{\check{\mathrm{H}}}^p(X, i(F)) \quad (p \in \mathbf{Z}).
$$
On prendra garde que l'on n'obtient pas ainsi un foncteur cohomologique sur la catégorie des $A$-faisceaux.

Étant donné un $A$-préfaisceau pseudo-injectif, le système projectif $\overline{\check{\Gamma}}(X, I)$ est directement strict. Par suite, le foncteur $\varprojlim \circ \overline{\check{\Gamma}} (X, .)$ est dérivable en un foncteur exact
$$
\bRd \check{\Gamma}(X, .) : \D^+(A-\prefsc(X)) \to \D^+(A-\mod),
$$
et on a un isomorphisme de foncteurs composés
$$
\bRd \check{\Gamma}(X, .) \isom \bRd \varprojlim \circ \bRd \overline{\check{\Gamma}} (X, .).
$$
Étant donné un objet $K$ de $\D^+(A-\prefsc(X))$, on pose
$$
\check{\mathrm{H}}^p(X, K) = \mathrm{H}^p(\bRd \check{\Gamma} (X, K)),
$$
et 
$$
\check{\mathrm{H}}^0 = \check{\Gamma}.
$$
Il est clair que lorsque $F$ parcourt la catégorie des $A$-préfaisceaux, le foncteur 
$$
F \mapsto \check{\mathrm{H}}^0(X, F) = \check{\Gamma}(X, F)
$$
est exact à gauche et $\check{\mathrm{H}}^p(X, F) = 0$ $(p < 0)$.
\vskip .3cm
{
Proposition {\bf 7.5.12}. --- \it Soit $X$ un topos dont l'objet final est quasicompact. Pour tout $K \in \D^+(X, A)$, on a des isomorphismes fonctoriels
$$
\bRd \overline{\Gamma}(X, K) \isom \bRd \overline{\check{\Gamma}}(X, \bRd i (K))
\leqno{(A)}
$$
$$
\bRd \Gamma(X, K) \isom \bRd \check{\Gamma}(X, \bRd i (K)).
\leqno{(B)}
$$
En particulier, on a pour tout $A$-faisceau $F$ des suites spectrales
$$
E^{p, q}_2 = \overline{\check{\mathrm{H}}}^p(X, \Rd^q i (F)) \Rightarrow \overline{\mathrm{H}}^{p+q}(X, F)
\leqno{(1)}
$$
$$
E^{p, q}_2 = \check{\mathrm{H}}^p(X, \Rd^q i (F)) \Rightarrow \mathrm{H}^{p+q}(X, F).
\leqno{(2)}
$$
Les morphismes canoniques $\overline{\check{\mathrm{H}}}^1(X, F) \to \overline{\mathrm{H}}^1(X, F)$ et $\check{\mathrm{H}}^1(X, F) \to \mathrm{H}^1(X, F)$ (resp. $\overline{\check{\mathrm{H}}}^2(X, F) \to \overline{\mathrm{H}}^2(X, F)$ et $\check{\mathrm{H}}^2(X, F) \to \mathrm{H}^2(X, F)$) déduits des suites spectrales (1) et (2) sont des isomorphismes (resp. des monomorphismes).
}
\vskip .3cm
{\bf Preuve} : L'isomorphisme (B) se déduit de (A) en appliquant le foncteur $\bRd \varprojlim$ aux deux membres. Pour voir (A), on peut supposer que $K$ est pseudo-injectif, donc que $i(K) = \bRd i(K)$ l'est également. Alors (A) se voit en appliquant l'égalité évidente
$$
\overline{\mathrm{H}}^0(X, F) = \overline{\check{\mathrm{H}}}^0(X, F),
$$
valable pour tout $A$-faisceau $F$, aux composants de $K$. Les suites spectrales (1) et (2) se déduisent de (A) et (B) respectivement par des arguments standards. Si maintenant $F$ est un $A$-faisceau, on voit en appliquant (SGA4 V (2.2)) et formule 23) aux composants de $F$ que
$$
\overline{\check{\mathrm{H}}}^0(X, \Rd^q i (F)) = 0 \quad (q \geq 1),
$$
d'où aussitôt 
$$
\check{\mathrm{H}}^0(X, \Rd^q i (F)) = 0 \quad (q \geq 1). 
$$
La dernière assertion de (7.5.12) en résulte aussitôt.

On peut introduire une autre notion de cohomologie de $\check{C}$ech de la manière suivante. Étant donné un topos $X$, la catégorie
$$
P = \underline{\Hom}(\mathbf{N}^\circ, X)
$$
est un topos, et le système projectif
$$
\mathbf{A} = (A/J^{n+1})_{n \in \mathbf{N}} = (A_n)_{n \in \mathbf{N}}
$$
est un Anneau de $P$. On définit sans peine un isomorphisme
$$
\pre-\mathbf{A}-\Mod_P \to \pre-\mathscr{E}(X, J),
$$
d'où un foncteur cohomologie de $\check{C}$ech
$$
\bRd \check{\Gamma}: \D^+(\pre-\mathscr{E}(X, J)) \to \D^+(A-\mod),
\leqno{(7.5.13)}
$$
défini sans hypothèse de quasicompacité sur l'objet final de $X$ 
\vskip .3cm
{
Proposition {\bf 7.5.14}. --- \it On suppose que l'objet final de $X$ est quasicompact. Alors le diagramme
\[\begin{tikzcd}
	{\D^+(\pre-\mathscr{E}(X, J))} && {\D^+(A-\prefsc(X)))} \\
	& {\D^+(A-\mod)}
	\arrow["{\D^+(p)}", from=1-1, to=1-3]
	\arrow["{(7.5.13)}"', from=1-1, to=2-2]
	\arrow["{\bRd \check{\Gamma}}", from=1-3, to=2-2]
\end{tikzcd}\]
dans lequel $p: \pre-\mathscr{E}(X, J) \to A-\prefsc(X)$ désigne le foncteur quotient cano\-nique, est commutatif.
}
\vskip .3cm
{\bf Preuve} : Évident, car il y a identité entre $\mathbf{A}$--Modules injectifs et $A$-préfaisceaux pseudo-injectifs.





\vskip .3cm
{\bf 7.6. Morphismes de CARTAN}.

Soient $K, L, M$ trois complexes de $A$-faisceaux sur un topos $X$. On suppose que l'une des conditions suivantes est réalisée pour les degrés :
\[\begin{tikzcd}
	K & L & M \\
	{-} & {-} & {+} \\
	b & b & \emptyset \\
	b & \emptyset & b & {.}
\end{tikzcd}\]
Alors, les morphismes (6.3.5) pour les composants permettent de définir de la fa\c{c}on habituelle un morphisme de $\K(X, A)$ : 
$$
\cHom^\bullet_A(K \otimes_A L, M) \to \cHom^\bullet_A(K, \cHom^\bullet_A(L, M)).
\leqno{(7.6.1)}
$$
Soient maintenant $K \in \D^-(X, A)$, $L \in \D^-(X, A)$ et $M \in \D^+(X, A)$, et choisissons des résolutions quasilibres $P$ et $Q$ de $K$ et $L$ respectivement, et une résolution flasque $R$ de $M$. Comme $P \otimes_A Q$ est quasilibre (5.8), le morphisme (7.6.1)
$$
\cHom^\bullet_A(P \otimes_A Q, R) \to \cHom^\bullet_A (P, \cHom_A(Q, R)),
$$
s'interprète comme un morphisme de $\D(X, A)$
$$
\bRd \cHom_A(K \boldsymbol{\otimes} L, M) \to \bRd \cHom_A(K, \bRd \cHom_A(L,M)), 
\leqno{(7.6.2)}
$$
et on vérifie facilement que (7.6.2) ne dépend pas des résolutions choisies, et dépend fonctoriellement de $K, L$ et $M$. Contrairement à ce qui se passe pour les faisceaux de $A$--modules, le morphisme (7.6.2) \emph{n'est pas en général un isomorphisme}~; nous verrons toutefois plus loin que c'est le cas si l'on fait des hypothèses de finitude convenables.

Supposons que le topos $X$ et l'anneau $A$ soient \emph{noethériens}. Alors on déduit de (7.6.2) des morphismes fonctoriels canoniques 
$$
\bRd \overline{\Hom}_A(K \boldsymbol{\otimes}L, M) \to \Rd \overline{\Hom}_A(K, \bRd \cHom_A(L, M)).
\leqno{(7.6.3)}
$$
$$
\bRd \Hom_A(K \boldsymbol{\otimes}L, M) \to \Rd \Hom_A(K, \bRd \cHom_A(L, M)).
\leqno{(7.6.4)}
$$
$$
\Hom_A(K \boldsymbol{\otimes}L, M) \to \Hom_A(K, \bRd \cHom_A(L, M)).
\leqno{(7.6.5)}
$$
Utilisant (7.4.18), on les obtient en appliquant respectivement les foncteurs $\bRd \overline{\Gamma}$, $\bRd \Gamma$, $\Hom_A(A, .)$ aux deux membres de (7.6.2).

Nous allons maintenant essayer de définir le morphisme (7.6.5) sans hypothèse de finitude sur $A$ ou $X$. Étant donnés deux complexes de $A$-faisceaux $K \in \K^+(X, A)$ et $L \in \K^b(X, A)$, le morphisme (6.3.13.1) permet de définir pour tout entier $n$ un morphisme de $A$-faisceaux
$$
K^n \to \bigoplus_p \cHom_A(L^p, K^n \otimes_A L^p) \subset \cHom^n_A(L, K \otimes_A L),
$$
d'où un morphisme de complexes 
$$
K \to \cHom^\bullet_A (L, K \otimes_A L).
\leqno{(7.6.6)}
$$
\vskip .3cm
{
Proposition {\bf 7.6.7}. --- \it Soient $K \in \D^+(X, A)$ et $L \in \D^-(X, A)_{\torf}$. On a un morphisme fonctoriel canonique
$$
K \to \bRd \cHom_A (L, K \boldsymbol{\otimes} L)
$$
qui ``coïncide'' avec l'identité de $K$ lorsque $L = A$.
}
\vskip .3cm
{\bf Preuve} : On peut supposer $L$ plat et borné. Choisissant alors une résolution quasilibre $L'$ de $L$ et une résolution flasque $F$ de $K \otimes L = K \boldsymbol{\otimes} L$, le morphisme annoncé est le composé de (7.6.6) et du morphisme canonique $\cHom^\bullet_A(L, K \otimes L) \to \cHom^\bullet_A(L', I)$. On laisse au lecteur le soin de voir que cela ne dépend pas des choix faits.
\vskip .3cm
{
Corollaire {\bf 7.6.8}. --- \it Soient $K \in \D^+(X, A)$, $L \in \D^-(X, A)_{\torf}$ et $M \in \D^+(X, A)$. On a un morphisme fonctoriel
$$
\Hom_A(K \boldsymbol{\otimes}L, M) \to \Hom_A(K, \bRd \cHom_A(L, M)).
$$
En particulier, on a un morphisme fonctoriel
$$
\Hom_A(K, M) \to \Hom_A (A, \bRd \cHom_A (K, M)).
$$
}
\vskip .3cm
{\bf Preuve} : Soit $u: K \boldsymbol{\otimes} L \to M$. On obtient un morphisme $K \to \bRd \cHom_A (L, M)$ en composant $\bRd \cHom_A(\id_L, u)$ avec (7.6.7).
\vskip .3cm
{\bf 7.6.9}. Soient $E, F, G$ trois complexes de $A$-faisceaux sur $X$. On suppose que les degrés vérifient l'une des conditions suivantes : 
\[\begin{tikzcd}
	E & {+} & {-} & b & \emptyset & b \\
	F & {-} & {+} & \emptyset & b & b \\
	G & {-} & {+} & b & b & \emptyset & {.}
\end{tikzcd}\]
Nous allons alors définir un morphisme de complexes fonctoriel
$$
\cHom^\bullet_A (E, F) \otimes_A G \to \cHom^\bullet_A(E, F \otimes_A G).
\leqno{(7.6.9.1)}
$$
Il est clair qu'il suffit de le définir pour les $A$-faisceaux, car alors on disposera pour tout triplet $(p, q, r)$ d'entiers d'un morphisme de $A$-faisceaux
$$
\cHom_A(E^p, F^q) \otimes_A G^r \to \cHom_A (E^p, F^q \otimes_A G^r),
$$
ce qui permet de définir de fa\c{c}on évidente le morphisme annoncé. Pla\c{c}ons-nous donc dans ce cas. Si $m$ et $n$ sont deux entiers, avec $m \geq n \geq 0$, on a un morphisme canonique de $A_n$--Modules
$$
\cHom_A(E_m, F_n) \otimes_A G_n \to \cHom_A(E_m, F_n \otimes_A G_n).
$$
Par passage à la limite inductive suivant $m$, on en déduit un morphisme de $A_n$--Modules
$$
\cHom_A(E, F)_n \otimes_A G_n \to \cHom_A (E, F \otimes_A G)_n,
$$
et la collection de ces morphismes pour les différents entiers $n$ définit le morphisme de $A$-faisceaux désiré (c'est même un morphisme de $\mathcal{E}(X, J)$).

Soient maintenant $E \in \D^-(X, A)$, $F \in \D^+(X, A)$, et $G \in \D(X, A)$. On se place dans l'un des cas suivants :
\begin{itemize}
    \item[(i)] $G \in \D^-(X, A)_{\torf}$.
    \item[(ii)] L'anneau $A$ est local régulier, $J$ est son idéal maximal et $G \in \D^+(X, A)$.
\end{itemize}
Nous allons alors définir un morphisme fonctoriel
$$
\bRd \cHom_A(E, F) \boldsymbol{\otimes}G \to \bRd \cHom_A(E, F \boldsymbol{\otimes} G).
\leqno{(7.6.9.2)}
$$
Dans chacun des cas envisagés, le complexe $G$ admet une résolution plate et bornée inférieurement $N$. Étant donnés une résolution quasilibre $L$ de $E$ et une résolution flasque $M$ de $F$, on a (7.6.9.1) un morphisme de complexes 
$$
\cHom^\bullet_A (L, M) \otimes_A N \to \cHom^\bullet_A (L, M \otimes_A N).
$$
Choisissant alors une résolution flasque $P$ de $M \otimes_A N$, on en déduit un morphisme de complexes
$$
\cHom^\bullet_A (L, M) \otimes_A N \to \cHom^\bullet_A (L, P),
$$
qui ne dépend pas, isomorphisme près dans $\D^+(X, A)$, des choix que l'on a faits. C'est celui-là que l'on prend.




\vskip .3cm
{\bf 7.7. Opérations externes}.

On suppose toujours fixés l'anneau $A$ et l'idéal $J$.
\vskip .3cm
{\bf 7.7.1}. Soit $f: X \to Y$ un morphisme de topos. Le foncteur image réciproque $f^*: A-\fsc(Y) \to A-\fsc(X)$, étant exact, admet un foncteur dérivé noté de même
$$
f^*: \D(Y, A) \to \D(X, A)\;.
$$
\vskip .3cm
{
Proposition {\bf 7.7.2}. --- \it Soient $E$ et $F \in \D(Y, A)$. 
\begin{itemize}
    \item[(i)] Lorsque $E$ et $F \in \D^-(Y, A)$, ou lorsque $A$ est un anneau local régulier et $J$ son idéal maximal, il existe un \emph{isomorphisme} canonique fonctoriel
    $$
    f^* (E) \boldsymbol{\otimes}_A f^* (F) \isomlong f^*(E \boldsymbol{\otimes}_A F),
    $$
    induisant, lorsque $E$ et $F$ sont des $A$-faisceaux, les morphismes (5.17.1) sur les objets de cohomologie.
    \item[(ii)] Lorsque $E \in \D^-(Y, A)$ et $F \in \D^+(Y, A)$, il existe un morphisme canonique fonctoriel
    $$
    f^* \bRd \cHom_A (E, F) \to \bRd \cHom_A (f^* E, f^* F),
    $$
    induisant, lorsque $E$ et $F$ sont des $A$-faisceaux, les morphismes (6.4.1.1) sur les objets de cohomologie.
\end{itemize}
}
\vskip .3cm
{\bf Preuve} : Montrons (i). On peut supposer $E$ et $F$ quasilibres, de sorte que (5.17.2) $f^* E$ et $f^* F$ le sont également. Par ailleurs, l'isomorphisme (5.17.1) (pour $i = 0$) permet de définir de fa\c{c}on classique un isomorphisme de complexes $f^* E \otimes_A f^* F \to f^*(E \otimes_A F)$, dont on vérifie sans peine qu'il répond à la question. Pour la partie (ii) on peut supposer que $E$ est quasilibre et $F$ flasque. Alors le morphisme (6.4.1.1) permet de définir, composant par composant, un morphisme de complexes
$$
f^* \cHom^\bullet_A(E, F) \to \cHom^\bullet_A (f^* E, f^* F).
$$
Choisissant alors une résolution flasque $L$ de $f^* F$, on en déduit un morphisme de complexes
$$
f^* \cHom^\bullet_A(E, F) \to \cHom^\bullet_A (f^* E, L)
$$
ne dépendant pas de $L$, à isomorphisme près dans $\D^+(X, A)$. Compte tenu du fait que $f^* E$ est quasilibre (5.17.2), on vérifie aussitôt qu'il répond à la question.
\vskip .3cm
{\bf 7.7.3}. Soit $f: X \to Y$ un morphisme \emph{quasicompact} de topos. Utilisant (4.2.4), on voit, à l'aide de résolutions flasques, que le foncteur $f_*: A-\fsc(X) \to A-\fsc(Y)$ est dérivable en un foncteur exact
$$
\bRd f_* = \bRd{}^+ f_*: \D^+(X, A) \to \D^+(Y, A).
$$
Il peut être utile de savoir prolonger ce foncteur à $\D(X, A)$. Pour cela, introduisons la définition suivante (cf. Séminaire Strasbourg-Heidelberg 66-67 Exp.2 2.5).
\vskip .3cm
{
Définition {\bf 7.7.4}. --- \it Soient $X$ un topos et $q$ un entier $\geq 0$. On dit que $X$ est de dimension topologique stricte $\leq q$ s'il vérifie les relations équivalentes suivantes.
\begin{itemize}
    \item[(i)] Tout faisceau abélien $F$ sur $X$ admet une résolution flasque de longueur $\leq q$.
    \item[(ii)] Pour tout objet $T$ de $X$, tout fermé $Z$ de $X/T$, et tout faisceau abélien $F$ sur $X$, on a 
    $$
    \mathrm{H}^{q+1}_Z(T, F|T) = 0.
    $$
    On appelle \emph{dimension topologique stricte} de $X$, et on note 
    $$
    \dimtops(X)
    $$
    la borne inférieure (éventuellement infinie) des entiers $q$ satisfaisant aux conditions (i) et (ii).
\end{itemize}
}
\vskip .3cm
Lorsque le topos $X$ est de dimension topologique stricte finie, tout complexe de $A$-faisceaux admet une résolution flasque, de sorte qu'on définit un foncteur dérivé droit
$$
\bRd f_*: \D(X, A) \to \D(Y, A),
$$
prolongeant le précédent, en utilisant ([H], I 5.3 $\gamma$).

Soit maintenant un autre morphisme quasicompact $g: Y \to Z$. Comme le foncteur $f_*$ transforme $A$-faisceau flasque en $A$-faisceau flasque, on a un isomorphisme
$$
\bRd{}^+ (g \circ f)_* \isomlong \bRd{}^+ g_* \circ \bRd{}^+ f_*,
\leqno{(7.7.5)}
$$
avec la condition de cocycles habituelle. Si de plus $X$ et $Y$ sont de dimension topologique stricte finie, cet isomorphisme se prolonge en un isomorphisme
$$
\bRd (g \circ f)_* \isomlong \bRd g_* \circ \bRd f_*.
\leqno{(7.7.5~\text{bis})}
$$
\vskip .3cm
{
Proposition {\bf 7.7.6}. --- \it Soit $f: X \to Y$ un morphisme de topos quasicompact. Étant donnés $E \in \D(Y, A)$ et $F \in \D^+(X, A)$, on a un \emph{isomorphisme} fonctoriel
$$
\Hom_A(f^* E, F) \isomlong \Hom_A (E, \bRd f_* (F)).
$$
}
\vskip .3cm
{\bf Preuve} : Analogue à celle de (7.5.9). Se ramenant au cas où $F$ est borné inférieurement et flasque, on définit, grâce à (4.3.2), un morphisme d'``adjonction''
$$
f^* \bRd f_* (F) \to F,
$$
d'où un morphisme de bifoncteurs cohomologiques
$$
\Hom_A (E, \bRd f_* (F)) \to \Hom_A (f^* E, F), 
$$
dont on veut prouver que c'est un isomorphisme. Par le way-out functor lemma, on peut pour cela supposer que $E \in \D^+(Y, A)$. Dans ce cas, on déduit de (4.3.1) un morphisme d'``adjonction''
$$
E \to \bRd f_* (f^* E),
$$
et on conclut en vérifiant que les composés canoniques sont les identités.
\vskip .3cm
{\bf 7.7.7}. Soient $f: X \to Y$ un morphisme quasicompact de topos, $E \in \D^-(Y, A)$ et $F \in \D^+(X, A)$. Nous allons définir un morphisme canonique fonctoriel
$$
\bRd \cHom_A (E, \bRd f_* (F)) \to \bRd f_* \bRd \cHom_A (f^*E, F).
$$
Pour cela, on peut supposer que $E$ est borné supérieurement et quasilibre, et que $F$ est borné inférieurement et flasque. Alors, on définit grâce à (6.4.4.1) un morphisme fonctoriel de complexes
$$
s: \cHom^\bullet_A (E, f_* (F)) \to f_* \cHom^\bullet_A (f^* E, F).
$$
Comme $f^* (E)$ est quasilibre (5.17.2), le complexe $\cHom^\bullet_A (f^* E, F)$ s'identifie à $\bRd \cHom_A (f^* E, F)$. Choisissant alors une résolution flasque
$$
u: \cHom^\bullet_A (f^* E, F) \to L,
$$
le morphisme désiré est le composé $f_*(u) \circ s$.
\vskip .3cm
{
Proposition {\bf 7.7.8}. --- \it On suppose $A$ noethérien. Soit $f: X \to Y$ un morphisme cohérent de topos, avec $X$ localement noethérien. (SGA4 VI 2.11). On a un \emph{isomorphisme} canonique fonctoriel
$$
\bRd f_* \bRd \cHom_A (f^* E, F) \isomlong \bRd \cHom_A (E, \bRd f_* (F)).
\leqno{(1)}
$$
Si de plus $X$ et $Y$ ont des objets finaux quasicompacts, on a des isomorphismes cano\-niques fonctoriels
$$
\bRd \overline{\Hom}_A (f^* E, F) \isomlong \bRd \overline{\Hom}_A (E, \bRd f_* (F)).
\leqno{(2)}
$$
$$
\bRd \Hom_A (f^* E, F) \isomlong \bRd \Hom_A (E, \bRd f_* (F)).
\leqno{(3)}
$$
}
\vskip .3cm
{\bf Preuve} : Pour définir (1), on peut supposer $E$ borné supérieurement et quasilibre, et $F$ borné inférieurement et pseudo-injectif. Comme $f$ est cohérent, il résulte de (6.4.5) que le morphisme de complexes $s$ de (7.7.7) est un isomorphisme. Par ailleurs, il résulte du lemme (7.4.19) que $\cHom^\bullet_A (f^* E, F)$ est flasque, d'où aussitôt l'assertion. 

Les isomorphismes (2) et (3) se déduisent de (1) en appliquant respectivement les foncteurs $\bRd \overline{\Gamma}(Y, .)$et $\bRd \Gamma(Y, .)$ aux deux membres de (1) et en utilisant (7.4.18).
\vskip .3cm
{\bf 7.7.9}. Soient $X$ un topos et $i: T \to T'$ un morphisme quasicompact (SGA4 VI 1.7) de $X$. Le foncteur (4.5.3)
$$
i_! : A-\fsc(T) \to A-\fsc(T').
$$
étant exact, est évidemment dérivable en un foncteur exact
$$
\bRd i_!: \D(T, A) \to \D(T', A).
$$
\vskip .3cm
{
Proposition {\bf 7.7.10}. --- \it Soient $E \in \D(T, A)$ et $F \in \D(T', A)$.
\begin{itemize}
    \item[(i)] On a un \emph{isomorphisme} fonctoriel
    $$
    \Hom_A (\bRd i_!(E), F) \isomlong \Hom_A (E, i^* (F)).
    $$
    \item[(ii)] Lorsque $E \in \D^-(T, A)$ et $F \in \D^+(T', A)$, il existe un morphisme fonctoriel
    $$
    \bRd \cHom_A (\bRd i_! (E), F) \to \bRd i_* \bRd \cHom_A (E, i^* F),
    $$
    qui est un \emph{isomorphisme} lorsque $i$ est cohérent en $X/T$ localement noethérien.
    \item[(iii)] Si $T$ et $T'$ sont quasicompacts, on a des \emph{isomorphismes} fonctoriels
    $$
    \bRd \overline{\Hom}_A (\bRd i_! (E), F) \isomlong \bRd \overline{\Hom}_A (E, i^* F)
    $$
    $$
    \bRd \Hom_A (\bRd i_! (E), F) \isomlong \bRd \Hom_A (E, i^* F).
    $$
    \item[(iv)] On a un \emph{isomorphisme} fonctoriel
    $$
    \bRd i_! (E \boldsymbol{\otimes} i^* (F)) \isomlong \bRd i_! (E) \boldsymbol{\otimes} F
    $$
    dans chacun des cas suivants : 
    \begin{itemize}
        \item[a)] $E \in \D^{-}(T, A)$ et $F \in \D^-(T', A)$.
        \item[b)] $E \in \D^{-}_{\torf}(T, A)$ et $F \in \D (T, A)$.
        \item[c)] L'anneau $A$ est local régulier, $J$ est son idéal maximal, $E \in \D^+(T, A)$ et $F \in \D^+(T', A)$.
    \end{itemize}
\end{itemize}
}
\vskip .3cm
{\bf Preuve} : Pour (i), on définit composant par composant, à partir de (4.5.5), des morphismes d'adjonction
$$
i_! i^* (F) \to F
$$
$$
E \to i^* i_! (E),
$$
et il est immédiat, composant par composant ; que les composés canoniques (cf. la preuve de (4.3.5)) sont des identités. Montrons (ii).

Pour définir le morphisme désiré, on peut supposer $E$ quasilibre et borné supérieurement, et $F$ flasque et borné inférieurement. Alors, on définit grâce  (6.4.7.1) un morphisme de complexes
$$
w: \cHom^\bullet_A (i_!(E), F) \to i_* \cHom^\bullet_A (E, i^* F).
$$
Comme $i_!(E)$ est quasilibre (5.18.5) et $i^* (F)$ flasque, on obtient   le morphisme annoncé en choisissant une résolution flasque $u$ de $\cHom^\bullet_A (E, i^* F)$ et en prenant le morphisme composé $i_* (u) \circ w$. Lorsque $i$ est cohérent, le morphisme $w$ est un isomorphisme (6.4.7), et lorsque $X/T$ est localement noethérien, le complexe $\cHom^\bullet_A (E, i^* F)$ est flasque, du moins lorsque $F$ est pris pseudo-injectif, ce qui est toujours possible (7.4.19). Montrons (iii). On peut comme précédemment prendre $E$ quasilibre et borné supérieurement, et $F$ pseudo-injectif et borné inférieurement. Alors l'isomorphisme de complexes 
$$
\overline{\Hom}^\bullet_A (E, i^* F) \isomlong \overline{\Hom}^\bullet_A (i_! E, F)
$$
fournit le premier isomorphisme annoncé. Le second s'en déduit en passant à la limite projective sur les composants. Enfin (iv) résulte immédiatement de l'isomorphisme (5.18.1) en prenant $E$ plat, ce qui est possible dans chacun des cas envisagés : alors, $i_! (E)$ est plat (5.18.5).
\vskip .3cm
{\bf 7.7.11}. Soient $X$ un topos, $U$ un ouvert de $X$ et $Y$ le topos fermé complémentaire de $U$ (SGA4 IV 3.3). On note $j: Y \to X$ le morphisme de topos canonique. Le foncteur cohomologique (4.6.1)
$$
(\Rd^p j^!)_{p \in \mathbf{Z}}: A-\fsc(X) \to A\fsc(Y)
$$
est, comme on le voit immédiatement composant par composant, \emph{effacé} (en degrés $> 0$) \emph{par les $A$-faisceaux flasques}. Ceci permet, au moyen de résolutions flasques, de définir un foncteur dérivé droit de $j^!$ : 
$$
\bRd j^! : \D^+(X, A) \to \D^+(Y, A),
$$
et on a pour tout $A$-faisceau $F$ sur $X$ :
$$
\Rd^p j^! (F) \isom \mathrm{H}^p(\bRd j^! (F)) \quad (p \in \mathbf{Z}).
$$
\vskip .3cm
{
Proposition {\bf 7.7.12}. --- \it Soient $E \in \D(X, A)$ et $F \in \D(Y, A)$.
\begin{itemize}
    \item[(i)] Si $E \in \D^+(X, A)$, on a un isomorphisme fonctoriel
    $$
    \Hom_A (\bRd j_* (F), E) \isomlong \Hom_A (F, \bRd j^!(E)).
    $$
    \item[(ii)] Si $E \in \D^+(X, A)$ et $F \in \D^-(Y, A)$, il existe un \emph{isomorphisme} fonctoriel
    $$
    \bRd \cHom_A (\bRd j_*(F), E) \isomlong \bRd j_* \bRd \cHom_A (F, \bRd j^! (E)).
    $$
    \item[(iii)] Si l'objet final de $X$ (donc aussi celui de $Y$) est quasicompact, alors il existe pour $E \in \D^+(X, A)$ et $F \in \D^-(Y, A)$ des isomorphismes fonctoriels
    $$
    \bRd \overline{\Hom}_A (\bRd j_*(F), E) \isomlong \bRd \overline{\Hom}_A (F, \bRd j^! (E)))
    $$
    $$
    \bRd \Hom_A (\bRd j_*(F), E) \isomlong \bRd \Hom_A (F, \bRd j^! (E))).
    $$
    \item[(iv)] On a un isomorphisme fonctoriel
    $$
    \bRd j_* (E \boldsymbol{\otimes} \bRd j^* (E)) \isomlong \bRd j_* (F) \boldsymbol{\otimes} E
    $$
    dans chacun des cas suivants :
    \begin{itemize}
        \item[a)] $E \in \D^-(X, A)$ et $F \in \D^-(Y, A)$.
        \item[b)] $F \in \D^-_{\torf}(Y, A)$.
        \item[c)] L'anneau $A$ est local régulier, $J$ est son idéal maximal, $E \in \D^+(X, A)$ et $F \in \D^+(Y, A)$.
    \end{itemize}
\end{itemize}
}
\vskip .3cm
{\bf Preuve} : Le foncteur $j_*$ étant exact, le foncteur $\bRd j_*$ est défini sur $\D(Y, A)$ en entier, ce qui donne un sens à certaines des expressions de l'énoncé. L'assertion (i) se voit comme l'assertion correspondante de (7.7.10), en utilisant (4.6.3). Montrons (ii). Pour définir le morphisme de l'énoncé, on peut supposer que $F$ est quasilibre et borné supérieurement, et que $E$ est pseudo-injectif et borné inférieurement. Alors le morphisme (6.4.8.1) permet de définir un isomorphisme de complexes
$$
h: \cHom^\bullet_A (j_* (F), E) \to j_* \cHom^\bullet_A(F, j^! (E)).
$$
Le complexe $j_* (F)$ est fortement plat (6.7.1), de sorte que l'on déduit de (6.7.2 (ii)) que $\cHom^\bullet_A (j_*(F), E) \isom \bRd \cHom_A (\bRd j_* (F), E)$. Par ailleurs, on voit par adjonction et grâce au fait que 
$$
j_*: \mathscr{E}(Y, J) \to \mathscr{E}(X, J)
$$
est un foncteur exact, que $j^!(E)$ est pseudo-injectif, de sorte que l'isomorphisme $h$ répond à la question. Les isomorphismes de (iii) se déduisent du précédent, grâce à (7.4.18), en appliquant aux deux membres les foncteurs $\bRd \overline{\Gamma}(X, .)$ et $\bRd \Gamma(X, .)$ respectivement. On peut aussi définir le premier directement à partir de l'isomorphisme (6.4.8.2). Enfin, l'assertion (iv) se voit comme l'assertion analogue de (7.7.10), en utilisant cette fois (5.19.1 (i)).
\vskip .3cm
{
Proposition {\bf 7.7.13}. --- \it Soit $K \in \D^+(X, A)$. On a un isomorphisme fonctoriel
$$
\bRd j^! (K) \isomlong j^* \bRd \cHom_A(j_*(A), K).
$$
}
\vskip .3cm
{\bf Preuve} : On peut supposer $K$ pseudo-injectif. Alors on déduit de (6.5.1.1) (pour $i = 0$) un isomorphisme de complexes
$$
j^!(K) \isomlong j^* \cHom^\bullet_A (j_*(A), K),
$$
qui répond à la question (comme dans la preuve de (7.7.12), on utilise le fait que $j_*(A)$ est fortement plat).
\vskip .3cm
{
Proposition {\bf 7.7.14}. --- \it Soient $X$ un topos, $U$ un ouvert de $X$, $Y$ le topos fermé complémentaire de $U$, $i: U \to X$ et $j: Y \to X$ les morphismes de topos cano\-niques . On suppose que $i$ est quasicompact. On a alors pour tout $E \in \D^+(X, A)$ des triangles exacts fonctoriels en $E$
\[\begin{tikzcd}
	& {\bRd j_*(j^* E)} \\
	{\bRd i_! i^* (E)} && E
	\arrow[from=2-1, to=2-3]
	\arrow[from=2-3, to=1-2]
	\arrow[dashed, from=1-2, to=2-1]
\end{tikzcd}\leqno{(I)}\]
\[\begin{tikzcd}
	& {\bRd i_*(i^* E)} \\
	{\bRd j_* \bRd j^! (E)} && {E.}
	\arrow[from=2-1, to=2-3]
	\arrow[from=2-3, to=1-2]
	\arrow["{d^\circ 1}"', dashed, from=1-2, to=2-1]
\end{tikzcd}\]
}
\vskip .3cm
{\bf Preuve} : On peut pour les définir supposer $E$ pseudo-injectif. Alors, on déduit de fa\c{c}on évidente de (4.6.4) des suites exactes de complexes répondant à la question.
\vskip .3cm
{
Définition {\bf 7.7.15}. --- \it Soient $X$ un topos, $U$ un ouvert de $X$, $Y$ le topos fermé complémentaire, $i: U \to X$ et $j: Y \to X$ les morphismes de topos canoniques. On pose pour tout objet $E$ de $\D^+(X, A)$ :
$$
\mathrm{H}^p_Y (X, E) = \Ext^p_A (j_*(A), E) \quad (p \in \mathbf{Z}).
$$
}
\vskip .3cm
{
Proposition {\bf 7.7.16}. --- \it Sous les hypothèses de (7.7.15), on a une suite exacte illi\-mitée
$$
\dots \to \mathrm{H}^p_Y (X, E) \to \mathrm{H}^p(X, E) \to \mathrm{H}^p(U, E) \to \mathrm{H}^{p+1}_Y(X, E) \to \dots.
$$
Si de plus l'objet final de $X$ est quasicompact, on a : 
\begin{itemize}
    \item[(i)] Une suite spectrale birégulière
    $$
    E^{p, q}_2 = \mathrm{H}^p(X, \mathrm{H}^q_Y(F)) \Rightarrow \mathrm{H}^{p+q}_Y (X, F),
    $$
    pour tout $A$-faisceau $F$ sur $X$.
    \item[(ii)] Des suites exactes
    $$
    0 \to \varprojlim^{(1)}_A \overline{\mathrm{H}}^{i-1}_Y (X, E) \to \mathrm{H}^i_Y(X, E) \to \varprojlim_A \overline{\mathrm{H}}^{i}_Y (X, E) \to 0,
    $$
    pour tout objet $E$ de $\D^+(X, A)$ et tout $i \in \mathbf{Z}$.
\end{itemize}
}
\vskip .3cm
{\bf Preuve} : Compte tenu de (7.7.10. (i)), la suite exacte illimitée est la suite exacte des $\Ext^\bullet(., E)$ déduite de la suite (4.6.4)
$$
0 \to i_!(A) \to A \to j_* (A) \to 0.
$$
Compte tenu de la définition (6.5.4), la suite spectrale (i) est conséquence immédiate de (7.7.12. (ii)). Enfin, les suites exactes (ii) sont une simple traduction de (7.4.16. a)).











% End
% Begin


















%%%%%%%%%%%%%%%%%%%%%%%%%%%%%%%%%%%%
\subsection*{8. Changement d'anneau.}
\addcontentsline{toc}{subsection}{8. Changement d'anneau}

\vskip .3cm
{\bf 8.1}. Soient $X$ un topos, $A$ et $B$ deux anneaux commutatifs unifères, $J$ et $K$ deux idéaux de $A$ et $B$ respectivement et
$$
u: A \to B
$$
un morphisme d'anneaux unifères, vérifiant $u(J) \subset K$. Comme il n'y aura pas d'ambiguïté, on appellera $A$-faisceaux les $(A, J)$-faisceaux et $B$-faisceaux les $(B, K)$-faisceaux (1.2).

Il est clair que tout $B$-faisceau $F$ est canoniquement muni d'une structure de $A$-faisceau d'où un foncteur exact
$$
\mathscr{E}(X, K) \to \mathscr{E}(X, J),
$$
qui fournit par passage au quotient un foncteur \emph{exact et conservatif}
$$
u_*: B-\fsc(X) \to A-\fsc(X),
\leqno{(8.1.1)}
$$
d'où un foncteur exact noté de même sur les catégories dérivées
$$
u_*: \D(X, B) \to \D(X, A).
\leqno{(8.1.2)}
$$
Étant donné un $A$-faisceau $F$, le $A$-faisceau $F \otimes_A u_*(B)$ est canoniquement muni d'une structure de $B$-faisceau grâce aux structures de $B$--Modules des composants du deuxième facteur, d'où un foncteur \emph{exact à droite}
$$
u^*: A-\fsc(X) \to B-\fsc(X).
\leqno{(8.1.3)}
$$
Utilisant des résolutions plates dans la catégorie des $A$-faisceaux, on en déduit un foncteur exact
$$
\bLd u^*: \D^-(X, A) \to \D^-(X, B).
\leqno{(8.1.4)}
$$
De plus, lorsque $A$ est un anneau local régulier et $J$ son idéal maximal, il résulte de (5.16) que le foncteur (8.1.4) se prolonge, toujours en utilisant des résolutions plates, en un foncteur noté de même
$$
\bLd u^*: \D(X, A) \to \D(X, B),
\leqno{(8.1.5)}
$$
qui envoie $\D^+(X, A)$ dans $\D^+(X, B)$.
\vskip .3cm
{
Proposition {\bf 8.1.6}. --- \it Soient $E \in \D(X, A)$ et $F \in \D(X, B)$.
\begin{itemize}
    \item[(i)] Si $E \in \D^-(X, A)$ (resp. $A$ est local régulier et $J$ est son idéal maximal), on a un isomorphisme fonctoriel de $A$-modules
    $$
    \Hom_B(\bLd u^* (E), F) \isomlong \Hom_A(E, u_*(F)).
    $$
    \item[(ii)] Si $E \in \D^-(X, A)$ et $F \in \D^+(X, B)$, on a un isomorphisme fonctoriel
    $$
    u_* \bRd \cHom_B (\bLd u^*(E), F) \isomlong \bRd \cHom_A(E, u_*(F)).
    $$
    \item[(iii)] Si le topos $X$ et l'anneau $A$ sont noethériens, et si $E \in \D^-(X, A)$ et $F \in \D^+(X, B)$, on a des isomorphismes fonctoriels
    $$
    u_* \bRd \overline{\Hom}_B(\bLd u^*(E), F) \isomlong \bRd \overline{\Hom}_A(E, u_*(F))
    $$
    et
    $$
    u_* \bRd \overline{\Hom}_B(\bLd u^*(E), F) \isomlong \bRd \Hom_A(E, u_*(F))
    $$
    dans $\D(A-\fsc(\pt))$ et $\D(A-\mod)$ respectivement, en convenant de noter encore $u: \D(B-\mod) \to \D(A-\mod)$ le foncteur restriction des scalaires.
\end{itemize}
}
\vskip .3cm
{\bf Preuve} : Montrons (i). Nous allons pour cela définir tout d'abord un morphisme d'``adjonction''
$$
E \to u_* \bLd u^* (E).
\leqno{(8.1.7)}
$$
Pour ce faire, on peut, dans chacun des cas envisagés, supposer $E$ plat, et on prend alors le morphisme ``extension des scalaires''
$$
E \to E \otimes_A u_* (B)
$$
déduit du morphisme évident de $A$-faisceaux $A \to u_* (B)$. Le morphisme (8.1.7) permet de définir de fa\c{c}on évidente un morphisme de bifoncteurs cohomologiques
$$
\Hom_B(\bLd u^* (E), F) \to \Hom_A(E, u^* (F)),
$$
et il s'agit de voir que c'est un isomorphisme. Quitte à découper $F$ en parties po\-sitive et négative, on est ramené à le voir lorsque $F \in \D^+(X, B)$ ou $F \in \D^-(X, B)$. Dans l'hypothèse non respée, le cas où $F \in \D^+(X, B)$ se ramène, au moyen du ``way-out functor lemma'', au cas où $F$ est borné, de sorte qu'on peut pour prouver l'assertion supposer que $F \in \D^-(X, B)$. Alors, on définit comme suit un morphisme d'``adjonction''
$$
\bLd u^* u_* F \to F.
\leqno{(8.1.8)}
$$
On choisit une résolution $A$-plate $P \to u_*(F)$, et la structure de $B$-faisceaux sur les composants de $u_*(F)$ permet d'en déduire de fa\c{c}on évidente un morphisme de complexes $u^*(P) \to F$ qui répond à la question. Dans l'hypothèse respée, le morphisme (8.1.8) se définit de même sans hypothèse de degrés sur $F$. Enfin, les morphismes composés canoniques déduits de (8.1.7) et (8.1.8) sont les identités, d'où (i). Pour montrer l'assertion (ii), on peut supposer $E$ quasilibre et $F$ flasque, ce qui implique que $u^*(E)$ et $u_*(F)$ sont respectivement quasilibre et flasque. Le lemme suivant permet alors de définir un isomorphisme de complexes
$$
u_* \cHom^\bullet_B(u^*(E), F) \isomlong \cHom^\bullet_A(E, u_*(F)),
$$
qui répond à la question.
\vskip .3cm
{
Lemme {\bf 8.1.9}. --- \it Étant donnés un $A$-faisceau $E = (E_n)_{n \in \mathbf{N}}$ et un $B$-faisceau $F = (F_n)_{n \in \mathbf{N}}$, il existe un isomorphisme fonctoriel
$$
u_* \cHom_B(u^*(E), F) \isomlong \cHom_A(E, u_*(F)).
$$
}
\vskip .3cm
On va le définir sur les composants. Soit $n$ un entier $\geq 0$; pour tout entier $m \geq n$, on a un isomorphisme évident de $A_m$--Modules
$$
\cHom_{B_m}(E_m \otimes_{A_m}B_m, F_n) \isomlong \cHom_{A_m}(E_m, F_n),
$$
et on obtient l'isomorphisme désiré sur les composants d'ordre $n$ en passant à la limite inductive suivant $m$.

Compte tenu de (7.4.18), l'assertion (iii) résulte de (ii) et des isomorphismes 
\[\begin{tikzcd}
	{\bRd \overline{\Gamma} \circ u_*} && {u_* \circ \bRd \overline{\Gamma}} \\
	{\bRd \Gamma \circ u_*} && {u_* \circ \bRd \Gamma,}
	\arrow["\sim", from=1-1, to=1-3]
	\arrow["\sim", from=2-1, to=2-3]
\end{tikzcd}\leqno{(8.1.10)}\]
qui proviennent immédiatemment du fait que la propriété pour un $A$--Module d'être flasque ne dépend pas de l'Anneau de base.
\vskip .3cm
{
Proposition {\bf 8.1.11}. --- \it
\begin{itemize}
    \item[(i)] Si $L$ est un $A$-faisceau plat, le $B$-faisceau $u^*(L)$ est plat. Si $E \in D^-(X, A)_{\torf}$, alors $\bLd u^*(E) \in \D^-(X, B)_{\torf}$.
    \item[(ii)] Soient $E$ et $F \in \D(X, A)$. On a un isomorphisme fonctoriel
    $$
    \bLd u^* (E \boldsymbol{\otimes}_A F) \isomlong \bLd u^* (E) \boldsymbol{\otimes}_B \bLd u^*(F)
    $$
    dans chacun des cas suivants~:
    \begin{itemize}
        \item $E$ et $F \in \D^-(X, A)$.
        \item $A$ et $B$ sont locaux réguliers, $J$ et $K$ sont leurs idéaux maximaux, et $E$ et $F \in \D^+(X, A)$, ou $E \in \D^b(X, A)$.
    \end{itemize}
    \item[(iii)] Soient $E \in \D(X, A)$ et $F \in \D(X, B)$. On a un isomorphisme fonctoriel
    $$
    u_* (\bLd u^* (E) \boldsymbol{\otimes}_B F) \isomlong u_*(F) \boldsymbol{\otimes}_A E,
    $$
    appelé \emph{formule de projection}, dans chacun des cas suivants :
    \begin{itemize}
        \item $E \in \D^-(X, A)$ et $F \in \D^-(X, B)$. 
        \item $E \in \D^-(X, A)_{\torf}$.
        \item $A$ et $B$ sont locaux réguliers, $J$ et $K$ sont leurs idéaux maximaux, $E \in \D^+(X, A)$ et $F \in \D^+(X, B)$.
    \end{itemize}
    \item[(iv)] Soient $E \in \D^-(X, A)$ et $F \in \D^+(X, A)$. On a un morphisme canonique fonctoriel
    $$
    \bLd u^* \bRd \cHom_A (E, F) \to \bRd \cHom_B (\bLd u^* (E), \bLd u^*(F))
    $$
    lorsque $A$ est un anneau local régulier et $J$ son idéal maximal. Ce morphisme est un \emph{isomorphisme} lorsque de plus $B$ est une $A$-algèbre finie et $K = JB$.
    \item[(v)] Soient $E \in \D^-(X, A)$ et $F \in \D^+(X, A)$. Lorsque l'objet final de $X$ est quasicompact, et que l'anneau $A$ est régulier et $J$ est son idéal maximal, on a des morphismes canoniques
    $$
    \bRd \overline{\Hom}_A(E, F) \to u_* \bRd \overline{\Hom}_B(\bLd u^* (E), \bLd u^*(F))
    $$
    $$
    \bRd \Hom_A(E, F) \to u_* \bRd \Hom_B(\bLd u^* (E), \bLd u^*(F)).
    $$
\end{itemize}
}
\vskip .3cm
{\bf Preuve} : Étant donnés un $A$-faisceau $L$ et un $B$-faisceau $M$, on définit composant par composant un isomorphisme ``de projection''
$$
u_* (u^*(L) \otimes_B M) \isomlong L \otimes_A u_* (M)
\leqno{(8.1.12)}
$$
qui nous servira dans la preuve de (iii). Lorsque $L$ est plat, on en déduit que $u^*(L)$ est plat, grâce à l'exactitude et à la conservativité du foncteur $u_*$ ; l'autre partie de (i) en résulte aussitôt. Pour voir (ii), on peut dans chacun des cas de l'énoncé prendre $E$ et $F$ plats, et alors on exhibe de fa\c{c}on évidente un isomorphisme de complexes qui répond à la question. De même pour (iii), compte tenu de l'isomorphisme (8.1.12). Montrons (iv). Par (8.1.6.(i)), il s'agit de définir un morphisme
$$
\bRd \cHom_A(E, F) \to u_* \bRd \cHom_B (\bLd u^* (E), \bLd u^*(F)),
$$
soit encore, par (8.1.6.(ii)), un morphisme
$$
\bRd \cHom_A(E, F) \to \bRd \cHom_A (E, u_* \bLd u^* (F)).
$$
On prend celui déduit de fa\c{c}on évidente du morphisme d'adjonction (8.1.7)~: $F \to u_* \bLd u^* (F)$. Notant $q$ le morphisme de l'énoncé, nous allons donner une des\-cription directe de $u_* (q)$. Il s'agit, compte tenu de (8.1.16.(iii)), d'un morphisme 
$$
B \boldsymbol{\otimes}_A \bRd \cHom_A (E, F) \to \bRd \cHom_A (E, B \boldsymbol{\otimes}_A F),
\leqno{(8.1.13)}
$$
qui n'est autre, comme on s'en assure aisément, que (7.6.9.2). Supposons maintenant que $B$ soit une $A$-algèbre finie et que $K = JB$. Pour voir que dans ce cas le morphisme $q$ est un isomorphisme, il suffit, d'après la conservativité du foncteur $u_*$, de montrer que (8.1.13) en est un. Quitte à remplacer $B$ par une résolution finie par des $A$-modules libres de type fini, on est ramené au cas où $B = A$, et alors l'assertion est évidente. Prouvons enfin (v). Nous allons le faire pour les $\bRd \overline{\Hom}$, la construction dans l'autre cas étant analogue. Appliquant le foncteur $u_* \circ \bRd \overline{\Gamma}$ au morphisme $q$, on obtient un morphisme
$$
u_* \bRd \overline{\Gamma} \bLd u^* \bRd \cHom_A (E, F) \to \bRd \cHom_B (\bLd u^* (E), \bLd u^* (F)).
\leqno{(8.1.14)}
$$
Par ailleurs, le morphisme d'adjonction $\id \xlongrightarrow{a} u_* \bLd u^*$ définit, grâce à(8.1.10), un morphisme
$$
\bRd \overline{\Gamma}(a): \bRd \overline{\Gamma} \bRd\cHom_A (E, F) \to u_* \bRd \overline{\Gamma} \bLd u^* \bRd \cHom_A(E, F)
\leqno{(8.1.15)}
$$
soit, d'après (7.4.18.(i)),
$$
\bRd \overline{\Hom}_A(E, F) \to u_* \bRd \overline{\Gamma} \bRd \cHom_A (E, F).
\leqno{(8.1.15)bis}
$$
Le morphisme annoncé est le composé de (8.1.14) et (8.1.15)bis.
\vskip .3cm
{
Proposition {\bf 8.1.16}. --- \it  
\begin{itemize}
    \item[(i)] Soit $f: X \to Y$ un morphisme de topos. Étant donné $E \in \D(Y, A)$, on a un \emph{isomorphisme} fonctoriel
    $$
    \bLd u^* f^* (E) \to f^* \bLd u^* (E)
    $$
    lorsque $E \in \D^-(Y, A)$ ou $A$ est local régulier et $J$ est son idéal maximal. 
    \item[(ii)] Soient $X$ un topos, $T$ et $T'$ deux objets de $X$ et $i: T \to T'$ un morphisme quasicompact. Étant donné $E \in \D(T, A)$, on a un \emph{isomorphisme} fonctoriel
    $$
    \bRd i_! \bLd u^* (E) \isomlong \bLd u^* \bRd i_! (E).
    $$
    lorsque $E \in \D^-(T, A)$, ou $A$ est local régulier et $J$ est son idéal maximal.
    \item[(iii)] Soient $X$ un topos, $U$ un ouvert de $X$, $Y$ le topos fermé complémentaire et $j: Y \to X$ le morphisme de topos canonique. Lorsque $A$ est régulier et $J$ est son idéal maximal, on a pour tout $E \in \D^+(X, A)$ un morphisme fonctoriel
    $$
    \bLd u^* \bRd j^! (E) \to \bRd j^! \bLd u^* (E).
    $$
    Ce morphisme est un \emph{isomorphisme} lorsque de plus $B$ est une $A$-algèbre finie et $K = JB$.
    \item[(iv)] Sous les hypothèses préliminaires de (iii), étant donné $E \in \D(Y, A)$, on a un \emph{isomorphisme} fonctoriel
    $$
    \bLd u^* \bRd j_* (E) \isomlong \bRd j_* \bLd u^* (E)
    $$
    lorsque $E \in \D^-(Y, A)$, ou $A$ est local régulier et $J$ est son idéal maximal.
\end{itemize}
}
\vskip .3cm
{\bf Preuve} : Montrons (i). Lorsque $E \in \D^-(Y, A)$, on peut le supposer quasilibre, de sorte que $f^*(E)$ l'est également (5.17.2). Comme $f^*(B) \isommap B$, l'isomorphisme évident de complexes
$$
B \otimes_A f^*(E) \to f^*(B \otimes_A E),
$$
défini composant par composant, répond à la question. Dans la deuxième hypothèse, on peut supposer $E$ plat, et on concluera comme précédemment si on prouve que $B \otimes_A f^* (E) \isom B \boldsymbol{\otimes}_A f^* (E)$. Comme le foncteur $\bLd u^*$ est de dimension cohomologique finie à gauche, il suffit pour cela (CD début page 43), de montrer que pour tout $A$-faisceau plat $M$ sur $Y$, on a $\cTor^A_i(B, f^*(M)) = 0$ $(i \geq 1)$, ce qui est immédiat à partir de (5.17.1). Pour voir (ii), on peut dans chacun des cas considérés supposer $E$ plat, de sorte que $i_!(E)$ l'est aussi (5.18.5.(i)). 

Alors, on construit, composant par composant à partir de (5.18.1), un isomorphisme de complexes
$$
i_!(E \otimes_A i^*(B)) \isomlong i_!(E) \otimes_A B
$$
qui répond à la question. L'assertion (iv) se montre de même à partir de (5.19). Prouvons (iii). D'après (7.7.13), il s'agit de définir un morphisme
$$
\bLd u^* j^* \bRd \cHom_A (j_* A, E) \to j^* \bRd \cHom_B (j_* B, \bLd u^*(E)),
\leqno{(8.1.17)}
$$
ou encore, compte tenu de (i), un morphisme
$$
j^* \bLd u^* \bRd \cHom_A (j_* A, E) \to j^* \bRd \cHom_B (\bLd u^* j_* A, \bLd u^* (E)).
\leqno{(8.1.18)}
$$
On prend celui déduit de (8.1.11.(iv)) par application du foncteur $j^*$. L'assertion d'isomorphie résulte également de (loc. cit.). 
\vskip .3cm
{
Définition {\bf 8.1.19}. --- \it Étant donnés deux idéotopes $(X, A, J)$ et $(Y, B, K)$, on appelle \emph{morphisme d'idéotopes} $(X, A, J) \to (Y, B, K)$ un couple $(f, u)$ formé d'un morphisme de topos $f: X \to Y$ et d'un morphisme d'anneaux unifères $u: B \to A$ vérifiant
$$
u(K) \subset J.
$$
}
\vskip .3cm
On vérifie immédiatement qu'on définit ainsi une catégorie, appelée \emph{catégorie des idéotopes}. Étant donné un morphisme d'idéotopes
$$
(f, u) = \varphi: (X, A, J) \to (Y, B, K),
$$
on note $\varphi^*$ et on appelle \emph{image réciproque} par $\varphi$ le foncteur exact 
$$
\varphi^* = u^* \circ f^* : \B-\fsc(Y) \to A-\fsc(X).
$$
De même, on note $\bLd \varphi^*$ le foncteur exact
$$
\bLd u^* \circ f^* : \D^-(Y, B) \to \D^-(X, A),
$$
qui se prolonge lorsque $B$ est régulier et $K$ est son idéal maximal, en un foncteur exact noté de même de $\D(Y, B)$ dans $\D(X, A)$.

De fa\c{c}on analogue, on définit les foncteurs \emph{images directes} par 
$$
\varphi_* = u_* \circ f_* : A-\fsc(X) \to B-\fsc(Y)
$$
et 
$$
\bRd \varphi_* = u_* \circ \bRd f_* : \D^+(X, A) \to \D^+(Y, B).
$$
On déduit sans peine de (7.7) et (8.1) des formalismes analogues pour les morphismes d'idéotopes, qu'on laisse au lecteur le soin d'expliciter, car nous n'en aurons pas besoin dans la suite.
 
\vskip .3cm
{\bf 8.2}. Nous allons maintenant expliciter dans quelques cas particuliers les résultats du numéro précédent.
\vskip .3cm
{
Proposition {\bf 8.2.1}. --- \it Soit $(X, A, J)$ un idéotope. Pour tout entier $n > 0$, les foncteurs
$$
(\id_A)_*: (A, J)-\fsc(X) \to (A, J^n)-\fsc(X)
$$
et
$$
(\id_A)^*: (A, J^n)-\fsc(X) \to (A, J)-\fsc(X)
$$
sont des équivalences quasi-inverses l'une de l'autre. En particulier, lorsque $A$ est noethérien, la catégorie $A-\fsc(X)$ ne dépend pas (à équivalence près) de l'idéal $J$, mais seulement de la topologie qu'il définit (ce qui justifie a posteriori la notation $A-\fsc(X)$ au lieu de $(A, J)-\fsc(X)$, étant alors sous-entendu que la lettre $A$ désigne un anneau topologique).
}
\vskip .3cm
{\bf Preuve} : Il est immédiat que les morphismes d'adjonction $\id \to u_* u^*$ et $u^*u_* \to \id$ sont des isomorphismes (on pose $u = \id$).

A partir de maintenant, on suppose donné un idéotope $(X, A, J)$, un entier $n \geq 0$, et on note $u: A \to A_n = A/J^{n+1}$ le morphisme d'anneaux canonique. L'anneau $A_n$ sera toujours supposé muni de l'idéal $J/J^{n+1}$. 
\vskip .3cm
{
Proposition {\bf 8.2.2}. --- \it Sous les conditions précédentes, le foncteur
$$
\bLd u^*: \D^-(X, A) \to \D^-(X, A_n)
$$
est \emph{conservatif}.
}
\vskip .3cm
{\bf Preuve} : Soit $p$ un morphisme de $\D^-(X, A)$ tel que $\bLd u^*(p)$ soit un isomorphisme, et montrons que $p$ est un isomorphisme. Comme $\bLd u^*$ transforme triangle exact en triangle exact, on est ramené, par considération du mapping-cylinder de $p$, à montrer que si $E$ est un objet de $\D^-(X, A)$ tel que $\bLd u^* (E)$ soit acyclique, alors $E$ est acyclique. Nous allons pour cela raisonner par l'absurde et supposer qu'il existe un plus grand entier $i$ tel que $\mathrm{H}^i(E) \neq 0$ ; quitte à tronquer $E$ et à translater les degrés, on peut d'ailleurs supposer que $E$ est à degrés $\leq 0$ et que $i = 0$, puis même que $E$ est plat. Alors il est clair que $\mathrm{H}^0(\bLd u^*(E)) = \mathrm{H}^0(E)/J^{n+1}\mathrm{H}^0(E)$, d'où $\mathrm{H}^0(E) = J^{n+1}\mathrm{H}^0(E)$ par hypothèse et $\mathrm{H}^0(E) = 0$ par le ``lemme de Nakayama'' (5.12). D'où la contradiction annoncée.

Nous allons maintenant comparer les $A_n$--Modules et les $A_n$-faisceaux. Tout d'abord, on sait (3.6) que le foncteur canonique (3.5.1)
$$
A_n-\Mod_X \to A_n-\fsc(X)
\leqno{(8.2.3)}
$$
est exact, pleinement fidèle, et que son image essentielle est une sous-catégorie exacte de $A_n-\fsc(X)$. Pour des raisons de commodité, nous noterons ici
$$
A_n-\fsc_0(X)
$$
cette sous-catégorie exacte de $A-\fsc(X)$. On en déduit que la sous-catégorie
$$
\D_0(X, A_n)
$$
de $\D(X, A_n)$ engendrée par les complexes dont la cohomologie appartient à $A_n-\fsc_0(X)$ est une sous-catégorie triangulée. Même remarque pour les catégories analogues $\D^*_0(X, A_n)$, avec des notations évidentes. Le foncteur (8.2.3) définit des foncteurs exacts
$$
\omega^*: \D^*(A_n-\Mod_X) \to \D^*_0(X, A_n).
\leqno{(8.2.4)}
$$
\vskip .3cm
{
Proposition {\bf 8.2.5}. --- \it  
\begin{itemize}
    \item[(i)] Le foncteur (8.2.4) induit un foncteur
    $$
    \D^-(A_n-\Mod_X)_{\torf} \to \D^-(X, A_n)_{\torf}.
    $$
    \item[(ii)] Soient $E$ et $F \in \D(A_n-\Mod_X)$. On a un isomorphisme fonctoriel
    $$
    \omega(E) \boldsymbol{\otimes} \omega(F) \isomlong \omega(E \boldsymbol{\otimes} F)
    $$
    lorsque $E$ et $F \in \D^-(A_n-\Mod_X)$, où $E \in \D^-(A_n-\Mod_X)_{\torf}$.
    \item[(iii)] Si $E \in \D^-(A_n-\Mod_X)$ et $F \in \D^+(A_n-\Mod_X)$, on a un isomorphisme fonctoriel
    $$
    \omega \bRd \cHom_{A_n}(E, F) \isomlong \bRd \cHom_{A_n}(\omega(E), \omega(F)).
    $$
\end{itemize}
}
\vskip .3cm
{\bf Preuve} : L'assertion (i) résulte de ce que un $A_n$--Module plat définit un $A_n$-faisceau fortement plat. Profitons d'ailleurs de l'occassion pour remarquer qu'un $A_n$--Module libre (sur un faisceau d'ensembles) définit un $A_n$-faisceau quasilibre. L'assertion (ii) se voit sans peine, en prenant $E$ plat. Pour (iii), on peut supposer $E$ quasilibre (i.e. à composants des $A_n$--Modules libres) et $F$ flasque. Alors $\omega(E)$ est quasilibre et $\omega(F)$ a des composants qui sont isomorphes à des $A_n$-faisceaux flasques, d'où aussitôt l'assertion.
\vskip .3cm
{
Proposition {\bf 8.2.6}. --- \it On suppose que l'objet final de $X$ est quasicompact. Alors le foncteur (8.2.4)
$$
\omega^+_0: \D^+(A_n-\Mod_X) \to \D^+_0(X, A_n)
$$
est une \emph{équivalence de catégories}.
}
\vskip .3cm
{\bf Preuve} : Résulte immédiatement de (7.1.2).
\vskip .3cm
{\bf Notations 8.2.7}. Dans la suite, on notera
$$
\alpha: A_n-\Mod_X \to A-\fsc(X)
$$
le foncteur canonique, composé de $u_*$ et de (8.2.3). On posera
$$
\alpha_* = u_* \circ \omega: \D(A_n-\Mod_X) \to \D(X, A).
$$
Avant de poursuivre, introduisons une nouvelle notion. Notant pour tout entier $p \geq 0$ par $u_p: A \to A_p$ le morphisme d'anneaux canonique, la sous-catégorie pleine de $\D^-(X, A)$ (resp. $\D^-(X, A)_{\torf}$) engendrée par les objets $E$ vérifiant
$$
\bLd u^*_p(E) \in \D_0(X, A_p) \quad (p \in \mathbf{N})
$$
est une sous-catégorie triangulée. On la note
$$
\D^-_0(X, A) \quad (\text{resp.}~\D^-_0(X, A)_{\torf}).
$$
Si $A$ est local régulier et $J$ est son idéal maximal, on définit de même la catégorie triangulée $\D_0(X, A)$. La cohérence de cette notation est donnée par la proposition suivante :
\vskip .3cm
{
Proposition {\bf 8.2.8}. --- \it Soit $E \in \D^-_0(X, A_n)$. Alors pour tout entier $p \leq n$, on a, en notant $v_p: A_n \to A_p$ le morphisme d'anneaux canonique,
$$
\bLd v^*_p(E) \in \D^-_0(X, A).
$$
}
\vskip .3cm
{\bf Preuve} : D'après ([H] I 7.3.), on peut supposer que $E$ est ``réduit au degré 0''. Notant de même le $A_n$--Module correspondant et $P$ une résolution de $E$ par des $A_n$--Modules plats, on a alors
$$
\bLd v^*_p (E) \quad \text{``=''} \quad  A_p \otimes_{A_n} P,
$$
d'où l'assertion.

Remarquons que l'intérêt de la catégorie $\D^-_0(X, A)$ vient de ce qu'elle contient, lorsque $X$ est localement noethérien, la catégorie triangulée correspondante formée des complexes à cohomologie des $A$-faisceaux \emph{constructibles}.

Supposons maintenant que l'objet final de $X$ soit quasicompact. Choisissant un foncteur quasi-inverse $\omega^{-1}$ de $\omega^+_0$ (8.2.6), nous noterons
$$
\bLd \alpha^* = \omega^{-1} \circ \bLd u^*: \D^b_0(X, A)_{\torf} \to \D^b(A_n-\Mod_X)_{\torf}.
\leqno{(8.2.9)}
$$
Le ``$\torf$'' en indice dans le second membre provient de (8.1.11.(i)) et de (8.2.5.(i)). Lorsque de plus l'anneau $A$ est local régulier et $J$ est son idéal maximal, on note de même le foncteur
$$
\bLd \alpha^* = \omega^{-1} \circ \bLd u^*: \D^+_0(X, A) \to \D^+(A_n-\Mod_X).
\leqno{(8.2.9)\text{bis}}
$$
\vskip .3cm
{
Proposition {\bf 8.2.10}. --- \it Soient $E$ et $F \in \D(X, A)$.
\begin{itemize}
    \item[(i)] Si $E$ et $F \in \D^-_0(X, A)$, alors $E \boldsymbol{\otimes}_A F \in \D^-_0(X, A)$. 
    \item[(ii)] Si l'anneau $A$ est local régulier et $J$ est son idéal maximal, et si $E$ et $F \in \D^+_0(X, A)$, alors $E \boldsymbol{\otimes}_A F \in \D^+_0(X, A)$.
    \item[(iii)] On suppose que l'anneau $A$ est local régulier et que $J$ est son idéal maximal. Alors, si $E \in \D^-_0(X, A)$ et $F \in \D^+_0(X, A)$, on a
    $$
    \bRd \cHom_A(E, F) \in \D^+_0(X, A).
    $$
\end{itemize}
}
\vskip .3cm
{\bf Preuve} : Montrons (i). D'après (8.1.11.(ii)), on est ramené à voir que si $L$ et $M \in \D^-_0(X, A_p)$ pour un entier $p \geq 0$, alors il en est de même pour $L \boldsymbol{\otimes}_{A_p} M$. On peut ([H] I 7.3.) supposer que $L$ et $M$ sont réduits au degré 0, et appartiennent donc à $A_p-\fsc_0(X)$. Alors l'assertion résulte de (8.2.5.(ii)). L'assertion (ii) résulte également sans peine de (8.2.6) et (8.2.5. (ii)). Montrons (iii). Grâce à (8.1.11. (iv)), on est ramené à voir l'assertion analogue dans $\D(X, A_p)$ pour tout entier $p \geq 0$. Par dévissage ([H] I 7.3.), on peut supposer que $E$ est réduit au degré 0, et alors l'assertion résulte de (8.2.6) et (8.2.5. (iii)).
\vskip .3cm
{
Proposition {\bf 8.2.11}. --- \it On suppose que l'objet final de $X$ est quasicompact. Si $E \in \D^b_0(X, A)_{\torf}$ et $F \in \D^+(A_n-\Mod_X)$, on a des isomorphismes fonctoriels
$$
\alpha_* \bRd \cHom_{A_n} (\bLd \alpha^*(E), F) \isomlong \bRd \cHom_A (E, \alpha_* F).
\leqno{(i)}
$$
$$
u_* \bRd \cHom_{A_n} (\bLd \alpha^*(E), F) \isomlong \bRd \Hom_A(E, \alpha_* F).
\leqno{(ii)}
$$
Si $E \in \D^b_0(X, A)$ et $F \in \D^+(A_n-\Mod_X)$, ou bien si $A$ est régulier, $J$ est son idéal maximal et $E \in \D^+_0(X, A)$ et $F \in D^+(A_n-\Mod_X)$, on a un isomorphisme fonctoriel
$$
u_* \Hom_{A_n} (\bLd \alpha^*(E), F) \isomlong \Hom_A(E, \alpha_* (F)).
\leqno{(iii)}
$$
}
\vskip .3cm
{\bf Preuve} : Conséquence immédiate des définitions (8.2.7), (8.2.9) et (8.2.9)bis, et de (8.1.6).
\vskip .3cm
{
Proposition {\bf 8.2.12}. --- \it On suppose que l'objet final de $X$ est quasicompact. Soient $E$ et $F \in \D(X, A)$ et $G \in \D(A_n-\Mod_X)$.
\begin{itemize}
    \item[(i)] On a un isomorphisme fonctoriel
    $$
    \bLd \alpha^* (E) \boldsymbol{\otimes}_{A_n} \bLd \alpha^*(F) \isomlong \bLd \alpha^*(E \boldsymbol{\otimes}_A F)
    $$
    lorsque $E \in \D^b_0(X, A)_{\torf}$ et $F \in \D^b_0(X, A)_{\torf}$, ou bien lorsque $A$ est local régulier, $J$ est son idéal maximal, $E \in \D^b_0(X, A)$ et $F \in \D^+_0(X, A)$.
    \item[(ii)] On suppose que $A$ est local régulier et que $J$ est son idéal maximal. Si $E \in \D^-_0(X, A)$ et $F \in \D^+_0(X, A)$, on a un isomorphisme fonctoriel
    $$
    \bLd \alpha^* \bRd \cHom_A (E, F) \isomlong \bRd \cHom_{A_n} (\bLd \alpha^* (E), \bLd \alpha^*(F)).
    $$
    \item[(iii)] On suppose que $G \in \D^b(A_n-\Mod_X)$ et $F \in \D^b_0(X, A)_{\torf}$, ou bien que $A$ est régulier d'idéal maximal $J$, et que $G \in \D^b(A_n-\Mod_X)_{\torf}$ (resp. $\D^+(A_n-\Mod_X)$) et $F \in \D^+_0(X, A)$ (resp. $\D^b_0(X, A)$). On a alors un isomorphisme fonctoriel
    $$
    \alpha_*(G \boldsymbol{\otimes}_{A_n} \bLd \alpha^*(F)) \isomlong \alpha_* (G) \boldsymbol{\otimes}_A F.
    $$
\end{itemize}
}
\vskip .3cm
{\bf Preuve} : Résulte formellement de (8.1.11), compte tenu des définitions (8.2.7), (8.2.9) et (8.2.9)bis.

Signalons enfin quelques propriétés de stabilité par morphismes des catégories $\D_0(X, A)$.
\vskip .3cm
{
Proposition {\bf 8.2.13}. --- \it Soient $A$ un anneau commutatif unifère et $J$ un idéal de $A$.
\begin{itemize}
    \item[(i)] Étant donné un morphisme de topos $f: X \to Y$, le foncteur $f^*$ induit un foncteur
    $$
    \D^-_0(Y, A) \to \D^-_0(X, A).
    $$
    Si de plus $A$ est local régulier et $J$ est son idéal maximal, il induit un foncteur
    $$
    \D_0(Y, A) \to \D_0(X, A).
    $$
    \item[(ii)] Soient $X$ un topos, $T$ et $T'$ deux objets de $X$ et $i: T \to T'$ un morphisme quasicompact. Alors le foncteur $i_!$ induit un foncteur
    $$
    \D^-_0(T, A) \to \D^-_0(T', A)
    $$
    $$
    (\text{resp}.~\D^-_0(T, A)_{\torf} \to \D^-_0(T', A)_{\torf}).
    $$
    Si de plus $A$ est local régulier et $J$ est son idéal maximal, il induit un foncteur
    $$
    \D_0(T, A) \to \D_0(T', A).
    $$
    \item[(iii)] Soient $X$ un topos, $U$ un ouvert de $X, Y$ le topos fermé complémentaire et $j: Y \to X$ le morphisme de topos canonique. Alors, si $A$ est local régulier et $J$ est son idéal maximal, le foncteur $\bRd j^!$ induit un foncteur
    $$
    \D^+_0(X, A) \to \D^+_0(Y, A).
    $$
    \item[(iv)] Sous les hypothèses préliminaires de (iii), le foncteur $\bRd j_*$ induit des foncteurs
    $$
    \D^-_0(Y, A) \to \D^-_0(X, A)
    $$
    $$
    \D^-_0(Y, A)_{\torf} \to \D^-_0(X, A)_{\torf}.
    $$
    Si de plus l'anneau $A$ est local régulier et $J$ est son idéal maximal, le foncteur $\bRd j_*$ induit un foncteur
    $$
    \D_0(Y, A) \to \D_0(X, A).
    $$
\end{itemize}
}
\vskip .3cm
{\bf Preuve} : Montrons par exemple (iii), les autres assertions se voyant de fa\c{c}on analogue. Étant données un entier $n \geq 0$ et $u: A \to A_n$ le morphisme d'anneaux canonique, il s'agit, d'après (8.1.16.(iii)), de prouver que $\bRd j^! (\bLd u^*(E)) \in \D^+_0(Y, A_n)$. Par hypothèse et d'après (8.2.6), il existe $F \in \D^+(A_n-\Mod_X)$ tel que $\bLd u^*(E) \isom \omega (F)$. On aura donc terminé si on prouve la commutativité du diagramme  
\[\begin{tikzcd}
	{\D^+(A_n-\Mod_X)} && {\D^+(X, A_n)} \\
	{\D^+(A_n-\Mod_Y)} && {\D^+(Y, A_n).}
	\arrow["\omega", from=1-1, to=1-3]
	\arrow["\omega", from=2-1, to=2-3]
	\arrow["{\bRd j^!}", from=1-3, to=2-3]
	\arrow["{\bRd j^!}"', from=1-1, to=2-1]
\end{tikzcd}\]
Or cela résulte immédiatement de ce que, comme on l'a déjà remarqué, un $A_n$--Module flasque définit un $A_n$-faisceau isomorphe à un $A_n$-faisceau flasque.



\vskip .3cm
{\bf 8.3. Parties multiplicatives}.
\vskip .3cm
{
Définition {\bf 8.3.1}. --- \it Étant donnés un topos $X$, un anneau $A$, un idéal $J$ de $A$ et une partie multiplicative $S$ de $A$, on appelle catégorie des $(A, J, S)$-faisceaux sur $X$ (ou, s'il n'y a pas de confusion possible, des $S^{-1}A$-faisceaux sur $X$), en on note $(A, J, S)-\fsc(X)$, ou plus simplement
$$
S^{-1}A-\fsc(X)
$$
la catégorie ainsi définie :
\begin{itemize}
    \item Ses objets sont les $A$-faisceaux. 
    \item Si $E$ et $F$ sont deux $A$-faisceau, un morphisme de $S^{-1}A$-faisceaux de $E$ dans $F$ est un élément de
    $$
    \Hom_{S^{-1} A}(E, F) = S^{-1} \Hom_A (E, F).
    $$
\end{itemize}
}
\vskip .3cm
On vérifie facilement (cf. SGA5 VI 1.4.3.) que la catégorie $S^{-1}A-\fsc(X)$ est abélienne, et s'identifie de fa\c{c}on plus précise à la catégorie abélienne quotient de $A-\fsc(X)$ par la sous-catégorie abélienne épaisse engendrée par les $A$-faisceaux annulés par un élément de $S$.

Soient maintenant $B$ un autre anneau commutatif unifère, muni d'un idéal $K$, et $u: A \to B$ un morphisme d'anneaux vérifiant $u(J) \subset K$. Munissant $B$ de la partie multiplicative définie par $S$, il est clair que le foncteur $u_*: B-\fsc(X) \to A-\fsc(X)$ transforme un $B$-faisceau annulé par un élément de $S$ en un $A$-faisceau vérifiant la même propriété. Par suite, il induit par passage au quotient un foncteur exact, noté de même, 
$$
u_*: S^{-1} B-\fsc(X) \to S^{-1}A-\fsc(X),
$$
qu'on peut également définir directement grâce à la forme explicite des morphismes donnée en (8.3.1). Bien entendu, le diagramme
\[\begin{tikzcd}
	{B-\fsc(X)} && {A-\fsc(X)} \\
	{S^{-1}B-\fsc(X)} && {S^{-1}A-\fsc(X)}
	\arrow[from=1-1, to=2-1]
	\arrow[from=1-3, to=2-3]
	\arrow["{u_*}", from=2-1, to=2-3]
	\arrow["{u_*}", from=1-1, to=1-3]
\end{tikzcd}\leqno{(8.3.2)}\]
dans lequel les flèches verticales sont les morphismes de projection canoniques, est commutatif.
\vskip .3cm
{
Définition {\bf 8.3.3}. --- \it Supposons donnés $(X, A, J, S)$ comme plus haut. Avec les mêmes conventions simplificatrices que précédemment, on note
$$
\D^*(X, S^{-1}A) \quad (* = +, -, b, \emptyset)
$$
la catégorie ainsi définie :
\begin{itemize}
    \item Ses objets sont ceux de $\D^*(X, A)$.  
    \item Si $E$ et $F \in \D^*(X, A)$, $\Hom_{S^{-1} A}(E, F) = S^{-1} \Hom_A (E, F)$.
\end{itemize}
}
\vskip .3cm
Il est évident que la catégorie $\D^*(X, S^{-1}A)$ est additive, et que le foncteur translation de $\D^*(X, A)$ définit un foncteur de même type dans $\D^*(X, S^{-1} A)$, appelé également foncteur translation. De plus, la définition des morphismes de $\D^*(X, S^{-1}A)$ fournit de fa\c{c}on évidente un foncteur additif
$$
\D^*(X, A) \to D^*(X, S^{-1}A),
\leqno{(8.3.4)}
$$
qui induit l'identité sur les objets. Munissant alors $\D^*(X, S^{-1}A)$ des triangles qui sont isomorphes à l'image par (8.3.4) d'un triangle de $\D^*(X, A)$, et du foncteur translation précédemment défini, on constate aisément que l'on obtient ainsi une catégorie triangulée telle que le foncteur (8.3.4) soit exact. Plus précisément, on vérifie que $\D^*(X, S^{-1}A)$ est la catégorie triangulée obtenue à partir de $D^*(X, A)$ en inversant les homothéties définies par des éléments de $S$.

Ces définitions étant ainsi posées, on laisse au lecteur le soin de s'assurer que le formalisme développé dans les numéros précédents pour les $A$-faisceaux s'étend, mutatis mutandis, aux $S^{-1}A$-faisceaux.  








% End

%Begin


%%%%%%%%%%%%%%%%%%%%%%%%%%%%%%%%%%%%%%%%%%%%%%%%%%%%%%%%%%%%%%%
\chapter*{\S \space II. --- CONDITIONS DE FINITUDE}
\addcontentsline{toc}{section}{II. Conditions de finitude}
\label{ch:2}
\section*{}

Dans tout ce chapitre, on fixe un anneau commutatif unifère \emph{noethérien} $A$ et un idéal $J$ de $A$. Sauf mention expresse du contraire, tous les topos considérés seront supposés \emph{localement noethériens} (SGA 4 VI 2.11.).

%%%%%%%%%%%%%%%%%%%%%%%%%%%%%%%%%%%%
\subsection*{1. Catégorie des $A$-faisceaux constructibles.}
\addcontentsline{toc}{subsection}{1. Catégorie des $A$-faisceaux constructibles}

Soit $X$ un topos localement noethérien.
\vskip .3cm
{
Définition {\bf 1.1}. --- \it On dit qu'un $A$-faisceau $F = (F_n)_{n \in \mathbf{N}}$ est $J$-\emph{adique constructible} s'il est $J$-adique (I 3.8.) et si pour tout $n \in \mathbf{N}$, le $A_n$--Module $F_n$ est constructible. On dit que $F$ est un $A$-\emph{faisceau constructible} s'il est isomorphe dans $A-\fsc(X)$ à un $A$-faisceau $J$-adique constructible. On appelle catégorie des $A$-faisceaux constructibles et on note
$$
A-\fscn(X) \quad \text{(``n'' pour ``noethérien'')}
$$
la sous-catégorie pleine de $A-\fsc(X)$ engendrée par les $A$-faisceaux constructibles.
}
\vskip .3cm
{
Proposition {\bf 1.2}. --- \it Soit $F = (F_n)_{n \in \mathbf{N}}$ un $A$-faisceau sur $X$. Les assertions suivantes sont équivalentes.
\begin{enumerate}
    \item[(i)] $F$ est un $A$-faisceau constructible.
    \item[(ii)] $F$ est de type strict (I 3.2.) et, notant $F'$ le $A$-faisceau strict associé à $F$ (I 3.3.), il existe localement une application croissante $\gamma \geq \id$ telle que $\chi_\gamma(F')$ (I 2.2) soit $J$-adique constructible.
    \item[(iii)] Pour tout entier $r \geq 0$, le $A$-faisceau $F \otimes_A A_r$ est de type constant (I 3.6.) associé à un $A_r$--Module constructible.
\end{enumerate}
}
\vskip .3cm
{\bf Preuve}~: Si $F$ vérifie (i), il résulte de (I 3.9.3. (i) $\Rightarrow$ (ii)) qu'il existe localement une telle application $\gamma$, avec $\chi_\gamma (F')$ $J$-adique. Mais $\chi_\gamma(F') \isom F$, donc $\chi_\gamma(F')$ est isomorphe à un $A$-faisceau $J$-adique constructible, d'où (ii) grâce à (I 3.9.1). L'assertion (ii) $\Rightarrow$ (iii) résulte aussitôt de ce que $F \otimes_A A_r \isom \chi_\gamma (F') \otimes_A A_r$. Pour voir que (iii) $\Rightarrow$ (i), on peut supposer que l'objet final de $X$ est quasicompact, et alors cela se voit comme l'assertion analogue de (I 3.9.3.).
\vskip .3cm
{
Corollaire {\bf 1.3}. --- \it Si $F = (F_n)_{n \in \mathbf{N}}$ est un $A$-faisceau strict et constructible, alors pour tout $n \in \mathbf{N}$, le $A_n$--Module $F_n$ est constructible.
}
\vskip .3cm
{\bf Preuve} : D'après (1.2.(i)), il existe localement une application croissante $\gamma \geq \id$ telle que $\chi_\gamma(F)$ soit $J$-adique constructible, donc à composants constructibles. L'assertion résulte alors de ce que le morphisme canonique de $\mathcal{E}(X, J): \chi_\gamma (F) \to F$ est un épimorphisme.
\vskip .3cm
{
Corollaire {\bf 1.4}. --- \it Pour qu'un $A$-faisceau annulé par une puissance de l'idéal $J$ soit constructible, il faut et il suffit qu'il soit de type constant associé à un $A$--Module constructible.
}
\vskip .3cm
{
Proposition {\bf 1.5}. --- \it 
\begin{itemize}
    \item[(i)] La propriété pour un $A$-faisceau d'être constructible est stable par restriction à un objet du topos, de nature locale, et la catégorie fibrée
    $$
    T \mapsto A-\fscn(T),
    $$
    où $T$ parcourt les objets de $X$, est un \emph{champ}.
    \item[(ii)] Notant $J-\adn(X)$ la sous-catégorie pleine de $\mathscr{E}(X, J)$ engendrée par les $A$-faisceaux $J$-adiques constructibles, le foncteur canonique
    $$
    J-\adn(X) \to A-\fscn(X)
    $$
    induit par (I 3.8.2) est une \emph{équivalence de catégories}.
    \item[(iii)] La catégorie $A-\fscn(X)$ est une sous-catégorie \emph{exacte} (i.e. stable par noyaux, conoyaux et extensions) de $A-\fsc(X)$. De plus, lorsque $X$ est noethérien, les objets de $A-\fscn(X)$ sont \emph{noethériens} (dans $A-\fscn(X)$).
\end{itemize}
}
\vskip .3cm
{\bf Preuve} : L'assertion (ii) est conséquence immédiate de (I 3.9.1.). Montrons (i). Le caractère local résulte par exemple de (1.2. (i) $\Leftrightarrow$ (ii)) et du caractère local de la propriété pour un $A$-faisceau d'être de type strict. Quant à la propriété de champ, elle provient de (ii) et de la propriété analogue, évidente, pour la catégorie fibrée $T \mapsto J-\adn(T)$. Montrons (iii). Pour voir la stabilité par noyaux et conoyaux, on se ramène grâce à (ii) au cas d'un morphisme $u: E \to F$ de $\mathcal{E}(X, J)$, avec $E$ et $F$ des $A$-faisceaux $J$-adiques constructibles, et alors l'assertion résulte, en se ramenant localement au cas où $X$ est noethérien, de (SGA5 V 5.2.1.). Pour montrer la stabilité par extensions, nous utiliserons le lemme suivant.
\vskip .3cm
{
Lemme {\bf 1.6}. --- \it Pour tout $A$-faisceau constructible $E$ et tout entier $n \geq 0$, le $A$-faisceau ${\cTor}^A_1(A_n, E)$ est de type constant, associé a un $A_n$--Module constructible.
}
\vskip .3cm
Il suffit de voir (1.4) qu'il est constructible. Or, notant $u: J^{n+1} \to A$ le morphisme de $A$-faisceaux canonique, on a un isomorphisme
$$
\cTor^A_1(A_n, E) \isom \Ker(u \otimes_A \id_E),
$$
d'où le lemme, car $J^{n+1} \otimes_A E$ est constructible, comme on voit aisément en se ramenant au cas où $E$ est $J$-adique constructible. 

Montrons comment le lemme entraîne la stabilité par extensions. Soit donc
$$
0 \to E \to F \to G \to 0
$$
une suite exacte de $A-\fsc(X)$, avec $E$ et $G$ constructibles, et montrons que $F$ l'est également. Pour cela, il suffit (1.2) de voir que pour tout entier $r \geq 0$, le $A$-faisceau $F \otimes_A A_r$ est de type constant associé à un $A_r$--Module constructible. Or on a une suite exacte
$$
\cTor^A_1 (A_r, G) \to A_r \otimes_A E \to A_r \otimes_A F \to A_r \otimes_A G \to 0,
$$
dans laquelle tous les termes, excepté éventuellement $A_r \otimes_A F$, sont de type constant et constructibles. Compte tenu du fait qu'un $A_r$--Module qui est extension de $A_r$--Modules constructibles est lui-même constructible, l'assertion résulte alors de (I 3.6). Il nous reste enfin à voir que la catégorie $A-\fscn(X)$ est noethérien lorsque $X$ est noethérien. Il suffit de le voir pour la catégorie équivalente $J-\adn(X)$, ce qui n'est autre que (SGA V 5.2.3.).
\vskip .3cm
{
Corollaire {\bf 1.6}. --- \it Notant $J-\Modn(X)$ la sous-catégorie abélienne épaisse de $A-\Mod_X$ engendrée par les $A$--Modules constructibles et localement annulés par une puissance de $J$, le foncteur
$$
J-\Modn(X) \to A-\fsc(X)
$$
induit par (I 3.5.1.) définit une équivalence avec la sous-catégorie abélienne épaisse de $A-\fsc(X)$ engendrée par les $A$-faisceaux de type constant et constructibles.
}
\vskip .3cm
On aura remarqué que dans l'énoncé (1.5), on a pris soin de préciser que les $A$-faisceaux constructibles sont noethériens \emph{dans $A-\fscn(X)$}. On pourrait croire qu'ils le sont aussi dans $A-\fsc(X)$.

Nous allons voir plus loin qu'il n'en est rien, mais donnons tout d'abord un cas où cette assertion est vraie.
\vskip .3cm
{
Proposition {\bf 1.7}. --- \it On suppose que l'idéal $J$ est maximal. Alors les assertions suivantes sont équivalentes pour un objet $F$ de $A-\fsc(\pt)$.
\begin{itemize}
    \item[(i)] $F$ est constructible.
    \item[(ii)] $F$ est noethérien.
\end{itemize}
}
\vskip .3cm
{\bf Preuve} : Montrons d'abord que (ii) $\Rightarrow$ (i). Comme la catégorie $A-\fscn(\pt)$ est noethérienne (1.5.(iii)), il suffit de montrer que tout sous-$A$-faisceau $E$ de $F$ est constructible. On se ramène immédiatement pour le voir au cas où $F$ est $J$-adique constructible et $E$ est un sous-système projectif de $E$. Mais alors les composants de $F$, donc aussi ceux de $E$, sont des $A$--modules artiniens, de sorte que $E$ vérifie la condition de Mittag-Leffler. Dans ces conditions, l'assertion résulte du lemme suivant, valable sans hypothèse spéciale sur le topos $X$ et le couple $(A, J)$, autre que celles de l'introduction.
\vskip .3cm
{
Lemme {\bf 1.8}. --- \it Soit $u: E \to F$ un monomorphisme de $A$-faisceaux. On suppose que $F$ est constructible et que $E$ est de type strict. Alors $E$ est constructible.
}
\vskip .3cm
On se ramène pour le voir au cas où $u$ est un monomorphisme de $\mathcal{E}(X, J)$ et $F$ est $J$-adique constructible, puis, quitte à remplacer $E$ par le $A$-faisceau strict associé, au cas où $E$ est strict. Alors le $A$-faisceau $G = \Coker(u)$ est $J$-adique (cf. le preuve de SGA5 V 3.2.4. (i)), et constructible, car ses composants sont des quotients des composants de $F$. Mais alors $E$, noyau du morphisme canonique $F \to G$, est constructible par (1.5.(iii)).

Montrons maintenant l'assertion (i) $\Rightarrow$ (ii) de la proposition.
\vskip .3cm
{
Lemme {\bf 1.9}. --- \it Un objet noethérien $F$ de $A-\fsc(\pt)$ est de type strict (i.e. vérifie la condition de Mittag-Leffler).
}
\vskip .3cm
Il est clair qu'il suffit de montrer la même assertion pour les $A$-faisceaux $F \otimes_A A_r$, qui sont également noethériens, de sorte que l'on est ramené au cas où $F$ est annulé par une puissance de $J$. Puis, utilisant la filtration (finie) de $F$ définie par les puissances de l'idéal $J$, on se ramène au cas où $F$ est annulé par $J$, et enfin au cas où $F$ est un système projectif de $(A/J)$-espaces vectoriels. Soit $n_0 \geq 0$ un entier, et montrons que la suite décroissante 
$$
\text{Im}(F_n \to F_{n_0})_{n \geq n_0}
$$
de $(A/J)$-espaces vectoriels est stationnaire. Posant 
$$
K_n = 
\begin{cases}
0 \quad (n < n_0) \\
\text{Im}(F_n \to F_{n_0}) \quad (n \geq n_0),
\end{cases}
$$
on définit, avec les morphismes de transition évidents, un $A$-faisceau quotient de $F$, donc noethérien. On est finalement ramené à voir qu'un $A$-faisceau noethérien $(V_p)_{p \in \mathbf{N}}$, dont les composants sont des $(A/J)$-espaces vectoriels et les morphismes de transition sont des monomorphismes, est essentiellement constant. Raisonnons par l'absurde, et supposons qu'il existe une infinité (dénombrable) de $V_p$ distincts. Quitte à renuméroter, on peut supposer que $V_i \neq V_j$ si $i \neq j$. Désignons alors, pour tout $i \geq 0$, par $X_i$ un supplémentaire de $V_{i + 1}$ dans $V_i$, et choisissons un élément non nul $e_i$ de $x_i$. On définit un sous-$A$-faisceau $W$ de $V = (V_p)_{p \in \mathbf{N}}$ en posant
$$
W_p = \bigoplus^\infty_{i = p} k e_i \quad (k = A/J),
$$
avec les morphismes de transitions évidents. Nous allons voir que le $A$-faisceau ainsi défini n'est pas noethérien, ce qui donnera la contradiction annoncée. Pour tout entier $p \geq 0$, notons $M_p$ le sous-espace vectoriel de $W_0$ ayant pour base les éléments
$$
e_{2^n - r} \quad (n, r \geq 0,\;\; 0 \leq r \leq p).
$$
Il est immédiat que pour tout couple $(p, q)$ d'entiers positifs distincts, on a $M_p \cap W_i \neq M_q \cap W_i$ pour tout entier $i \geq 0$. Considérant alors pour tout entier $p \geq 0$ le système projectif, noté $(M_p \cap W)_\cdot$ défini par
$$
(M_p \cap W)_n = M_p \cap W_n \quad (n \geq 0)
$$
avec les morphismes de transition évidents, on a une suite croissante
$$
(M_0 \cap W)_\cdot \subset (M_1 \cap W)_\cdot \subset \dots \subset (M_p \cap W)_\cdot \subset \dots 
$$
de sous-$A$-faisceaux de $W$. Le lemme suivant entraîne qu'elle n'est pas stationnaire.
\vskip .3cm
{
Lemme {\bf 1.10}. --- \it Soit $V = (V_n)_{n \in \mathbf{N}}$ un système projectif d'objets d'une catégorie abélienne $C$, dont les morphismes de transition sont des monomorphismes. Pour tout sous-objet $X$ de $V_0$, on pose
$$
(X \cap V)_\cdot = (X \cap V_n)_{n \in \mathbf{N}}.
$$
Si $L$ et $M$ sont deux sous-objets de $V_{0}$, avec $L \subset M$, les assertions suivantes sont équivalentes.
\begin{itemize}
    \item[(i)] $(M \cap W)_\cdot/(L \cap W)_\cdot$ est essentiellement nul. 
    \item[(ii)] Il existe un entier $p \geq 0$ tel que $M \cap V_p = L \cap V_p$.
\end{itemize}
Lorsque (i) et (ii) sont satisfaites, on a $M \cap V_q = L \cap V_q$ pour $q$ assez grand.
}
\vskip .3cm
Il est clair que (ii) $\Rightarrow$ (i). Inversement, si (i) est vérifiée, il existe un entier $p \geq 0$ tel que le morphisme canonique
$$
(M \cap V_p)/(L \cap V_p) \to M/L
$$
soit nul. Autrement dit , $M \cap V_p \subset L$, d'où $M \cap V_p \subset L \cap V_p$. L'inclusion en sens opposée étant évidente, l'assertion en résulte. 

Achevons la preuve de l'assertion (ii) $\Rightarrow$ (i) de (1.7). Il s'agit de voir (1.2) que pour tout entier $r \geq 0$, le $A$-faisceau $F \otimes_A A_r$ est de type constant associé à un $A_r$--Module constructible, ce qui permet de se ramener au cas où $F$ est annulé par une puissance de $J$. Utilisant la filtration de $F$ définie par les puissances de $J$, on peut même supposer que $F$ est annulé par $J$. Finalement, utilisant (1.9), on a à montrer que si un système projectif strict de $(A/J)$-espaces vectoriels est noethérien en tant que $A$-faisceau, alors il est essentiellement constant et sa limite projective est un $(A/J)$-espace vectoriel de dimension finie. Cela se voit immédiatement par l'absurde, et est laissé en exercice au lecteur.

J'ignore s'il est toujours vrai qu'un $A$-faisceau noethérien est constructible. Par contre, la proposition suivante, intéressante en soi, montre qu'en général un $A$-faisceau constructible n'est pas noethérien.
\vskip .3cm
{
Proposition {\bf 1.11}. --- \it On suppose que $X$ soit le topos étale d'un schéma de Jacobson noté de même. Soit $F$ un $A$-faisceau sur $X$. Les assertions suivantes sont équivalentes:
\begin{itemize}
    \item[(i)] $F$ est noethérien et constructible.
    \item[(ii)] Il existe un nombre fini de points fermés $(x_i,\dots, x_d)$ de $X$ et pour tout $i \in [1, d]$ un $A$-faisceau $F_i$ noethérien et constructible sur le topos ponctuel tels que, notant $j_{x_i}: x_i \to X$ les immersions canoniques, on ait 
    $$
    F \isom \bigoplus_{1 \leq i \leq d} (j_{x_i})_*(F_i).
    $$
\end{itemize}
}
\vskip .3cm
{\bf Preuve} : Pour voir que (ii) $\Rightarrow$ (i), il suffit de voir que les $A$-faisceaux $(j_{x_i})_*(F_i)$ sont constructibles et noethériens. Le caractère constructible se voit en se ramenant au cas où $F_i$ est $J$-adique constructible, en utilisant l'exactitude du foncteur $(j_{x_i})_*$.

Le caractère noethérien résulte immédiatement de l'adjonction naturelle entre les foncteurs $(j_{x_i})^*$ et $(j_{x_i})_*$. Montrons que (i) $\Rightarrow$ (ii). On peut supposer que $F = (F_n)_{n \in \mathbf{N}}$ est $J$-adique constructible, et nous allons alors voir qu'il n'existe qu'un nombre fini $(x_1, \dots, x_d)$ de points fermés de $X$ tels que $(F_0)_{x_i} \neq 0$. Notons pour tout point fermé $x$ de $X$ par $j_x: X \to X$ l'immersion fermée canonique. Pour tout famille finie $Y = (x_1, \dots, x_m)$ de points fermés de $X$, l'inclusion $Y \to X$ définit un épimorphisme canonique
$$
F_0 \to \bigoplus_{1 \leq i \leq m} (j_{x_i})_* (j_{x_i})^*(F_0).
$$
Supposons alors qu'il existe une infinité dénombrable $(x_i)_{i \in \mathbf{N}}$ de points fermés de $X$, avec $F_{x_i} \neq 0$. Définissant un $A$-faisceau $G$ par
$$
G_n = \bigoplus_{0 \leq p \leq n} (j_{x_p})_*(j_{x_p})^*(F_0),
$$
avec les morphismes de transition évidents (identité sur les termes communs et $0$ ailleurs), on a un épimorphisme
$$
\overline{F_0} \to G \to 0
$$
du $A$-faisceau constant $\overline{F}_0$ défini par $F_0$ sur $G$. Il s'ensuit que $G$ est un quotient de $F$, donc est noethérien. On obtient une contradiction en définissant une suite croissante non stationnaire $(F^q G)_{q \in \mathbf{N}}$ de sous-$A$-faisceaux de $G$. Pour cela, on pose
$$
(F^q G)_n = 
\begin{cases}
    G_n \quad (n \leq q) \\
    \bigoplus_{0 \leq i \leq q} (j_{x_i})_*(j_{x_i})^*(F_0) \quad (n > q),
\end{cases}
$$
avec les morphismes de transition évidents. Ceci dit, soient donc $(x_1, \dots, x_d)$ les seuls points fermés du support de $F_0$, et $U$ l'ouvert complémentaire de leur réunion. Nous allons voir que $F|U = 0$. Il en résultera, d'après la suite exacte (I 4.6.4.(i)), que 
$$
F \isom \bigoplus_{1 \leq i \leq d} (j_{x_i})_*(j_{x_i})^*(F),
$$
de sorte qu'il suffira de prouver que pour tout $i \in (1, d)$ le $A$-faisceau $(j_{x_i})^*(F)$ est constructible et noethérien. Qu'il soit constructible est évident; comme $F$ est noethérien, son facteur direct $(j_{x_i})_*(j_{x_i})^*(F)$ l'est également, de sorte que le caractère noethérien de $(j_{x_i})^*(F)$ se voit en utilisant l'adjonction naturelle entre $(j_{x_i})_*$ et $(j_{x_i})^*$. Montrons donc que $F | U = 0$. Comme $F/JF \isom \overline{F}_ 0$, il suffit, d'après le lemme de Nakayama (I 5.12.) de voir que $F_0 | U = 0$. En effet, comme $X$ est un schéma de Jacobson, il existerait sinon (SGA4 VIII 3.13.) un point fermé $x$ de $X$ contenu dans $U$ tel que $(j_x)^*(F_0) \neq 0$.

\vskip .3cm
{\bf 1.12}. Notant $\hat{A}$ le complété de $A$ pour la topologie $J$-adique, on définit un foncteur exact et pleinement fidèle (EGA $0_I$ 7.8.2)
$$
\hat{A}-\modn \to A-\fsc(\pt)
\leqno{(1.12.1)}
$$
en associant à tout $\hat{A}$-module de type fini $M$ le système projectif
$$
(M/J^{n+1}M)_{n \in \mathbf{N}}.
$$
Ce foncteur se factorise de manière évidente en un foncteur
$$
\hat{A}-\modn \to A-\fscn(\pt).
\leqno{(1.12.2)}
$$
Il résulte aisément de (EGA $0_I$ 7.2.9. et 7.8.2.) que ce foncteur est une \emph{équivalence de catégories}, un foncteur quasi-inverse étant d'ailleurs fourni par la limite projective.

Si maintenant $a: \pt \to X$ est un point du topos $X$, il est clair que le foncteur fibre défini par $a$
$$
A-\fsc(X) \to A-\fsc(\pt)
$$
envoie $A-\fscn(X)$ dans $A-\fscn(\pt)$, d'où grâce à l'équivalence (1.12.2) un foncteur exact, appelé encore foncteur fibre associé à $a$,
$$
\mathscr{E}_a: A-\fscn(X) \to \hat{A}-\modn,
\leqno{(1.12.3)}
$$
qui est obtenu de fa\c{c}on précise en composant le foncteur fibre ordinaire et le foncteur limite projective $A-\fscn(\pt) \to \hat{A}-\modn$.

Rappelons enfin (SGA4 VI \quad) qu'un topos localement noethérien à suffisamment de points.
\vskip .3cm
{
Proposition {\bf 1.12.4}. --- \it La collection des foncteurs fibres
$$
\mathscr{E}_a: A-\fscn(X) \to \hat{A}-\modn,
$$
où a parcourt les points du topos $X$, est \emph{conservative}. En particulier, pour qu'une suite de $A$-faisceaux constructibles
$$
F' \xlongrightarrow{u} F \xlongrightarrow{v} F''
\leqno{(S)}
$$
soit exacte, il faut et il suffit que les suites correspondantes 
$$
\mathscr{E}_a(F') \xlongrightarrow{\mathscr{E}_a(u)} \mathscr{E}_a(F) \xlongrightarrow{\mathscr{E}_a(v)} \mathscr{E}_a(F'')
\leqno{(S_a)}
$$
soient exactes.
}
\vskip .3cm
{\bf Preuve} : On est ramené à voir l'assertion analogue pour les foncteurs fibres évidents $J-\adn(X) \to J-\adn(\pt)$, qui se voit composant par composant à partir de la conservativité des foncteurs fibres pour les $A$--Modules.
\vskip .3cm
{\bf 1.13}. Soit $f: X \to Y$ un morphisme de topos localement noethériens. Comme l'image réciproque d'un $A$--Module constructible est un $A$--Module constructible, le foncteur image réciproque
$$
f^*: \mathcal{E}(Y, J) \to \mathcal{E}(X, J)
$$
envoie évidemment $J-\adn(Y)$ dans $J-\adn(X)$. On en déduit aussitôt que le foncteur image réciproque (I 4.1.1.) envoie $A-\fscn(Y)$ dans $A-\fscn(X)$. D'où un diagramme commutatif
\[\begin{tikzcd}
	{J-\adn(Y)} && {A-\fscn(Y)} \\
	{J-\adn(X)} && {A-\fscn(X)}
	\arrow["{f^*}", from=1-3, to=2-3]
	\arrow["{f^*}"', from=1-1, to=2-1]
	\arrow["{\varphi_Y}", from=1-1, to=1-3]
	\arrow["{\varphi_X}"', from=2-1, to=2-3]
\end{tikzcd}\]
dans lequel $\varphi_X$ et $\varphi_Y$ désignent les équivalences canoniques.

Si $i: T \to T'$ est un morphisme quasi-compact entre objets d'un topos localement noethérien $X$, on voit de même que le foncteur $i_!$ induit un foncteur exact
$$
i_!: A-\fscn(T) \to A-\fscn(T').
\leqno{(1.13.1)}
$$
Enfin, étant donné un ouvert $U$ d'un topos localement noethérien et $Y$ le topos (également localement noethérien) fermé complémentaire de $U$, si on note $j: Y \to X$ le morphisme de topos canonique, le foncteur exact $j_*$ induit un foncteur exact
$$
j_*: A-\fscn(Y) \to A-\fscn(X).
\leqno{(1.13.2)}
$$
\vskip .3cm
{
Définition {\bf 1.14}. --- \it On dit qu'un $A$-faisceau $J$-adique constructible (1.1) sur $X$ $F = (F_n)_{n \in \mathbf{N}}$ est \emph{constant tordu} (resp. par abus de langage, \emph{localement libre}), si pour tout entier $n \geq 0$, le $A_n$--Module $F_n$ est localement constant (resp. localement libre). On dit qu'un $A$-faisceau sur $X$ est \emph{constant tordu constructible} (resp. \emph{localement libre constructible}) s'il est isomorphe dans $A-\fsc(X)$ à un $A$-faisceau $J$-adique constructible constant tordu (resp. localement libre). On note
$$
A-\fsct(X)
$$
la sous-catégorie pleine de $A-\fscn(X)$, donc aussi de $A-\fsc(X)$, engendrée par les $A$-faisceaux constant tordus constructibles.
}
\vskip .3cm
Nous allons maintenant énoncer pour les $A$-faisceaux constants tordus constructibles (resp. localement libres constructibles) un certain nombre de résultats analogues à des assertions déjà données pour les $A$-faisceaux constructibles. Nous ne donnerons pratiquement pas de démonstrations, et signalerons surtout les points possibles de divergence.  
\vskip .3cm
{
Proposition {\bf 1.15}. --- \it Soit $F$ un $A$-faisceau sur $X$. Les assertions suivantes sont équivalentes.
\begin{itemize}
    \item[(i)] $F$ est un $A$-faisceau constant tordu constructible (resp. localement libre constructible). 
    \item[(ii)] $F$ est de type strict et, notant $F'$ le $A$-faisceau strict associé à $F$, il existe localement une application croissante $\gamma \geq \id: \mathbf{N} \to \mathbf{N}$ telle que $\chi_\gamma(F')$ soit $J$-adique constructible constant tordu (resp. localement libre).
    \item[(iii)] Pour tout entier $r \geq 0$, le $A$-faisceau $F \otimes_A A_r$ est de type fini constant associé à un $A_r$--Module localement constant constructible (resp. localement libre constructible).
\end{itemize}
}
\vskip .3cm
{
Corollaire {\bf 1.16}. --- \it Soit $F = (F_n)_{n \in \mathbf{N}}$ un $A$-faisceau de type $J$-adique (par exemple, constructible). On suppose que pour tout $n \in \mathbf{N}$, le $A_n$--Module $F_n$ est localement constant constructible. Alors, $F$ est constant tordu constructible.
}
\vskip .3cm
{\bf Preuve} : Comme la catégorie des $A$--Modules localement constants constructibles est stable par images dans $A-\Mod_X$, on peut, quitte à remplacer $F$ par le système projectif strict associé, supposer que $F$ est strict. Par hypothèse, il existe alors (I 3.11) localement une application croissante $\gamma \geq \id: \mathbf{N} \to \mathbf{N}$ telle que $\chi_\gamma(F)$ soit $J$-adique. Mais l'hypothèse sur $F$ entraîne que les composants de $\chi_\gamma(F)$ sont localement constants constructibles, d'où l'assertion.
\vskip .3cm
{
Proposition {\bf 1.17}. --- \it 
\begin{itemize}
    \item[(i)] La propriété pour un $A$-faisceau d'être constant tordu constructible (resp. localement libre constructible) est stable par restriction à un objet du topos et de nature locale. La catégorie fibrée
    $$
    T \mapsto A-\fsct(T),
    $$
    où $T$ parcourt les objets de $X$, est un \emph{champ}.
    \item[(ii)] Notant $J-\adt(X)$ la sous-catégorie pleine de $\mathcal{E}(X, J)$ engendrée par les $A$-faisceaux $J$-adiques constants tordus constructibles, le foncteur canonique
    $$
    J-\adt(X) \to A-\fsct(X)
    $$
    induit par (I 3.8.2) est une équivalence de catégories.
    \item[(iii)] La catégories $A-\fsct(X)$ est une sous-catégorie \emph{exacte} de $A-\fsc(X)$. De plus, lorsque $X$ n'a qu'un nombre fini de composantes connexes (par exemple, est noethérien), les objets de $A-\fsc(X)$ sont noethériens (dans $A-\fsct(X)$).
\end{itemize}
}
\vskip .3cm
{\bf Preuve} : Seule l'assertion (iii) mérite quelque attention. Pour la stabilité par noyaux et conoyaux, on se ramène au cas d'un morphisme $u: E \to F$ de $\mathcal{E}(X, J)$. Les systèmes projectifs $\Ker(u)$ et $\Coker(u)$ ont des composants localement constants constructibles, et sont constructibles (1.5.(iii)), de sorte que l'assertion résulte de (1.16). La stabilité se voit comme l'assertion analogue de (1.5), en utilisant le fait (cf.1.6.) que pour tout $A$-faisceau constant tordu constructible $E$ et tout entier $n \geq 0$, le $A$-faisceau $\cTor^A_1(A_n, E)$ est de type constant, associé à un $A_n$--Module localement constant constructible. D'après (1.6), il suffit pour cela de voir qu'il est constant tordu, ce qui, vu que ses composants sont localement constants, résulte une nouvelle fois de (1.16). Pour la dernière assertion, rappelons (SGA4 VI \quad) que les composantes connexes d'un topos localement noethérien sont, par définition, les ouverts connexes maximaux du topos. On peut supposer que $X$ est connexe. Soit donc $(E^n)_{n \in \mathbf{N}}$ une suite croissante de sous-$A$-faisceaux constants tordus constructibles d'un $A$-faisceau constant tordu constructible $E$, et montrons qu'elle est stationnaire. Supposons $X$ non vide, et choisissons un ouvert noethérien non vide $U$ de $X$. Par (1.5.(iii)), la suite des $E^n|U$ est stationnaire; il existe donc un entier $q$ tel que $E^p|U = E^q|U$ pour $p \geq q$, ou encore $(E^p/E^q)|U = 0$. On est donc ramené à voir que si un $A$-faisceau $J$-adique constant tordu constructible est nul au-dessus d'un ouvert non vide d'un topos localement noethérien connexe, il est nul. Cela résulte immédiatement de l'assertion analogue pour les $A$--Modules, appliqués à ses composants.
\vskip .3cm
{
Corollaire {\bf 1.18}. --- \it Notant $J-\Modt(X)$ la sous-catégorie abélienne épaisse de $A-\Mod_X$ engendrée par les $A$--Modules localement constants constructibles et annulés par une puissance de $J$, le foncteur
$$
J-\Modt(X) \to A-\fsc(X)
$$
induit par (I 3.5.1.) définit une équivalence avec la sous-catégorie abélienne épaisse de $A-\fsc(X)$ engendrée par les $A$-faisceaux de type constant et constants tordus constructibles.
}
\vskip .3cm
{
Proposition {\bf 1.19}. --- \it 
\begin{itemize}
    \item[(i)] Soit $0 \to L' \xlongrightarrow{u} L \xlongrightarrow{v} L'' \to 0$ une suite exacte de $\mathcal{E}(X, J)$. Si $L$ et $L''$ (resp. $L'$ et $L''$) sont $J$-adiques localement libres constructibles, il en est de même de $L'$ (resp. $L$).
    \item[(ii)] Soit $0 \to L' \xlongrightarrow{u} L \xlongrightarrow{v} L'' \to 0$ une suite exacte de $A-\fsc(X)$. Si $L$ et $L''$ (resp. $L'$ et $L''$) sont des $A$-faisceaux localement libres constructibles, il en est de même de $L'$ (resp. $L$).
\end{itemize}
}
\vskip .3cm
{\bf Preuve} : Montrons (i). Comme il est clair que les composants de $L'$ (resp. $L$) sont localement libres constructibles, on a seulement à voir que $L'$ (resp. $L$) est $J$-adique. Dans le cas respé, cela résulte de (SGA5 V 3.1.3.(iii)). Dans le cas non respé, on a pour tout entier $n \geq 0$ un diagramme commutatif exact
\[\begin{tikzcd}
	& {L'_{n+1}/J^{n+1}L'_{n+1}} && {L_{n+1}/J^{n+1}L_{n+1}} && {L''_{n+1}/J^{n+1}L''_{n+1}} & 0 \\
	0 & {L'_n} && {L_n} && {L''_n} & {0,}
	\arrow["{\mu}","{\sim}"', from=1-4, to=2-4]
	\arrow["\lambda", from=1-2, to=2-2]
	\arrow["{\overline{u}_{n+1}}", from=1-2, to=1-4]
	\arrow["{u_n}", from=2-2, to=2-4]
	\arrow[from=2-1, to=2-2]
	\arrow["{\overline{v}_{n+1}}", from=1-4, to=1-6]
	\arrow["{v_n}", from=2-4, to=2-6]
	\arrow["\nu", "{\sim}"', from=1-6, to=2-6]
	\arrow[from=2-6, to=2-7]
	\arrow[from=1-6, to=1-7]
\end{tikzcd}\]
dans lequel les flèches verticales sont déduites de fa\c{c}on évidente des morphismes de transition. Comme $L''_{n+1}$ est un $A_{n+1}$--Module localement libre, $\overline{u}_{n+1}$ est un monomorphisme, donc $\lambda$ est un isomorphisme d'où l'assertion. Montrons maintenant (ii), et tout d'abord l'assertion non respée. On peut supposer (1.17.(ii)) que $L$
 et $L''$ sont $J$-adiques localement libres constructibles et que $v$ est l'image d'un morphisme de $\mathcal{E}(X, J)$; alors l'assertion résulte de (i) non respée. Prouvons maintenant l'assertion respée. Comme elle est de nature locale (1.17.(i)), on peut supposer $X$ noethérien, et bien sûr $L''$ $J$-adique localement libre constructible. Alors il existe une application croissante $\gamma \geq \id: \mathbf{N} \to \mathbf{N}$ telle que $v$ e¿soit l'image d'un morphisme 
 $$
 \chi_\gamma(L) \to L''
 $$
 de $\mathcal{E}(X, J)$, qui, comme $v$ est un épimorphisme et $L''$ est strict, est un épimorphisme. On est ainsi ramené au cas où $X$ est noethérien, la suite exacte en question est l'image d'une suite exacte de $\mathcal{E}(X, J)$, et $L''$ est $J$-adique localement libre constructible. Comme $L'$ et $L''$ vérifient la condition de Mittag-Leffler, il en est de même de $L$ (EGA $0_{III}$ 13.2.1.); quitte à remplacer $L$ par le système projectif strict associé, on peut donc supposer $L$ strict. Alors (SGA5 V 3.1.3.) $L'$ est strict. Par suite, il existe une application croissante $\gamma\geq \id: \mathbf{N} \to \mathbf{N}$ telle que $\chi_\gamma(L')$ soit $J$-adique localement libre constructible. Comme les composants de $L''$ sont localement libres, la suite
 $$
 0 \to \chi_\gamma(L') \xlongrightarrow{\chi_\gamma(u)} \chi_\gamma(L) \xlongrightarrow{\chi_\gamma(v)} \chi_\gamma(L'') \to 0
 $$
 est exacte. On peut donc supposer que $L'$ et $L''$ sont tous les deux $J$-adiques localement libres constructibles, et alors l'assertion résulte de (i) respé.
\vskip .3cm
{\bf 1.20}. Nous allons maintenant expliciter la structure de la catégorie $A-\fsc(X)$, lorsque le topos $X$ est connexe. Rappelons tout d'abord quelques faits concernant le pro-groupe fondamental d'un topos. Étant donné un pro-groupe strict
$$
G = (G_i)_{i \in I},
$$
on définit comme suit un topos, noté
$$
\B_G
$$
et appelé \emph{topos classifiant} de $G$. Un objet de $\B_G$, appelé encore $G$-\emph{ensemble}, est un ensemble $M$ muni d'une application
$$
p: M \to \varinjlim_i \Hom(G_i, M)
$$
$$
\qquad\qquad\qquad\qquad m \mapsto (g_i \mapsto g_i m \quad \text{pour}~i~\text{``assez grand''})
$$
telle que l'on ait
$$
g_i (g'_i m) = (g_i g'_i)m \quad \text{pour}~i~\text{``assez grand''}.  
$$
Autrement dit, $M$ admet une filtration par des $G_i$-ensembles $(i \in I)$, avec compatibilité des diverses opérations. Un morphisme de $G$-ensembles $M \to N$ est une application $u: M \to N$ qui rend le diagramme
\[\begin{tikzcd}
	M && {\varinjlim_i \Hom(G_i, M)} \\
	N && {\varinjlim_i \Hom(G_i, N)}
	\arrow["u"', from=1-1, to=2-1]
	\arrow["p", from=1-1, to=1-3]
	\arrow["p", from=2-1, to=2-3]
	\arrow["{\varinjlim_i \Hom(\id, u)}", from=1-3, to=2-3]
\end{tikzcd}\]
commutatif.

De la même manière, étant donné un anneau $B$, on définit la notion de $(B, G)$-\emph{module}, en exigeant que l'application structurale
$$
p: M \to \varinjlim_i \Hom(G_i, M)
$$
soit $B$-linéaire, lorsque l'on munit le second membre de la structure de $B$--Module déduite de fa\c{c}on évidente de celle de $M$. Autrement dit, un $(B, G)$--Module n'est autre qu'un $B$--Module sur le topos $\B_G$.

Le topos $\B_G$ est localement noethérien (cf. SGA4 VI 1.33.) et n'admet (à isomorphisme près) qu'un seul point, à savoir le foncteur qui associe à tout $G$-ensemble $M$ l'ensemble sous-jacent.

Étant données maintenant un topos connexe $X$ (non nécessairement localement noethérien) et un point
$$
a: \pt \to X,
$$
on définit, à isomorphisme près dans la catégorie des pro-groupes, un pro-groupe strict 
$$
\pi_1(X, a),
$$
appelé pro-groupe fondamental de $X$ en $a$, et une équivalence de catégories
$$
\Elc(X) \xlongrightarrow{\approx} \B_{\pi_1(X, a)}
\leqno{(1.20.1)}
$$
de la catégorie des faisceaux d'ensembles localement constants sur $X$, avec le topos classifiant de $\pi_1(X, a)$. De plus, notant $c$ le point canonique du topos classifiant du pro-groupe fondamental, le diagramme 
\[\begin{tikzcd}
	{\Elc(X)} && {\B_{\pi_1(X, a)}} \\
	& \Ens
	\arrow["{a^*}"', from=1-1, to=2-2]
	\arrow["{c^*}", from=1-3, to=2-2]
	\arrow["{(1.20.1)}", from=1-1, to=1-3]
\end{tikzcd}\]
est commutatif (à isomorphisme près).

Étant donné un anneau commutatif unifère $B$, le foncteur (1.20.1) définit une équivalence
$$
B-\Modlc(X) \xlongrightarrow{\approx} \B-\Mod(\B_{\pi_1(X, a)}),
\leqno{(1.20.2)}
$$
où $B-\Modlc(X)$ désigne la catégorie des $B$--Modules localement constants sur $X$.

Si $f: X \to Y$ est un morphisme de topos, le morphisme composé $b = f \circ a$ est un point de $Y$, et on définit fonctoriellement en les données, un morphisme de pro-groupes
$$
\pi_1(f): \pi_1(X, a) \to \pi_1(Y, b)
$$
tel que le foncteur image réciproque
$$
f^*: \Elc(Y) \to \Elc(X)
$$
corresponde dans les équivalences (1.20.1) à la restriction du pro-groupe structural.

Soit $X$ un topos localement noethérien connexe, et choisissons un point $a: \pt \to X$ de $X$. Par simple extension aux systèmes projectifs, le foncteur (1.20.2) définit une équivalence
$$
J-\adt(X) \xlongrightarrow{\approx} J-\adt(\B_{\pi_1(X, a)}) = J-\adn(\B_{\pi_1(X, a)}).
\leqno{(1.20.3)}
$$
\vskip .3cm
{
Proposition {\bf 1.20.4}. --- \it 
\begin{itemize}
    \item[(i)] Soient $X$ un topos localement noethérien connexe et $a$ un point de $X$. On a une \emph{équivalence} canonique, définie à isomorphisme près,
    $$
    A-\fsct(X) \xlongrightarrow{\omega_X} A-\fsct(\B_{\pi_1(X, a)}) = A-\fscn(\B_{\pi_1(X, a)}).
    $$
    Le foncteur fibre défini par $a$ (1.12.3)
    $$
    \mathscr{E}_a: A-\fsct(X) \to \hat{A}-\modn 
    $$
    est \emph{conservatif}.
    \item[(ii)] Soient $X$ et $Y$ deux topos localement noethériens, et $f: X \to Y$ un morphisme. Pour tout $A$-faisceau constant tordu constructible $F$ sur $Y$, le $A$-faisceau $f^*(F)$ est constant tordu constructible. Supposons maintenant que $X$ et $Y$ soient connexes, choisissons un point $a$ de $X$ et posons $b = f \circ a$. Alors le diagramme
    \[\begin{tikzcd}
	{A-\fsct(Y)} && {A-\fsct(\B_{\pi_1(Y, b)})} \\
	{A-\fsct(X)} && {A-\fsct(\B_{\pi_1(X, a)})} & {,}
	\arrow["{f^*}"', from=1-1, to=2-1]
	\arrow["{\omega_Y}", from=1-1, to=1-3]
	\arrow["{\omega_X}"', from=2-1, to=2-3]
	\arrow["\Res", from=1-3, to=2-3]
    \end{tikzcd}\]
    dans lequel le foncteur $\Res$ désigne la restriction du pro-groupe structural, est commutatif à isomorphisme près.
\end{itemize}
}
\vskip .3cm
{\bf Preuve} : L'équivalence $\omega_X$ se déduit de fa\c{c}on évidente de (1.20.3), en utilisant l'équivalence (1.17.(ii)). Comme l'``unique'' foncteur fibre du topos $\B_{\pi_1(X, a)}$ est conservatif (1.12.4), la conservativité annoncée en résulte aussitôt. L'assertion (ii) est conséquence immédiate de l'assertion analogue pour les Modules localement constants, rappelée plus haut.
\vskip .3cm
{
Corollaire {\bf 1.20.5}. --- \it Soient $X$ un schéma localement noethérien connexe et $a$ un point géométrique de $X$. Notant encore $\pi_1(X, a)$ le groupe fondamental de $X$ en $a$, muni de sa topologie canonique, on a une équivalence canonique (à isomorphisme près)
$$
A-\fsct(X) \xlongrightarrow{\approx} \hat{A}-\modn(\pi_1(X, a)),
$$
où la deuxième membre désigne la catégorie des $\hat{A}$--Modules de type fini munis d'une opération continue de $\pi_1(X, a)$ pour la topologie $J$-adique. De plus, si
$$
f: X \to Y
$$
est un morphisme de schémas localement noethériens connexes, alors, munissant $Y$ du point géométrique $b = f \circ a$, le diagramme 
\[\begin{tikzcd}
	{A-\fsct(Y)} && {\hat{A}-\modn(\pi_1(Y, b))} \\
	{A-\fsct(X)} && {\hat{A}-\modn(\pi_1(X, a))} & {,}
	\arrow["{f^*}"', from=1-1, to=2-1]
	\arrow["\approx", from=1-1, to=1-3]
	\arrow["\approx"', from=2-1, to=2-3]
	\arrow["\Res", from=1-3, to=2-3]
\end{tikzcd}\]
dans lequel les flèches horizontales désignent les équivalences canoniques et $\Res$ est le foncteur restriction des scalaires déduit de $\pi_1(f): \pi_1(X, a) \to \pi_1(Y, b)$, est commutatif (à isomorphisme près).
}
\vskip .3cm
{\bf Preuve} : Seule la première assertion demande une démonstration. Pour cela, il n'y a qu'à transcrire la preuve de (SGA5 VI 1.2.5).

Dans l'énoncé suivant, nous appellerons sous-topos localement fermé d'un topos $X$ un couple $(U, Y)$ formé d'un ouvert $U$ de $X$ et du topos fermé complémentaire (relativement à $U$) d'un ouvert $V$ de $U$. Il est clair qu'il revient au même de se donner deux ouverts emboîtés $U$ et $V$ de $X$. On définit les opérations de restriction à un sous-topos localement fermé $(U, Y)$ comme composées des restrictions à $U$ puis à $Y$. Étant donné un autre ouvert $U'$ de $X$, on note
$$
U' \cap (U, Y)
$$
et on appelle intersection de $U'$ avec $(U, Y)$ le sous-topos localement fermé $(U \times U', Y')$ de $U'$, où $Y'$ désigne le topos fermé de $U \times U'$ complémentaire de $V \times U'$.

Étant donnés un topos $X$ et une famille finie $(U_i, Y_i)_{1 \leq i \leq p}$ de sous-topos localement fermés de $X$, on dira que $X$ est \emph{réunion} des $(U_i, Y_i)$ si, notant pour tout $i$ par $V_i$ l'ouvert de $U_i$ dont $Y_i$ est le complémentaire, on a les relations
$$
X = \bigcup_i U_i
$$
$$
\bigcap_i (V_i) = \emptyset
$$
et si pour toute partition $[1, q] = S \cup T$ de $[1, q]$ la relation 
$$
\bigcap_S (V_S) \cap \bigcup_T (U_t) = \emptyset
$$
implique soit que $T$ est vide, soit que $U_t = \emptyset$ pou tout $t \in T$.
\vskip .3cm
{
Proposition {\bf 1.21}. --- \it Soient $X$ un topos localement noethérien et $F$ un $A$-faisceau sur $X$. Les assertions suivantes sont équivalentes.
\begin{itemize}
    \item[(i)] $F$ est un $A$-faisceau constructible.
    \item[(ii)] Tout ouvert noethérien de $X$ est réunion d'un nombre fini de sous-topos localement fermés $Z_i = (U_i, Y_i)$ au-dessus desquels l'image réciproque de $F$ est un $A$-faisceau constant tordu constructible.
    \item[(iii)] $X$ admet un recouvrement par des ouverts, qui sont réunions finies des sous-topos localement fermés, au-dessus desquels l'image réciproque de $F$ est un $A$-faisceau constant tordu constructible.
\end{itemize}
}
\vskip .3cm
{\bf Preuve}~: Il est évident que (ii) $\Rightarrow$ (iii). Pour voir que (i) $\Rightarrow$ (ii), on peut supposer $X$ noethérien et $F$ $J$-adique constructible, et alors (SGA5 V 5.1.6) le gradué strict $\grs(F)$ est un $\gr_J(A)$--Module constructible. D'après la structure des Modules constructibles sur un topos noethérien (SGA4 VI \quad), le topos $X$ admet un recouvrement fini par des sous-topos localement fermés au-dessus desquels l'image réciproque de $\grs(F)$ est un $\gr_J(A)$--Module localement constant constructible. Au dessus de ces sous-topos localement fermés, les composants de $\grs(F)$ sont localement constants constructibles, et par suite $F$ est $J$-adique constant tordu constructible. Montrons que (iii) $\Rightarrow$ (i). Comme l'assertion est locale, on peut supposer que $X$ est noethérien et réunion finie de sous-topos localement fermés $Z_i = (U_i, Y_i)$ $(1 \leq i \leq q)$ au-dessus desquels $F$ est constant tordu constructible. En particulier, les $F | Z_i$ vérifiant la condition de Mittag-Leffler, et il résulte sans peine du lemme suivant que $F$ la vérifie également.
\vskip .3cm
{
Proposition {\bf 1.22}. --- \it Si un topos $X$ est réunion d'un nombre fini de sous-topos localement fermés $Z_m = (U_m, Y_m)$ $(1 \leq m \leq q)$, alors, notant $j_m: Y_m \to X$ les morphismes de topos canoniques, les foncteurs
$$
(j_m)^*: A-\Mod_X \to A-\Mod_{Y_m}
$$
forment une famille \emph{conservative}.
}
\vskip .3cm
Comme ces foncteurs sont exacts, il s'agit de voir que si un $A$--Module $M$ vérifie $(j_m)^*(M) = 0$ pour tout $m$, alors $M = 0$. Nous allons voir cette assertion par récurrence sur $q$, le cas où $q = 1$ étant évident. Nous allons pour cela noter $V_m$ l'ouvert de $U_m$ dont $Y_m$ est le complémentaire, et $i_m: V_m \to U_{m}$, $k_m: V_m \to X$ et $l_m: U_m \to X$ les morphismes canoniques. L'hypothèse de récurrence appliquée au topos fermé $K_m$ complémentaire de $U_m$ dans $X$ montre que pour tout $m$ le morphisme canonique
$$
(l_m)_! (M | U_m) \to M
$$
est un isomorphisme. Par ailleurs le fait que $(j_m)^*(M) = 0$ implique que le morphisme canonique
$$
(i_m)_! (M | V_m) \to M | U_m
$$
est également un isomorphisme. Il est donc de même du morphisme canonique $(k_m)_!(M | V_m) \to M$, et par suite (SGA4 IV 2.6)
$$
M \isom M \otimes_A (k_m)_! (A).
$$
Par récurrence, on en déduit que 
$$
M \isomlong M \otimes_A \bigotimes_m (k_m)_! (A).
$$
Mais, notant $k: \prod_m (V_m) \to e_X$ le morphisme canonique, on a (SGA4 IV 2.13.b) de 1))
$$
\bigotimes_m (k_m)_! (A) \isomlong k_!(A),
$$
d'où l'assertion, puisque par hypothèse le produit des $V_m$ est vide.

Sachant que $F$ vérifie la condition de Mittag-Leffler, on peut, quitte à le remplacer par le système projectif strict associé, supposer qu'il est strict. Alors (1.15.(ii)), il existe pour tout $i$ une application croissante $\gamma_i \geq \id: \mathbf{N} \to \mathbf{N}$ telle que $\chi_{\gamma_i}(F | Z_i)$ soit $J$-adique constructible. Posant $\gamma = \sup(\gamma_i)$, on voit que $\chi_\gamma(F)$ est $J$-adique, en utilisant (1.22), et constructible, d'où l'assertion.

Dans l'énoncé suivant, étant donné un sous-topos localement fermé $(U, Y)$ d'un topos localement noethérien $X$, et $i: Y \to X$, $j: Y \to U$, $k: U \to X$ les morphismes de topos canoniques, nous noterons $i_!$ le foncteur
$$
i_!: A-\fsc(Y) \to A-\fsc(X)
$$
le morphisme composé de $k_!$ et $j_*$. On s'assure aisément qu'il ne dépend pas (à isomorphisme près) de $U$, ce qui permet d'ôter ce dernier des notations. Le foncteur $i_!$ ainsi défini est exact et transforme $A$-faisceau constructible en $A$-faisceau constructible.
\vskip .3cm
{
Proposition {\bf 1.23}. --- \it Soient $X$ un topos noethérien et $F$ un $A$-faisceau constructible sur $X$. Il existe dans $\mathscr{E}(X, J)$, donc aussi dans $A-\fsc(X)$, une filtration finie de $F$ dont les quotients consécutifs sont de la forme $i_!(G)$, où $i: Y \to X$ est le morphisme structural d'un sous-topos localement fermé $(U, Y)$ de $X$, et $G$ un $A$-faisceau constant tordu constructible sur $Y$. Lorsque $X$ est le topos étale d'un schéma noethérien, noté de même, on peut prendre pour sous-topos localement fermés de $X$ les topos étales de schémas réduits associés à des parties localement fermées irréductibles de $X$. 
}
\vskip .3cm
{\bf Preuve} : Par récurrence noethérienne, on est ramené à prouver l'assertion en la supposant vraie pour out sous-topos fermé de $X$, différent de $X$. L'argument de la preuve de (SGA4 IX 2.5.), de nature formelle, s'applique aux topos généraux et montre, compte tenu de (1.21), qu'il existe un ouvert non vide $U$ de $X$ tel que $F U$ soit constant tordu constructible. Notons alors $Y$ le topos fermé complémentaire de $U$, et $i: U \to X$ et $j: Y \to X$ les morphismes de topos canoniques. On a alors (I 4.6.4.(i)) une suite exacte de $\mathscr{E}(X, J)$
$$
0 \to i_!(F | U) \to F \to j_* (F | Y) \to 0.
$$
L'assertion étant vraie sur $Y$ pour $F | Y$, par hypothèse de récurrence, on en déduit aussitôt qu'elle est vraie pour $F$. Dans le cas où $X$ est le topos étale d'un schéma, les sous-topos localement fermés de $X$ correspondent aux schémas réduits associés à des parties localement fermées de $X$, et on peut dans la preuve prendre pour $U$ un ouvert irréductible de $X$. 
\vskip .3cm
{
Proposition {\bf 1.24}. --- \it Soient $X$ un topos localement noethérien, et $E$ et $F$ deux $A$-faisceaux sur $X$.
\begin{itemize}
    \item[(i)] Si $E$ et $F$ sont constructibles (resp. constants tordus constructibles), les $A$-faisceaux
    $$
    {\cTor}^A_p(E, F) \quad (p \in \mathbf{Z})
    $$
    sont constructibles (resp. constants tordus constructibles). Si de plus l'anneau $A$ est régulier de dimension $r$, on 
    $$
    {\cTor}^A_p(E, F) = 0 \quad \text{pour}~p \geq r+1.
    $$
    \item[(ii)] Supposons maintenant que $X$ soit connexe, et soit $a$ un point de $X$. Lorsque $E$ et $F$ sont constants tordus constructibles, on a, avec les notations de (1.20.4), des isomorphismes de bifoncteurs cohomologiques
    $$
    \omega_X({\cTor}^A_p(E, F)) \isomlong {\cTor}^A_p(\omega_X(E), \omega_X(F))
    \leqno{(1.24.1)}
    $$
    et
    $$
    \mathscr{E}_a({\cTor}^A_p(E, F)) \isomlong \Tor^{\hat{A}}_p(\mathscr{E}_a(E), \mathscr{E}_a(F)).
    \leqno{(1.24.2)}
    $$
    De plus, lorsque $X$ est le topos étale d'un schéma localement noethérien, notant $M$ et $N$ les $\hat{A}$--Modules de type fini munis d'une opération continue de $\pi_1(X, a)$ correspondant à $E$ et $F$ (1.20.5), les $A$-faisceaux
    $$
    {\cTor}^A_p(E, F) \quad (p \in \mathbf{Z})
    $$
    correspondant aux $\hat{A}$--Modules de type fini
    $$
    \Tor^{\hat{A}}_p(M, N),
    $$
    munis de l'opération ``diagonale'' de $\pi_1(X, a)$.
\end{itemize}
}
\vskip .3cm
{\bf Preuve} : Supposons tout d'abord que $E$ et $F$ sont constants tordus constructibles, et montrons que les $A$-faisceaux ${\cTor}^A_p(E, F)$ le sont également. On peut pour cela supposer $E$ et $F$ $J$-adiques constants tordus constructibles. Pour tout entier $p$, la définition de ${\cTor}^A_p(E, F)$ (I 5.1) montre que ce $A$-faisceau a des composants localement constants constructibles, de sorte qu'il suffit (1.16) de voir qu'il est de type $J$-adique. On peut supposer $X$ connexe; soit alors $a$ un point de $X$. Posant alors $M = \mathscr{E}_a(E)$ et $N = \mathscr{E}_a(F)$, foncteur fibre 
$$
\mathcal{E}(X, J) \to \mathcal{E}(\pt, J)
$$
défini par $a$ associe au $A$-faisceau ${\cTor}^A_p(E, F)$ le système projectif
$$
(\Tor^{A_n}_p(M/J^{n+1}M, N/J^{n+1}N))_{n \in \mathbf{N}},
$$
et il suffit, vu la conservativité du foncteur fibre (habituel) défini par $a$ sur les $A$--Modules localement constants, de vérifier que ce dernier est de type $J$-adique. Choisissons pour cela une résolution libre de type fini
$$
P \to M
$$
du $\hat{A}$--Module $M$. Convenant de poser pour tout $\hat{A}$--Module de type fini $L$
$$
\mathbf{L} = (L/J^{n+1}L)_{n \in \mathbf{N}},
$$
il résulte de (4.1.4) que $\mathbf{P} \to \mathbf{M}$ est une résolution quasilibre de $\mathbf{M}$. Par suite (I 5.11.(i)), on a dans $A-\fsc(\pt)$ un isomorphisme canonique
$$
{\cTor}^A_p(\mathbf{M}, \mathbf{N}) \isom \mathrm{H}_p(\mathbf{P} \otimes_A \mathbf{N}).
\leqno{(1.24.3)}
$$
Mais les composants du complexe $\mathbf{P} \otimes_A N$ sont des $A$-faisceaux $J$-adiques constructibles, donc ses objets de cohomologie sont des $A$-faisceaux constructibles (1.5.(iii)), d'où l'assertion. Par ailleurs, le foncteur limite projective
$$
A-\fscn(\pt) \to \hat{A}-\modn 
$$
est exact (1.12.2) et commute au produit tensoriel (EGA $0_{III}$ 7.3.4), de sorte que (1.24.2) s'obtient par passage à la limite projective à partir de (1.24.3). Lorsque $X$ est un schéma connexe et $a$ un point géométrique de $X$, ce qui précède montre en tout cas que l'application canonique 
$$
\Tor^A_p(M, N) \to \varprojlim_n \Tor^{A_n}_p(M/J^{n+1}M, N/J^{n+1}N)
$$
est un isomorphisme topologique. Par ailleurs, il est immédiat que, munissant le premier membre de l'opération diagonale de $\pi_1(X, a)$ et le second membre de la limite projective des opérations diagonales, c'est un morphisme de $\pi_1(X, a)$--modules. Terminons la preuve de (ii), en exhibant l'isomorphisme (1.24.1). Il suffit pour cela de remarquer que le foncteur (1.20.2) ``commute aux $\cTor_i$'' (SGA4 IV) ce qui permet, vu la définition (I 5.1.) de définir (1.24.1) sur les composants. Montrons maintenant (i). Si $E$ et $F$ sont constructibles, on sait (1.21) que $X$ admet un recouvrement par des ouverts, qui sont réunions finies de sous-topos localement fermés $(Z_i)_{i \in I}$ au-dessus desquels $E$ et $F$ sont constants tordus constructibles. Mais 
$$
\cTor^A_p(E, F)|Z_i \isom \cTor^A_p(E|Z_i, F|Z_i) \quad (i \in I, p \in \mathbf{Z}),
$$
et par suite, d'après (ii), les restrictions aux $Z_i$ des $A$-faisceaux $\cTor^A_p(E, F)$ sont des $A$-faisceaux constants tordus constructibles, ce qui entraîne qu'ils sont constructibles (1.21). Montrons enfin que si $A$ est régulier de dimension $r$, on a 
$$
\cTor^A_p(E, F) \quad (p \geq r+1)~\text{dans}~A-\fsc(X).
$$
On peut supposer $X$ noethérien, et il s'agit alors de voir que les systèmes projectifs $\cTor^A_p(E, F)$ $(p \geq r+1)$ sont essentiellement nuls. Grâce à (1.22), il suffit de vérifier cette assertion au-dessus des sous-topos localement fermés de $X$ sur lesquels $E$ et $F$ sont constants tordus constructibles. On est ainsi ramené au cas où $E$ et $F$sont constants tordus constructibles. Supposant de plus $X$ connexe et choisissant un point $a$ de $X$, l'assertion résulte alors de (1.24.2) et du fait que $\hat{A}$ est régulier de dimension $r$ (EGA $0_{IV}$ 17.3.8.1).
\vskip .3cm
{
Proposition {\bf 1.25}\footnote{Pour les assertions respées, $J$ doit être supposé maximal.}. --- \it Soient $X$ un topos localement noethérien et $E$ un $A$-faisceau constructible (resp. constant tordu constructible) sur $X$.
\begin{itemize}
    \item[(i)] Les assertions suivantes sont équivalentes :
    \begin{itemize}
        \item[a)] $E$ est plat.
        \item[b)] $E$ est fortement plat (resp. localement libre constructible).
    \end{itemize}
    Si de plus $J$ est un idéal maximal de $A$, elles équivalent à :
    \begin{itemize}
        \item[c)] $E$ est presque plat (I 5.14).
    \end{itemize}
    \item[(ii)] Si $E$ est $J$-adique, les assertions suivantes sont équivalentes :
    \begin{itemize}
        \item[a)] $E$ est plat.
        \item[b)] Pour tout entier $n \geq 0$, le n$^\text{ème}$ composant $E_n$ de $E$ est un $A_n$--Module plat (resp. localement libre constructible).
    \end{itemize}
    \item[(iii)] Si $A$ est un anneau local régulier de dimension $r$ et $J$ est son idéal maximal, alors
    $$
    \cTor^A_p(E, F) = 0 \quad (p \geq r+1)
    $$
    pour tout $A$-faisceau $F$. Si de plus $F$ est presque plat,
    $$
    \cTor^A_p(E, F) = 0 \quad (p \geq 1).
    $$
\end{itemize}
}
\vskip .3cm
{\bf Preuve} : Montrons (ii). L'assertion b) $\Rightarrow$ a) a déjà été vue (I 5.6.); l'assertion a) $\Rightarrow$ b) s'obtient en écrivant que pour toute suite exacte $0 \to M' \to M \to M'' \to 0$ de $A_n$--Modules, la suite correspondante 
$$
0 \to M' \otimes_A E \to M \otimes_A E \to M'' \otimes_A E \to 0
$$
est exacte. On déduit aussitôt de (ii) l'équivalence des assertions a) et b) de (i), de sorte qu'il suffit de voir que c) $\Rightarrow$ a). Autrement dit, nous avons à montrer que pour tout $A$-faisceau $F$, les systèmes projectifs $\cTor^A_p(E, F)$ $(p \geq 1)$ sont essentiellement nuls, lorsqu'on se restreint à des ouverts noethériens. Grâce à (1.22) et (1.21), on peut supposer $X$ noethérien connexe et $E$ constant tordu constructible. Par ailleurs, la catégorie $A-\fsc(X)$ ne changeant pas lorsque $A$ est remplacé par $A_J$, on peut supposer que $A$ est local noethérien. Choisissant alors un point $a$ de $X$, le $\hat{A}$--module $M$ correspondant à $E$ dans l'équivalence $\epsilon_a$ (1.20.4) vérifie (1.24.2)
$$
\cTor^{\hat{A}}_1(A/J, M) = 0,
$$
donc est libre (Bourbaki.Alg.Comm. II 3 Cor.2), et par suite $E$ est localement libre constructible, d'où l'assertion. Montrons (iii). Comme tout $A$-faisceau admet (I 5.16) une résolution de longueur $r$ par des $A$-faisceaux presque plats, on peut supposer que $F$ est presque plat. Comme précédemment, on se ramène au cas où $A$ est local noethérien, $X$ noethérien connexe et $E$ constant tordu constructible. Ayant choisi un point $a$ de $X$, soit $M$ le $\hat{A}$--module de type fini correspondant à $E$. Nous allons voir que
$$
\cTor^A_p(E, F) = 0 \quad (p \geq 1)
$$
par récurrence croissante sur la dimension de $M$. Lorsque dim$(M) = 0$, $M$ est annulé par une puissance de $J$, et l'assertion résulte de (I 5.13). Supposons maintenant l'assertion vraie pour dim$(M) = d \geq 0$ et montrons qu'elle est vraie pour dim$(M) = d+1$. Le sous-$\hat{A}$--module $M'$ de $M$ formé des éléments annulés par une puissance de l'idéal $J$ correspond au plus grand sous-$A$-faisceau constant tordu constructible $E'$ de $E$ annulé par une puissance de $J$. Posons $E'' = E/E'$ et $M'' = M/M'$. Le $\hat{A}$--module $M''$ correspond à $E''$, et on a une suite exacte
$$
\cTor^A_i(E', F) \to \cTor^A_i(E, F) \to \cTor^A_i(E'', F) \quad (i \geq 1),
$$
qui montre, compte tenu de ce que $E'$ est annulé par une puissance de $J$, qu'il suffit de prouver l'assertion pour $E''$. Mais prof$(M'') > 0$ et par suite il existe un élément $u$ de $J$ tel que la multiplication par $u$ soit un monomorphisme de $E''$. On en déduit pour tout $i \geq 1$ une suite exacte
$$
\cTor^A_i(E'', F) \xlongrightarrow{u} \cTor^A_i(E'', F) \to \cTor^A_i(E''/uE'', F).
$$
Mais le $\hat{A}$--module correspondant à $E''/uE''$, à savoir $M''/uM''$, est de dimension $d$ (EGA $0_{IV}$ 16.3.4), donc
$$
\cTor^A_i(E''/uE'', F) = 0 \quad (i \geq 1)
$$
par hypothèse de récurrence, et par suite
$$
\cTor^A_i(E'', F) = u\cTor^A_i(E'', F),
$$
ce qui permet de conclure par le lemme de Nakayama (I 5.12).
\vskip .3cm
{
Proposition {\bf 1.26}. --- \it Soient $X$ un topos localement noethérien, et $E$ et $F$ deux $A$-faisceaux sur $X$. 
\begin{itemize}
    \item[(i)] Si $E$ et $F$ sont constant tordus constructibles, les $A$-faisceaux
    $$
    \cExt^p_A(E, F) \quad (p \in \mathbf{Z})
    $$
    sont constant tordus constructibles. Lorsque $X$ est connexe, le choix d'un point $a$ de $X$ définit, avec les notations de (1.20.4), des isomorphismes de bifoncteurs cohomologiques
    $$
    \omega_X \cExt^p_A(E, F) \isomlong \cExt^p_A(\omega_X E, \omega_X F).
    \leqno{(1.26.1)}
    $$
    $$
    \mathscr{E}_a \cExt^p_A(E, F) \isomlong \cExt^p_A(\mathscr{E}_aE, \mathscr{E}_a F).
    \leqno{(1.26.2)}
    $$
    De plus, lorsque $X$ est le topos étale d'un schéma localement noethérien, notant $M$ et $N$ les $\hat{A}$--modules de type fini munis d'une opération continue  de $\pi_1(X, a)$ correspondant à $E$ et $F$ (1.20.5), les $A$-faisceaux
    $$
    \cExt^p_A(E, F) \quad (p \in \mathbf{Z})
    $$
    correspondent aux $\hat{A}$--modules de type fini
    $$
    \Ext^p_{\hat{A}}(M, N),
    $$
    munis de l'opération ``diagonale'' de $\pi_1(X, a)$.
    \item[(ii)] Si $E$ est constant tordu constructible et $F$ constructible, les $A$-faisceaux $\cExt^p_A(E, F)$ sont constructibles.
    \item[(iii)] On suppose que l'anneau $A$ est local régulier de dimension $r$ et que $J$ est son idéal maximal. Alors, si $E$ est constant tordu constructible, on a 
    $$
    \cExt^p_A(E, F) = 0 \quad (p \geq r+1).
    $$
    \item[(iv)] Supposons que pour toute $A$-algèbre de type fini $B$ annulée par une puissance de $J$, et toute couple $(M, N)$ de $B$--Modules constructibles, les $B$--Modules
    $$
    \cExt^p_B(M, N) \quad (p \in \mathbf{Z})
    $$
    soient constructibles. Alors, lorsque $E$ et $F$ sont constructibles, les $A$-faisceaux $\cExt^p_A(E,F)$ sont constructibles.
\end{itemize}
}
\vskip .3cm
{\bf Preuve} : Montrons (i). Comme le $A$-faisceau $\cExt^p_A(E, F)$ a des composants localement constants constructibles, il suffit pour voir qu'il est constant tordu constructible, de montrer qu'il est de type $J$-adique (1.16). On peut supposer $X$ connexe; soit alors $a$ un point de $X$. Posant $M = \epsilon_a(E)$ et $N = \epsilon_a(F)$, le foncteur fibre
$$
\mathcal{E}(X, J) \to \mathcal{E}(\pt, J)
$$
défini par $a$ associe au $A$-faisceau $\cExt^p_A(E, F)$ le système projectif
$$
(\varinjlim_{m \geq n}\Ext^p_{A_m}(M/J^{m+1}M, N/J^{n+1}N))_{n \in \mathbf{N}},
$$
et il suffit, vu la conservativité du foncteur fibre (habituel) défini par $a$ sur les $A$--Modules localement constants, de vérifier que ce dernier est de type $J$-adique. Avec les notations de la preuve de (1.24.(i)), on a dans $A-\fsc(\pt)$ un isomorphisme canonique
$$
\cExt^p_A(\mathbf{M}, \mathbf{N}) \isom \mathrm{H}^p (\cHom^\bullet_A(\mathbf{P}, \mathbf{N})),
\leqno{(1.26.3)}
$$
défini grâce à (I 7.3.11). Mais (SGA5 VI 1.3.3) les composants de $\cHom^\bullet_A(\mathbf{P}, \mathbf{N})$ sont $J$-adiques constructibles, d'où aussitôt l'assertion.

Les assertions restantes de la partie (i) se montrent à partir de là en calquant la preuve des assertions analogues de (1.24). Montrons (ii). D'après (1.21), on peut supposer que $F$ est également constant tordu constructible, et alors (ii) résulte de (i). Pour voir (iii), on peut supposer que $X$ est connexe. Choisissant alors un point $a$ de $X$, nous allons raisonner par récurrence croissante sur la dimension du $\hat{A}$--module de type fini $M$ associé à $E$. Si dim$(M) = 0$, le $A$-faisceau $E$ est défini par un $A$-Module localement constant constructible annulé par une puissance de $J$, que, quitte à localiser, on peut même supposer constant. Alors, toute résolution de longueur $r$ de $M$ par des $\hat{A}$-modules libres de type fini définit une résolution localement libre constructible et de longueur $r$ du $A$-faisceau $E$. On conclut dans ce cas grâce à (I 7.3.11). Supposons maintenant l'assertion vraie lorsque dim$(M) = d \geq 0$ et montrons qu'elle est vraie pour dim$(M) = d+1$. Le sous-$\hat{A}$--module $M'$ de $M$ formé des éléments annulés par une puissance de l'idéal $J$ correspond au plus grand sous-$A$-faisceau constant tordu constructible $E'$ de $E$ annulé par une puissance de $J$. Posons $E'' = E/E'$ et $M'' = M/M'$. Le $\hat{A}$--module $M''$ correspond à $E''$ et on a une suite exacte
$$
\cExt^p_A(E'', F) \to \cExt^p_A(E, F) \to \cExt^p_A(E', F) \quad (p \in \mathbf{Z}),
$$
qui montre, compte tenu de ce que $E'$ est annulé par une puissance de $J$, qu'il suffit de prouver l'assertion pour $E''$. Mais prof$(M'') > 0$, de sorte qu'il existe un élément $u$ de $J$ tel que la multiplication par $u$ définisse un monomorphisme de $E''$. On en déduit pour tout $p \in \mathbf{Z}$ une suite exacte
$$
\cExt^p_A(E'', F) \xlongrightarrow{\times u} \cExt^p_A(E'', F) \xlongrightarrow{\delta} \cExt^{p+1}_A(E''/uE'', F).
$$
L'hypothèse de récurrence implique alors que 
$$
\cExt^p_A(E'', F) = u \cExt^p_A(E'', F) \quad (p \geq r+1),
$$
et on conclut par le lemme de Nakayama (I 5.12). Pour prouver (iv), nous allons tout d'abord supposer que $E$ est plat et $J$-adique, donc que pour tout entier $n \geq 0$, le $n^{\text{ème}}$ composant $E_n$ de $E$ est un $A_n$--Module constructible et plat (1.25). Dans ce cas, nous allons utiliser la notation suivante. Soit $M$ un $A_p$--Module. Il résulte de (I 6.5.2) que pour tout entier $q \geq p$, le morphisme canonique
\[\begin{tikzcd}
	{\cExt^i_{A_q}(E_q, M)} && {\cExt^i_{A_p}(E_p, M)} && {(i \geq 0)}
	\arrow[from=1-3, to=1-1]
\end{tikzcd}\]
est un isomorphisme. Posant pour tout entier $i \geq 0$
$$
T^i(M) = \varinjlim_{q \geq p}\cExt^i_{A_q}(E_q, M),
$$
il est clair qu'on obtient un foncteur cohomologique de la catégorie des $A$--Modules annulés par une puissance de $J$ dans elle-même. De plus, les hypothèses faites assurent que lorsque $M$ est constructible, les $A$--Modules $T^i(M)$ sont constructibles. Rappelons enfin qu'avec ces notations, le $A$-faisceau $\cExt^i_A(E, F)$ est identique au système projectif
$$
(T^i(F_n))_{n \in \mathbf{N}}.
$$
Pour voir l'assertion dans ce cas, on peut supposer que $F$ est $J$-adique. Alors, compte tenu de lemme d'Artin-Rees (SGA5 V 4.2.6) et du lemme de Shih (SGA5 V A3.2), il suffit de montrer que pour tout entier $m \geq 0$, le $\gr_J(A)$--Module
$$
T^m(\grs(F)) \isom \cExt^m_{A_0}(E_0, \grs(F)),
$$
dans lequel $\gr_J(A)$ opère par l'intermédiaire du deuxième argument, est noethérien. Mais il résulte de (I 6.5.2) que 
$$
\cExt^m_{A_0}(E_0, \grs(F)) \isom \cExt^m_{\gr_J(A)}(E_0 \otimes_{A_0}\gr_J A, \grs(F)),
$$
ce qui permet de conclure grâce à l'hypothèse de l'énoncé et au théorème de Hilbert (SGA5 V 5.1.4). Montrons maintenant comment on peut se ramener en général au cas où $E$ est plat. On se ramène facilement au cas où $X$ est noethérien, de sorte que (1.23) $E$ admet une filtration finie dont les quotients consécutifs sont de la forme $i_!(G)$, où $i: Y \to X$ est le morphisme structural d'un sous-topos localement fermé de $X$ et $G$ est un $A$-faisceau constant tordu constructible sur $Y$, de sorte qu'on peut supposer $E$ de la forme $i_!(G)$. Lorsque $i$ est une immersion fermée, on a (I 7.7.12)
$$
\bRd \cHom_A(i_!(G), F) \isomlong i_* \bRd \cHom_A(G, \bRd i^!(F)),
$$
de sorte que d'après (ii), on est ramené à voir que $\Rd i^!(F)$ est à cohomologie formée de $A$-faisceaux constructibles. Mais (I 7.7.13)
$$
\bRd i^!(F) \isomlong i^* \bRd\cHom_A(i_*(A), F),
$$
d'où l'assertion dans ce cas, car $i_*(A)$ est un $A$-faisceau plat et constructible. Dans le cas où $i$ n'est pas une immersion fermée, on l'écrit sous la forme 
$$
i = k \circ j,
$$
où $j$ est une immersion fermée et $k$ une immersion ouverte. On a alors 
$$
\bRd \cHom_A(i_!G, F) \isom \bRd k_* \bRd\cHom_A(j_*(G), k^*(F)),
$$
de sorte que, d'après ce qui a été vu dans le cas d'une immersion fermée, il suffit de montrer que si $k$ est définie par l'ouvert $U$ de $X$ et $P$ est un complexe borné inférieurement de $A$-faisceaux sur $U$ dont les objets de cohomologie sont des $A$-faisceaux constructibles, les objets de la cohomologie de $\bRd k_*(P)$ sont également des $A$-faisceaux constructibles. Or, notant $t: Z \to X$ l'immersion fermée complémentaire de $k$, on a un tringle exact (I 7.7.14)
\[\begin{tikzcd}
	& {\bRd k_*(P)} \\
	{\bRd t_*\bRd t^!k_!(P)} && {k_!(P)}
	\arrow[from=2-1, to=2-3]
	\arrow[from=2-3, to=1-2]
	\arrow[dashed, from=1-2, to=2-1]
\end{tikzcd}\]
qui montre qu'il suffit de voir que 
$$
\bRd t^!k_!(P) \isom t^*\bRd\cHom_A(t_*(A), k_!(P))
$$
est à cohomologie constructible, ce qui nous ramène à nouveau au cas d'une immersion fermée.
\vskip .3cm
{\bf Exemple 1.27}. Les hypothèses de (iv) sont notamment réalisées (SGA5 I Appendice 6) lorsque $X$ est le topos étale d'un schéma localement noethérien, lorsqu'on dispose de la résolution des singularités et de la pureté au sens fort (SGA5 I Appendice 4.4). C'est le cas notamment lorsque $X$ est de dimension $\leq 1$, ou lorsque $X$ est excellent de caractéristique nulle, ou localement de type fini sur un corps et de dimension $\leq 2$.

Nous allons maintenant nous intéresser à quelques propriétés particulières aux anneaux de valuation discrète.
\vskip .3cm
{
Proposition {\bf 1.28}. --- \it On suppose que $A$ est un anneau de valuation discrète et que $J$ est son idéal maximal. Étant donné un $A$-faisceau constructible $F$ sur $X$, les assertions suivantes sont équivalentes:
\begin{itemize}
    \item[(i)] $F$ est plat. 
    \item[(ii)] $F$ est sans torsion, i.e. pour tout élément $a$ de $A$, l'homothétie 
    $$
    a_F: F \to F
    $$
    est un monomorphisme.
    \item[(iii)] Étant donnée une uniformisante locale $u$ de $A$, l'endomorphisme $u_F$ est un monomorphisme.
\end{itemize}
}
\vskip .3cm
{\bf Preuve} : L'équivalence de (ii) et (iii) est évidente, et il est immédiat que (i) $\Rightarrow$ (iii). Montrons que (iii) $\Rightarrow$ (i). D'après (1.25.(i)), il suffit de voir que 
$$
{\cTor}^A_i(A/uA, E) = 0,
$$
ce qui se voit sans peine sur la suite exacte des ${\cTor}^A_i(., E)$ associée à la suite exacte $0 \to A \xlongrightarrow{u} A \to A/uA \to 0$.
\vskip .3cm
{
Proposition {\bf 1.29}. --- \it On suppose que $A$ est de valuation discrète, que $J$ est son idéal maximal, et que $X$ est le topos étale d'un schéma noethérien (resp. localement noethérien). Alors, pour tout $A$-faisceau constructible (resp. constant tordu constructible) $F$ sur $X$, il existe une suite exacte
$$
0 \to L' \to L \to F \to 0
$$
de $A-\fscn(X)$, avec $L$ et $L'$ deux $A$-faisceaux \emph{constructibles et plats} (resp. \emph{localement libres constructibles}).
}
\vskip .3cm
{\bf Preuve} : Il résulte de (1.28) que tout sous-$A$-faisceau constructible d'un $A$-faisceau constructible et plat est plat. Utilisant (1.5.(iii)), (resp. 1.17.(iii)), on voit donc qu'il nous suffit de prouver l'existence d'un épimorphisme $L \to F \to 0$, avec $L$ constructible et plat (resp. localement libre constructible).
\begin{itemize}
    \item[a)] Supposons tout d'abord que $F$ soit associé à un $A_d$--Module localement constant constructible ($d$ entier $\geq 0$) et montrons l'assertion respée dans ce cas. Quitte à décomposer $X$ en ses composantes connexes (ouvertes), on peut le supposer connexe. Alors, choisissant un point $a$ de $X$, le $A$-faisceau $F$ correspond à une représentation continue  de $\pi_1(X, a)$ dans un $A_d$--module de type fini $S$. Le groupe $\pi_1(X, a)$ opérant par un quotient fini $G$ sur $S$, il existe un épimorphisme de $(G, A_d)$--modules
    $$
    T \to S \to 0,
    $$
    avec $T$ un $\hat{A}(G)$--module libre de type fini, qui peut être aussi considéré comme un $(\pi_1(X, a), \hat{A})$--module continu. L'assertion en résulte grâce à (1.20.5), puisque le $A$-faisceau constant tordu correspondant à $T$ est localement libre. On remarquera que dans cette partie on n'a pas utilisé que $A$ est de valuation discrète. 
    \item[b)] Montrons maintenant l'assertion dans le cas où $F$ est associé à un $A_d$--Module constructible. D'après (SGA4 IX 2.14.(ii)), il existe une famille finie
    $$
    (p_i: X_i \to X)_{i \in I}
    $$
    de morphismes finis, pou tout $i$ un $A_d$--Module constant $C_i$ sur $X_i$, et un monomorphisme 
    $$
    F \xlongrightarrow{\lambda} \prod_i (p_i)_*(C_i).
    $$
    Or, d'après a), il existe des épimorphismes de $A$-faisceaux $P_i \to C_i \to 0$, avec $P_i$ constructible et plat, d'où un épimorphisme
    $$
    \prod_i(p_i)_*(P_i) \xlongrightarrow{\mu} \prod_i (p_i)_*(C_i) \to 0,
    $$dont la source est un $A$-faisceau constructible et sans torsion, donc plat. Considérons alors un diagramme cartésien
    \[\begin{tikzcd}
	P && {\prod_i(p_i)_*(P_i)} \\
	F && {\prod_i(p_i)_*(C_i).}
	\arrow["\beta"', from=1-1, to=2-1]
	\arrow["\lambda", from=2-1, to=2-3]
	\arrow["\alpha", from=1-1, to=1-3]
	\arrow["\mu", from=1-3, to=2-3]
    \end{tikzcd}\]
    Des arguments catégoriques généraux montrent que $\alpha$ est un monomorphisme et $\beta$ un épimorphisme; de plus, $P$ est constructible, et plat puisque $\alpha$ est un monomorphisme. On remarquera qu'on a seulement utilisé que $X$ est localement noethérien, et que l'argument montre plus généralement que, sans hypothèse sur $A$, $F$ est quotient d'un $A$-faisceau constructible et sans torsion.
    \item[c)] Passons au cas général. La catégorie $A-\fscn(X)$ (resp. $-\fsct(X)$) est noethérien (1.5.(iii) resp. 1.17.(iii)); par suite, $u$ désignant une uniformisante locale de $A$, la famille de sous-$A$-faisceaux constructibles (resp. constants tordus constructibles)
    $$
    _{u^{n}} F = \Ker(F \xlongrightarrow{u^n} F)
    $$
    admet un plus grand élément, soit $_{u^{d}} F$, dans $A-\fsc(X)$. Le $A$-faisceau $M = u^dF$ est sans torsion et $F/u^dF$ est isomorphe au $A$-faisceau associé   à un $(A/u^dA)$--Module constructible (resp. localement constant constructible). D'après b) (resp a)), il existe un épimorphisme
    $$
    \gamma: P \to F/u^dF,
    $$
    avec $F$ un $A$-faisceau constructible et plat (resp. localement libre constructible). Désignant par $L$ le produit fibré de $P$ et $F$ au-dessus de $F/u^dF$, le diagramme commutatif exact évident
    \[\begin{tikzcd}
	0 & M & L & P & 0 \\
	0 & M & F & {F/u^dF} & 0
	\arrow[from=1-1, to=1-2]
	\arrow[from=1-2, to=1-3]
	\arrow[from=1-3, to=1-4]
	\arrow[from=1-4, to=1-5]
	\arrow[from=2-4, to=2-5]
	\arrow[from=2-3, to=2-4]
	\arrow[from=2-2, to=2-3]
	\arrow[from=2-1, to=2-2]
	\arrow["\id"', from=1-2, to=2-2]
	\arrow["\delta"', from=1-3, to=2-3]
	\arrow["\gamma"', from=1-4, to=2-4]
    \end{tikzcd}\]
    montre que $\delta$ est un épimorphisme et que $L$, extension de deux $A$-faisceaux constructibles et plats (resp. localement libres constructibles) est lui-même plat (resp. localement libre constructible).
\end{itemize}


% End

% Begin








%%%%%%%%%%%%%%%%%%%%%%%%%%%%%%%%%%%%
\subsection*{2. Conditions de finitude dans les catégories dérivées.}
\addcontentsline{toc}{subsection}{2. Conditions de finitude dans les catégories dérivées}

Soit $X$ un topos localement noethérien.
\vskip .3cm
{
Définition {\bf 2.1}. --- \it On dit qu'un complexe $E$ de $A$-faisceaux sur $X$ est \emph{à cohomologie constructible} (resp. \emph{constante tordue constructible}) si tous ses objets de cohomologie sont des $A$-faisceaux constructibles (resp. constants tordus constructibles). 
}
\vskip .3cm
La sous-catégorie $A-\fscn(X)$ étant exacte dans $A-\fsc(X)$ (1.5.(iii)), les sous-catégories pleines
$$
\K^*_c(X, A) \quad \text{et} \quad \D^*_c(X, A) \quad (* = \emptyset, + -~\text{ou}~b)
$$
de $\K^*(X, A)$ et $\D^*(X, A)$ respectivement engendrées par les complexes à cohomologie constructible sont des sous-catégories triangulées; de plus, $\D^*_c(X, A)$ s'obtient par inversion des quasi-isomorphismes à partir de $\K^*_c(X, A)$. De même, on définit des catégories triangulées
$$
\K^*_t(X, A) \quad \text{et} \quad \D^*_t(X, A) \quad (* = \emptyset, + -~\text{ou}~b)
$$
à partir des complexes à cohomologie constants tordue constructible et $\D^*_t(X, A)$ s'obtient à partir de $\K^*_t(X, A)$ en inversant les quasi-isomorphismes.
\vskip .3cm
{
Définition {\bf 2.2}. --- \it On dit qu'un complexe $E$ de $A$-faisceaux sur $X$ est \emph{pseudocohérent} s'il est à cohomologie localement bornée supérieurement et constante tordue constructible. On dit qu'il est \emph{parfait} si de plus il est localement de tor-dimension finie.
}
\vskip .3cm
Comme $A-\fsct(X)$ est une sous-catégorie exacte de $A-\fsc(X)$, il est clair que les sous-catégories pleines
$$
\K_{\text{coh}}(X, A) \quad \text{et} \quad \K_{\parf}(X, A)
$$
de $\K(X, A)$ engendrées respectivement par les complexes pseudocohérents et parfaits sont des sous-catégories triangulées vérifiant les inclusions 
$$
\K_{\parf}(X, A) \subset \K_{\text{coh}}(X, A) \subset \K_t(X, A).
$$
On définit de même des catégories triangulées
$$
\D_{\text{coh}}(X, A) \quad \text{et} \quad \D_{\parf}(X, A)
$$
vérifiant les inclusions 
$$
\D_{\parf}(X, A) \subset \D_{\text{coh}}(X, A) \subset \D_t(X, A).
$$
De plus la catégorie $\D_{\parf}(X, A)$ (resp. $\D_{\text{coh}}(X, A)$) est obtenue à partir de $\K_{\parf}(X, A)$ (resp. $\K_{\text{coh}}(X, A)$) par inversion des quasi-isomorphismes. Enfin, on utilisera également les notations
$$
\D^b_{\parf}(X, A) = (\D^b_t(X, A))_{\text{torf}}
$$
et
$$
\D^b_{\text{coh}}(X, A) = \D^b_t(X, A).
$$
Avant de poursuivre, nous allons expliciter certaines de ces notions dans le cas où $X$ est le topos ponctuel. Dans ce cas, le foncteur additif
$$
M \mapsto (M/J^{n+1}M)_{n \in \mathbf{N}}
$$
de la catégorie des $\hat{A}$--modules de type fini dans $A-\fsc(\pt)$ est exact et permet donc de définir par prolongement aux complexes un foncteur exact
$$
\D^b(\hat{A}-\modn) \to \D^b_c(\pt, A).
\leqno{(2.3.1)}
$$
De plus, comme tout complexe parfait de $\hat{A}$--modules est équivalent à un complexe borné de $\hat{A}$--modules projectifs de type fini, le foncteur (2.3.1) induit un foncteur exact
$$
\D_{\parf}(\hat{A}-\modn) \to \D_{\parf}(\pt, A).
\leqno{(2.3.2)}
$$
\vskip .3cm
{
Proposition {\bf 2.3}. --- \it Les foncteurs 
$$
\D^b(\hat{A}-\modn) \to \D^b_c(\pt, A)
\leqno{(2.3.1)}
$$
$$
\D_{\parf}(\hat{A}-\modn) \to \D_{\parf}(\pt, A)
\leqno{(2.3.2)}
$$
ci-dessus sont des \emph{équivalences de catégories}.
}
\vskip .3cm
{\bf Preuve} : Comme le foncteur (2.3.1) commute évidemment au produit tensoriel et est conservatif, il est clair qu'un complexe dont l'image par (2.3.1) est de tor-dimension finie est lui-même de tor-dimension finie. Il nous suffit donc de montrer que (2.3.1) est une équivalence. Notons pour cela $U$ la sous-catégorie pleine de $\K^b_c(\pt, A)$ engendrée par les complexes bornés à cohomologie constructible et dont les composants sont essentiellement stricts, i.e. vérifiant la condition de Mittag-Leffler. Comme la catégorie $\hat{A}-\modn$ s'identifie à une sous-catégorie pleine de $A-\fsc(\pt)$, il est clair qu'on a une suite de foncteurs d'``inclusion''
$$
\K^b(\hat{A}-\modn) \xlongrightarrow{p} U \xlongrightarrow{q} \K^b_c(\pt, A).
$$
Nous allons voir successivement que lorsqu'on inverse les quasi-isomorphismes, les foncteurs $p$ et $q$ deviennent des équivalences. Pour le voir pour $p$, il suffit (CD I 4.2.(b)) de montrer qu'étant donné un objet $E$ de $U$, il existe un quasi-isomorphisme
$$
M \to E,
$$
avec $M$ un objet de $\K^b(\hat{A}-\modn)$. Appliquant (EGA $0_{III}$ 11.9.1), on est ramené à montrer qu'étant donnés un objet $F$ de $A-\fsc(\pt)$ vérifiant la condition de Mittag-Leffler et un épimorphisme de $A$-faisceaux
$$
F \xlongrightarrow{u} P \to 0,
$$
avec $P$ un $\hat{A}$--module de type fini, il existe un $\hat{A}$--module de type fini $Q$ et un morphisme $v: Q \to F$ tels que le composé $u v$ soit un épimorphisme. Quitte à remplacer $F = (F_n)_{n \in \mathbf{N}}$ par le système projectif strict associé, on peut supposer qu'il est strict. Alors, les morphismes de $\hat{A}$--modules canoniques
$$
\varprojlim (F) \to F_n \quad (n \in \mathbf{N})
$$
sont des épimorphismes. Choisissons alors un sous-$\hat{A}$--module de type fini $Q$ de $\varprojlim (F)$ tel que la projection $Q \to F_0$ soit un épimorphisme. Alors le morphisme composé $Q \to P$ induit un épimorphisme $Q/JQ \to P/JP$, donc est un épimorphisme d'après le lemme de Nakayama. Montrons maintenant que le foncteur $q$ induit une équivalence après inversion des quasi-isomorphismes. Étant donné un objet $K$ de $\K^b_c(\pt, A)$, on sait (I 6.6.3) qu'il existe un quasi-isomorphisme
$$
K \overset{w}\simeq L
$$
où $L$ est un complexe borné inférieurement et dont les composants sont directement stricts, donc vérifient la condition de Mittag-Leffler. Nous allons voir que, quitte à tronquer $L$, on peut le remplacer par un complexe borné et dont les composants vérifient la condition de Mittag-Leffler, ce qui achèvera la démonstration d'après (CD I 4.2. (c) et (d)). Si $p$ est un entier tel que $K^q = 0$ $(q \geq p)$, le morphisme $w_p$ se factorise $\Ker(d^p_L)$, et, quitte à tronquer $L$ au degré $p$, il nous suffit de voir que $\Ker(d^p_L)$ vérifie la condition de Mittag-Leffler, ce qui est immédiat puisqu'il est isomorphe dans $A-\fsc(\pt)$ au système projectif $\text{Im}(L^{p-1})$, lui-même quotient du système projectif strict $L^{p-1}$.
\vskip .3cm
{
Proposition {\bf 2.4}. --- \it Le bifoncteur dérivé du produit tensoriel induit des bifoncteurs
\begin{itemize}
    \item[(i)] $\D^-_\lambda(X, A) \times \D^-_\lambda(X, A) \to \D^-_\lambda(X, A)$ \quad ($\lambda = c$ ou $t$).
    \item[(ii)] $\D^b_\lambda(X, A)_{\torf} \times \D^+_\lambda(X, A) \to \D^+_\lambda(X, A)$ \quad ($\lambda = c$ ou $t$).
    \item[(iii)] $\D_{\coh}(X, A) \times \D_{\coh}(X, A) \to \D_{\coh}(X, A)$.
    \item[(iv)] $\D_{\parf}(X, A) \times \D_{\parf}(X, A) \to \D_{\parf}(X, A)$.
\end{itemize}
Si de plus l'anneau $A$ est local régulier d'idéal maximal $J$, le bifoncteur (I 7.2.4) induit des bifoncteurs
\begin{itemize}
    \item[(v)] $\D^*_\lambda (X, A) \times \D^*_\lambda (X, A) \to \D^*_\lambda (X, A)$, avec $* = b$ ou $+$, et $\lambda = c$ ou $t$. 
\end{itemize}
}
\vskip .3cm
{\bf Preuve} : Notons respectivement $E$ et $F$ les complexes à droite et à gauche dans le premier membre. Pour (i), (iii) et (v), on se ramène au moyen du ``way-out functor lemma'' (H I 7.1) au cas où $E$ et $F$ sont réduits au degré 0, et alors on conclut par (1.24.(i)). Pour (ii), on se ramène par way-out functor lemma au cas où $F$ est borné et alors, compte tenu de ce que $E$ est de tor-dimension finie, l'assertion est conséquence de (i). Enfin, la partie (iv) résulte de (iii) et du fait que le produit tensoriel dérivé de deux complexes de tor-dimension finie est lui-même de tor-dimension finie.
\vskip .3cm
{
Proposition {\bf 2.5}. --- \it Le bifoncteur $\bRd \cHom_A$ induit des bifoncteurs
\begin{itemize}
    \item[(i)] $(\D^-_t(X, A))^\circ \times \D^+_\lambda(X, A) \to \D^+_\lambda(X, A)$ \quad $(\lambda = c$~ou~$t)$.
\end{itemize}
Lorsque $A$ est local régulier d'idéal maximal $J$, il induit des bifoncteurs exacts
\begin{itemize}
    \item[(ii)] $(\D^b_t(X, A))^\circ \times \D^b_\lambda(X, A) \to \D^b_\lambda(X, A)$ \quad $(\lambda = \emptyset, c$~ou~$t)$.
\end{itemize}
Enfin, supposons que pour toute $A$-algèbre de type fini $B$ annulée par une puissance de $J$, et tout couple $(M, N)$ de $B$--Modules constructibles, les $B$--Modules $\cExt^p_B(M, N)$ $(p \in \mathbf{N})$ soient constructibles. Alors, le bifoncteur $\bRd \cHom_A$ induit un bifoncteur
\begin{itemize}
    \item[(iii)] $(\D^-_c(X, A))^\circ \times \D^+_c(X, A) \to \D^+_c(X, A)$.
\end{itemize}
}
\vskip .3cm
{\bf Preuve} : Soient $E \in \D^-(X, A)$ et $F \in D^+(X, A)$. Pour voir (i), on se ramène par le way-out functor lemma au cas où $E$ et $F$ sont réduits au degré $0$, et alors l'assertion résulte de (1.26. (i) et (ii)). L'assertion (ii) se déduit sans peine de (i) et (1.26.(iii)). Enfin, l'assertion (iii) se voit de même que (i), en utilisant cette fois (1.26.(iv)).
\vskip .3cm
{
Proposition {\bf 2.6}. --- \it Soient $K \in \D^-_c(X, A)$, $L \in \D^-(X, A)$ et $M \in \D^+(X, A)$. Alors, le morphisme de Cartan
$$
\bRd \cHom_A(K \boldsymbol{\otimes}_A L, M) \to \bRd \cHom_A(K, \bRd \cHom_A(L, M))
\leqno{(I 7.6.2)}
$$
est un \emph{isomorphisme}. Si de plus $X$ est noethérien, les morphismes
$$
\bRd \overline{\Hom}_A(K \boldsymbol{\otimes}_A L, M) \to \bRd \overline{\Hom}_A(K, \bRd \cHom_A(L, M))
\leqno{(I 7.6.3)}
$$
$$
\bRd \Hom_A(K \boldsymbol{\otimes}_A L, M) \to \bRd \Hom_A(K, \bRd \cHom_A(L, M))
\leqno{(I 7.6.4)}
$$
$$
\Hom_A(K \boldsymbol{\otimes}_A L, M) \to \Hom_A(K, \bRd \cHom_A(L, M))
\leqno{(I 7.6.5)}
$$
sont aussi des \emph{isomorphismes}.
}
\vskip .3cm
{\bf Preuve} : La définition des trois derniers morphismes à partir du premier au moyen de (I 7.4.18) montre qu'il suffit de voir que (I 7.6.2) est un isomorphisme. On peut pour cela supposer $L$ quasilibre et $M$ flasque. Ceci dit, les foncteurs exacts
$$
\bRd \cHom_A(K \boldsymbol{\otimes}_A L, .) \quad \text{et} \quad \bRd \cHom_A(K, \bRd \cHom_A(L, .))
$$
de $\D^+(X, A)$ dans $\D^+(X, A)$ possèdent la propriété de ``décalage à droite'' ([H] I 7), ce qui permet de se ramener au cas où $M$ est de plus réduit au degré 0. Dans ce cas, fixant $K$ et $M$, les foncteurs exacts
$$
\bRd \cHom_A(K \boldsymbol{\otimes}_A ., M) \quad \text{et} \quad \bRd \cHom_A(K, \bRd \cHom_A(., M))
$$
possèdent également la propriété de décalage à droite, ce qui permet de se ramener au cas où de plus $L$ est réduit au degré 0. Enfin, un dernier argument de décalage permet de supposer que $K$ est réduit au degré 0 et que $K^0$ est un $A$-faisceau constructible. Pour montrer l'assertion dans ce dernier cas, on peut, quitte à localiser, supposer $X$ noethérien. Alors, il est immédiat que $K$ est quasi-isomorphe à un complexe quasilibre borné supérieurement, tel que pour tout $n \in \mathbf{Z}$, le $A$-faisceau $K^n$ ait ses composants constructibles. Finalement, on peut supposer $K$ et $L$ réduits au degré 0, quasilibres, que $K^0$ a des composants constructibles et que $L$ est flasque. Alors, il résulte de (I 6.3.8) que le morphisme de complexes (I 7.6.1) est un isomorphisme, d'où l'assertion.
\vskip .3cm
{
Proposition {\bf 2.7}. --- \it Soient $E \in \D^-_{\parf}(X, A)$, $F \in \D^+(X, A)$ et $G \in \D(X, A)$. Le morphisme 
$$
m: \bRd \cHom_A(E, F) \boldsymbol{\otimes}_A G \to \bRd \cHom_A(E, F \boldsymbol{\otimes}_A G)
\leqno{(I 7.6.9.2)}
$$
est un \emph{isomorphisme} dans chacun des cas suivants:
\begin{itemize}
    \item[(i)] L'anneau $A$ est local régulier d'idéal maximal $J$, et $G \in \D^+(X, A)$.
    \item[(ii)] $F \in \D^b_c(X, A)$ et $G \in \D^-_c(X, A)_{\torf}$.
\end{itemize}
}
\vskip .3cm
{\bf Preuve} : Pla\c{c}ons-nous d'abord dans le cas (i). Par dévissage, on se ramène au cas où $F$ et $G$ sont bornés. Alors les deux membres sont à cohomologie bornée supérieurement (2.5.(ii)). Notant alors $u: A \to A/J$ le morphisme d'anneaux canonique, il nous suffit (I 8.2.2) de montrer que $\bLd u^* (m)$ est un isomorphisme. Utilisant (I 8.1.11.(ii) et (iv)), on voit qu'on peut remplacer $A$ par $A/J$. En effet, le complexe $\bLd u^* (E)$ est de tor-dimension finie (I 8.1.11.(i)) et il est immédiat que sa cohomologie est constante tordue constructible (nous reviendrons d'ailleurs plus loin sur ce point). Ceci dit, on peut supposer $X$ quasicompact; alors, l'équivalence (I 8.2.6) permet de se ramener à l'assertion analogue dans la catégorie des $(A/J)$--Modules (SGA6 I 7.6). Dans l'hypothèse (ii), la cohomologie des deux membres est constructible, ce qui (1.12.4) de vérifier l'assertion sur les fibres. Utilisant (I 6.4.2), on est ainsi ramené au cas où $X$ est le topos ponctuel. Mais alors, grâce à (2.3), c'est une conséquence immédiate de l'assertion analogue pour les $\hat{A}$--modules de type fini.  

Pour énoncer le corollaire suivant, on posera pour tout $E$ appartenant à $D^-_{\parf}(X, A)$
$$
\check{E} = \bRd \cHom_A(E, A).
$$
Il est clair que lorsque $A$ est local régulier d'idéal maximal $J$, on a 
$$
E \in \D^+_{\parf}(X, A),
$$
mais j'ignore si c'est vrai sans hypothèse sur l'anneau $A$.
\vskip .3cm
{
Corollaire {\bf 2.8}. --- \it Soient $E \in \D^-_{\parf}(X, A)$ et $F \in \D^+(X, A)$. Le morphisme canonique
$$
m: \check{E} \boldsymbol{\otimes}_A F \to \bRd \cHom_A(E, F)
$$
est un isomorphisme lorsque $A$ est local régulier d'idéal maximal $J$, ou lorsque $F \in \D^b_c(X, A)_{\torf}$. C'est le cas en particulier lorsque $E \in \D^-_{\parf}(X, A)$ et $F \in \D^b_{\parf}(X, A)$, de sorte que le complexe
$$
\bRd \cHom_A(E, F)
$$
est \emph{parfait} lorsque de plus $A$ est local régulier d'idéal maximal $J$.
}
\vskip .3cm
{\bf 2.9}. Supposons maintenant pour simplifier que $A$ est local régulier d'idéal maximal $J$. Soient $E, E' \in \D^-(X, A)$ et $F, F' \in D^+(X, A)$. On suppose que $E \in \D^-_c(X, A)$. Nous allons définir un morphisme fonctoriel 
$$
\bRd \cHom_A(E, F) \boldsymbol{\otimes}_A \bRd \cHom_A(E', F') \to \bRd \cHom_A(E \boldsymbol{\otimes}_A E', F \boldsymbol{\otimes}_A F').
\leqno{(2.9.1)}
$$
Considérons pour cela le diagramme
\[\begin{tikzcd}
	{\bRd \cHom_A(E, F) \boldsymbol{\otimes}_A\bRd\cHom_A(E', F')} && {\bRd \cHom_A(E \boldsymbol{\otimes}_AE', F\boldsymbol{\otimes}_AF') } \\
	{\bRd \cHom_A(E, F) \boldsymbol{\otimes}_A\bRd\cHom_A(E', F'))} && {\bRd \cHom_A(E, \bRd \cHom_A(E', F \boldsymbol{\otimes}_A F')).}
	\arrow["{(1)}"', from=1-1, to=2-1]
	\arrow["{(3)}"', from=1-3, to=2-3]
	\arrow["{(2)}", from=2-1, to=2-3]
	\arrow[dashed, from=1-1, to=1-3]
\end{tikzcd}\]
La flèche (1) n'est autre que (I 7.6.9.2), qui existe puisque $\bRd\cHom_a(E', F') \in \D^+(X, A)$. La flèche (2) est obtenue en appliquant le foncteur $\bRd \cHom_A(E, .)$ à la flèche (I 7.6.2), pour $E'$, $F'$ et $F$. Enfin, la flèche (3) est le morphisme de Cartan (I 7.6.9.2). Comme $E \in \D^-_c(X, A)$, cette dernière est un isomorphisme (2.6), ce qui permet de définir (2.9.1) comme l'unique flèche (en pointillé) rendant le diagramme ci-dessus commutatif.

Si maintenant on a aussi $E' \in \D^-_c(X, A)$, on définit, en échangeant les rôles de $E$ et $E'$, et $F$ et $F'$ respectivement, un autre flèche et on vérifie qu'elle coïncide avec la première.
\vskip .3cm
{
Proposition {\bf 2.9.2}. --- \it On suppose que l'anneau $A$ est local régulier d'idéal maximal $J$. Soient $E, E' \in \D^-_{\parf}(X, A)$ et $F, F' \in \D^+(X, A)$. Alors, le morphisme canonique
$$
\bRd \cHom_A(E, F) \boldsymbol{\otimes}_A \bRd \cHom_A(E', F') \to \bRd \cHom_A(E \boldsymbol{\otimes}_A E', F \boldsymbol{\otimes}_A F')
\leqno{(2.9.1)}
$$
est un \emph{isomorphisme}.
}
\vskip .3cm
{\bf Preuve} : D'après (2.7), les flèches (1) et (2) du diagramme ci-dessus sont des isomorphismes.
\vskip .3cm
{\bf 2.10}. Supposons maintenant le topos $X$ \emph{noethérien}, et que l'anneau $A$ est local régulier d'idéal maximal $J$. Étant donné $E \in \D^b_{\parf}(X, A)$, le complexe $\check{E}$ appartient aussi à $\D^b_{\parf}(X, A)$, et le morphisme de Cartan (I 7.6.5)
$$
\Hom_A(\check{E} \boldsymbol{\otimes}_A E, A) \to \Hom_A(\check{E}, \check{E})
$$
est une bijection. En particulier, l'identité de $\check{E}$ correspond à un morphisme 
$$
E \boldsymbol{\otimes}_A \check{E} \to A.
\leqno{(2.10.1)}
$$
\vskip .3cm
{
Proposition {\bf 2.10.2}. --- \it Soit $F \in \D^+(X, A)$. Il existe un morphisme fonctoriel en $F$
$$
E \boldsymbol{\otimes}_A \bRd \cHom_A (E, F) \to F,
$$
qui ``coïncide'' avec (2.10.1) lorsque $F = A$. 
}
\vskip .3cm
{\bf Preuve} : En tensorisant par l'identité de $E$ l'isomorphisme (2.8) on obtient un isomorphisme
$$
a: E \boldsymbol{\otimes}_A \check{E} \boldsymbol{\otimes}_A F \isomlong E \boldsymbol{\otimes}_A \bRd \cHom_A(E, F).
$$
Par ailleurs, on définit, en tensorisant par l'identité de $F$ le morphisme (2.10.1), un morphisme
$$
b: E \boldsymbol{\otimes}_A \check{E} \boldsymbol{\otimes}_A F \to F.
$$
Le morphisme annoncé est le composé $b \circ a^{-1}$.

Lorsque $F \in \D^b_c(X, A)$, il en est de même de $\bRd \cHom_A(E, F)$, et le morphisme de Cartan
$$
\Hom_A\bRd \cHom_A(E, F) \boldsymbol{\otimes}_A E, F) \to \Hom_A(\bRd \cHom_A (E, F), \bRd \cHom_A(E, F))
$$
est une bijection. On laisse alors au lecteur le soin de vérifier que le morphisme (2.10.2) correspond à l'identité de $\bRd \cHom_A(E, F)$ dans cette bijection.
\vskip .3cm
{
Proposition {\bf 2.10.3}. --- \it Soit $E \in \D^b_{\parf}(X, A)$. Pour tout $F \in \D^+(X, A)$, il existe un morphisme canonique
$$
E \to \bRd \cHom_A(\bRd \cHom_A(E, F), F).
$$
}
\vskip .3cm
{\bf Preuve} : On prend l'image de (2.10.2) par le morphisme de Cartan
$$
\Hom_A(E \boldsymbol{\otimes} \bRd \cHom_A(E, F), F) \to \Hom_A(E, \bRd \cHom_A(\bRd \cHom_A (E, F), F)).
$$
En particulier, pour $F = A$, on déduit de (2.10.3) un morphisme 
$$
E \to (\check{E})~\check{}.
\leqno{(2.10.4)}
$$
\vskip .3cm
{
Proposition {\bf 2.10.5}. --- \it Soit $E \in \D^b_{\parf}(X, A)$. Le morphisme
$$
E \to (\check{E})~\check{}
$$
ci-dessus est un \emph{isomorphisme}.
}
\vskip .3cm
{\bf Preuve} : Comme les deux membres sont à cohomologie constructible, on est ramené à vérifier l'assertion sur les fibres (1.12.4).

Grâce à (I 6.4.2), on peut alors supposer que $X$ est le topos ponctuel. Enfin, la proposition (2.3) montre que dans ce cas, l'assertion (2.10.5) est conséquence de l'assertion analogue pour les complexes parfaits de $\hat{A}$--modules (SGA 6 I 7.2). 
\vskip .3cm
{\bf 2.11. Trace et cup-produit}.

On suppose que $A$ est local régulier et que $J$ est son idéal maximal. Étant donné $E \in \D_{\parf}(X, A)$, nous allons définir un morphisme \emph{trace}
$$
\tr: \Hom_A(E, E) \to \Gamma(X, A),
$$
satisfaisant au formalisme développé dans (SGA6 I 8), à l'exception de l'additivité qui est d'ailleurs énoncée de fa\c{c}on erronée dans (loc.cit.).

Supposons tout d'abord que $X$ soit noethérien et $E \in \D^b_{\parf}(X, A)$. Alors il existe une flèche naturelle
$$
\bRd \cHom_A(E, E) \to A,
\leqno{(2.11.1)}
$$
composée de l'isomorphisme inverse de (2.8) $\bRd \cHom_A(E, E) \isomlong \check{E} \boldsymbol{\otimes}_A E$ et du morphisme (2.10.1). Appliquant à (2.11.1) le foncteur $\mathrm{H}^0(X, .)$, on obtient, compte tenu de l'isomorphisme de Cartan, une application $A$-linéaire $\Hom_A(E, E) \to \Gamma(X, A)$, qui est le morphisme trace annoncé lorsque $X$ est noethérien. Dans le cas général, comme le préfaisceau
$$
U \mapsto \mathrm{H}^0(U, A)
$$
est un faisceau (I 3.9), les morphismes traces précédemment définis sur les ouverts noethériens de $X$ se recollent pour fournir le morphisme trace annoncé. Il est immédiat de vérifier que, sur les fibres, il induit, compte tenu de l'équivalence (2.3.2), le morphisme trace défini dans (SGA6 I 8).

Plus généralement, étant donnés $E$ et $F \in \D_{\parf}(X, A)$, un accouplement
$$
(\quad , \quad): \Hom_A(E, F) \otimes_{\mathrm{H}^0(X, A)}\Hom_A(F, E) \to \mathrm{H}^0(X, A). 
$$
que nous appellerons \emph{cup-produit}. Pour cela, on se ramène comme précédemment à le définir lorsque $X$ est noethérien et les complexes $E$ et $F$ sont bornés. Dans ce cas, on dispose d'un homomorphisme canonique
$$
\bRd \cHom_A(E, F) \boldsymbol{\otimes}_A \bRd \cHom_A(F, E) \to A,
\leqno{(2.11.2)}
$$
que l'on construit comme suit. D'après (2.8), il s'agit de définir un accouplement
$$
\check{E} \boldsymbol{\otimes}_A F \boldsymbol{\otimes}_A \check{F} \boldsymbol{\otimes}_A E \to A.
$$
On prend le produit tensoriel des accouplements (2.10.1) associés à $E$ et $F$ respectivement. Montrons maintenant comment déduire le cup-produit de (2.11.2). D'après l'isomorphisme de Cartan (2.6), il s'agit, étant donnés deux morphismes
$$
u: A \to \bRd \cHom_A(E, F)
$$
et
$$
v: A \to \bRd \cHom_A(F, E),
$$
d'en définir un de $A$ dans $A$. On prend le morphisme composé de (2.11.1) et de $u \boldsymbol{\otimes}_A v$. 
\vskip .3cm
{
Proposition {\bf 2.11.3}. --- \it Soient $E$ et $F \in \D_{\parf}(X, A)$.
\begin{itemize}
    \item[(i)] Étant donnés deux morphismes $u: E \to F$ et $v: F \to E$, on a 
    $$
    (u, v) = (v, u) = \tr(v \circ u) = \tr(u \circ v).
    $$
    \item[(ii)] Étant donnés un morphisme $u: E \to E$ et un isomorphisme $s: E \to F$, on a 
    $$
    \tr(s \circ u \circ s^{-1}) = \tr(u).
    $$
    \item[(iii)] Étant donnés deux morphismes $u: E \to E$ et $v: F \to F$, on a : 
    $$
    \tr(u \boldsymbol{\otimes} v) = \tr(u) \tr(v).
    $$
\end{itemize}
}
\vskip .3cm
{\bf Preuve} : Il est clair que (i) $\Rightarrow$ (ii). Par passage aux fibres, et compte tenu de l'équivalence (2.3.2), les assertions (i) et (iii) résultent des assertions analogues pour les complexes parfaits de $\hat{A}$--modules (SGA6 I 8.3 et 8.7).
\vskip .3cm
{\bf 2.12}. Nous allons maintenant expliciter pour la commodité des références un certain nombre de compatibilités de la notion de constructibilité avec les opérations externes, qui sont pour la plupart évidentes et ont déjà été utilisées librement dans les numéros précédents.
\vskip .3cm
{
Proposition {\bf 2.12.1}. --- \it Soit $f: X \to Y$ un morphisme de topos localement noethériens. 
\begin{itemize}
    \item[(i)] Le foncteur $f^*$ transforme $A$-faisceau constructible (resp. constant tordu constructible) en $A$-faisceau constant tordu constructible. 
    \item[(ii)] Le foncteur $f^*: \D(Y, A) \to \D(X, A)$ induit des foncteurs 
    $$
    \D^*_\lambda(Y, A) \to \D^*_\lambda(X, A) \quad (* = \emptyset, +, -, b~ \text{et}~ \lambda= c, t).
    $$
    ($A$ rég., $J$ id. max.) $\D_{\parf}(Y, A) \to \D_{\parf}(X, A)$.
    $$
    \D_{\coh}(Y, A) \to \D_{\coh}(X, A).
    $$
    \item[(iii)] Étant données $E \in \D^-_t(Y, A)$ et $F \in \D^+(Y, A)$, le morphisme canonique (I 7.7.2.(ii))
    $$
    f^* \bRd \cHom_A(E, F) \to \bRd \cHom_A(f^* E, f^* F)
    $$
    est un \emph{isomorphisme}.
\end{itemize}
}
\vskip .3cm
{\bf Preuve} : Dans le cas constant tordu constructible, l'assertion (i) a été déjà vue (1.20.4). Dans le cas constructible, on se ramène au cas où $F$ est $J$-adique constructible, où c'est immédiat. L'assertion (ii) résulte immédiatement de (i); pour la perfection, on suppose $A$ régulier d'idéal maximal $J$, car je ne sais pas en général si l'image réciproque d'un $A$-faisceau plat est un $A$-faisceau plat. Enfin, (iii) résulte de (I 6.4.2).  
\vskip .3cm
{
Proposition {\bf 2.12.2}. --- \it Soient $X$ un topos localement noethérien, $T$ et $T'$ deux objets de $X$, et $f: T \to T'$ un morphisme quasicompact.
\begin{itemize}
    \item[(i)] Le foncteur (I 7.7.9) $\bRd f_!: \D(T, A) \to \D(T', A)$ induit des foncteurs
    $$
    \D^*_c(T, A) \to \D^*_c(T', A) \quad (* = \emptyset, -, +~\text{ou}~b). 
    $$
    \item[(ii)] Si $f$ est une immersion ouverte, le foncteur $f^*$ induit un foncteur
    $$
    f^*: \D_{\parf}(T', A) \to \D_{\parf}(T, A).
    $$
\end{itemize}
}
\vskip .3cm
{\bf Preuve} : Comme le foncteur $f_!$ est exact et transforme $A$--Module constructible en $A$-Module constructible (SGA4 \quad), il transforme $A$-faisceau $J$-adique constructible en $A$-faisceau $J$-adique constructible, d'où aussitôt (i). L'assertion (ii) provient de ce que l'on sait dans ce cas (I 5.18.5 (ii)) que le foncteur $f^*$ transforme $A$-faisceau plat en $A$-faisceau plat.
\vskip .3cm
{
Proposition {\bf 2.12.3}. --- \it Soient $X$ un topos, $U$ un ouvert de $X$ et $j: Y \to X$ l'immersion fermée complémentaire.
\begin{itemize}
    \item[(i)] Le foncteur $\bRd j_*: \D(Y, A) \to \D(X, A)$ induit des foncteurs
    $$
    \D^*_c(Y, A) \to \D^*_c(X, A) \quad (* = \emptyset, -, +~\text{ou}~b).
    $$
    \item[(ii)] Le foncteur $j^*$ induit un foncteur
    $$
    j^*: \D_{\parf}(X, A) \to \D_{\parf}(Y, A).
    $$
    \item[(iii)] On suppose que pour toute $A$-algèbre de type fini $B$ annulée par une puissance de $J$, et tout couple $(M, N)$ de $B$--Modules constructibles les $B$--Modules $\cExt^p_B(M, N)$ $(p \in \mathbf{N})$ soient constructibles. Alors le foncteur $\bRd j^!$ (I 7.7.11) induit un foncteur exact
    $$
    \bRd j^!: \D^+_c(X, A) \to \D^+_c(Y, A).
    $$    
\end{itemize}
}
\vskip .3cm
{\bf Preuve} : Les assertions (i) et (ii) se voient comme les assertions analogues de (2.12.2), en utilisant (I 5.19.1 (ii)) pour la deuxième.Quant à (iii), elle résulte, compte tenu de l'isomorphisme (I 7.7.13) de (2.5 (iii)).

\vskip .3cm
{\bf 2.13. Changement d'anneau}. 

Soient $X$ un topos localement noethérien, $A$ et $B$ deux anneaux commutatifs unifères noethériens, $J$ et $K$ deux idéaux de $A$ et $B$ respectivement et $u: A \to B$ un morphisme d'anneaux unifères, tel que $u(J) \subset K$. On utilise par ailleurs librement les notations de (I 8.1).
\vskip .3cm
{
Proposition {\bf 2.13.1}. --- \it 
\begin{itemize}
    \item[(i)] Le foncteur $\bLd u^*: \D^-(X, A) \to \D^-(X, B)$ induit des foncteurs exacts
    \[\begin{tikzcd}
	{\D^-_c(X, A)} && {\D^-_c(X, B)} \\
	{\D^-_t(X, A)} && {\D^-_t(X, B)} \\
	{\D^-_{\parf}(X, A)} && {\D^-_{\parf}(X, B).}
	\arrow[from=1-1, to=1-3]
	\arrow[from=2-1, to=2-3]
	\arrow[from=3-1, to=3-3]
    \end{tikzcd}\]
    Si de plus $A$ est régulier d'idéal maximal $J$, il induit des foncteurs
    \[\begin{tikzcd}
	{\D^*_c(X, A)} && {\D^*_c(X, B)} && {(*= \emptyset, +~\text{ou}~b)} \\
	{\D^*_t(X, A)} && {\D^*_t(X, B)} && {(*= \emptyset, +~\text{ou}~b)} \\
	{\D^*_{\parf}(X, A)} && {\D^*_{\parf}(X, B)} && {(*= \emptyset, +~\text{ou}~b).}
	\arrow[from=1-1, to=1-3]
	\arrow[from=2-1, to=2-3]
	\arrow[from=3-1, to=3-3]
    \end{tikzcd}\]
    \item[(ii)] Si $B$ est une $A$-algèbre finie, le foncteur $u_*: \D(X, B) \to \D(X, A)$ induit des foncteurs exacts
    \[\begin{tikzcd}
	{\D_c(X, B)} && {\D_c(X, A)} \\
	{\D_t(X, B)} && {\D_t(X, A).}
	\arrow[from=1-1, to=1-3]
	\arrow[from=2-1, to=2-3]
    \end{tikzcd}\]
\end{itemize}
}
\vskip .3cm
{\bf Preuve} : Montrons (i). L'assertion concernant les complexes parfaits découle de celle concernant les complexes à cohomologie constante tordue constructible et de (I 8.1.11 (i)). Montrons par exemple que si $E \in \D_c(X, A)$, alors $\bLd u^* (E) \in \D_c(X, B)$. Dans chacun des cas envisagés, on est ramené grâce à ([H], I 7.3) au cas où $E$ est réduit au degré 0, associé à un $A$-faisceau constructible noté de même. Il s'agit alors de voir que pour tout $p \in \mathbf{N}$, le $B$-faisceau
$$
\cTor^A_p(B, E) = (\cTor^{A_n}_p(B_n, E_n))_{n \in \mathbf{N}} = (F_n)_{n\in \mathbf{N}}
$$
est constructible. Pour tout $n \in \mathbf{N}$, le calcul de $F_n$ au moyen d'une résolution plate et constructible de $E_n$ montre que c'est un $B_n$--Module constructible. Il nous suffit donc de voir que $\cTor^A_p(B, E)$ est de type $J$-adique. Pour cela, on se ramène grâce à (1.21) et (1.22) au cas où $E$ est $J$-adique constant tordu constructible. Supposons alors $X$ connexe, et choisissons un point $a$ de $X$; comme le foncteur fibre défini par $a$ est conservatif pour les $B$--Modules localement constants, on est ramené à voir l'assertion pour la fibre de $\cTor^A_p(B, E)$. On peut donc supposer que $X$ est le topos ponctuel. Utilisant alors une résolution libre de type fini du $\check{A}$--module de type fini associé à $E$ (1.20.5), on se ramène au cas où $E$ est localement libre constructible et $J$-adique, et alors l'assertion est immédiate, car
$$
\cTor^A_p(B, E) = 0 \quad (p \geq 1) \quad \text{et} \quad B \otimes_A E \isom B^r,
$$
pour un $r \in \mathbf{N}$. Montrons (ii), dans le cas constructible par exemple. Si $E$ est un $B$-faisceau $J$-adique constructible, ses composants sont des $A$--Modules constructibles ($B$ est une $A$-algèbre finie), donc il est aussi constructible en tant que $A$-faisceau ; d'où l'assertion.

Dans la suite du numéro, nous noterons pour tout entier $n \geq 0$
$$
u_n: A \to (A/J^{n+1})
$$
le morphisme d'anneaux canonique. On définit un bifoncteur cohomologique
$$
(\widehat{\Ext}^i(. , .))_{i \in {\textbf{Z}}}: \D^-(X, A) \times \D^-(X, A) \to \mathcal{E}(\pt, J),
$$
en posant pour tout couple $(E, F)$ d'objets de $\D^-(X, A)$ et tout $i \in \mathbf{Z}$
$$
\widehat{\Ext}^i(E, F) = (\Ext^i_{A_n}(\bLd u^*_n(E), \bLd u^*_n(F)))_{n \in \mathbf{N}}
$$
Lorsque $A$ est régulier, d'idéal maximal $J$, le bifoncteur cohomologique précédent se prolonge en un bifoncteur cohomologique
$$
\D(X, A) \times \D(X, A) \to \mathcal{E}(\pt, J),
$$
de manière évidente.
\vskip .3cm
{
Théorème {\bf 2.13.2}. --- \it On suppose $A$ régulier d'idéal maximal $J$, que le topos $X$ est noethérien de dimension topologique stricte finie, et qu'il vérifie de plus les deux propriétés suivantes.
\begin{itemize}
    \item[(i)] Pour toute $A$-algèbre de type fini $B$ annulée par une puissance de $J$, et tout couple $(M, N)$ de $B$--Modules constructibles, les $B$--Modules $\cExt^p_B(M, N)$ $(p \in \mathbf{N})$ sont constructibles.
    \item[(ii)] Pour tout $A$-algèbre de type fini $B$ annulée par une puissance de $J$ et tout $B$--Modules constructible $M$, les $B$--modules
    $$
    \mathrm{H}^p(X, M) \quad (p \in \mathbf{N})
    $$
    sont de type fini.
\end{itemize}
Alors, étant données $E \in \D^-_c(X, A)$ et $F \in \D^+_c(X, A)$, les systèmes projectifs
$$
\widehat{\Ext}^i_A(E, F) \quad (i \in \mathbf{Z})
$$
sont des $A$-faisceaux \emph{constructibles} sur le topos ponctuel, et les applications canoniques évidentes
$$
\Ext^i_A(E, F) \to \varprojlim_n \Ext^i_{A_n}(\bLd u^*_n(E), \bLd u^*_n(F))
$$
sont des \emph{bijections}. En particulier, les $\hat{A}$--modules
$$
\Ext^i_A(E, F)
$$
sont \emph{de type fini}.
}
\vskip .3cm
{\bf Preuve} : Les isomorphismes de Cartan (2.6)
$$
\mathrm{H}^i(X, \bRd \cHom_{A_n}(\bLd u^*_n(E), \bLd u^*_n(F))) \isomlong \Ext^i_{A_n}(\bLd u^*_n(E), \bLd u^*_n(F))
$$
et l'isomorphisme (I 8.1.11 (iv))
$$
\bLd u^*_n \bRd \cHom_A(E, F) \isomlong \bRd \cHom_{A_n}(\bLd u^*_n (E), \bLd u^*_n(F)))
$$
montrent que 
$$
\widehat{\Ext}^i_A(E, F) \isom \widehat{\Ext}^i_A(A, \bRd \cHom_A(E, F)).
$$
Or le complexe $\bRd \cHom_A(E, F)$ est à cohomologie constructible (2.5 (iii)) donc, pour voir que $\widehat{\Ext}^i_A(E, F)$ est un $A$-faisceau constructible, on peut supposer que $E = A$. Dans ce cas, la suite spectrale canonique
$$
E^{p, q}_2 = \widehat{\Ext}^p_A(A, \mathrm{H}^q(F)) \Rightarrow \widehat{\Ext}^{p+q}_A(E, F),
$$
construite de fa\c{c}on habituelle au moyen des couples exacts, permet de se ramener de plus au cas où $F$ est réduit au degré 0 et défini par un $A$-faisceau constructible noté de même. Posant pour simplifier pour tout entier $i$ et tout $A$-faisceau $M$,
$$
\hat{\mathrm{H}}^i(M) = \widehat{\Ext}^i_A(A, M),
$$
on doit alors prouver le lemme suivant.
\vskip .3cm
{
Lemme {\bf 2.13.3}. --- \it Pour tout $A$-faisceau constructible $M$ sur $X$, les $A$-faisceaux $\hat{\mathrm{H}}^i(M)$ sont constructibles.
}
\vskip .3cm
Avant de le faire, dégageons le résultat préliminaire suivant.
\vskip .3cm
{
Lemme {\bf 2.13.4}. --- \it Désignant par $d$ la dimension topologique stricte de $X$, on a pour tout $A$-faisceau $N$ sur $X$
$$
\hat{\mathrm{H}}^i(N) = 0 \quad (i \geq d+1).
$$
}
\vskip .3cm
En effet, étant donné un entier $p \geq 0$, le complexe $\bLd u^*_p(N)$ a sa cohomologie concentré en degrés $\leq 0$. Comme $\mathrm{H}^i(X, G) = 0$ \quad $(i \geq d+1)$ pour tout $A_p$-faisceau $G$, on a une suite spectrale birégulière 
$$
\mathrm{H}^i(X, \Ld u^j_p(N)) \Rightarrow \hat{H}^{i+j}(N)_p
$$
qui permet aussitôt de conclure.

Montrons maintenant (2.13.3). Nous allons le voir par récurrence croissante sur l'entier
$$
\text{dim}_A(M) =~\text{dim}_A(A/\ann(M)), 
$$
appelé dimension de $M$. Si dim$_A(M) = 0$, le $A$-faisceau $M$ est annulé par une puissance de $J$, donc il existe un entier $n \geq 0$ tel que $M$ soit défini par un $A_n$--Module constructible. Soit alors $P$ une résolution à gauche de $M$ par des $A_n$--Modules de la forme $i_!(A_n)$, où $i: T \to e_X$ désigne l'unique morphisme d'un objet noethérien de $X$ dans l'objet final. Grâce à (2.13.4), on a une suite spectrale birégulière
$$
E^{p, q}_1 = \widehat{\Ext}^q_A(A, P^p) \Rightarrow \widehat{\Ext}^{p+q}_A(A, M),
$$
qui permet de se ramener au cas où $M$ est de la forme $i_!(A_n)$. Dans ce cas, soit $L$ une résolution à gauche de $A_n$ par des $\hat{A}$--modules libres de type fini. A nouveau, la suite spectrale
$$
E^{p, q}_1 = \hat{\mathrm{H}}^q(i_!(L^p)) \Rightarrow \hat{\mathrm{H}}^{p+q}(i_!(A_n)),
$$
permet de supposer que $M$ est de la forme $i_!(L^p)$ pour un entier $p$, donc est constructible et \emph{plat}. Il est alors clair que pour tout $i$
$$
\hat{\mathrm{H}}^i(M) = (\mathrm{H}^i(X, M_n))_{n \in \mathbf{N}}.
$$
Dans ce cas, l'hypothèse (ii) de l'énoncé permet de conclure grâce au lemme de SHIH (SGA5 V 5.3.1). Soit maintenant $r$ un entier $\geq 0$, et supposons l'assertion vraie pour tous les $A$-faisceau constructibles de dimension $\leq r$. Nous allons voir qu'elle est vraie pour tout $A$-faisceau constructible $M$ de dimension $r+1$. Comme $r+1 > 0$, il existe un élément $a$ de $J$ qui n'appartient à aucun idéal premier minimal dans $\Ass(A/\ann(M))$; on a alors (EGA $0_{IV}$ 16.3.4)
$$
\text{dim}_A(A/(\ann(M) + aA)) = r.
$$
Comme $M$ est noethérien dans la catégorie des $A$-faisceaux constructibles, la suite croissante des sous-$A$-faisceaux
$$
\Ker(a^q: M \to M)
$$
de $M$ est stationnaire. Quitte à remplacer $a$ par une puissance de $a$, on peut donc supposer que l'homothétie
$$
a: M/\Ker(a) \to M/\Ker(a)
$$
est un monomorphisme. La suite exacte
$$
0 \to \Ker(a) \to M \to M/\Ker(a) \to 0
$$
donne lui à une suite exacte illimitée
$$
\dots \hat{\mathrm{H}}^{i-1}(M/\Ker(a)) \to \hat{\mathrm{H}}^i(\Ker(a)) \to \hat{\mathrm{H}}^i(M) \to \hat{\mathrm{H}}^i(M/\Ker(a)) \to \hat{\mathrm{H}}^{i+1}(\Ker(a)) \dots 
$$
Mais $\Ker(a)$ est de dimension $\leq r$ par construction de $a$. Utilisant l'hypothèse de récurrence et le caractère exact de la sous-catégorie des $A$-faisceaux constructibles, on peut donc remplacer $M$ par $M/\Ker(a)$ donc supposer que la multiplication par $a$ définit un monomorphisme sur $M$. Nous allons alors montrer que $\hat{\mathrm{H}}^i(M)$ est constructible par récurrence décroissante sur l'entier $i$, compte tenu de (2.13.4). D'après (loc.cit.), l'assertion est évidente pour $i \geq d+1$. Supposons donné un entier $p$ tel qu'elle soit vraie pour $i \geq p+1$, et montrons qu'elle est vraie pour $p$. Pour tout entier $q$, on a une suite exacte
$$
0 \to M \xlongrightarrow{a^q} M \to M/a^qM \to 0,
$$
d'où une suite exacte illimitée
$$
\dots \to \hat{\mathrm{H}}^p(M) \xlongrightarrow{a^q} \hat{\mathrm{H}}^p(M) \to \hat{\mathrm{H}}^p(M/a^qM) \to \hat{\mathrm{H}}^{p+1}(M) \to \hat{\mathrm{H}}^{p+1}(M) \dots 
$$
Comme dim$_A(M/a^qM) \leq r$, l'hypothèse de récurrence sur la dimension et celle sur l'exposant entraînent que pour tout entier $q > 0$, le $A$-faisceau
$$
\hat{\mathrm{H}}^p(M)/a^q \hat{\mathrm{H}}^p(M)
$$
est constructible. Mais par ailleurs, on a une suite exacte
$$
J^q \hat{\mathrm{H}}^p(M) / a^q \hat{\mathrm{H}}^p(M) \to \hat{\mathrm{H}}^p(M)/a^q \hat{\mathrm{H}}^p(M) \to \hat{\mathrm{H}}^p(M)/J^q \hat{\mathrm{H}}^p(M) \to 0
$$
d'où résulte, comme le terme de gauche vérifie la condition de Mittag-Leffler (c'est un quotient de $J^q \otimes_A (\hat{\mathrm{H}}^p(M)/a^q \hat{\mathrm{H}}^p(M))$) et celui du milieu est constructible, que pour tout entier $q > 0$, le $A$-faisceau
$$
\hat{\mathrm{H}}^p(M) / J^q \hat{\mathrm{H}}^p(M)
$$
est constructible (cf. la preuve de SGA5 V 3.2.4 (i)). L'assertion résulte alors de (1.2 (iii)) et (1.6).

Ceci dit, posant pour tout $i \in \mathbf{Z}$
$$
\ext^i_A(E, F) = \varprojlim_p \Ext^i_A (\bLd u^*_p (E), \bLd u^*_p(F)),
$$
on obtient, grâce à (EGA $0_{III}$ 13.2.2), un bifoncteur cohomologique
$$
\D^-_c(X, A)^\circ \times \D^+_c(X, A) \to \hat{A}-\modn, 
$$
et il reste à voir que le morphisme canonique de bifoncteurs cohomologiques
$$
\Ext^i_A(E, F) \to \ext^i_A(E, F)
$$
est un isomorphisme. On se ramène comme précédemment au cas où $E = A$.

On dispose alors, par la voie des couples exacts, de deux suites spectrales 
$$
\mathbf{E}' \qquad 'E^{p, q}_2 = \Ext^p_A(A, \mathrm{H}^q(F)) \Rightarrow \Ext^p_A(A, F)
$$
$$
\mathbf{E}'' \qquad ''E^{p, q}_2 = \ext^p_A(A, \mathrm{H}^q(F)) \Rightarrow \ext^p_A(A, F)
$$
et d'un morphisme naturel $\mathbf{E}' \to \mathbf{E}''$. Ces deux suites spectrales sont birégulières; pour la première, c'est évident (son support est dans un cadran supérieur droit), et pour la seconde cela résulte du lemme ci-dessous.
\vskip .3cm
{
Lemme {\bf 2.13.5}. --- \it Soit $M$ un $A$-faisceau sur $X$. On a 
$$
\ext^i_A(A, M) = 0 \quad \text{pour} \quad i < -2~\text{dim}(A).
$$
}
\vskip .3cm
En effet, pour tout entier $m$, le complexe $\bLd u^*_m(M)$ est acyclique en dimensions $< -2~\text{dim}(A)$, puisque la catégorie des $A$-faisceaux est de tor-dimension $\leq 2~\text{dim}(A)$ (I 5.16).

La comparaison des deux suites spectrales précédentes permet alors de se ramener au cas où $F$ est réduit au degré 0 est défini par un $A$-faisceau constructible noté de même. Nous allons alors encore une fois raisonner sur la dimension de $F$. Lorsque dim$_A(F) = 0$, l'assertion résulte, au moyen des dévissages utilisés pour prouver la constructibilité des $\mathrm{H}^i(F)$, du lemme suivant, appliqué à un $A$-faisceau constructible et plat de la forme $i_!(A)$, où $i: T \to e_X$ est l'unique morphisme d'un objet noethérien de $X$ dans l'objet final.
\vskip .3cm
{
Lemme {\bf 2.13.6}. --- \it Étant donné un $A$-faisceau constructible $M$ sur $X$, l'application canonique
$$
\mathrm{H}^i(X, M) \to \varprojlim \overline{\mathrm{H}}^i(X, M) \quad (i \in \mathbf{Z})
$$
est une bijection.
}
\vskip .3cm
En effet, le système projectif $\overline{\mathrm{H}}^{i-1}(X, M)$ vérifie la condition de Mittag-Leffler, d'après l'hypothèse (ii) et le lemme de SHIH. Le lemme résulte donc de (I 7.4.16).

Soit maintenant $r$ un entier $\geq 0$, et supposons l'assertion vraie pour tous les $A$-faisceaux constructibles de dimension $\leq r$. Nous allons voir qu'elle est vraie pour tout $A$-faisceau constructible $F$ de dimension $r+1$. On choisit un élément $a$ comme plus haut. Utilisant le lemme des 5, on se ramène à vérifier l'assertion lorsque de plus la multiplication par $a$ définit un monomorphisme de $F$. Nous allons alors montrer l'assertion par récurrence décroissante sur l'entier $i$, en commen\c{c}ant par voir qu'elle est vraie pour les grandes valeurs de $i$. Pour $i \geq d+2$, il résulte de (I 7.4.16) que
$$
\mathrm{H}^i(X, F) = 0,
$$
et nous allons prouver que $h^i(X, F) = \ext^i_A(A, F)$ est également nul.

Pour cela, la suite exacte
$$
0 \to F \xlongrightarrow{a} F \to F/aF \to 0
\leqno{(2.13.7)}
$$
fournit une suite exacte
$$
\mathrm{h}^i(X, F) \xlongrightarrow{a} \mathrm{h}^i(X, F) \to \mathrm{h}^i(X, F/aF).
$$
L'hypothèse de récurrence sur la dimension montre que 
$$
\mathrm{h}^i(X, F/aF) = \mathrm{H}^i(X, F/aF) = 0,
$$
la dernière égalité provenant d'une nouvelle application de (I 7.4.16). Par suite $\mathrm{h}^i(X, F) = a \mathrm{h}^i(X, F)$, d'où $\mathrm{h}^i(X, F) = 0$ par le lemme de Nakayama. Supposons maintenant donné un entier $p$ tel que le morphisme canonique
$$
\mathrm{H}^i(X, F) \to \mathrm{h}^i(X, F)
$$
soit un isomorphisme pour $i \geq p+1$, et montrons que c'est un isomorphisme pour $i = p$. La suite exacte (2.13.7) fournit un diagramme commutatif exact
\[\begin{tikzcd}
	{\mathrm{H}^p(X, F)} & {\mathrm{H}^p(X, F)} & {\mathrm{H}^p(X, F/aF)} & {\mathrm{H}^{p+1}(X, F)} \\
	{\mathrm{h}^p(X, F)} & {\mathrm{h}^p(X, F)} & {\mathrm{h}^p(X, F/aF)} & {\mathrm{h}^{p+1}(X, F),}
	\arrow[from=1-1, to=2-1]
	\arrow[from=1-2, to=2-2]
	\arrow["u", from=1-3, to=2-3]
	\arrow["v", from=1-4, to=2-4]
	\arrow[from=1-3, to=1-4]
	\arrow[from=2-3, to=2-4]
	\arrow[from=2-2, to=2-3]
	\arrow[from=1-2, to=1-3]
	\arrow["a", from=1-1, to=1-2]
	\arrow["a", from=2-1, to=2-2]
\end{tikzcd}\]
dans lequel les flèches verticales sont les flèches canoniques. Par hypothèse de récurrence sur la dimension (resp. sur l'entier $i$), la flèche $u$ (resp. $v$) est un isomorphisme : par suite, la flèche canonique
$$
\mathrm{H}^p(X, F)/a\mathrm{H}^p(X, F) \to \mathrm{h}^p(X, F)/a\mathrm{h}^p(X, F)
$$
est un isomorphisme de $\hat{A}$--modules. On sait que $\mathrm{h}^p(X, F)$ est un $\hat{A}$--module de type fini, et il résulte du théorème de SHIH (SGA5 V A 3.2) et de (2.13.6) que $\mathrm{H}^p(X, F)$ est également un $\hat{A}$--module de type fini. On conclut alors par le lemme de Nakayama.
\vskip .3cm
{\bf Remarques 2.13.7}.
\begin{itemize}
    \item[(i)] Supposons par exemple que $X$ soit le topos étale d'un schéma de type fini sur un corps séparablement clos ou fini. Il est alors conjecturé que les conditions (i) et (ii) de (2.13.2) sont réalisées dans chacun des cas suivants :
    \begin{itemize}
        \item[a)] Les $A_n$ sont des groupes abéliens de torsion, et $X$ est propre sur $K$. 
        \item[b)] Les $A_n$ sont des groupes abéliens de torsion première à la caractéristique de $k$. 
    \end{itemize}
    Elles sont en tout cas démontrées lorsque de plus le corps $k$ est de caractéristique 0, ou bien dim$(X) \leq 2$.
    \item[(ii)] Lorsque $A$ est un anneau de valuation discrète et $X$ le topos étale d'un schéma noethérien, on peut préciser (2.13.2) en montrant que les systèmes projectifs
    $$
    \widehat{\Ext}^i_A(E, F)
    $$
    vérifient la condition de Mittag-Leffler-Artin-Rees (SGA5 V 2.2.1). 

    En effet, reprenant le dévissage utilisé dans la preuve de (2.13.2), on se ramène à le voir pou $E = A$, et $F$ le complexe de degré 0 associé à un $A$-faisceau constructible noté de même. Alors, utilisant (1.29), on se ramène au cas où $F$ est constructible et \emph{plat}, ce qui permet alors de conclure directement grâce au lemme de SHIH (SGA5 V 3.1 et A 3.2) et à l'hypothèse (ii).
    \item[(iii)] L'hypothèse (i) de (2.13.2) sert uniquement pour montrer que $\bRd \cHom_A (E, F)$ est à cohomologie constructible. Elle est donc en particulier inutile lorsque $E \in \D^-_t(X, A)$, vu (2.5.(i)).
\end{itemize}
\vskip .3cm
{\bf 2.13.8}. Sans hypothèse particulière sur $X$ ou $A$, il résulte de (2.13.1) que 
$$
\D^-_c(X, A) \hookrightarrow \D^-_0(X, A) \quad \text{(I 8.2.7).}
$$
On en déduit pour tout entier $n \geq 0$ un foncteur exact (8.2.9)
$$
\bLd \alpha^*_n : \D^b_c(X, A)_{\torf} \to \D^b_c(A_n-\Mod_X)_{\torf}
$$
lorsque $X$ est noethérien.

Si maintenant $A$ est régulier et local d'idéal maximal $J$, on a même une inclusion
$$
\D_c(X, A) \hookrightarrow \D_0(X, A),
$$
qui permet, lorsque de plus $X$ est noethérien, de définir pour tout entier $n \geq 0$ un foncteur exact (8.2.9.bis)
$$
\Ld \alpha^*_n: \D^+_c(X, A) \to \D^+_c(A_n-\Mod_X).
$$
\vskip .3cm
{
Proposition {\bf 2.13.9}. --- \it Sous les hypothèses de (2.13.2), les applications canoniques 
$$
\Ext^i_A(E, F) \to \varprojlim_n \Ext^i_{A_n} (\bLd \alpha^*_n(E), \bLd \alpha^*_n(F))
$$
sont des isomorphismes de $\hat{A}$--modules.
}
\vskip .3cm
{\bf Preuve} : Par définition des foncteurs $\bLd \alpha^*_n$, on a 
$$
\Ext^i_{A_n} (\bLd \alpha^*_n(E), \bLd \alpha^*_n(F))~\text{``=''}~\Ext^i_{A_n}(\bLd u^*_n(E), \bLd u^*_n(F)).
$$
\vskip .3cm
{\bf Remarques 2.13.10}. La proposition (2.13.9) ramène en pratique l'étude ces complexes de $A$-faisceaux à cohomologie constructible à celle des complexes de $A_n$--Modules constructibles $(n \geq 0)$, et sera de ce fait un instrument privilégié pour obtenir, à partir d'énoncés sur les $A_n$--Modules, les énoncés correspondants pour les $A$-faisceaux.

La proposition suivante généralise la dernière assertion de (2.13.2).
\vskip .3cm
{
Proposition {\bf 2.14}. --- \it Soit $X$ un topos noethérien. On suppose que 
\begin{itemize}
    \item[(i)] Pour tout $A$-algèbre de type fini $B$ annulée par une puissance de $J$, et tout couple $(M, N)$ de $B$--Modules constructibles, les $B$--Modules $\cExt^p_B(M, N)$ soient constructibles. 
    \item[(ii)] Pour tout $A$-algèbre de type fini $B$ annulée par une puissance de $J$ et tout $B$--Module constructible $M$, les $B$--modules
    $$
    \mathrm{H}^p(X, M) \quad (p \in \mathbf{N})
    $$
    sont de type fini.
\end{itemize}
Alors, étant donnés $B \in \D^-_c(X, A)$ et $F \in \D^+_c(X, A)$, ls systèmes projectifs
$$
\overline{\Ext}^i_A(E, F) \quad (i \in \mathbf{Z})
$$
(I 6.2.1) sont des $A$-faisceaux \emph{constructibles} sur le topos ponctuel et les applications canoniques (I 7.4.16)
$$
\Ext^i_A(E, F) \to \varprojlim \overline{\Ext}^i_A(E, F)
$$
sont des bijections. En particulier, les $\hat{A}$--modules $\Ext^i_A(E, F)$ sont de \emph{type fini}.
}
\vskip .3cm
{\bf Preuve} : Montrons que les $A$-faisceaux $\overline{\Ext}^i_A(E, F)$ sont constructibles, les autres assertions en découlant sans peine grâce à la suite exacte (I 7.4.16). On a l'isomorphisme de Cartan (2.6)
$$
\Rd \overline{\Hom}_A(E, F) \isom \Rd \overline{\Hom}_A(A, \bRd \cHom_A(E, F))
$$
qui permet, puisque $\bRd \cHom_A(E, F)$ est à cohomologie constructible (2.5.(iii)), de se ramener au cas où $E = A$. Alors la suite spectrale birégulière canonique
$$
E^{p, q}_2 = \overline{\mathrm{H}}^p(X, \mathrm{H}^q(F)) \Rightarrow \overline{\mathrm{H}}^{p+q}(X, F)
$$
montre qu'il suffit de voir l'assertion dans le cas où $F$ est le complexe de degré 0 associé à un $A$-faisceau constructible noté de même.

Alors l'hypothèse (ii) et le lemme de SHIH (SGA5 V 3.1) permettent de conclure.
\vskip .3cm
{\bf Remarques 2.15}. Comme pour (2.13.2), l'hypothèse (i) a servi uniquement pour assurer que le complexe $\bRd \cHom_A(E, F)$ est à cohomologie constructible. Elle est donc inutile en particulier dans le cas où $E \in \D^-_t(X, A)$.









% End

%Begin







%%%%%%%%%%%%%%%%%%%%%%%%%%%%%%%%%%%%%%%%%%%%%%%%%%%%%%%%%%%%%%%
\chapter*{\S \space III. --- APPLICATIONS AUX SCHÉMAS}
\addcontentsline{toc}{section}{III. Applications aux schémas}
\label{ch:3}
\section*{}

Le texte qui suit ayant un caractère essentiellement provisoire (cf. l'appendice basé sur une construction de \emph{Deligne}), nous ferons toutes les hypothèses simplificatrices qui nous paraîtront nécessaires pour la clarté de l'exposé.

Soit $\ell$ un nombre premier. On fixe comme précédemment un anneau noethérien $A$ et un idéal propre $J$ de $A$. On suppose de plus que $A$ est une $\mathbf{Z}_{\ell}$-algèbre et que $J$ contient $\ell A$. Pour simplifier (cf. supra), tous les schémas considérés sont \emph{noethériens}.

%%%%%%%%%%%%%%%%%%%%%%%%%%%%%%%%%%%%
\subsection*{1. Opérations externes.}
\addcontentsline{toc}{subsection}{1. Opérations externes}

\vskip .3cm
{\bf 1.1}. Soient $X$ et $Y$ deux schémas noethériens, et $f: X \to Y$ un morphisme séparé de type fini. On définit comme suit un foncteur exact
$$
\bRd f_!: \D(X, A) \to \D(Y, A),
\leqno{(1.1.1)}
$$
appelé \emph{image directe à supports propres}. D'après Nagata et Mumford il existe une factorisation
\[\begin{tikzcd}
	X && Z \\
	& Y & {,}
	\arrow["q", from=1-3, to=2-2]
	\arrow["f"', from=1-1, to=2-2]
	\arrow["i", hook, from=1-1, to=1-3]
\end{tikzcd}\]
où $i$ est une immersion ouverte et $q$ un morphisme propre. On pose alors
$$
\bRd f_! = \bRd q_* \circ \bRd i_!.
$$
On vérifie, grâce à la technique de factorisation de \emph{Lichtenbaum} (SGA4 XVIII \quad), que le résultat ne dépend pas, à isomorphisme près, de la factorisation choisie.

La même technique de factorisation montre que si $g: Y \to Z$ est un autre morphisme séparé de type fini, on a un isomorphisme
$$
\bRd (g \circ f)_! \isomlong \bRd g_! \circ \bRd f_!,
\leqno{(1.1.2)}
$$
avec la condition de cocycles habituelle pour un triple de morphismes.
\vskip .3cm
{
Définition {\bf 1.1.3}. --- \it Si $E$ est un $A$-faisceau sur $X$ (resp. un objet de $\D(X, A)$), on pose pour tout $p \in \mathbf{Z}$
$$
\Rd^pf_!(E) = \mathrm{H}^p(\bRd f_!(E)).
$$
}
\vskip .3cm
On obtient ainsi un foncteur cohomologique qui n'est pas en général (sauf bien sûr si le morphisme $f$ est propre) le foncteur cohomologique dérivé de $\Rd^\circ f_!$.

\vskip .3cm
{\bf 1.1.4}. Il est clair que si $F = (F_n)_{n \in \mathbf{N}}$ est un $A$-faisceau, on a pour tout $p \in \mathbf{Z}$
$$
\Rd^p f_! (F) = (\Rd^p f_! (F_n))_{n \in \mathbf{N}}.
$$
\vskip .3cm
{
Proposition {\bf 1.1.5}. --- \it Soit
\[\begin{tikzcd}
	{X'} && X \\
	{Y'} && Y
	\arrow["f", from=1-3, to=2-3]
	\arrow["g"', from=2-1, to=2-3]
	\arrow["{g'}", from=1-1, to=1-3]
	\arrow["{f'}"', from=1-1, to=2-1]
\end{tikzcd}\]
un carré cartésien de schémas noethériens.
\begin{enumerate}
    \item[(i)] ({\bf Théorème de changement de base propre}) Si $f$ (donc $f'$) est séparé de type fini, on a pour tout $E \in \D^+(X, A)$ un isomorphisme canonique fonctoriel
    $$
    g^* \bRd f_! (E) \isomlong \bRd f'_! (g')^* (E).
    $$
    \item[(ii)] ({\bf Théorème de changement de base lisse}) Si $\ell$ est premier aux caractéristiques résiduelles de $Y$ et $g$ est lisse, on a pour tout $E \in \D^+(X, A)$ un isomorphisme canonique fonctoriel
    $$
    g^* \bRd f_* (E) \isomlong \bRd (f')_* (g')^* (E).
    $$
\end{enumerate}
}
\vskip .3cm
{\bf Preuve} : Montrons (ii). Utilisant l'adjonction entre image directe et image réciproque (I 7.7.6), on construit comme dans (SGA4 XVII) (voir aussi SGA4 XII 4), un morphisme fonctoriel
$$
g^* \bRd f_* (E) \to \bRd (f')_* (g')^* (E).
$$
Pour voir que c'est un isomorphisme, on se ramène par ``way-out functor lemma'' au cas où $E$ est de degré 0, et il suffit alors de montrer que les morphismes 
$$
g^* \Rd^i f_* (E) \to \Rd^i (f')_*(g')^* (E) \quad (i \in \mathbf{N})
$$
correspondants de $\mathcal{E}(X, J)$ sont des isomorphismes. Cela se voit sur les composants, grâce au théorème de changement de base lisse sur les $A_n$--Modules (SGA4 XII 1.1). Montrons (i). On construit tout d'abord un morphisme fonctoriel
$$
g^* \bRd f_! (E) \to \bRd (f')_! (g')^* (E),
\leqno{(1.1.6)}
$$
en paraphrasant la construction faite pour les $A_n$--Modules (SGA4 XVII \quad). Pour cela, choisissant une factorisation $f = q \circ i$, avec $i$ une immersion ouverte et $q$ un morphisme propre, on se ramène à faire la construction lorsque $f$ est propre, ou bien est une immersion ouverte; on vérifie ensuite de f\c{c}on standard que le résultat ne dépend pas de la factorisation choisie. Lorsque $f$ est une immersion ouverte, les morphismes analogues pour les $A_n$--Modules $(n \in \mathbf{N})$ définissent de fa\c{c}on évidente un isomorphisme $g^* f_! \isomlong (f')_! (g')^*$ de foncteurs exacts
$$
\mathcal{E}(X, J) \to \mathcal{E}(Y', J),
$$
d'où par passage au quotient, un isomorphisme de foncteurs exacts
$$
A-\fsc(X) \to A-\fsc(Y'),
$$
qui fournit à son tour un isomorphisme de foncteurs exacts
\[\begin{tikzcd}
	{\D(X, A)} && {\D(Y', A).}
	\arrow[""{name=0, anchor=center, inner sep=0}, "{g^* \circ f_!}", curve={height=-12pt}, from=1-1, to=1-3]
	\arrow[""{name=1, anchor=center, inner sep=0}, "{(f')_! \circ (g')^*}"', curve={height=12pt}, from=1-1, to=1-3]
	\arrow["\sim", shorten <=3pt, shorten >=3pt, from=0, to=1]
\end{tikzcd}\]
Lorsque $f$ est un morphisme propre, on utilise la même construction que pour (ii). Pour montrer enfin que le morphisme (1.1.6) ainsi construit est un isomorphisme, on se ramène au cas où $E$ est de degré 0, et alors l'assertion résulte, comme pour (ii), de l'assertion analogue pour les $A_n$--Modules (SGA4 XVII \quad).
\vskip .3cm
{
Proposition {\bf 1.1.7} (Formule de projection). --- \it Soient $f: X \to Y$ un morphisme séparé de type fini entre schémas noethériens, $E \in \D^-(X, A)$ et $F \in \D^-(X, A)$. On a un isomorphisme canonique fonctoriel
\[\begin{tikzcd}
	{\bRd f_! (E \boldsymbol{\otimes}f^* (F)) } & {\bRd f_! (E) \boldsymbol{\otimes} F. }
	\arrow["\sim"', from=1-2, to=1-1]
\end{tikzcd}\]
}
\vskip .3cm
{\bf Preuve} : Nous utiliserons le lemme suivant.
\vskip .3cm
{
Lemme {\bf 1.1.8}. --- \it Si $d$ est un entier majorant la dimension des fibres de $f$, on a pour tout $A$-faisceau $M$ sur $X$
$$
\Rd^if_! (M) = 0 \quad (i > 2d).
$$
}
\vskip .3cm
(Résulte immédiatement de l'assertion analogue pour les composants de $M$).

Choisissant une compactification de $f$, on se ramène à montrer (1.1.7) successivement lorsque $f$ est une immersion ouverte, ou un morphisme propre. Dans le premier cas, ce n'est autre que (I 7.7.10.(iv)). Dans le second cas, on définit un morphisme 
$$
\bRd f_* (E) \boldsymbol{\otimes} F \to \bRd f_* (E \boldsymbol{\otimes} f^* F)
\leqno{(1.1.9)}
$$
sur le modèle de (J.L. Verdier: The Lefschetz fixed point formula in étale cohomology, in ``Conference on local fields held at Driebergen'' preuve de 3.2), en se ramenant à $F$ plat et $E$ $f_*$-acyclique (ce qui est possible grâce à 1.1.8). Enfin, pour voir que (1.1.9) est un isomorphisme, on se ramène par les dévissages habituels au cas où $E$ et $F$ sont réduits au degré 0 et $F$ plat, et alors l'assertion résulte de la formule de projection pour les $A_n$--Modules $(n \in \mathbf{N})$, appliquée aux composants de $E$ et $F$.
\vskip .3cm
{
Proposition {\bf 1.1.10} (Formule de Künneth). --- \it Considérons un diagramme cartésien de schémas noethériens
\[\begin{tikzcd}
	& {X \times_Z Y} \\
	X && Y \\
	& Z & {,}
	\arrow["p"', from=1-2, to=2-1]
	\arrow["q", from=1-2, to=2-3]
	\arrow["g", from=2-3, to=3-2]
	\arrow["f"', from=2-1, to=3-2]
\end{tikzcd}\]
et posons $h = f \circ p = g \circ q$. Si $E \in \D^-(X, A)$ et $F \in D^-(Y, A)$, on a un isomorphisme canonique fonctoriel
$$
\bRd f_! (E) \boldsymbol{\otimes} \bRd g_! (F) \isomlong \bRd h_! (p^* E \boldsymbol{\otimes}q^* F).
$$
}
\vskip .3cm
{\bf Preuve} : Formellement la même que celle de l'assertion correspondante pour les faisceaux de $A_n$--Modules $(n \in \mathbf{N})$ (SGA4 XVII). De (1.1.7) appliqué à $f$, on déduit un isomorphisme
$$
\bRd f_! (E) \boldsymbol{\otimes} \bRd g_! (F) \isomlong \bRd f_! (E \boldsymbol{\otimes}f^* \bRd g_! (F)).
\leqno{(1.1.10.1)}
$$
Le théorème de changement de base propre pour $f$ (1.1.5.(i)) montre que 
$$
f^* \bRd g_! (F) \isomlong \bRd p_! q^* (F). 
\leqno{(1.1.10.2)}
$$
Comparant avec (1.1.10.1), on a donc
$$
\bRd f_! (E) \boldsymbol{\otimes} \bRd g_! (F) \isomlong \bRd f_! (E \boldsymbol{\otimes} \bRd p_! q^* (F)).
$$
La formule de projection (1.1.7) pour le morphism $p$ montre que
$$
E \boldsymbol{\otimes} \bRd p_! q^* (F) \isomlong \bRd p_! (p^* E \boldsymbol{\otimes} q^* F),
$$
d'où
$$
\bRd f_!(E \boldsymbol{\otimes} \bRd p_!q^* (F)) \isomlong \bRd f_! \bRd p_! (p^* E \boldsymbol{\otimes} q^* F),
$$
et le résultat annoncé puisque $f \circ p = h$.
\vskip .3cm
{
Proposition {\bf 1.1.11}. --- \it Soient $X$ et $Y$ deux schémas noethériens, $f: X \to Y$ un morphisme séparé de type fini, et $E \in \D(X, A)$.
\begin{itemize}
    \item[(i)] Si $E \in \D_c(X, A)$, alors $\bRd f_! (E) \in \D_c(Y, A)$.
    \item[(ii)] Si $E \in \D^-(X, A)_{\text{torf}}$, alors $\bRd f_! (E) \in \D^-(Y, A)_{\text{torf}}$.
    \item[(iii)] Supposons que $f$ soit propre et lisse, et que $\ell$ soit premier aux caractéristiques résiduelles de $Y$. Alors:
    \[\begin{tikzcd}
	{E \in \D_t(X, A)} && {\bRd f_!(E) \in \D_t(Y, A).} \\
	{E \in \D_{\text{parf}}(X, A)} && {\bRd f_!(E) \in \D_{\text{parf}}(Y, A).}
	\arrow[shorten <=10pt, shorten >=10pt, Rightarrow, from=1-1, to=1-3]
	\arrow[shorten <=9pt, shorten >=9pt, Rightarrow, from=2-1, to=2-3]
    \end{tikzcd}\]
\end{itemize}
}
\vskip .3cm
{\bf Preuve} : Montrons (i). Grâce à (1.1.8), on peut supposer $E$ de degré 0 associé à un $A$-faisceau $J$-adique constructible. Alors, l'assertion est essentiellement (SGA5 VI 2.2.2). Pour la première partie de (iii), on est ramène de même à voir que si $E$ est un $A$-faisceau $J$-adique constant tordu constructible, les $A$-faisceaux $\Rd^p f_* (E)$ $(p \in \mathbf{N})$ sont constants tordus constructibles. Cela se voit comme (SGA5 V 2.2.2), en utilisant le lemme de SHIH (SGA5 V A 3.2) et la stabilité des catégories des faisceaux abéliens localement constants constructibles par images directes supérieures (SGA4 XVI 2.2). L'assertion (ii) résulte sans peine de (1.1.7), et on en déduit aussitôt la deuxième partie de (iii) (compte tenu de la première).
\vskip .3cm
{\bf 1.2}. Soient $X$ et $Y$ deux schémas noethériens de caractéristique résiduelles premières à $\ell$, et $f: X \to Y$ un morphisme quasiprojectif. On suppose que $Y$ admet un Module inversible ample. On définit alors comme suit un foncteur exact
$$
\bRd f^!: \D^+(Y, A) \to \D^+(X, A).
\leqno{(1.2.1)}
$$
D'après (EGA II 5.3.3), il existe une factorisation
\[\begin{tikzcd}
	X && {{\mathbb P}^r_Y} \\
	& Y & {,}
	\arrow["f"', from=1-1, to=2-2]
	\arrow["q", from=1-3, to=2-2]
	\arrow["j", hook, from=1-1, to=1-3]
\end{tikzcd}\]
avec $j$ une immersion. On en déduit aussitôt une factorisation
\[\begin{tikzcd}
	X && U \\
	& Y & {,}
	\arrow["f"', from=1-1, to=2-2]
	\arrow["p", from=1-3, to=2-2]
	\arrow["i", hook, from=1-1, to=1-3]
\end{tikzcd}\]
où $i$ est une immersion fermée et $p$ un morphisme lisse équidimensionel de dimension $r$. Avec les notations de (SGA5 VI 1.3.4), on pose alors pour tout $F \in \D^+(X, A)$
$$
\bRd f^!(F) = \bRd i^! (p^* F \boldsymbol{\otimes}_{\mathbf{Z}_\ell}\mathbf{Z}_\ell(r))[2r],
$$
où le foncteur $\bRd i^!$ a été défini en (I 7.7.11). Pour avoir que cette définition ne dépend pas, à isomorphisme près, des choix faits, on est ramené, grâce à la technique de factorisation de \emph{Lichtenbaum}, à prouver le théorème de \emph{pureté cohomologique} suivant.
\vskip .3cm
{
Proposition {\bf 1.2.3}. --- \it Soient $S$, $X$, $Y$ trois schémas noethériens de caractéristique résiduelles premières à $\ell$, et
\[\begin{tikzcd}
	Y && X \\
	& S
	\arrow["f", from=1-3, to=2-2]
	\arrow["g"', from=1-1, to=2-2]
	\arrow["j", hook, from=1-1, to=1-3]
\end{tikzcd}\]
un $S$-couple lisse (SGA4 XVI 3.1) purement de codimension $d$. Pour tout $F \in \D^+(S, A)$, on a un isomorphisme canonique fonctoriel (classe fondamentale locale)
\[\begin{tikzcd}
	{\bRd j^!(f^* F)} && {g^* (F) \boldsymbol{\otimes}_{\mathbf{Z}_\ell}\mathbf{Z}_\ell (-d)[-2d].}
	\arrow["\sim"', from=1-3, to=1-1]
\end{tikzcd}\leqno{(1.2.3.1)}\]
}
\vskip .3cm
{\bf Preuve} : Par (I 7.7.12), il s'agit de définir un morphisme
$$
\bRd j_* (g^* F \boldsymbol{\otimes}_{\mathbf{Z}_\ell}\mathbf{Z}_\ell(-d)[-2d]) \to f^* F,
$$
soit, d'après la formule de projection (I 7.7.12 (iv)),
$$
f^* F \boldsymbol{\otimes}_{\mathbf{Z}_\ell} \bRd j_* (\mathbf{Z}_\ell(-d)[-2d]) \to f^* F.
$$
On est ainsi ramené à définir (1.2.3.1) dans le cas où $A = \mathbf{Z}_\ell = F$. Il s'agit alors d'exhiber un morphisme
$$
\mathbf{Z}_\ell \to \bRd j^! (\mathbf{Z}_\ell (d))[2d].
$$
Mais on sait, d'après l'assertion analogue (SGA4 VI 3) pour les composantes, que
$$
\Rd^sj^! (\mathbf{Z}_\ell (d)) = 0 \quad \text{pour}~s<2d,
$$
de sorte qu'il suffit d'exhiber un morphisme de $\mathbf{Z}_\ell$-faisceaux
$$
\mathbf{Z}_\ell \to \Rd^{2d}j^!(\mathbf{Z}_\ell(d)).
$$
On prend le système projectif des morphismes classes fondamentales correspondants
$$
\mathbf{Z}/\ell^{n + 1} \mathbf{Z} \to \Rd^{2d}j^!(\boldsymbol{\mu}^{\otimes d}_{\ell^{n + 1}}).
$$
Enfin, pour voir que (1.2.3.1) est un isomorphisme, on peut supposer que $F$ est réduit au degré 0, associé à un $A$-faisceau noté de même.
Alors, l'assertion résulte du théorème de pureté cohomologique pour les $A_n$--Modules $(n \in \mathbf{N})$, appliqué aux composants de $F$.
\vskip .3cm
{\bf Notation 1.2.4}. Si $F$ est un $A$-faisceau sur $Y$ (resp. un objet de $\D^+(Y, A)$), on pose pour tout $p \in \mathbf{Z}$
$$
\Rd^p f^! (F) = \mathrm{H}^p(\bRd f^! F).
$$
\vskip .3cm
{\bf 1.2.5}. Si $X, Y, Z$ sont trois schémas noethériens admettant des Modules inversibles amples, et $f: X \to Y$ et $g: Y \to Z$ deux morphismes quasiprojectifs, on a un isomorphisme
$$
\bRd (g \circ f)^! \isomlong \bRd g^! \circ \bRd f^!,
$$
avec la condition de cocycles habituelle pour un triple de tels morphismes. 

Cela se voit, comme dans le cas usuel des faisceaux abéliens de torsion, par la méthode de factorisation de Lichtenbaum.
\vskip .3cm
{
Proposition {\bf 1.2.5} (Formule d'induction). --- \it Sous les hypothèses préliminaires de (1.2), soient $E \in \D^-_c (Y, A)$ et $F \in \D^+(X, A)$. On a un isomorphisme canonique fonctoriel
$$
\bRd f^! \bRd {\cHom}_A (E, F) \isomlong \bRd{\cHom}_A(f^* E, \bRd f^! F).
$$
}
\vskip .3cm
{\bf Preuve} : Si $f$ est une immersion fermée, on a (I 7.7.13) un isomorphisme
$$
\bRd f^! \bRd{\cHom}_A (E, F) \isomlong f^* \bRd{\cHom}_A (f_* A, \bRd{\cHom}_A(E, F)),
$$
soit, d'après l'isomorphisme de Cartan ($E$ et $f_* A$ sont à cohomologie constructible)
$$
\bRd f^! \bRd{\cHom}_A (E, F) \isomlong f^* \bRd{\cHom}_A (E, \bRd{\cHom}_A(f_* A, F)).
$$
Utilisant à nouveau (I 7.7.3), on a 
$$
\bRd f^! \bRd{\cHom}_A (E, F) \isomlong f^* \bRd{\cHom}_A (E, f^* \bRd f^! F),
$$
d'où, d'après l'adjonction entre $f^*$ et $f_*$ (I 7.7.6)
\[\begin{tikzcd}
	{\bRd f^! \bRd{\cHom}_A (E, F) } && {f^*f_* \bRd{\cHom}_A (f^* E, \bRd f^! F)} \\
	{} && {\bRd{\cHom}_A(f^* E, \bRd f^! F).}
	\arrow["\sim", from=1-1, to=1-3]
	\arrow["\sim"{pos=0.7}, shorten <=44pt, shorten >=9pt, from=2-1, to=2-3]
\end{tikzcd}\]
Lorsque $f$ est lisse et équidimensionnel de dimension $r$, on a 
$$
\bRd f^! \bRd{\cHom}_A (E, F) \isomlong f^* \bRd{\cHom}_A (E, F) \boldsymbol{\otimes}_{\mathbf{Z}_\ell}\mathbf{Z}_\ell(r)[2r];
$$
de (I 7.7.2 (ii)), on déduit alors aussitôt un morphisme ``canonique''
$$
\bRd f^! \bRd{\cHom}_A (E, F) \to \bRd{\cHom}_A(f^* E, \bRd f^! F).
$$
Pour voir que c'est un isomorphisme, on peut supposer que $E$ et $F$ sont réduits au degré 0 et que $\mathrm{H}^0(E)$ est un $A$-faisceau constructible. Il s'agit alors de voir que les morphismes canoniques
$$
f^* {\cExt}^p_A (E, F) \to {\cExt}^p_A(f^* E, f^*F)
\leqno{(\text{I}~6.4.1.1)}
$$
sont des isomorphismes~; vu leur définition, cela est conséquence immédiate de l'assertion analogue pour les $A_n$--Modules (SGA4 XVIII). Enfin, dans le cas général, on choisit une factorisation $f = p \circ i$ du type (1.2.2). Des deux cas précédents, on déduit des isomorphismes 
$$
\bRd f^! \bRd{\cHom}_A (E, F) \isomlong \bRd p^! \bRd i^! \bRd{\cHom}_A (E, F) \isomlong \bRd p^! \bRd{\cHom}_A (i^* E, \bRd i^! F)
$$
$$
\isomlong \bRd{\cHom}_A (p^* i^* E, \bRd p^! \bRd i^! F) \isomlong \bRd{\cHom}_A (f^* E, \bRd f^! F).
$$
On assure ensuite, comme d'habitude, que l'isomorphisme composé ne dépend pas de la factorisation choisie.
\vskip .3cm
{\bf 1.3}. Soient $u: A \to B$ une $A$-algèbre et $K$ un idéal de $B$ tel que $u(J) \subset K$. On utilise dans l'énoncé suivant les notations de (I 8). 
\vskip .3cm
{
Proposition {\bf 1.3.1}. --- \it Soit $f: X \to Y$ un morphisme séparé de type fini entre schémas noethériens.
\begin{enumerate}
    \item[1)] Soit $E \in \D(X, A)$. On a un isomorphisme canonique
    $$
    \bLd u^* \bRd f_! (E) \isomlong \bRd f_! \bLd u^* (E),
    $$
    lorsque $E \in \D^{-}(X, A)$, ou lorsque $A$ est local régulier et $J$ est son idéal maximal.
    \item[2)] Pla\c{c}ons-nous maintenant dans le cas où $Y$ admet un Module inversible ample. On suppose de plus que $\ell$ est premier aux caractéristiques résiduelles de $Y$, que l'anneau $A$ est local régulier et que $J$ est son idéal maximal. Alors pour tout $F \in \D^+(Y, A)$, on a un morphisme canonique fonctoriel
    $$
    \bLd u^* \bRd f^! (F) \isomlong \bRd f^! \bLd u^* (F),
    $$
    qui est un \emph{isomorphisme} lorsque $B$ est une $A$-algèbre finie et $K = JB$.
\end{enumerate}
}
\vskip .3cm
{\bf Preuve} : Montrons 1), et définissons d'abord un morphisme 
$$
\bLd u^* \bRd f_! (E) \to \bRd f_! \bRd u^* (E).
\leqno{(1.3.1.1)}
$$
D'après (I. 8.1.6), il suffit dans chacun des cas considérés de définir un morphisme
$$
\bRd f_! (E) \to u_* \bRd f_! \bLd u^*(E).
\leqno{(1.3.1.2)}
$$
Mais il est immédiat que $u_* \bRd f_! \isom \bRd f_! u_*$, de sorte que l'on définit (1.3.1.2) en appliquant le foncteur $\bRd f_!$ au morphisme d'adjonction (I 8.1.7)
$$
E \to u_* \bLd u^* (E).
$$
Pour voir que (1.3.1.1) est un isomorphisme, on se ramène, par le way-out functor lemme, au cas où $E \in \D^-(X, A)$. Alors, grâce à la conservativité du foncteur $u_*$, il s'agit de montrer que le morphisme canonique
$$
B \boldsymbol{\otimes}_A \bRd f_! (E) \to \bRd f_!(B \boldsymbol{\otimes}_A E)
$$
est un isomorphisme, ce qui résulte de (1.1.7). Montrons 2). Pour définir un morphisme
$$
\bLd  u^* \bRd f^! (F) \to \bRd f^! \bLd  u^*(F),
\leqno{(1.3.1.3)}
$$
on se ramène encore, grâce à (I 8.1.6), à définir un morphisme
$$
\bRd f^! (F) \to u_* \bRd f^! \bLd  u^*(F).
\leqno{(1.3.1.4)}
$$
On a évidemment $u_* \bRd f^! \isom \bRd f^!u_*$; on prend pour (1.3.1.4) l'image par $\bRd f^!$ du morphisme d'adjonction (I 8.1.7). Pour voir que (1.3.1.3) est un isomorphisme, on se ramène, après avoir choisi une ``lissification'' (1.2.2), à le faire successivement pour une immersion fermée et un morphisme lisse équidimensionnel. Dans le premier cas, ce n'est autre que (I 8.1.16 (iii)). Dans le second, on se ramène aussitôt à (I 8.1.16 (i)).








% End

% Begin









%%%%%%%%%%%%%%%%%%%%%%%%%%%%%%%%%%%%
\subsection*{2. Dualité.}
\addcontentsline{toc}{subsection}{2. Dualité}

Dans tout ce paragraphe, tous les schémas considérés sont de caractéristiques résiduelles premières à $\ell$.

\vskip .3cm
{\bf 2.1}. Soient $X$ et $Y$ deux schémas noethériens et $f: X \to Y$ un morphisme quasiprojectif. On suppose que $Y$ admet un Module inversible ample et on se propose de définir un morphisme ``trace''
$$
\Tr_f: \bRd f_! \bRd f^! \to \id 
\leqno{(2.1.1)}
$$
entre foncteurs de $\D(Y, A)$ dans $\D(Y, A)$.  

Lorsque $f$ est une immersion fermée, on dispose d'un tel morphisme, à savoir le morphisme d'adjonction déduit de (I 7.7.12 (i)).

Lorsque $f$ est un morphisme lisse équidimensionnel de dimension $r$, il s'agit de définir pour tout $F \in \D(Y, A)$ un morphisme fonctoriel
$$
\bRd f_! (f^* F \boldsymbol{\otimes}_{\mathbf{Z}_\ell}\mathbf{Z}_\ell(r)[2r]) \to F.
\leqno{(2.1.2)}
$$
Comme $A \boldsymbol{\otimes}_{\mathbf{Z}_\ell}\mathbf{Z}_\ell(r)$ est localement libre constructible, on définit sur le modèle de (1.1.7), mais sans hypothèse de degré sur $F$, un isomorphisme de ``projection''
$$
\bRd  f_! (\mathbf{Z}_\ell(r))[2r]\boldsymbol{\otimes}_{\mathbf{Z}_\ell} F \isomlong \bRd f_! (f^* F \boldsymbol{\otimes}_{\mathbf{Z}_\ell}\mathbf{Z}_\ell(r))[2r],
$$
ce qui ramène à faire la construction de (2.1.2) dans le cas où $A = \mathbf{Z}_\ell = F$. Dans ce cas, comme $\Rd^i f_! = 0$ pour $i > 2d$ (1.1.8), il s'agit d'exhiber un morphisme ``trace''
$$
\Rd^{2r} f_! (\mathbf{Z}_\ell(r)) \to \mathbf{Z}_\ell.
$$
On prend le système projectif des morphismes traces ``habituels''
$$
\Rd^{2r}f_! (\boldsymbol{\mu}^{\otimes r}_{\ell^{n+1}}) \to \mathbf{Z}/\ell^{n+1}\mathbf{Z}.
$$
Dans le cas général, on choisit pour définir (2.1.1) une factorisation $f = p \circ i$ du type (1.2.2). Désignant par
\[\begin{tikzcd}
	{u: \bRd i_! \bRd i^!} & \id \\
	{v: \bRd p_! \bRd p^!} & \id
	\arrow[from=1-1, to=1-2]
	\arrow[from=2-1, to=2-2]
\end{tikzcd}\]
les morphismes traces définis par les méthodes précédentes pour $i$ et $p$ respectivement, on définit $\Tr_f$ par la commutativité du diagramme 
\[\begin{tikzcd}
	{\Rd f_! \Rd f^!} && {\Rd p_! \Rd i_! \Rd i^! \Rd p^!} \\
	\id && {\Rd p_! \Rd p^!} & {.}
	\arrow["\sim", from=1-1, to=1-3]
	\arrow["v", from=2-3, to=2-1]
	\arrow["{\Tr_f}"', from=1-1, to=2-1]
	\arrow["{\Rd p_! (u \Rd p^!)}", from=1-3, to=2-3]
\end{tikzcd}\]
On s'assure ensuite, de la fa\c{c}on habituelle, que le résultat ne dépend pas de la factorisation choisie.

\vskip .3cm
{\bf 2.2}. Sous les hypothèses précédentes, on se propose maintenant de définir, pour $E \in \D^-(X, A)$ et $F \in \D^+(Y, A)$ un morphisme ``canonique'' fonctoriel
$$
\bRd f_* \bRd {\cHom}_A (E, \bRd f^! F) \to \bRd {\cHom}_A(\bRd f_! E, F).
\leqno{(2.2.1)}
$$
Pour cela, nous allons d'abord définir, pour $L \in \D^-(X, A)$ et $M \in \D^+(X, A)$ un morphisme fonctoriel
$$
\bRd f_* \bRd {\cHom}_A (L, M) \to \bRd {\cHom}_A (\bRd f_! L, \bRd f_! M).
\leqno{(2.2.2)}
$$
On prendra alors pour (2.2.1) le morphisme composé
\[\begin{tikzcd}
	& {\bRd {\cHom}_A(\bRd f_! E, \bRd f_! \bRd f^! F)} \\
	{\bRd f_* \bRd {\cHom}_A(E, \bRd f^! F)} && {\bRd {\cHom}_A(\bRd f_! E, F)}
	\arrow["{\bRd {\cHom}_A (\id, \Tr_f)}", from=1-2, to=2-3]
	\arrow["{(2.2.2)}", from=2-1, to=1-2]
\end{tikzcd}\]
Il reste à définir (2.2.2). Lorsque $f$ est une immersion ouverte, le foncteur $f_!$ commute aux limites inductives filtrantes, et permet donc de définir pour tout couple $(E, F)$ de $A$-faisceaux sur $X$ un morphisme fonctoriel
$$
f_! {\cHom}_A (E, F) \to {\cHom}_A (f_! E, f_! F),
\leqno{(2.2.3)}
$$
à partir des morphismes analogues dans la catégorie des $A$--Modules. Pour définir (2.2.2) dans ce cas, on peut supposer $L$ quasilibre et $M$ flasque, de sorte que ${\cHom}^{\bullet}_A(L, M)$ est flasque. Le morphisme (2.2.3) fournit par fonctorialité un morphisme de complexes
$$
f_! {\cHom}^{\bullet}_A (L, M) \to {\cHom}^{\bullet}_A (f_! L, f_! M).
\leqno{(2.2.4)}
$$
Choisissant une résolution quasilibre $P \to f_! L$ et une résolution flasque $f_! M \to Q$, on prend pour (2.2.2) le composé de (2.2.4) et du morphisme canonique
$$
{\cHom}^{\bullet}_A(f_! L, f_! M) \to {\cHom}^{\bullet}_A (P, Q).
$$
Lorsque $f$ est propre, il s'agit de définir un morphisme
$$
\bRd f_* \bRd {\cHom}_A (L, M) \to \bRd {\cHom}_A(\bRd f_* L, \bRd f_* M).
$$
La construction que nous allons faire de (2.2.5) vaut plus généralement pour un morphisme quasicompact et quasiséparé. Cette dernière hypothèse implique que le foncteur $f_*$ commute aux limites inductives filtrantes, et permet donc comme précédemment de définir pour tout couple $(E, F)$ de $A$-faisceaux sur $X$ un morphisme fonctoriel 
$$
f_* {\cHom}_A (E, F) \to {\cHom}_A (f_* E, f_* F),
\leqno{(2.2.6)}
$$
à partir des morphismes analogues dans la catégorie des $A$--Modules. Pour définir (2.2.5), on peut supposer $L$ quasilibre et $M$ flasque, de sorte que ${\cHom}^{\bullet}_A(L, M)$ est flasque. Le morphisme (2.2.6) fournit par fonctorialité un morphisme de complexes
$$
f_* {\cHom}^{\bullet}_A (L, M) \to {\cHom}^{\bullet}_A (f_* L, f_* M).
\leqno{(2.2.7)}
$$
On prend pour (2.2.5) le composé de (2.2.7) et du morphisme
$$
{\cHom}^{\bullet}_A (f_* L, f_* M) \to {\cHom}^{\bullet}_A (P, f_* M)
$$
déduit d'une résolution quasilibre $P \to f_* L$ de $f^* L$.

Enfin, dans le cas général, on choisit une compactification $f = q \circ i$ de $f$, et on définit (2.2.2) de fa\c{c}on évidente à partir des morphismes déjà définis pour $i$ et $q$ respectivement. Bien, entendu, on s'assure que le résultat ne dépend pas des choix faits, et notamment de la compactification choisie.
\vskip .3cm
{
Proposition {\bf 2.2.8}. --- \it Sous les hypothèses préliminaires de (2.1), soient $E \in \D^-(X, A)$ et $F \in \D^+(Y, A)$. On a des isomorphismes canoniques fonctoriels :
\begin{itemize}
    \item[(i)] $\bRd f_* \bRd {\cHom}_A (E, \bRd f^! F) \isomlong \bRd {\cHom}_A (\bRd f_! E, F)$.
    \item[(ii)] $\bRd \overline{\Hom}_A(E, \bRd f^! F) \isomlong \bRd \overline{\Hom}_A (\bRd f_! E, F)$.
    \item[(iii)] $\bRd \Hom_A(E, \bRd f^! F) \isomlong \bRd \Hom_A (\bRd f_! E, F)$.
    \item[(iv)] $\Hom_A(E, \bRd f^! F) \isomlong \Hom_A(\bRd f_! E, F)$.
\end{itemize}
}
\vskip .3cm
{\bf Preuve} : Nous allons voir que (2.2.1) est un isomorphisme. Les autres assertions en résulteront en appliquant aux deux membres les foncteurs respectifs $\bRd \overline{\Gamma}(Y, .)$, $\bRd \Gamma(Y, .)$ (I 7.4.10) et $\Hom_A(A, .)$ d'après (I 7.4.18). Pour voir que (2.2.1) est un isomorphisme, on se ramène par le way-out functor lemma au cas où $E$ et $F$ sont les complexes de degré 0 associés à des $A$-faisceaux notés de même. Les constructions aboutissant à la définition de (2.2.1) peuvent alors être faites au moyen de résolutions flasques ou quasilibres dans $\mathcal{E}(X, J)$ et $\mathcal{E}(Y, J)$. Si $E = (E_n)_{n \in \mathbf{N}}$ et $F = (F_n)_{n \in \mathbf{N}}$, l'assertion résulte alors aussitôt du fait que les morphismes de dualité
$$
\bRd f_* \bRd {\cHom}_{A_m}(E_m, \bRd f^! F_n) \to \bRd {\cHom}_{A_m}(\bRd f_! E_m, F_n) \quad (m, n \in \mathbf{N}; m \geq n)
$$
sont des isomorphismes, et de ce que les foncteurs $\Rd^i f_*$ $(i \in \mathbf{Z})$ commutent aux limites inductives filtrantes.
\vskip .3cm
{\bf 2.3}. À partir de maintenant, on suppose, pour simplifier, que $A = \mathbf{Z}_\ell$ et $J = \ell \mathbf{Z}_\ell$. Étant donné un complexe $K \in \D^+(X, \mathbf{Z}_\ell)$, on pose pour tout $F \in \D^-(X, \mathbf{Z}_\ell)$
$$
\D_K(F) = \bRd {\cHom}_{\mathbf{Z}_\ell}(F, K).
$$
\vskip .3cm
{
Définition {\bf 2.3.1}. --- \it Soit $X$ un schéma noethérien. On dit qu'un complexe $K$ de $\mathbf{Z}_\ell$-faisceaux sur $X$ est \emph{dualisant} si pour tout $F \in \D^b_c(X, \mathbf{Z}_\ell)$, on a $\D_K(F) \in \D^b_c(X, \mathbf{Z}_\ell)$, et si le morphisme ``de Cartan''
$$
F \to \D_{K^\circ}\D_K (F)
\leqno{(2.3.2)}
$$
que l'on en déduit est un \emph{isomorphisme}.
}
\vskip .3cm
Explicitons (2.3.2). Comme $D_K(F) \in \D^b_c(X, \mathbf{Z}_\ell)$, le morphisme (I 7.6.5)
$$
\Hom_{\mathbf{Z}_\ell}(\D_K (F) \boldsymbol{\otimes} F, K) \to \Hom_{\mathbf{Z}_\ell}(\D_K(F), \D_K(F))
$$
est un isomorphisme (II 2.6). L'image inverse de l'identité de $\D_K(F)$ définit un morphisme
$$
F \boldsymbol{\otimes} \D_K(F) \to K.
$$
Comme $F \in \D^b_c(X, \mathbf{Z}_\ell)$, une nouvelle application de l'isomorphisme de Cartan permet d'en déduire le morphisme (2.3.2) annoncé.
\vskip .3cm
{
Proposition {\bf 2.3.2} (Formules d'échange). --- \it Soient $X$ et $Y$ deux schémas noethériens de caractéristique résiduelles premières à $\ell$, et $f: X \to Y$ un morphisme quasiprojectif. On suppose que $Y$ admet un Module inversible ample. Étant donné $K_Y \in \D^+(Y, \mathbf{Z}_\ell)$, on pose
$$
K_X = \bRd f^!(K_Y), \quad \D_X = \D_{K_X}, \quad \D_Y = \D_{K_Y}.
$$
\begin{itemize}
    \item[a)] Il existe, pour $F \in \D^-(X, \mathbf{Z}_\ell)$, un isomorphisme fonctoriel
        $$
        \bRd f_* \D_X(F) \isomlong \D_Y \Rd f_! (F).
        \leqno{(i)}
        $$
        Si $K_X$ et $K_Y$ sont dualisants et $F \in \D^b_c (X, \mathbf{Z}_\ell)$, on a un isomorphisme fonctoriel
        $$
        \bRd f_! \D_X (F) \isomlong \D_Y \bRd f_* (F).
        \leqno{(ii)}
        $$
    \item[b)] Il existe, pour $\D^-(Y, \mathbf{Z}_\ell)$, un isomorphisme fonctoriel
        $$
        \bRd f^! \D_Y(F) \isomlong \D_X(f^* F).
        \leqno{(i)}
        $$
        Si $K_X$ et $K_Y$ sont dualisants et $F \in \D^b_c (Y, \mathbf{Z}_\ell)$, on a un isomorphisme fonctoriel
        $$
        f^* \D_Y (F) \isomlong \D_X \bRd f^! (F).
        \leqno{(ii)}
        $$
\end{itemize}
}
\vskip .3cm
{\bf Preuve} : Formellement identique à celle de (SGA5 I 1.12), dont d'ailleurs (2.3.2) n'est qu'une paraphrase.
\vskip .3cm
{
Proposition {\bf 2.3.3}. --- \it Soient $X$ un schéma noethérien de caractéristiques résiduelles premières à $\ell$.
\begin{itemize}
    \item[(i)] Si $X$ est régulier, de dimension finie, et satisfait aux conditions de (SGA5 I 3.4.1), le complexe $\mathbf{Z}_\ell$ est dualisant sur $X$.
    \item[(ii)] Si $X$ est régulier excellent de caractéristique 0, et admet un Module inversible ample, alors pour tout morphisme quasiprojectif $f: T \to X$, le complexe $\bRd f^!(\mathbf{Z}_\ell)$ est dualisant sur $T$.
    \item[(iii)] Soient $k$ un corps et $f: X \to S = \Spec(k)$ un morphisme quasiprojectif, avec dim$(X) \leq 2$. Alors $\bRd f^!(\mathbf{Z}_\ell)$ est dualisant sur $X$.
\end{itemize}
}
\vskip .3cm
{\bf Preuve} : Montrons par exemple (ii), les autres assertions se prouvant de fa\c{c}on essentiellement identique, à partir des énoncés correspondants de (SGA5 I). Montrons tout d'abord que si $E \in \D^b_c(T, \mathbf{Z}_\ell)$, alors $\bRd {\cHom}_{\mathbf{Z}_\ell}(E, \bRd f^!(\mathbf{Z}_\ell)) \in \D^b_c(T, \mathbf{Z}_\ell)$.
\vskip .3cm
{
Lemme {\bf 2.3.4}. --- \it Si $F \in \D^+_c(X, \mathbf{Z}_\ell)$, alors $\bRd f^!(F) \in \D^+_c(T, \mathbf{Z}_\ell)$.
}
\vskip .3cm
On se ramène à le voir lorsque $F$ est un $\mathbf{Z}_\ell$-faisceau constructible. Alors cela résulte de la fa\c{c}on habituelle (SGA5 VI) du lemme de Shih, et de l'énoncé analogue pour les $\mathbf{Z}/\ell^{n+1}\mathbf{Z}$--Modules constructibles $(n \in \mathbf{N})$ et pour les $\mathbf{Z}_\ell[T]$--Modules constructibles. 

Comme les hypothèses de (II 2.5 (iii)) sont réalisées (SGA5 I 3.3.1) il résulte du lemme que 
$$
\bRd {\cHom}_{\mathbf{Z}_\ell}(E, \bRd f^! (\mathbf{Z}_\ell)) \in \D^+_c(T, \mathbf{Z}_\ell).
$$
Pour voir qu'il est borné, on peut supposer que $E$ est un $\mathbf{Z}_\ell$-faisceau constructible. Alors, on peut prendre une résolution quasilibre (resp. flasque) de $E$ (resp. $\bRd f^! (\mathbf{Z}_\ell)$) ``canonique'' dans $\mathscr{E}(T, \ell \mathbf{Z}_\ell)$, en ce sens que c'est un système projectif de résolutions quasilibres (resp. flasques) des composants. Comme la dimension quasi-injective des $\bRd f^! (\mathbf{Z}/\ell^{n+1}\mathbf{Z})$ est indépendante de $n$ (preuve de SGA5 I 3.4.3), l'assertion en résulte aussitôt. Il reste à voir que, posant
$$
K = \bRd f^! (\mathbf{Z}_\ell)
$$
le morphisme canonique $E \to \D_K \circ \D_K (E)$ est un isomorphisme. Pour cela, désignant par $u_0: \mathbf{Z} \to \mathbf{Z}/\ell \mathbf{Z}$ le morphisme d'anneaux canonique, il suffit (II 8.2.2) de voir que le morphisme correspondant
$$
\bLd u^*_0 (E) \to \bLd u^*_0 \D_K \circ \D_K (E)
\leqno{(2.3.5)}
$$
est un isomorphisme. Posant $L = \bLd u^*_0 (K)$, les différentes compatibilités exposées en (I 8) montrent que (2.3.5) s'identifie au morphisme canonique
$$
\bLd u^*_0(E) \to \D_L \circ \D_L(\bLd u^*_0(E)).
$$
Comme $L = \bRd f^! (\mathbf{Z}/\ell \mathbf{Z})$ (1.3.1;2), et $\bLd u^*_0(E) \in \D^b_c(X, \mathbf{Z}/\ell \mathbf{Z})$ (II 2.13.1), l'assertion résulte alors de (SGA5 I 3.4.3). On aurait pu également utiliser (II 2.13.9), qui éclaire bien la situation.
\vskip .3cm
{
Proposition {\bf 2.3.6}. --- \it Soient $k$ un corps séparablement clos de caractéristique différente de $\ell$, $S = \Spec(k)$ et $f: X \to S$ et $g: Y \to S$ deux $S$-schémas quasiprojectifs, d'où un diagramme commutatif évident 
\[\begin{tikzcd}
	& {X\times_S Y} \\
	X && Y \\
	& S && {.}
	\arrow["h"', from=1-2, to=3-2]
	\arrow["q", from=1-2, to=2-3]
	\arrow["p"', from=1-2, to=2-1]
	\arrow["f"', from=2-1, to=3-2]
	\arrow["g", from=2-3, to=3-2]
\end{tikzcd}\]
On suppose que les schémas de type fini sur $S$ et de dimension $\leq \text{dim}(X) + \text{dim}(Y)$ sont fortement désingularisables (SGA5 3.1.5) ce qui a lieu notamment si car$(k) = 0$, ou si $k$ est parfait et dim$(X \times_S Y) \leq 2$. On pose
$$
K_X = \bRd f^! (\mathbf{Z}_\ell), \quad K_Y = \bRd g^! (\mathbf{Z}_\ell), \quad K_{X \times_S Y} = \bRd h^! (\mathbf{Z}_\ell).
$$
Ces complexes sont dualisants pour $X, Y$ et $X \times_S Y$ respectivement, et on note $\D_X$, $\D_Y$, $\D_{X \times_S Y}$ les foncteurs dualisants correspondants.

Alors
\begin{itemize}
    \item[a)] Il existe un isomorphisme canonique
    $$
    p^* K_X \boldsymbol{\otimes} q^* K_Y \isomlong K_{X \times_S Y}.
    $$
    \item[b)] Pour $F \in \D^b_c(X, \mathbf{Z}_\ell)$ et $G \in \D^b_c(X, \mathbf{Z}_\ell)$, il existe un isomorphisme canonique fonctoriel
    $$
    p^* \D_X(F) \boldsymbol{\otimes} q^* \D_Y(G) \isomlong D_{X \times_S Y}(p^* F \boldsymbol{\otimes}q^* G).
    $$
\end{itemize}
}
\vskip .3cm
{\bf Preuve} : L'énoncé est une paraphrase de (SGA5 III 3.1). Le fait que $K_X$, $K_Y$ et $K_{X \times_S Y}$ soient dualisants résulte, sur le modèle de la preuve de (2.3.3), de (SGA5 I App.7.5). Utilisant (II 2.13.9), et diverses compatibilités évidentes, les assertions a) et b) résultent par simple passage à la limite des assertions correspondantes (SGA5 III 3.1) pour les $(\mathbf{Z}/\ell^{n+1}\mathbf{Z})$--Modules $(n \in \mathbf{N})$.

Avant d'énoncer la proposition suivante, précisons quelques définitions et notations de (I 8.3). Soit $X$ un schéma noethérien. On définit la catégorie, notée 
$$
\mathbf{Q}_\ell-\fsc(X)
$$
des \emph{$\mathbf{Q}_\ell$-faisceaux} sur $X$, au moyen de la partie multiplicative $\mathbf{Z}_\ell - 0$ de $\mathbf{Z}_\ell$ (I 8.3). Étant donné $d \in \mathbf{Z}$ le $\mathbf{Z}_\ell$-faisceau $\mathbf{Z}_\ell(d)$ définit un $\mathbf{Q}_\ell$-faisceau, noté de préférence
$$
\mathbf{Q}_\ell(d).
$$
Comme on l'a indiqué dans (I 8.3), on étend sans peine aux $\mathbf{Q}_\ell$-faisceaux le formalisme développé pour les $\mathbf{Z}_\ell$-faisceaux. Ainsi, on dit qu'un $\mathbf{Q}_\ell$-faisceau $F$ est \emph{constructible} (resp. \emph{constant tordu constructible}) s'il est isomorphe à l'image d'un $\mathbf{Z}_\ell$-faisceau du même type. La sous-catégorie pleine, notée $\mathbf{Q}_\ell-\fscn(X)$ (reps. $\mathbf{Q}_\ell-\fsct(X)$), de $\mathbf{Q}_\ell-\fsc(X)$ engendrée par les $\mathbf{Q}_\ell$-faisceaux constructibles (resp. constants tordus constructibles) est \emph{exacte}. De même, si $K$ est un objet de $\D(X, \mathbf{Q}_\ell)$, on dit que $K$ est à cohomologie constructible (resp. constante tordu constructible) s'il est isomorphe dans $\D(X, \mathbf{Q}_\ell)$ à un complexe de $\mathbf{Z}_\ell$-faisceaux constructible (resp. constant tordu constructible). On note
$$
\D_c(X, \mathbf{Q}_\ell) \quad (\text{resp}. \D_t(X, \mathbf{Q}_\ell))
$$
la sous-catégorie triangulée pleine de $\D(X, \mathbf{Q}_\ell)$ définie par les complexes à cohomologie constructible (resp. constante tordue constructible).
\vskip .3cm
{
Proposition {\bf 2.3.7} (Dualité locale). --- \it Soient $X$ un schéma quasiprojectif et lisse de dimension $d$ sur un corps séparablement clos, et $x$ un point fermé de $X$. Pour tout $F \in \D^b_c(X, \mathbf{Q}_\ell)$, on a une dualité parfaite entre espaces vectoriels de dimension finie sur $\mathbf{Q}_\ell$
$$
\cExt^{2d-i}_{\mathbf{Q}_\ell}(F, \mathbf{Q}_\ell(d))_x \times \mathrm{H}^i_x(F) \to \mathbf{Q}_\ell.
$$
}
\vskip .3cm
{\bf Preuve} : On convient d'identifier, comme on l'a fait dans l'énoncé, un $\mathbf{Q}_\ell$-faisceau constructible ponctuel et le $\mathbf{Q}_\ell$-espace vectoriel de dimension finie correspondant (SGA5 VI 1.4.3). Soit
$$
i: x \hookrightarrow X
$$
l'immersion fermée canonique. Posons $K_X = \mathbf{Z}_\ell(d)[2d]$ et $K_x = \mathbf{Z}_\ell$; ce sont des complexes dualisants pour $X$ et $x$ respectivement (2.3.3) et l'on a (1.2.3)
$$
\bRd i^!(K_X) \isomlong K_x \quad (\text{canoniquement}).
$$
Notant $\D_X = \D_{K_X}$, $\D_x = \D_{K_x}$, la formule d'induction complémentaire fournit un isomorphisme (2.3.2 b) (ii))
$$
i^* \D_X (F) \isomlong \D_x \bRd i^! (F).
$$
Avec des notations évidentes, on a donc
$$
\D_X(F)_x \isomlong \bRd \cHom_{\mathbf{Q}_\ell}(\bRd \Gamma_x(F), \mathbf{Q}_\ell).
$$
L'assertion en résulte aussitôt, grâce au fait que $\mathbf{Q}_\ell$ est un injectif dans la catégorie des $\mathbf{Q}_\ell$-espaces vectoriels.
\vskip .3cm
{\bf Remarque 2.3.8}. Nous avons seulement donné ici la variante la moins technique du théorème de dualité locale, et renvoyons le lecteur à (SGA5 I 4) pour des énoncés plus généraux.
\vskip .3cm
{\bf 2.4}. Repla\c{c}ons-nous sous les hypothèses préliminaires de (2.3.6). Nous allons indiquer brièvement comment les constructions de (SGA5 III 3) se transposent dans notre cadre et permettent de démontrer un théorème de \emph{Lefschetz-Verdier}.
\vskip .3cm
{
Proposition {\bf 2.4.1}. --- \it Soient $F \in \D^b_c (X, \mathbf{Z}_\ell)$, $G \in \D^b_c (Y, \mathbf{Z}_\ell)$.
\begin{itemize}
    \item[a)] Il existe un isomorphisme canonique fonctoriel
    $$
    \bRd {\cHom}_{\mathbf{Z}_\ell}(p^* F, q^* G) \isomlong p^* \D_X(F) \boldsymbol{\otimes} q^* G.
    $$
    \item[b)] Il existe un accouplement parfait canonique
    $$
    \bRd {\cHom}_{\mathbf{Z}_\ell}(p^* F, \bRd q^! G) \times \bRd {\cHom}_{\mathbf{Z}_\ell}(q^* G, \bRd p^! F) \to K_{X \times_S Y}.
    $$
\end{itemize}
}
\vskip .3cm
{\bf Preuve} : Formellement identique à celle de (SGA5 III 3.2), à partir de (2.3.2) et (2.3.6). On notera que, comme tous les complexes entrant en jeu sont à cohomologie constructible, on dispose sans restriction de l'isomorphisme de Cartan (II 2.6).
\vskip .3cm
{
Proposition {\bf 2.4.2}. --- \it Si $F \in \D^-_c (X, \mathbf{Z}_\ell)$, $G \in \D^+_c (Y, \mathbf{Z}_\ell)$, on a des isomorphismes canoniques 
\begin{itemize}
    \item[(i)] $\bRd h_* \bRd {\cHom}_{\mathbf{Z}_\ell}(p^* F, \bRd q^! G) \isomlong \bRd {\cHom}_{\mathbf{Z}_\ell}(\bRd p_! (F), \bRd q_* (G))$. 
    \item[(ii)] $\Hom_{\mathbf{Z}_\ell}(p^* F, \bRd q^! G) \isomlong \Hom_{\mathbf{Z}_\ell}(\bRd p_! (F), \bRd q_* (G))$.
\end{itemize}
}
\vskip .3cm
{\bf Preuve} : La preuve de l'assertion (i) est formellement identique à celle de (SGA5 III 2.2.1). On en déduit (ii) en appliquant aux deux membres le foncteur $\Hom_{\mathbf{Z}_\ell}(\mathbf{Z}_\ell, .)$ (I 7.4.6).

Supposons maintenant que $X$ et $Y$ soient \emph{propres} sur $S$, et soient $F \in \D^b_c(X, \mathbf{Z}_\ell)_{\text{torf}}$, $G \in \D^b_c(Y, \mathbf{Z}_\ell)_{\text{torf}}$. Les complexes $\bRd f_* (F)$ et $\bRd g_* (G)$ sont \emph{parfaits} : en effet, (1.1.11 (i) et (ii)), ils appartiennent à $\D^b_c (S, \mathbf{Z}_\ell)_{\text{torf}}$ et, comme le corps $k$ est séparablement clos, $\D_c(S, \mathbf{Z}_\ell) = \D_t(S, \mathbf{Z}_\ell)$.

Donnons-nous de plus deux familles $\phi$ et $\psi$ de supports sur $X \times_S Y$. On construit alors comme suit un diagramme
\[\begin{tikzcd}
    & \phantom{XXXXX}
    \mathclap{\mathrm{H}^0_\phi(X \times_S Y, \bRd {\cHom}(p^* F, \bRd q^! G)) \times \mathrm{H}^0_\psi(X \times_S Y, \bRd {\cHom}(q^* G, \bRd p^! F))}
    \phantom{XXXXX}\\
    \phantom{XXXXX}
    \mathclap{\Hom(\bRd f_* F, \bRd g_* G) \times \Hom(\bRd g_* G, \bRd f_* F)}
    \phantom{XXXXX}
    && {\mathrm{H}^0_{\phi \cap \psi}(X \times_S Y, K_{X \times_S Y})} \\
    & {\mathrm{H}^0(S, \mathbf{Z}_\ell)}
    \arrow["{(b)}", from=1-2, to=2-3]
    \arrow["{(d)}", from=2-3, to=3-2]
    \arrow["{(a)}"', from=1-2, to=2-1]
    \arrow["{(c)}"', from=2-1, to=3-2]
  \end{tikzcd}\leqno{(2.4.3)}
\]
Compte tenu de l'isomorphisme (2.4.2 (ii)), la flèche (a) n'est autre que la restriction du support. La flèche (b) résulte sans peine de l'accouplement (2.4.1 b)). La flèche (c) est le cup-produit défini en (II 2.11). Enfin, la flèche s'obtient immédiatement à partir du morphisme trace
$$
\bRd h_* (K_{X \times_S Y}) \to \mathbf{Z}_\ell.
$$
\vskip .3cm
{
Théorème {\bf 2.4.4} (Lefschetz-Verdier). --- \it Le diagramme (2.4.3) ci-dessus est commutatif. 
}
\vskip .3cm
{\bf Preuve} : Utilisant les notations de (II 2.13.9), il suffit de voir que pour tout $n \in \mathbf{N}$, le diagramme déduit de (2.4.3) après application du foncteur $\bLd (\alpha_n)^*$ est commutatif. Comme le foncteur $\bLd (\alpha_n)^*$ commute à toutes les opérations usuelles, l'assertion résulte donc de (SGA 5 III 3.3) pour les $(\mathbf{Z}/\ell^{n+1}\mathbf{Z})$--Modules $(n \in \mathbf{N})$.






% End
%Begin









%%%%%%%%%%%%%%%%%%%%%%%%%%%%%%%%%%%%
\subsection*{3. Formalisme des fonctions $L$.}
\addcontentsline{toc}{subsection}{3. Formalisme des fonctions $L$}

Soit $p$ un nombre premier $\neq \ell$. On note $f$ l'élément de Frobenius $u \mapsto u^p$ $(u \in \overline{\mathbf{F}}_p)$, qui est un générateur topologique du groupe de Galois $\Gal(\overline{\mathbf{F}}_p/\mathbf{F}_p)$.

Étant donné un schéma $X$ de type fini sur $\mathbf{F}_p$, on note $X^\circ$ l'ensemble des points fermés de $X$, et, pour tout $x \in X^\circ$, on désigne par $d(x)$ le degré résiduel de $x$. Choisissant pour tout $x \in X^\circ$ un point géométrique $\overline{x}$ au-dessus de $x$, on rappelle (SGA 5 XV 3) que la fonction $L$ d'un $\mathbf{Q}_\ell$-faisceau constructible $F$ sur $X$ est définie par la formule
$$
L_F(f) = \prod_{x \in X^\circ} (1/\det (1 - f_{F_{\overline{x}}}^{-d(x)}t^{d(x)})).
\leqno{(3.0)}
$$
Grâce à la propriété de multiplicativité de (SGA 5 XV 3.1 a)), on peut prolonger cette définition à $\D^b_c(X, \mathbf{Q}_\ell)$, en posant pour tout $E \in \D^b_c(X, \mathbf{Q}_\ell)$
$$
L_E(f) = \prod_{i \in \mathbf{Z}} (L_{\mathrm{H}^i(E)}(t))^{(-1)^i}.
\leqno{(3.1)}
$$
\vskip .3cm
{
Proposition {\bf 3.2}. --- \it Soit $X$ un schéma de type fini sur $\mathbf{F}_p$.
\begin{enumerate}
    \item[a)] Pour tout triangle exact
    \[\begin{tikzcd}
	& {E''} \\
	{E'} && E
	\arrow[from=2-1, to=2-3]
	\arrow[from=2-3, to=1-2]
	\arrow[dashed, from=1-2, to=2-1]
    \end{tikzcd}\]
    de $\D^b_c(X, \mathbf{Q}_\ell)$, on a 
    $$
    L_E(t) = L_{E'}(t) L_{E''}(t).
    $$
    En particulier, pour tout $m \in \mathbf{Z}$, on a 
    $$
    L_{E[m]}(t) = (L_E(t))^{(-1)^m}.
    $$
    \item[b)] Soient $Y$ un sous-schéma fermé de $X$, et $U = X - Y$ l'ouvert complémentaire. On a 
    $$
    L_E = L_{E | U} L_{E | Y},
    $$
    pour tout $E \in \D^b_c(X, \mathbf{Q}_\ell)$.
    \item[c)] Soit $h: X \to S$ un morphisme de schémas de type fini sur $\mathbf{F}_p$. Pour tout $E \in \D^b_c(X, \mathbf{Q}_\ell)$, on a 
    $$
    L_E = \prod_{s \in S^\circ} L_{E | X_s}.
    $$
\end{enumerate}
}
\vskip .3cm
{\bf Preuve} : Immédiat à partir des assertions analogues pour les objets de cohomologie (SGA5 XV 3.1).
\vskip .3cm
{
Proposition {\bf 3.3}. --- \it Soient $X$ un schéma de type fini sur $\mathbf{F}_p$, $g: X \to \mathbf{F}_p$ le morphisme structural et $E \in \D^b_c(X, \mathbf{Q}_\ell)$. Alors
$$
L_E = L_{\bRd g_!(E)}.
$$
En particulier, $L_E$ est une fraction rationnelle.
}
\vskip .3cm
{\bf Preuve} : On peut supposer que $E$ est un $\mathbf{Q}_\ell$-faisceau constructible, et alors l'assertion n'est autre que (SGA5 XV 3.2).
\vskip .3cm
{
Corollaire {\bf 3.4}. --- \it Soit $h: X \to S$ un morphisme de schémas de type fini sur $\mathbf{F}_p$. Pour tout $E \in \D^b_c(X, \mathbf{Q}_\ell)$, on a
$$
L_E = L_{\bRd h_! (E)}.
$$
}
\vskip .3cm
Nous allons maintenant déduire de (3.3) une \emph{équation fonctionnelle} pou les fonctions $L$, du moins si $X$ est projectif sur $\mathbf{F}_p$.
\vskip .3cm
{
Définition {\bf 3.5}. --- \it Soient $g: X \to \mathbf{F}_p$ un schéma de type fini sur $\mathbf{F}_p$, et $\overline{X} = X \times_{\mathbf{F}_p} \overline{\mathbf{F}}_p$. Pour tout $E \in \D^b_c(X, \mathbf{Q}_\ell)$, on pose  
$$
\chi(E) = \rang (\bRd g_! E) = \sum_{i \in \mathbf{Z}} (-1)^i [\mathrm{H}^i_c (\overline{X}, \overline{E}):\mathbf{Q}_\ell],
$$
$$
\delta(E) = \det (\bRd g_! (E)) = \prod_{i \in \mathbf{Z}} (\det f_{\mathrm{H}^i_c(\overline{X}, \overline{E})})^{(-1)^i},
$$
où $\overline{E}$ désigne l'image inverse de $E$ au-dessus de $\overline{X}$.
}
\vskip .3cm
D'après les propriétés d'additivité et de multiplicativité respectives de la trace et du déterminant dans la catégorie des $\mathbf{Q}_\ell$-espaces vectoriels de dimension finie, il es clair que pour tout triangle exact
\[\begin{tikzcd}
	& {E''} \\
	{E'} && {E,}
	\arrow[from=2-1, to=2-3]
	\arrow[from=2-3, to=1-2]
	\arrow[dashed, from=1-2, to=2-1]
\end{tikzcd}\]
on a 
$$
\chi(E) = \chi(E') + \chi(E'').
\leqno{(3.5.1)}
$$
$$
\delta(E) = \delta(E')  \delta(E'').
\leqno{(3.5.2)}
$$
En particulier, pour tout $m \in \mathbf{Z}$ et tout $E \in \D^b_c(X, \mathbf{Q}_\ell)$,
$$
\chi(E[m]) = (-1)^m \chi(E) \quad \text{et} \quad \delta (E[m]) = (\delta(E))^{(-1)^m}.
$$
\vskip .3cm
{
Proposition {\bf 3.6}. --- \it Soit $g: X \to \mathbf{F}_p$ un schéma projectif sur $\mathbf{F}_p$. On pose $K_X = \bRd g^! (\mathbf{Q}_\ell)$, et $\D_X = \bRd {\cHom}_{\mathbf{Q}_\ell}(. , K_X)$. Alors, pour tout $E \in \D^b_c(X, \mathbf{Q}_\ell)$, on a l'identité
$$
L_{\D_X(E)}(t) = (-t)^{-\chi(E)} \delta(E) L_E(t^{-1}).
$$
}
\vskip .3cm
{\bf Preuve} : Le second membre a un sens d'après (3.3). Posons $S = \Spec \mathbf{F}_p$ et $\D_S = \bRd {\cHom}_{\mathbf{Q}_\ell}(. , \mathbf{Q}_\ell)$. D'après (2.3.2 a)), on a 
$$
\bRd g_* (\D_X E) \isomlong \D_S \bRd g_* (E),
$$
donc (3.3)
$$
L_{\D_X}(E) = L_{\D_S(\bRd g_* (E))}.
$$
Comme $L_E = L_{\bRd g_* (E)}$ (3.3), l'assertion résultera du lemme suivant
\vskip .3cm
{
Lemme {\bf 3.7}. --- \it Si $F \in \D^b_c(S, \mathbf{Q}_\ell)$, on a :
$$
L_{\D_S(F)}(t) = (-t)^{\chi(F)} \delta(F) L_F(t^{-1}).
$$
}
\vskip .3cm
D'après les propriétés d'additivité et de multiplicativité (3.5.1) et (3.5.2), on peut supposer que $F \in \mathbf{Q}_\ell-\fscn (S)$. Alors $F$ correspond (SGA5 VII 1.4.2) à un $\mathbf{Q}_\ell$-espace vectoriel de dimension finie $V$ muni d'une opération continue $f_V$ du Frobenius, et le $\mathbf{Q}_\ell$-faisceau $\D_S (F) = {\cHom}_{\mathbf{Q}_\ell}(F, \mathbf{Q}_\ell)$ correspond (II 1.26) au $\mathbf{Q}_\ell$-espace vectoriel $V^{\vee}$ muni de l'opération continue $(f^{\vee}_V)^{-1}$ du Frobenius. Il suffit alors de montrer que, étant donnés un corps $K$, un $K$-espace vectoriel de dimension finie $V$ et un automorphisme $u$ de $V$, on a l'identité
$$
1/\det(1 - u^{-1} t) = (-t)^{-\text{dim}(V)} \det(u)/\det(1 - ut^{-1})
\leqno{(3.8)}
$$
dans $K(t)$. On peut pour cela supposer $K$ algébriquement clos, donc $u$ triangulable, puis, grâce aux propriétés de multiplicativité du déterminant, que dim$(V) = 1$. Alors $u$ est l'homothétie définie par un scalaire non nul $\lambda$, et (3.8) est l'identité évidente
$$
1/(1-(t/\lambda)) = (-\lambda/t)/(1-(\lambda/t)).
$$
Bien entendu, la formule (3.6) ne présente d'intérêt en pratique que si l'on dispose d'une expression simple pour $\D_X(E)$. Nous allons maintenant donner des cas où il en est ainsi.
\vskip .3cm
{
Proposition {\bf 3.9}. --- \it On suppose $X$ quasiprojectif, lisse et purement de dimension $n$ sur $\mathbf{F}_p$. Posant pour tout $E \in \D^b_c(X, \mathbf{Q}_\ell)$
$$
E^\vee = {\cHom}^{\bullet}_{\mathbf{Q}_\ell}(E, \mathbf{Q}_\ell),
$$
on a un isomorphisme
$$
\D_X(E) \isom E^\vee (n) [2n]
$$
dans chacun des cas suivants
\begin{enumerate}
    \item[(i)] $E \in \D^b_t(X, \mathbf{Q}_\ell)$
    \item[(ii)] $X$ est une courbe, et $E$ est un $\mathbf{Q}_\ell$-faisceau constructible de la forme $i_*(M)$, où $i: U \hookrightarrow X$ est l'inclusion d'un ouvert dense de $X$ et $M \in \mathbf{Q}_\ell-\fsct(U)$.
\end{enumerate}
}
\vskip .3cm
{\bf Preuve} : Comme $\D_X(E) = \bRd {{\cHom}}_{\mathbf{Q}_\ell}(E, \mathbf{Q}_\ell(n))[2n]$, le cas (i) résulte du lemme suivant.
\vskip .3cm
{
Lemme {\bf 3.9.1}. --- \it Étant donnés un schéma noethérien $X$, $F \in \mathbf{Q}_\ell-\fsct(X)$ et $G \in \mathbf{Q}_\ell-\fscn(X)$, on a:
$$
{\cExt}^j_{\mathbf{Q}_\ell}(F, G) = 0 \quad (j \geq 1).
$$
}
\vskip .3cm
Il s'agit de voir que si $F \in \mathbf{Z}_\ell-\fsct(X)$ et $G \in \mathbf{Z}_\ell-\fscn(X)$, les $\mathbf{Z}_\ell$-faisceaux $\cExt^j_{\mathbf{Z}_\ell}(F, G)$ $(j \geq 1)$ sont annulés par une puissance de $\ell$. D'après (I 6.4.2) et (II 1.2.1), on peut, quitte à se restreindre à des parties localement fermées convenables de $X$, supposer que $G \in \mathbf{Z}_\ell-\fsct(X)$. Alors, compte tenu de (II 1.26), l'assertion résulte de l'assertion analogue, bien connue, pour les $\mathbf{Z}_\ell$--Modules de type fini. Montrons (ii). 

Il s'agit de voir que
$$
P^j = {\cExt}^j_{\mathbf{Q}_\ell}(E, \mathbf{Q}_\ell(1)) = 0 \quad (j \geq 1).
$$
Comme $M$ est constante tordu constructible, il résulte du cas (i) que $P^j | U = 0$. Il nous suffit donc de voir que pour tout point fermé $x$ de $Y = X - U$ et tout point géométrique $\overline{x}$ au-dessus de $x$, on a $P^j_x = 0$. Le pendant pour les $\mathbf{Q}_\ell$-faisceaux de la variante (SGA5 I 4.6.2) du théorème de dualité locale fournit un accouplement parfait
$$
{\cExt}^j_{\mathbf{Q}_\ell}(E, \mathbf{Q}_\ell(1)) \times \bH^{2-j}_{\overline{x}}(E) \to \mathbf{Q}_{\ell},
\leqno{(3.9.2)}
$$
avec (SGA5 I 4.5.1)
$$
\bH^{2-j}_{\overline{x}}(E) = (\bH^{2-j}_x(E))_{\overline{x}}.
$$
Comme le morphisme d'adjonction canonique
$$
E \to i_* i^* (E)
$$
est un isomorphisme, il résulte de la première suite exacte de (SGA4 V 4.5) que 
$$
\bH^0_x(E) = \bH^1_x(E) = 0,
$$
d'où aussitôt le résultat annoncé.

Ceci dit, lorsque $X$ est projectif sur $\mathbf{F}_\ell$, la formule (3.6) prend la forme
$$
L_{E^\vee}(p^{-n} t) = (-1)^{- \chi (E)} \delta(E) L_E (t^{-1}),
\leqno{(3.10)}
$$
dans chacun des cas de (3.9). Compte tenu de (3.2 a)), cela résulte du lemme suivant.
\vskip .3cm
{
Lemme {\bf 3.11}. --- \it Soient $X$ un schéma de type fini sur $\mathbf{F}_p$, et $F \in \D^b_c(X, \mathbf{Q}_\ell)$. Posant $F(j) = F \otimes \mathbf{Q}_\ell (j)$ $(j \in \mathbf{Z})$, on a la relation
$$
L_{F(j)}(t) = L_F (p^{-j} t).
$$
}
\vskip .3cm
D'après les propriétés de multiplicativité (3.2 a)), on peut pour le voir supposer que $F$ est un $\mathbf{Q}_\ell$-faisceau constructible; alors, comme le Frobenius opère sur $\mathbf{Q}_\ell(j) \isom \mathbf{Q}_\ell$ (non canoniquement) par l'homothétie de rapport $p^{-j}$, l'assertion est immédiate sur la définition (3.0).

Supposons maintenant qu'on ait de plus un isomorphisme
$$
E^\vee \isomlong E(\rho) \quad \text{pour un}~\rho \in \mathbf{Z}.
$$
Alors la formule (3.10) prend la forme
$$
L_E (p^{-n-\rho}t) = (-t)^{-\chi(E)}\delta(E) L_E(t^{-1}),
$$
ou encore, après avoir posé $q = n + \rho$ et fait le changement de variable $t \mapsto t^{-1}$,
$$
L_E(1/qt) = (-t)^{\chi(E)}\delta(E)L_E(t).
\leqno{(3.12)}
$$
\vskip .3cm
{\bf Remarque 3.13}. Sous les hypothèses de (3.9), l'existence d'un tel entier $p$ est assurée dans les cas suivants
\begin{enumerate}
    \item[cas~(i)] $E \isomlong \mathbf{Q}_\ell(m)$ \quad pour un $m \in \mathbf{Z}$, et alors $\rho = -2m$.
    \item[cas~(ii)] $M \isomlong \mathbf{Q}_\ell(m)$ \quad pour un $m \in \mathbf{Z}$, et alors $\rho = -2m$.
\end{enumerate}
(Pour ce dernier cas, il est immédiat que
$$
i_*(M^\vee) \isom (i_*(M))^\vee. \quad )
$$
Explicitons enfin une relation importante entre les entiers $\chi(E)$ et $\delta(E)$.
\vskip .3cm
{
Proposition {\bf 3.14}. --- \it Soient $X$ un schéma projectif et lisse purement de dimension $n$ sur $\mathbf{Z}_p$ et $E \in \D^b_c(X, \mathbf{Q}_\ell)$. On suppose qu'il existe un entier $m$ tel que 
$$
\D_X(E) \isomlong E(m),
$$
et on pose $q= p^m$. Alors, on a l'égalité
$$
\delta(E)^2 = q^{\chi(E)}.
$$
}
\vskip .3cm
{\bf Preuve} : La substitution $t \mapsto 1/qt$ dans (3.12) fournit l'équation fonctionnelle
$$
L_E(t) = (-1/qt)^{\chi(E)}\delta(E)L_E(1/qt).
\leqno{(3.12~\text{bis})}
$$
Multipliant (3.12) et (3.12 bis) membre à membre, on obtient l'identité 
$$
L_E(t)L_E(1/qt) = q^{-\chi(E)}(\delta(E))^2 L_E(t) L_E(1/qt),
$$
d'où aussitôt la relation désirée, compte tenu du fait que $L_E$ n'est pas identiquement nulle, comme il est clair sur sa définition (3.0).










% End

%%%%%%%%%%%%%%%%%%%%%%%%%%%%%%%%%%%%%%%%%





\newpage
%\thispagestyle{empty}
\mbox{}
\thispagestyle{empty}
\begin{tikzpicture}[remember picture,overlay]
    \draw[line width=10pt,color=DarkKhaki]
        ([shift={(-0.5\pgflinewidth,-0.5\pgflinewidth)}]current page.north west)
        rectangle
        ([shift={(0.5\pgflinewidth,0.5\pgflinewidth)}]current page.south east);
\end{tikzpicture}



\end{document}




%End
